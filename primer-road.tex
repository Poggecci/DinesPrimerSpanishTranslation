
\renewcommand{\domkap}{\sort{\brcolor{Roads}}}
\nbbbbbb{Road Transport}\label{p-ch:Road Transport}\label{Chapter:Road Transport}
\pos{\minitoc}{}

\bbbbb{The Road Transport Domain}

\begynd
\pind Our universe of discourse in this \pos{chapter}{example} is \nyl the road
transport domain. 
\pind Not a specific one, but ``a generic road transport domain''.
\afslut

\fotoiv{50mm}{rts1}{rts2}{rts3}{rts4}{Road System
  Components}{rts-figs}

\nbbbb{Naming}
 
%\RSLatex
%   type RTS
%\endRSLatex
\bp
\>\ \kw{type} RTS
\ep
\vspace*{2mm}

\nbbbb{Rough Sketch}
\begynd
\pind The generic road transport domain that we have in mind consists of
\begynd
\pind a road net (aggregate) and
\pind an aggregate of vehicles
\pind such that the road net serves to convey vehicles.
\afslut
\pind We consider the road net to consist of
\begynd
\pind hubs, i.e., street intersections, \nyl or just street segment
      connection points, and
\pind links, i.e., street segments between adjacent hubs.
\afslut
\pos{\psno}{\mnewfoil}
\pind We consider the aggregate of vehicles to include
\begynd
\pind in addition to vehic\ysfchg{l}es, i.e., automobiles,
\pind a department of motor vehicles (DMVs),
\pind zero or more bus companies, each with zero, one or more buses, and
\pind vehicle associations, each with 
\begynd
\pind zero, one or more members 
\pind who are owners of zero, one or more vehicles\finalchanges{\footnotemark}\ \eox\
\afslut
\afslut
\afslut
\footnotetext{\LLLL \finalchanges{This ``rough'' narrative fails to narrate what
  \pos{hubs, links, vehicles, DMVs, bus companies, buses and  vehicle
    associations are. In presenting it here, as we are, we rely on
    your a priori understanding of these terms. But that is
    dangerous\,! The danger, if we do not painstakingly narrate and
    formalise what we mean by all these terms, then readers (software
    designers, etc.) may make erroneous assumptions.}{...}}}
\noindent

\nbbbbb{External Qualities}\HHHH\label{p-chapter3-external-qualities}

\noindent
\sort{A Road Transport System, I -- Manifest External Qualities:}{ 
\begynd
\pind Our intention is that the manifest external qualities of a road
      transport system are those of its
\begynd
\pind roads, 
\begynd
\pind their \sort{hub}s\pos{\footnotemark}{} i.e., road (or street) intersections, and
\pind their \sort{link}s, i.e., the roads (streets) between hubs, and
\afslut
\pind \sort{vehicle}s, i.e., automobiles -- that ply the roads --
\begynd
\pind the buses, trucks, private cars, bicycles, etc. \eox\
\afslut
\afslut
\afslut}\pos{\footnotetext{\LLLL We have \sort{highlighted}
  certain endurant sort names -- as they will re-appear in rather many
  upcoming examples.}}{}

\nbbbb{A Road Transport System, II -- Abstract External Qualities}{%
\begynd
\pind Examples of what could be considered abstract external qualities
      of a road transport domain are:
\begynd
\pind the aggregate of all hubs and all links, 
\pind the aggregate of all buses, say into bus companies,
\pind the aggregate of all bus companies into public transport, and
\pind the aggregate of all vehicles into a department of vehicles.
\afslut
\pind Some of these aggregates may, at first be treated as abstract.
\pind Subsequently, in our further analysis \& description we may
      decide \nyl to consider some of them as concretely manifested in, for
      example, actual
\begynd
\pind departments of roads.
\afslut
\afslut}

\nbbbb{Transport System Structure}{%
\begynd
\pind A transport system is modeled as structured into
\begynd
\pind a \sfsl{road net structure} and 
\pind an \sfsl{automobile structure}.
\afslut
\pind The \sfsl{road net structure} is then structured as a pair:
\begynd
\pind a \sfsl{structure of hubs} and
\pind a \sfsl{structure of links}.
\afslut
\pind These latter structures are then modeled as set of hubs, \nyl
      respectively links.
\afslut
}

\mnewfoil

\begynd
\pind We could have modeled the road net \sfsl{structure}
\begynd
\pind as a \sfsl{composite part}
\pind with \sfsl{unique identity, mereology} and \sfsl{attributes}
\pind which could then serve to model 
\pind a \texttt{road net authority}.
\afslut
\pind And we could have modeled the automobile \sfsl{structure}
\begynd
\pind as a \sfsl{composite part}
\pind with \sfsl{unique identity, mereology} and \sfsl{attributes}
\pind which could then serve to model 
\pind a \texttt{department of vehicles} \eox
\afslut
\afslut

\nbbbb{Atomic Road Transport Parts}\HHHH
\noindent
\begynd
\pind From one point of view all of the following can be considered
      atomic parts:
\begynd
\pind hubs, 
\pind links\footnote{\LLLL Hub {\IS} street intersection; link
      {\IS} street segments with no intervening hubs.}, and
\pind automobiles.
\afslut
\afslut


\nbbbb{Compound Road Transport Parts}
\bbb{The Composites}\HHHH

\pos{\begin{multicols}{2}}{}
\begin{enumerate}\setei
\item \label{p-x-pts-000} There is the \sfsl{universe of discourse},
      \textsf{UoD}.
\savei\end{enumerate}
\noindent
It is structured into \ptsidxi{UoD}{x-pts-000}
\begin{enumerate}\setei  
\item \label{p-x-pts-020} a \sfsl{road net}, \textsf{RN}, and
  \ptsidxi{RN}{x-pts-020} 
\item \label{p-x-pts-030} a \sfsl{fleet of vehicles},
  \textsf{FV}. \ptsidxi{FV}{x-pts-030} 
  \savei\end{enumerate}
\pos{\end{multicols}}{}
\noindent
Both are structures. \dotfill %\footsize
\pos{\begin{multicols}{2}}{}
%\RSLatex
%type
%&\ref{p-x-pts-000}&  UoD  axiom all uod:UoD :- is_structure(uod).
%&\ref{p-x-pts-020}&  RN   axiom all rn:RN :- is_structure(rn).
%&\ref{p-x-pts-030}&  FV   axiom all fv:FV :- is_structure(fv).
%value
%&\ref{p-x-pts-020}&  obs_RN: UoD -> RN &\psfidxi{obs\_RN}{x-pts-020}&
%&\ref{p-x-pts-030}&  obs_FV: UoD -> FV &\psfidxi{obs\_FV}{x-pts-030}\eox&
%\endRSLatex
\bp
\kw{type}\\
\ref{p-x-pts-000}\ \ UoD\ \ \kw{axiom} {\ALL} uod:UoD {\RDOT} is\_structure(uod).\\
\ref{p-x-pts-020}\ \ RN\ \ \ \kw{axiom} {\ALL} rn:RN {\RDOT} is\_structure(rn).\\
\ref{p-x-pts-030}\ \ FV\ \ \ \kw{axiom} {\ALL} fv:FV {\RDOT} is\_structure(fv).\\
\kw{value}\\
\ref{p-x-pts-020}\ \ obs\_RN: UoD {\RIGHTARROW} RN \psfidxi{obs\_RN}{x-pts-020}\\
\ref{p-x-pts-030}\ \ obs\_FV: UoD {\RIGHTARROW} FV \psfidxi{obs\_FV}{x-pts-030}\eox
\ep
\pos{\end{multicols}}{}
\normalsize\rm 
%%  LocalWords:  UoD pts FV uod rn fv DMVs priori formalise endurant

\mnewfoil

\hDBfigure{rts}{\pos{6}{10}cm}{A Road Transport System Compounds and Structures}{rts.fig}

\nbbb{The Part Parts}\HHHH
\begin{enumerate}\setei
\item \label{p-x-pts-060} The structure of hubs is a set, \textsf{sH}, \ptsidxi{sH}{x-pts-060} 
                        of atomic hubs, \textsf{H}. \ptsidxi{H}{x-pts-060} 
\item \label{p-x-pts-070} The structure of links is a set, \textsf{sL}, of atomic
                        links, \textsf{L}.  \ptsidxi{sL}{x-pts-070}  \ptsidxi{L}{x-pts-070} 
\item \label{p-x-pts-080} The structure of buses is a set, \textsf{sBC},
                        of composite bus companies, \textsf{BC}.
                        \ptsidxi{sBC}{x-pts-080}   \ptsidxi{BC}{x-pts-080}
\item \label{p-x-pts-085} The composite bus companies, \textsf{BC}, are sets of
                        buses, \textsf{sB}. \ptsidxi{BC}{x-pts-085} \ptsidxi{B}{x-pts-085} 
\item \label{p-x-pts-090} The structure of private automobiles is a set, \textsf{sA},
                        of atomic automobiles, \textsf{A}.\ptsidxi{sA}{x-pts-090} \ptsidxi{A}{x-pts-090} 
\savei\end{enumerate}\footsize\HHHH
\pos{\psno}{\mnewfoil} 
%\RSLatex
%type
%&\ref{p-x-pts-060}&  H, sH = H-set  axiom all h:H :- is_atomic(h)
%&\ref{p-x-pts-070}&  L, sL = L-set  axiom all l:L :- is_atomic(l)   
%&\ref{p-x-pts-080}&  BC, BCs = BC-set  axiom all bc:BC :- is_composite(bc)
%&\ref{p-x-pts-085}&  B, Bs = B-set  axiom all b:B :- is_atomic(b)
%&\ref{p-x-pts-090}&  A, sA = A-set  axiom all a:A :- is_atomic(a)
%value
%&\ref{p-x-pts-060}&  obs_sH: SH -> sH &\psfidxi{obs\_sH}{x-pts-060}&
%&\ref{p-x-pts-070}&  obs_sL: SL -> sL      &\psfidxi{obs\_sL}{x-pts-070}&
%&\ref{p-x-pts-080}&  obs_sBC: SBC -> BCs   &\psfidxi{obs\_sBC}{x-pts-080}&
%&\ref{p-x-pts-085}&  obs_Bs: BCs -> Bs    &\psfidxi{obs\_Ms}{x-pts-085}&
%&\ref{p-x-pts-090}&  obs_sA: SA -> sA &\psfidxi{obs\_sA}{x-pts-090}\eox&
%\endRSLatex \RSLatex
\bp
\kw{type}\\
\ref{p-x-pts-060}\ \ H, sH {\EQ} H\kw{-set}\ \ \kw{axiom} {\ALL} h:H {\RDOT} is\_atomic(h)\\
\ref{p-x-pts-070}\ \ L, sL {\EQ} L\kw{-set}\ \ \kw{axiom} {\ALL} l:L {\RDOT} is\_atomic(l)\ \ \ \\
\ref{p-x-pts-080}\ \ BC, BCs {\EQ} BC\kw{-set}\ \ \kw{axiom} {\ALL} bc:BC {\RDOT} is\_composite(bc)\\
\ref{p-x-pts-085}\ \ B, Bs {\EQ} B\kw{-set}\ \ \kw{axiom} {\ALL} b:B {\RDOT} is\_atomic(b)\\
\ref{p-x-pts-090}\ \ A, sA {\EQ} A\kw{-set}\ \ \kw{axiom} {\ALL} a:A {\RDOT} is\_atomic(a)\\
\kw{value}\\
\ref{p-x-pts-060}\ \ obs\_sH: SH {\RIGHTARROW} sH \psfidxi{obs\_sH}{x-pts-060}\\
\ref{p-x-pts-070}\ \ obs\_sL: SL {\RIGHTARROW} sL\ \ \ \ \ \ \psfidxi{obs\_sL}{x-pts-070}\\
\ref{p-x-pts-080}\ \ obs\_sBC: SBC {\RIGHTARROW} BCs\ \ \ \psfidxi{obs\_sBC}{x-pts-080}\\
\ref{p-x-pts-085}\ \ obs\_Bs: BCs {\RIGHTARROW} Bs\ \ \ \ \psfidxi{obs\_Ms}{x-pts-085}\\
\ref{p-x-pts-090}\ \ obs\_sA: SA {\RIGHTARROW} sA \psfidxi{obs\_sA}{x-pts-090}\eox
\ep
\normalsize\rm
%%  LocalWords:  sH pts sL busses sBC sB sA BCs bc Bs xpui et nique

\nbbbb{The Transport System State}

\pos{\vspace*{1.5mm}\smallish}{} \LLLL\HHHH

\begin{enumerate}\setei
\item \label{p-srares-000} Let there be given a universe of discourse,
                         $rts$ \ysfchg{(road transport system)}. It is an example of a state.
\savei\end{enumerate}
\noindent From that state we can calculate other states.

\begin{enumerate}\setei
\item \label{p-srares-020} The set of all hubs, $hs$. \vidxi{$hs$}{srares-020} 
\item \label{p-srares-030} The set of all links, $ls$. \vidxi{$ls$}{srares-030} 
\item \label{p-srares-040} The set of all hubs and links, $hls$. \vidxi{$hls$}{srares-040} 
\item \label{p-srares-050} The set of all bus companies, $bcs$. \vidxi{$bcs$}{srares-050} 
\item \label{p-srares-060} The set of all buses, $bs$. \vidxi{$bs$}{srares-060}
\item \label{p-srares-080} The set of all private automobiles, $as$. \vidxi{$as$}{srares-080} 
\item \label{p-srares-090} The set of all parts, $ps$.\vidxi{$ps$}{srares-090} 
\savei\end{enumerate}\pos{\footsize}{}
\pos{\psno}{\mnewfoil}
%\RSLatex
%value
%&\ref{p-srares-000}&   &$rts$&:UoD &\cf{srares-000}&
%&\ref{p-srares-020}&   &$hs$&:H-set is:H-set is obs_sH(obs_SH(obs_RN(&$rts$&)))
%&\ref{p-srares-030}&   &$ls$&:L-set is:L-set is obs_sL(obs_SL(obs_RN(&$rts$&)))
%&\ref{p-srares-040}&   &$hls$&:(H|L)-set is &$hs$&union&$ls$& 
%&\ref{p-srares-050}&   &$bcs$&:BC-set is obs_BCs(obs_SBC(obs_FV(obs_RN(&$rts$&))))
%&\ref{p-srares-060}&   &$bs$&:B-set is union{obs_Bs(bc)|bc:BC:-bc isin &$bcs$&} 
%&\ref{p-srares-080}&   &$as$&:A-set is obs_BCs(obs_SBC(obs_FV(obs_RN(&$rts$&))))  
%&\ref{p-srares-090}&   &$ps$&:(UoB|H|L|BC|B|A)-set is &$rts$&union&$hls$&union&$bcs$&union&$bs$&union&$as$&  
%\endRSLatex    
\bp
\kw{value}\\
\ref{p-srares-000}\ \ \ $rts$:UoD \cf{srares-000}\\
\ref{p-srares-020}\ \ \ $hs$:H\kw{-set} {\IS}:H\kw{-set} {\IS} obs\_sH(obs\_SH(obs\_RN($rts$)))\\
\ref{p-srares-030}\ \ \ $ls$:L\kw{-set} {\IS}:L\kw{-set} {\IS} obs\_sL(obs\_SL(obs\_RN($rts$)))\\
\ref{p-srares-040}\ \ \ $hls$:(H{\BAR}L)\kw{-set} {\IS} $hs${\UNION}$ls$ \\
\ref{p-srares-050}\ \ \ $bcs$:BC\kw{-set} {\IS} obs\_BCs(obs\_SBC(obs\_FV(obs\_RN($rts$))))\\
\ref{p-srares-060}\ \ \ $bs$:B\kw{-set} {\IS} {\UNION}{\LBRACE}obs\_Bs(bc){\BAR}bc:BC{\RDOT}bc {\ISIN} $bcs${\RBRACE} \\
\ref{p-srares-080}\ \ \ $as$:A\kw{-set} {\IS} obs\_BCs(obs\_SBC(obs\_FV(obs\_RN($rts$))))\ \ \\
\ref{p-srares-090}\ \ \ $ps$:(UoB{\BAR}H{\BAR}L{\BAR}BC{\BAR}B{\BAR}A)\kw{-set} {\IS} $rts${\UNION}$hls${\UNION}$bcs${\UNION}$bs${\UNION}$as$\ \ 
\ep

\dbeat{%%%%%%%%%%%%%%%%%%%%%%%%%%%%%%%%%%%%%%%%%%%%%%%%%%%%%%%%%%%
\noindent
\bmcolor{Indexed States:}
\dotfill\label{p-example-states}\pos{\footnotesize}{\HHHH}

\noindent
\begynd
\pind We shall
\afslut
\begin{enumerate}\setei
\item \label{p-x.state-000} index bus companies,
\item \label{p-x.state-010} index buses, and
\item \label{p-x.state-020} index automobiles
\savei\end{enumerate} using the unique identifiers of these
parts.\pos{\smallish}{\LLLL}
\pos{\psno}{\mnewfoil}
%\RSLatex
%type
%&\ref{p-x.state-000}&  BC&$_{ui}$&
%&\ref{p-x.state-010}&  B&$_{ui}$&
%&\ref{p-x.state-020}&  A&$_{ui}$& 
%value 
%&\ref{p-x.state-000}&  &$ibcs$&:BC&$_{ui}&$-set is { bc&$_{ui}$& | bc:BC,bc:BC&$_{ui}$&:BC&$_{ui}$& :- &bc&isin&$bcs$&/\&$ui$&=uid_BC(bc) }
%&\ref{p-x.state-010}&  &$ibs$&:B&$_{ui}&$-set is { b&$_{ui}$& | b:B,b:B&$_{ui}$&:B&$_{ui}$& :- &b&isin&$bs$&/\&$ui$&=uid_B(b) }
%&\ref{p-x.state-020}&  &$ias$&:A&$_{ui}&$-set is { a&$_{ui}$& | a:A,a:A&$_{ui}$&:A&$_{ui}$& :- &a&isin&$as$&/\&$ui$&=uid_A(a) }
%\endRSLatex 
\bp
\kw{type}\\
\ref{p-x.state-000}\ \ BC$_{ui}$\\
\ref{p-x.state-010}\ \ B$_{ui}$\\
\ref{p-x.state-020}\ \ A$_{ui}$ \\
\kw{value} \\
\ref{p-x.state-000}\ \ $ibcs$:BC$_{ui}$\kw{-set} {\IS} {\LBRACE} bc$_{ui}$ {\BAR} bc:BC,bc:BC$_{ui}$:BC$_{ui}$ {\RDOT} bc{\ISIN}$bcs${\WEDGE}$ui${\EQ}uid\_BC(bc) {\RBRACE}\\
\ref{p-x.state-010}\ \ $ibs$:B$_{ui}$\kw{-set} {\IS} {\LBRACE} b$_{ui}$ {\BAR} b:B,b:B$_{ui}$:B$_{ui}$ {\RDOT} b{\ISIN}$bs${\WEDGE}$ui${\EQ}uid\_B(b) {\RBRACE}\\
\ref{p-x.state-020}\ \ $ias$:A$_{ui}$\kw{-set} {\IS} {\LBRACE} a$_{ui}$ {\BAR} a:A,a:A$_{ui}$:A$_{ui}$ {\RDOT} a{\ISIN}$as${\WEDGE}$ui${\EQ}uid\_A(a) {\RBRACE}
\ep
}%%%%%%%%%%%%%%%%%%%%%%%%%%%%%%%%%%%%%%%%%%%%%%%
\normalsize\rm
%%  LocalWords:  rts hs srares hls bcs busses bs ui bc ps UoD sH sL
%%  LocalWords:  BCs SBC FV ibcs uid ibs ias UoB ub dentifiers uic ap


\nbbbbb{Internal Qualities}

\bbbb{Unique Identifiers}\label{p-ch2.Unique Identifiers}\label{p-ch2.uid.1}

\pos{\begin{multicols}{2}}{}
\smallish\LLLL\HHHH\begin{enumerate}\setei
\item \label{p-x-ui-000} We assign unique identifiers to all parts.
\item \label{p-x-ui-004} By a road identifier we shall mean a link or a
                       hub identifier.
  \utidxi{R\_UI}{x-ui-004}\utidxi{L\_UI}{x-ui-004}\utidxi{H\_UI}{x-ui-004}   
\item \label{p-x-ui-006} By a vehicle identifier we shall mean a bus or an
                       automobile identifier.
                       \utidxi{V\_UI}{x-ui-006}\utidxi{B\_UI}{x-ui-006}\utidxi{A\_UI}{x-ui-006}  
\item \label{p-x-ui-010} Unique identifiers uniquely identify all parts. 
\begin{enumerate}
\item \label{p-x-ui-020} All hubs have distinct [unique] identifiers.
\item \label{p-x-ui-030} All links have distinct identifiers. 
\item \label{p-x-ui-040} All bus companies have distinct
                       identifiers. 
\item \label{p-x-ui-050} All buses of all bus companies have distinct 
                       identifiers.
\item \label{p-x-ui-060} All automobiles have distinct identifiers.
\item \label{p-x-ui-070} All parts have  distinct identifiers.
\end{enumerate}
\savei\end{enumerate}
\pos{\end{multicols}}{}
\pos{\psno}{\mnewfoil}\HHHH
\pos{\begin{multicols}{2}}{}
%\RSLatex
%type  
%&\ref{p-x-ui-000}&   H_UI, L_UI, BC_UI, B_UI, A_UI &\utidxi{H\_UI}{x-ui-000}\utidxi{L\_UI}{x-ui-006}\utidxi{BC\_UI}{x-ui-006}\utidxi{B\_UI}{x-ui-006}\utidxi{A\_UI}{x-ui-006}&
%&\ref{p-x-ui-004}&   R_UI = H_UI | L_UI &\utidxi{R\_UI{\EQ}H\_UI\protect{\BAR}L\_UI}{x-ui-004}&
%&\ref{p-x-ui-006}&   V_UI = B_UI | A_UI &\utidxi{V\_UI{\EQ}B\_UI\protect{\BAR}A\_UI}{x-ui-006}&
%value
%&\ref{p-x-ui-020}&   uid_H: H -> H_UI &\ufidxi{uid\_H}{x-ui-020}&
%&\ref{p-x-ui-030}&   uid_L: H -> L_UI  &\ufidxi{uid\_L}{x-ui-030}& 
%&\ref{p-x-ui-040}&   uid_BC: H -> BC_UI    &\ufidxi{uid\_BC}{x-ui-040}&
%&\ref{p-x-ui-050}&   uid_B: H -> B_UI   &\ufidxi{uid\_B}{x-ui-050}&
%&\ref{p-x-ui-060}&   uid_A: H -> A_UI  &\ufidxi{uid\_A}{x-ui-060}&
%\endRSLatex \RSLatex
\bp
\kw{type}\ \ \\
\ref{p-x-ui-000}\ \ \ H\_UI, L\_UI, BC\_UI, B\_UI, A\_UI \utidxi{H\_UI}{x-ui-000}\utidxi{L\_UI}{x-ui-006}\utidxi{BC\_UI}{x-ui-006}\utidxi{B\_UI}{x-ui-006}\utidxi{A\_UI}{x-ui-006}\\
\ref{p-x-ui-004}\ \ \ R\_UI {\EQ} H\_UI {\BAR} L\_UI \utidxi{R\_UI{\EQ}H\_UI\protect{\BAR}L\_UI}{x-ui-004}\\
\ref{p-x-ui-006}\ \ \ V\_UI {\EQ} B\_UI {\BAR} A\_UI \utidxi{V\_UI{\EQ}B\_UI\protect{\BAR}A\_UI}{x-ui-006}\\
\kw{value}\\
\ref{p-x-ui-020}\ \ \ uid\_H: H {\RIGHTARROW} H\_UI \ufidxi{uid\_H}{x-ui-020}\\
\ref{p-x-ui-030}\ \ \ uid\_L: H {\RIGHTARROW} L\_UI\ \ \ufidxi{uid\_L}{x-ui-030} \\
\ref{p-x-ui-040}\ \ \ uid\_BC: H {\RIGHTARROW} BC\_UI\ \ \ \ \ufidxi{uid\_BC}{x-ui-040}\\
\ref{p-x-ui-050}\ \ \ uid\_B: H {\RIGHTARROW} B\_UI\ \ \ \ufidxi{uid\_B}{x-ui-050}\\
\ref{p-x-ui-060}\ \ \ uid\_A: H {\RIGHTARROW} A\_UI\ \ \ufidxi{uid\_A}{x-ui-060}
\ep
\pos{\end{multicols}}{}

\nbbb{Extract Parts from Their Unique Identifiers}
\begin{enumerate}\setei
\item \label{p-xpui-000} From the unique identifier of a part we can
                       retrieve, $\wp$, the part having that
                       identifier.\xfidxi{$\wp$}{xpui-000} 
\savei\end{enumerate}\label{p-wp}\footsize\HHHH
%\RSLatex
%type 
%&\ref{p-xpui-000}& P = H | L | BC | B | A
%value
%&\ref{p-xpui-000}&  &$\wp$&: H_UI->H | L_UI->L | BC_UI->BC | B_UI->B | A_UI->A
%&\ref{p-xpui-000}&  &$\wp$&(ui) is let p:(H|L|BC|B|A):-&p&isin&$ps$&/\uid_P(p)=ui in p end 
%\endRSLatex 
\bp
\kw{type} \\
\ref{p-xpui-000} P {\EQ} H {\BAR} L {\BAR} BC {\BAR} B {\BAR} A\\
\kw{value}\\
\ref{p-xpui-000}\ \ $\wp$: H\_UI{\RIGHTARROW}H {\BAR} L\_UI{\RIGHTARROW}L {\BAR} BC\_UI{\RIGHTARROW}BC {\BAR} B\_UI{\RIGHTARROW}B {\BAR} A\_UI{\RIGHTARROW}A\\
\ref{p-xpui-000}\ \ $\wp$(ui) {\IS} \kw{let} p:(H{\BAR}L{\BAR}BC{\BAR}B{\BAR}A){\RDOT}p{\ISIN}$ps${\WEDGE}uid\_P(p){\EQ}ui \kw{in} p \kw{end} 
\ep
\noindent

\nbbb{All Unique Identifiers of a Domain}\label{p-All Unique Identifiers of a Domain}\label{p-ui-sets}\LLll
We can calculate:
\begin{enumerate}\setei
\item \label{p-uic-000} the $s$et, $h_{ui}s$, of $u$nique $h$ub
                      $i$dentifiers; \uividxi{$h_{ui}s$}{uic-000}
\item \label{p-uic-010} the $s$et, $l_{ui}s$, of $u$nique $l$ink
                      $i$dentifiers; \uividxi{$l_{ui}s$}{uic-010}
\item \label{p-uic-011a} the $m$ap, $hl_{ui}m$, from $u$nique $h$ub
                      $i$dentifiers to the $s$et of $u$nique $l$ink
                      $i$identifiers of the links connected to the
                      zero, one or more identified hubs,
                      \uividxi{$hl_{ui}m$}{uic-011a}  
\item \label{p-uic-011b} the $m$ap, $lh_{ui}m$, from $u$nique $l$ink
                      $i$dentifiers to the $s$et of $u$nique $h$ub
                      $i$identifiers of the two hubs connected to the 
                      identified link; \uividxi{$lh_{ui}m$}{uic-011b}
\item \label{p-uic-011} the $s$et, $r_{ui}s$, of all $u$nique hub and
                      link, i.e., 
                      $r$oad $i$dentifiers;
                      \uividxi{$r_{ui}s$}{uic-011} 
\item \label{p-uic-020} the $s$et, $bc_{ui}s$, of $u$nique $b$us
                      $c$ompany
                      $i$dentifiers;\uividxi{$bc_{ui}s$}{uic-020}
%\pos{\psno}{\mnewfoil}
\item \label{p-uic-030} the $s$et, $b_{ui}s$, of $u$nique $b$us
  $i$dentifiers;\uividxi{$b_{ui}s$}{uic-030} 
\item \label{p-uic-040} the $s$et, $a_{ui}s$, of $u$nique private
                      $a$utomobile
                      $i$dentifiers;\uividxi{$a_{ui}s$}{uic-040}
\item \label{p-uic-041} the $s$et, $v_{ui}s$, of $u$nique bus and
                      automobile, i.e., \uividxi{$v_{ui}s$}{uic-041}
                      $v$ehicle $i$dentifiers; 
\item \label{p-uic-042} the $m$ap, $bcb_{ui}m$, from $u$nique $b$us
                      $c$ompany $i$dentifiers to the $s$et 
                      of its  $u$nique $b$us $i$dentifiers;
                      and \uividxi{$bcb_{ui}m$}{uic-042} 
\item \label{p-uic-043} the ($b$ijective) $m$ap, $bbc_{ui}bm$, from
                      $u$nique $b$us $i$dentifiers to 
                      their $u$nique $b$us $c$ompany
                      $i$dentifiers.\uividxi{$bbc_{ui}bm$}{uic-043} 
\savei\end{enumerate}
\pos{\psno}{\mnewfoil}\LLLL
%\RSLatex  
%value
%&\ref{p-uic-000}&   &$h_{ui}s$&:H_UI-set is {uid_H(h)|h:H:-h isin &$hs$&}
%&\ref{p-uic-010}&   &$l_{ui}s$&:L_UI-set is {uid_L(l)|l:L:-l isin &$ls$&} 
%&\ref{p-uic-011}&   &$r_{ui}s$&:R_UI-set is &$h_{ui}s$&union&$l_{ui}s$&
%&\ref{p-uic-011a}&   &$hl_{ui}m$&:(H_UI-m->L_UI-set) is 
%&\ref{p-uic-011a}&       [h_ui+>luis|h_ui:H_UI,luis:L_UI-set:-h_&ui&isin&$h_{ui}s$&/\(_,luis,_)=mereo_H(`eta(h_ui))]  &[cf.\,Item\,\ref{p-mereo-000}]&
%&\ref{p-uic-011b}&   &$lh_{ui}m$&:(L+UI-m->H_UI-set) is 
%&\ref{p-uic-011b}&       [l_ui+>huis | h_ui:L_UI,huis:H_UI-set :- l_&ui&isin&$l_{ui}s$& /\ (_,huis,_)=mereo_L(`eta(l_ui))]  &[cf.\,Item\,\ref{p-mereo-010}]&
%&\ref{p-uic-020}&   &$bc_{ui}s$&:BC_UI-set is {uid_BC(bc)|bc:BC:-bc isin &$bcs$&}  
%&\ref{p-uic-030}&   &$b_{ui}s$&:B_UI-set is union{uid_B(b)|b:B:-b isin &$bs$&}
%&\ref{p-uic-040}&   &$a_{ui}s$&:A_UI-set is {uid_A(a)|a:A:-a isin &$as$&}
%&\ref{p-uic-041}&   &$v_{ui}s$&:V_UI-set is &$b_{ui}s$& union &$a_{ui}s$&
%&\ref{p-uic-042}&   &$bcb_{ui}m$&:(BC_UI-m->B_UI-set) is 
%&\ref{p-uic-042}&      [ bc_ui +> buis | bc_ui:BC_UI, bc:BC :- &bc&isin&$bcs$& /\ bc_ui=uid_BC(bc) /\ (_,_,buis)=mereo_BC(bc) ]  
%&\ref{p-uic-043}&   &$bbc_{ui}bm$&:(B_UI-m->BC_UI) is 
%&\ref{p-uic-043}&      [ b_ui +> bc_ui | b_ui:B_UI,bc_ui:BC_ui :- bc_ui=dom&$bcb_{ui}m$&/\b_&ui&isin&$bcb_{ui}m$&(bc_ui) ]
%\endRSLatex
\bp
\kw{value}\\
\ref{p-uic-000}\ \ \ $h_{ui}s$:H\_UI\kw{-set} {\IS} {\LBRACE}uid\_H(h){\BAR}h:H{\RDOT}h {\ISIN} $hs${\RBRACE}\\
\ref{p-uic-010}\ \ \ $l_{ui}s$:L\_UI\kw{-set} {\IS} {\LBRACE}uid\_L(l){\BAR}l:L{\RDOT}l {\ISIN} $ls${\RBRACE} \\
\ref{p-uic-011}\ \ \ $r_{ui}s$:R\_UI\kw{-set} {\IS} $h_{ui}s${\UNION}$l_{ui}s$\\
\ref{p-uic-011a}\ \ \ $hl_{ui}m$:(H\_UI{\MARROW}L\_UI\kw{-set}) {\IS} \\
\ref{p-uic-011a}\ \ \ \ \ \ \ {\LBRACKET}h\_ui{\MAPSTO}luis{\BAR}h\_ui:H\_UI,luis:L\_UI\kw{-set}{\RDOT}h\_ui{\ISIN}$h_{ui}s${\WEDGE}({\UNDERLINE},luis,{\UNDERLINE}){\EQ}mereo\_H($\eta$(h\_ui)){\RBRACKET}\ \ [cf.\,Item\,\ref{p-mereo-000}]\\
\ref{p-uic-011b}\ \ \ $lh_{ui}m$:(L{\PLUS}UI{\MARROW}H\_UI\kw{-set}) {\IS} \\
\ref{p-uic-011b}\ \ \ \ \ \ \ {\LBRACKET}l\_ui{\MAPSTO}huis {\BAR} h\_ui:L\_UI,huis:H\_UI\kw{-set} {\RDOT} l\_ui{\ISIN}$l_{ui}s$ {\WEDGE} ({\UNDERLINE},huis,{\UNDERLINE}){\EQ}mereo\_L($\eta$(l\_ui)){\RBRACKET}\ \ [cf.\,Item\,\ref{p-mereo-010}]\\
\ref{p-uic-020}\ \ \ $bc_{ui}s$:BC\_UI\kw{-set} {\IS} {\LBRACE}uid\_BC(bc){\BAR}bc:BC{\RDOT}bc {\ISIN} $bcs${\RBRACE}\ \ \\
\ref{p-uic-030}\ \ \ $b_{ui}s$:B\_UI\kw{-set} {\IS} {\UNION}{\LBRACE}uid\_B(b){\BAR}b:B{\RDOT}b {\ISIN} $bs${\RBRACE}\\
\ref{p-uic-040}\ \ \ $a_{ui}s$:A\_UI\kw{-set} {\IS} {\LBRACE}uid\_A(a){\BAR}a:A{\RDOT}a {\ISIN} $as${\RBRACE}\\
\ref{p-uic-041}\ \ \ $v_{ui}s$:V\_UI\kw{-set} {\IS} $b_{ui}s$ {\UNION} $a_{ui}s$\\
\ref{p-uic-042}\ \ \ $bcb_{ui}m$:(BC\_UI{\MARROW}B\_UI\kw{-set}) {\IS} \\
\ref{p-uic-042}\ \ \ \ \ \ {\LBRACKET} bc\_ui {\MAPSTO} buis {\BAR} bc\_ui:BC\_UI, bc:BC {\RDOT} bc{\ISIN}$bcs$ {\WEDGE} bc\_ui{\EQ}uid\_BC(bc) {\WEDGE} ({\UNDERLINE},{\UNDERLINE},buis){\EQ}mereo\_BC(bc) {\RBRACKET}\ \ \\
\ref{p-uic-043}\ \ \ $bbc_{ui}bm$:(B\_UI{\MARROW}BC\_UI) {\IS} \\
\ref{p-uic-043}\ \ \ \ \ \ {\LBRACKET} b\_ui {\MAPSTO} bc\_ui {\BAR} b\_ui:B\_UI,bc\_ui:BC\_ui {\RDOT} bc\_ui{\EQ}\kw{dom}$bcb_{ui}m${\WEDGE}b\_ui{\ISIN}$bcb_{ui}m$(bc\_ui) {\RBRACKET}
\ep

\pos{\normalsize}{\HHHH}

\nbbb{Uniqueness of Road Net Identifiers}\LLLL
\begynd
\pind We must express the following axioms:
\afslut
\begin{enumerate}\setei
\item \label{p-x-ui-020d} All hub identifiers are distinct.
\item \label{p-x-ui-030d} All link identifiers are distinct.
\item \label{p-x-ui-040d} All bus company identifiers are distinct.
\item \label{p-x-ui-050d} All bus identifiers are distinct.
\item \label{p-x-ui-060d} All private automobile identifiers are distinct.
\item \label{p-x-ui-070d} All part identifiers are distinct.
\savei\end{enumerate}\HHHH
%\RSLatex
%axiom
%&\ref{p-x-ui-020d}&   card&\,$hs$& = card&\,$h_{ui}s$&
%&\ref{p-x-ui-030d}&   card&\,$ls$& = card&\,$l_{ui}s$&  
%&\ref{p-x-ui-040d}&   card&\,$bcs$& = card&\,$bc_{ui}s$& 
%&\ref{p-x-ui-050d}&   card&\,$bs$& = card&\,$b_{ui}s$&
%&\ref{p-x-ui-060d}&   card&\,$as$& = card&\,$a_{ui}s$&
%&\ref{p-x-ui-070d}&   card&\,&{&$h_{ui}s$&union&$l_{ui}s$&union&$bc_{ui}s$&union&$b_{ui}s$&union&$a_{ui}s$&} 
%&\ref{p-x-ui-070d}&      = card&\,$h_{ui}s$&+card&\,$l_{ui}s$&+card&\,$bc_{ui}s$&+card&\,$b_{ui}s$&+card&\,$a_{ui}s$  \eox&
%\endRSLatex
\bp
\kw{axiom}\\
\ref{p-x-ui-020d}\ \ \ \kw{card}\,$hs$ {\EQ} \kw{card}\,$h_{ui}s$\\
\ref{p-x-ui-030d}\ \ \ \kw{card}\,$ls$ {\EQ} \kw{card}\,$l_{ui}s$\ \ \\
\ref{p-x-ui-040d}\ \ \ \kw{card}\,$bcs$ {\EQ} \kw{card}\,$bc_{ui}s$ \\
\ref{p-x-ui-050d}\ \ \ \kw{card}\,$bs$ {\EQ} \kw{card}\,$b_{ui}s$\\
\ref{p-x-ui-060d}\ \ \ \kw{card}\,$as$ {\EQ} \kw{card}\,$a_{ui}s$\\
\ref{p-x-ui-070d}\ \ \ \kw{card}\,{\LBRACE}$h_{ui}s${\UNION}$l_{ui}s${\UNION}$bc_{ui}s${\UNION}$b_{ui}s${\UNION}$a_{ui}s${\RBRACE} \\
\ref{p-x-ui-070d}\ \ \ \ \ \ {\EQ} \kw{card}\,$h_{ui}s${\PLUS}\kw{card}\,$l_{ui}s${\PLUS}\kw{card}\,$bc_{ui}s${\PLUS}\kw{card}\,$b_{ui}s${\PLUS}\kw{card}\,$a_{ui}s$  \eox
\ep

\nbbbb{Mereology}\label{p-ch2.Mereology}\label{p-refMereology of Parts}\label{p-m1}
\bbb{Mereology Types and Observers}
\rm\smallish\LLLL\rm
\begin{enumerate}\setei
\item \label{p-mereo-000} The mereology of hubs is a pair: (i) the set  
  of all bus and automobile identifiers\footnotemark, and (ii) the set of unique
  identifiers of the links that it is connected to and the set of all
  unique identifiers of all vehicles (buses and private
  automobiles).\footnotemark
\item \label{p-mereo-010} The mereology of links is a pair: (i) the 
  set of all bus and automobile identifiers, and  (ii)  the
  set of the two distinct hubs they are connected to. 
\item \label{p-mereo-020} The mereology of a bus company is a set the
  unique identifiers of the buses operated by 
  that company. 
\item \label{p-mereo-030} The mereology of a bus is  a pair: (i) 
  the set of the one single
  unique identifier of the bus company it is operating for, and (ii)
  the unique 
  identifiers of all links and hubs\footnotemark.  
\item \label{p-mereo-040} The mereology of an automobile is the set of
  the unique 
  identifiers of all links and hubs\footnotemark. 
  \savei\end{enumerate}
\mnewfoil
\begin{multicols}{2}
%\RSLatex 
%type
%&\ref{p-mereo-000}&   H_Mer = V_UI-set><L_UI-set   &\mtidxi{H\_Mer{\EQ}V\_UI\kw{-set}{\TIMES}L\_UI\kw{-set}}{mereo-000}& 
%&\ref{p-mereo-010}&   L_Mer = V_UI-set><H_UI-set   &\mtidxi{L\_Mer{\EQ}V\_UI\kw{-set}{\TIMES}H\_UI\kw{-set}}{mereo-010}&
%&\ref{p-mereo-020}&   BC_Mer = B_UI-set      &\mtidxi{BC\_Mer{\EQ}B\_UI\kw{-set}}{mereo-020}&
%&\ref{p-mereo-030}&   B_Mer = BC_UI><R_UI-set  &\mtidxi{B\_Mer{\EQ}BC\_UI{\TIMES}R\_UI\kw{-set}}{mereo-030}&
%&\ref{p-mereo-040}&   A_Mer = R_UI-set &\mtidxi{A\_Mer{\EQ}R\_UI\kw{-set}}{mereo-040}&
%value
%&\ref{p-mereo-000}&   mereo_H: H -> H_Mer &\mfidxi{mereo\_H}{mereo-000}&
%&\ref{p-mereo-010}&   mereo_L: L -> L_Mer &\mfidxi{mereo\_L}{mereo-010}&   
%&\ref{p-mereo-020}&   mereo_BC: BC -> BC_Mer &\mfidxi{mereo\_BC}{mereo-020}&   
%&\ref{p-mereo-030}&   mereo_B: B -> B_Mer &\mfidxi{mereo\_B}{mereo-030}&   
%&\ref{p-mereo-040}&   mereo_A: A -> A_Mer  &\mfidxi{mereo\_A}{mereo-040}& 
%\endRSLatex
\bp
\kw{type}\\
\ref{p-mereo-000}\ \ \ H\_Mer {\EQ} V\_UI\kw{-set}{\TIMES}L\_UI\kw{-set}\ \ \ \mtidxi{H\_Mer{\EQ}V\_UI\kw{-set}{\TIMES}L\_UI\kw{-set}}{mereo-000} \\
\ref{p-mereo-010}\ \ \ L\_Mer {\EQ} V\_UI\kw{-set}{\TIMES}H\_UI\kw{-set}\ \ \ \mtidxi{L\_Mer{\EQ}V\_UI\kw{-set}{\TIMES}H\_UI\kw{-set}}{mereo-010}\\
\ref{p-mereo-020}\ \ \ BC\_Mer {\EQ} B\_UI\kw{-set}\ \ \ \ \ \ \mtidxi{BC\_Mer{\EQ}B\_UI\kw{-set}}{mereo-020}\\
\ref{p-mereo-030}\ \ \ B\_Mer {\EQ} BC\_UI{\TIMES}R\_UI\kw{-set}\ \ \mtidxi{B\_Mer{\EQ}BC\_UI{\TIMES}R\_UI\kw{-set}}{mereo-030}\\
\ref{p-mereo-040}\ \ \ A\_Mer {\EQ} R\_UI\kw{-set} \mtidxi{A\_Mer{\EQ}R\_UI\kw{-set}}{mereo-040}\\
\kw{value}\\
\ref{p-mereo-000}\ \ \ mereo\_H: H {\RIGHTARROW} H\_Mer \mfidxi{mereo\_H}{mereo-000}\\
\ref{p-mereo-010}\ \ \ mereo\_L: L {\RIGHTARROW} L\_Mer \mfidxi{mereo\_L}{mereo-010}\ \ \ \\
\ref{p-mereo-020}\ \ \ mereo\_BC: BC {\RIGHTARROW} BC\_Mer \mfidxi{mereo\_BC}{mereo-020}\ \ \ \\
\ref{p-mereo-030}\ \ \ mereo\_B: B {\RIGHTARROW} B\_Mer \mfidxi{mereo\_B}{mereo-030}\ \ \ \\
\ref{p-mereo-040}\ \ \ mereo\_A: A {\RIGHTARROW} A\_Mer\ \ \mfidxi{mereo\_A}{mereo-040} 
\ep
\end{multicols}
\addtocounter{footnote}{-3}
\footnotetext{This is just another
  way of saying that the meaning of hub mereologies involves
  the unique identifiers of all the vehicles that might pass through the hub
  \texttt{is\_of\_interest} to it.}
\addtocounter{footnote}{1}
\footnotetext{The link identifiers designate the
  links, zero, one or more,  that a hub is connected to
  \texttt{is\_of\_interest} to both the hub and that these links is
  \texttt{interested} in the hub.}
\addtocounter{footnote}{1}
\footnotetext{--- that the bus might pass through}
\addtocounter{footnote}{1}
\footnotetext{--- that the automobile might pass through}

\nbbb{Invariance of Mereologies}\HHHH
\begynd
\pind For mereologies one can usually express some invariants.
\begynd
\pind Such invariants express \sfsl{``law-like properties''}, 
\pind facts which are indisputable.
\afslut
\afslut

\nbb{Invariance of Road Nets}{ %
\begynd
\pind The observed mereologies must express identifiers of the state
of such for road nets:
\afslut
%\RSLatex 
%axiom 
%&\ref{p-mereo-000}&   all (vuis,luis):H_Mer :- luis<<=&$l_{ui}s$& /\ vuis=&$v_{ui}s$&
%&\ref{p-mereo-010}&   all (vuis,huis):L_Mer :- vuis=&$v_{ui}s$& /\ huis<<=&$h_{ui}s$& /\ card&\ysfchg{ }huis\,=\,&2
%&\ref{p-mereo-020}&   all buis:H_Mer :- buis = &$b_{ui}s$&
%&\ref{p-mereo-030}&   all (bc_ui,ruis):H_Mer:-bc_&ui&isin&$bc_{ui}s$&/\ruis=&$r_{ui}s$&
%&\ref{p-mereo-040}&   all ruis:A_Mer :- ruis=&$r_{ui}s$&
%\endRSLatex
\bp
\kw{axiom} \\
\ref{p-mereo-000}\ \ \ {\ALL} (vuis,luis):H\_Mer {\RDOT} luis{\SUBSETEQ}$l_{ui}s$ {\WEDGE} vuis{\EQ}$v_{ui}s$\\
\ref{p-mereo-010}\ \ \ {\ALL} (vuis,huis):L\_Mer {\RDOT} vuis{\EQ}$v_{ui}s$ {\WEDGE} huis{\SUBSETEQ}$h_{ui}s$ {\WEDGE} \kw{card}\ysfchg{ }huis\,=\,2\\
\ref{p-mereo-020}\ \ \ {\ALL} buis:H\_Mer {\RDOT} buis {\EQ} $b_{ui}s$\\
\ref{p-mereo-030}\ \ \ {\ALL} (bc\_ui,ruis):H\_Mer{\RDOT}bc\_ui{\ISIN}$bc_{ui}s${\WEDGE}ruis{\EQ}$r_{ui}s$\\
\ref{p-mereo-040}\ \ \ {\ALL} ruis:A\_Mer {\RDOT} ruis{\EQ}$r_{ui}s$
\ep
\pos{\psno}{\mnewfoil}
\begin{enumerate}\setei
\item \label{p-mereo-axiom-000} For all hubs, $h$, and links, $l$, in the same road net,
\item \label{p-mereo-axiom-010} if the hub $h$ connects to link $l$ then link $l$ connects to hub $h$. 
\savei\end{enumerate}
%\RSLatex
%axiom
%&\ref{p-mereo-axiom-000}&   all h:H,l:L :- h isin &$hs$& /\ l isin &$ls$& =>
%&\ref{p-mereo-axiom-000}&      let (_,luis)=mereo_H(h), (_,huis)=mereo_L(l)
%&\ref{p-mereo-axiom-010}&      in uid_L(l)isin&luis& is uid_H(h)isin&huis& end 
%\endRSLatex
\bp
\kw{axiom}\\
\ref{p-mereo-axiom-000}\ \ \ {\ALL} h:H,l:L {\RDOT} h {\ISIN} $hs$ {\WEDGE} l {\ISIN} $ls$ {\DBLRIGHTARROW}\\
\ref{p-mereo-axiom-000}\ \ \ \ \ \ \kw{let} ({\UNDERLINE},luis){\EQ}mereo\_H(h), ({\UNDERLINE},huis){\EQ}mereo\_L(l)\\
\ref{p-mereo-axiom-010}\ \ \ \ \ \ \kw{in} uid\_L(l){\ISIN}luis {\IS} uid\_H(h){\ISIN}huis \kw{end} 
\ep
\pos{\psno}{\mnewfoil}
\HHHH
\vspace*{2mm}
\begin{enumerate}\setei
\item \label{p-mereo-axiom-200} For all links, $l$, and hubs, $h_a, h_b$, in the same road net,
\item \label{p-mereo-axiom-210} if the $l$ connects to hubs $h_a$
  and $h_b$, then  $h_a$ and $h_b$ both connects to link $l$.
\savei\end{enumerate}\footsize\HHHH
%\RSLatex
%axiom
%&\ref{p-mereo-axiom-200}&   all h_a,h_b:H,l:L :- {h_a,h_b} <<= &$hs$& /\ l isin &$ls$& =>
%&\ref{p-mereo-axiom-200}&      let (_,luis)=mereo_H(h), (_,huis)=mereo_L(l)
%&\ref{p-mereo-axiom-210}&      in uid_L(l)isin&luis& is uid_H(h)isin&huis& end 
%\endRSLatex
\bp
\kw{axiom}\\
\ref{p-mereo-axiom-200}\ \ \ {\ALL} h\_a,h\_b:H,l:L {\RDOT} {\LBRACE}h\_a,h\_b{\RBRACE} {\SUBSETEQ} $hs$ {\WEDGE} l {\ISIN} $ls$ {\DBLRIGHTARROW}\\
\ref{p-mereo-axiom-200}\ \ \ \ \ \ \kw{let} ({\UNDERLINE},luis){\EQ}mereo\_H(h), ({\UNDERLINE},huis){\EQ}mereo\_L(l)\\
\ref{p-mereo-axiom-210}\ \ \ \ \ \ \kw{in} uid\_L(l){\ISIN}luis {\IS} uid\_H(h){\ISIN}huis \kw{end} 
\ep
}

\nbb{Possible Consequences of a Road Net Mereology}{%
\begin{enumerate}\setei
%\item \label{p-eks-mereo-000} are the nets acyclic\,? 
\item \label{p-eks-mereo-010} Are there [isolated] units from which one
  can not ``reach'' other units\,? 
\item \label{p-eks-mereo-020} Does the net consist of two or more
  ``disjoint'' nets\,?
\item \label{p-eks-mereo-030} Et cetera.
\savei\end{enumerate}
\noindent
\begynd
\pind We leave it to the reader to narrate and formalise the above properly.
\afslut
}

\nbb{Fixed and Varying Mereology}%{%
\begynd
\pind Let us consider a road net. 
\begynd
\pind If hubs and links never change ``affiliation'', that is:
\begynd
\pind hubs are in fixed relation to zero one or more links, and
\pind links are in a fixed relation to exactly two hubs
\pind then the mereology is
      a \sfsl{fixed mereology}. 
\afslut
\pos{\psno}{\mnewfoil}      
\pind If, on the other hand
\begynd
\pind hubs may be inserted into or removed from the net, and/or
\pind links may be removed from or inserted between any two existing hubs,
\pind then the mereology  is
      a \sfsl{varying mereology}. 
\afslut
\afslut
\afslut
%}
 
\nbbbb{Attributes}\label{p-road-Attributes}\label{p-Attributes.1}
\bbb{Hub Attributes}{ %\input{eks-attrs}
\noindent\LLLL\HHHH
\begynd
\pind We treat some attributes of the hubs of a road net.
\afslut
\begin{enumerate}\setei
\item \label{p-h-attr-000} There is a hub state.
\begynd
\pind  It is a set of pairs, \textsf{(l$_f$,l$_t$)}, of link identifiers,
\begynd
\pind  where these link identifiers are in the mereology of the hub. 
\afslut
\pind The meaning of the hub  state
\begynd
\pind in which, e.g., \textsf{(l$_f$,l$_t$)} is an element, 
\pind is that the hub is open,  \bgcolor{``green''}, 
\pind for traffic $f$rom link \textsf{l$_f$} $t$o link \textsf{l$_t$}.  
\pind If a hub state is empty
\pind then the hub is closed, i.e., \brcolor{``red''} 
\pind for traffic from any
                         connected links to any other connected links.
\afslut
\afslut
\mnewfoil
\item \label{p-h-attr-010} There is a hub state space.
\begynd
\pind It is a set of hub states. 
\pind The current hub state must be in its state space.
\pind The meaning of the hub state space is 
\begynd
\pind that its states are all those the hub can attain.
\afslut
\afslut
\item \label{p-hub-traffic} Since we can think rationally about it,
\begynd
\pind it can be described, hence we can model, as an
                          attribute of hubs, a history of its traffic:
\begynd
\pind the recording, per unique bus  and automobile identifier, 
\pind of the time ordered presence in the hub of these vehicles.   
\afslut
\pind Hub history is an \sfsl{event history}.
\afslut
\savei\end{enumerate}
\pos{\psno}{\mnewfoil}\LLll
%\RSLatex
%type
%&\ref{p-h-attr-000}&  H`Sigma = (L_UI><L_UI)-set
%axiom
%&\ref{p-h-attr-000}&  all h:H :- obs_H`Sigma(h) isin obs_H`Omega(h)
%type
%&\ref{p-h-attr-010}&  H`Omega = H`Sigma-set
%&\ref{p-hub-traffic}&  H_Traffic
%&\ref{p-hub-traffic}&  H_Traffic = (A_UI|B_UI) -m-> (&$\mathbb{TIME}$& >< VPos)-list
%axiom
%&\ref{p-hub-traffic}&  all ht:H_Traffic,ui:(A_UI|B_UI) :- 
%&\ref{p-hub-traffic}&     ui isin dom ht => time_ordered(ht(ui))
%value 
%&\ref{p-h-attr-000}&  attr_H`Sigma: H -> H`Sigma
%&\ref{p-h-attr-010}&  attr_H`Omega: H -> H`Omega 
%&\ref{p-hub-traffic}&  attr_H_Traffic: H -> H_Traffic
%value
%&\ref{p-hub-traffic}&   time_ordered: (&$\mathbb{TIME}$& >< VPos)-list -> Bool
%&\ref{p-hub-traffic}&   time_ordered(tvpl) is ...
%\endRSLatex 
\bp
\kw{type}\\
\ref{p-h-attr-000}\ \ H$\Sigma$ {\EQ} (L\_UI{\TIMES}L\_UI)\kw{-set}\\
\kw{axiom}\\
\ref{p-h-attr-000}\ \ {\ALL} h:H {\RDOT} obs\_H$\Sigma$(h) {\ISIN} obs\_H$\Omega$(h)\\
\kw{type}\\
\ref{p-h-attr-010}\ \ H$\Omega$ {\EQ} H$\Sigma$\kw{-set}\\
\ref{p-hub-traffic}\ \ H\_Traffic\\
\ref{p-hub-traffic}\ \ H\_Traffic {\EQ} (A\_UI{\BAR}B\_UI) {\MARROW} ($\mathbb{TIME}$ {\TIMES} VPos)$^{\ast}$\\
\kw{axiom}\\
\ref{p-hub-traffic}\ \ {\ALL} ht:H\_Traffic,ui:(A\_UI{\BAR}B\_UI) {\RDOT} \\
\ref{p-hub-traffic}\ \ \ \ \ ui {\ISIN} \kw{dom} ht {\DBLRIGHTARROW} time\_ordered(ht(ui))\\
\kw{value} \\
\ref{p-h-attr-000}\ \ attr\_H$\Sigma$: H {\RIGHTARROW} H$\Sigma$\\
\ref{p-h-attr-010}\ \ attr\_H$\Omega$: H {\RIGHTARROW} H$\Omega$ \\
\ref{p-hub-traffic}\ \ attr\_H\_Traffic: H {\RIGHTARROW} H\_Traffic\\
\kw{value}\\
\ref{p-hub-traffic}\ \ \ time\_ordered: ($\mathbb{TIME}$ {\TIMES} VPos)$^{\ast}$ {\RIGHTARROW} \kw{Bool}\\
\ref{p-hub-traffic}\ \ \ time\_ordered(tvpl) {\IS} {\DOTDOTDOT}
\ep
\noindent
\begynd
\pind In Item\,\vref{p-hub-traffic} we model the time-ordered sequence
      of traffic as a discrete sampling, i.e., {\MARROW}, rather than as a
      continuous function, {\RIGHTARROW}.
\afslut
}

\nbbb{Invariance of Traffic States}{%
\begin{enumerate}\setei
\item \label{p-h-attr-030} The link identifiers of hub states must be in
                         the set, $l_{ui}s$, of the road net's link
                         identifiers.
\savei\end{enumerate}
%\RSLatex
%axiom
%&\ref{p-h-attr-030}&  all h:H :- h isin &$hs$& => 
%&\ref{p-h-attr-030}&      let h`sigma = attr_H`Sigma(h) in 
%&\ref{p-h-attr-030}&      all (l&$_{ui}i$&,li&$_{ui}i'$&):(L_UI><L_UI) :- (l&$_{ui}i$&,l&$_{ui}i'$&) isin h`sigma => {l&$_{ui_i}$&,l&$_{ui_i}'$&} <<= l&$_{ui}s$& end
%\endRSLatex
\bp
\kw{axiom}\\
\ref{p-h-attr-030}\ \ {\ALL} h:H {\RDOT} h {\ISIN} $hs$ {\DBLRIGHTARROW} \\
\ref{p-h-attr-030}\ \ \ \ \ \ \kw{let} h$\sigma$ {\EQ} attr\_H$\Sigma$(h) \kw{in} \\
\ref{p-h-attr-030}\ \ \ \ \ \ {\ALL} (l$_{ui}i$,li$_{ui}i'$):(L\_UI{\TIMES}L\_UI) {\RDOT} (l$_{ui}i$,l$_{ui}i'$) {\ISIN} h$\sigma$ {\DBLRIGHTARROW} {\LBRACE}l$_{ui_i}$,l$_{ui_i}'${\RBRACE} {\SUBSETEQ} l$_{ui}s$ \kw{end}
\ep

%%  LocalWords:  ui hs attr li fh th BPos atHub onLink RegNo vel acc
  
}%

\nbbb{Link Attributes}{%%%%%%%%
We show just a few attributes.

\pos{\bmcii}{}
\begin{enumerate}\setei
\item \label{p-l-attr-000} There is a link state. It is a set of pairs,
                         \textsf{(h$_f$,h$_t$)}, of distinct hub identifiers,
                         where these hub identifiers are in the
                         mereology of the link. The meaning of a link
                         state in which \textsf{(h$_f$,h$_t$)} is an
                         element is that the link is open, 
                         \bgcolor{``green''}, for traffic $f$rom hub
                         \textsf{h$_f$} $t$o hub \textsf{h$_t$}.  Link
                         states can have either 0, 1 or 2 elements.
\item \label{p-l-attr-010} There is a link state space. It is a set of
                         link states.  The meaning of the link state
                         space is that its states are all those the
                         which the link can attain. The current link state must be
                         in its state space. If a link state space is
                         empty then the link is (permanently)
                         closed. If it has one element then it is a
                         one-way link. If a one-way link, $l$, is imminent
                         on a hub whose mereology designates that
                         link, then the link is a ``trap'', i.e., a
                         ``blind cul-de-sac''. 
\mnewfoil
\item \label{p-link-traffic} Since we can think rationally about it, it
                          can be described, hence it can model, as an
                          attribute of links a  history of its traffic:
                          the recording, per unique bus  and
                          automobile identifier, of the time ordered
                          positions along the link (from one hub to
                          the next) of these vehicles.  
\item \label{p-l-attr-030} The hub identifiers of link states must be in
                         the set, $h_{ui}s$, of the road net's hub
                         identifiers. 
\savei\end{enumerate}
\mnewfoil\LLLL
%\RSLatex
%type
%&\ref{p-l-attr-000}&  L`Sigma = H_UI-set
%axiom
%&\ref{p-l-attr-000}&  all l`sigma:L`Sigma:-card l`sigma=2
%&\ref{p-l-attr-000}&  all l:L :- obs_L`Sigma(l) isin obs_L`Omega(l)
%type
%&\ref{p-l-attr-010}&  L`Omega = L`Sigma-set
%&\ref{p-link-traffic}&  L_Traffic
%&\ref{p-link-traffic}&  L_Traffic = (A_UI|B_UI) -m-> (&$\mathbb{T}$&><(H_UI><Frac><H_UI))-list      
%&\ref{p-link-traffic}&  Frac = Real, axiom frac:Fract :- 0<frac<1
%value 
%&\ref{p-l-attr-000}&  attr_L`Sigma: L -> L`Sigma
%&\ref{p-l-attr-010}&  attr_L`Omega: L -> L`Omega
%&\ref{p-link-traffic}&  attr_L_Traffic: : -> L_Traffic
%axiom
%&\ref{p-link-traffic}&  all lt:L_Traffic,ui:(A_UI|B_UI):-ui isin dom ht => time_ordered(ht(ui))
%&\ref{p-l-attr-030}&  all l:L :- l isin &$ls$& => 
%&\ref{p-l-attr-030}&     let l`sigma = attr_L`Sigma(l) in all (h&$_{ui}i$&,h&$_{ui}i'$&):(H_UI><K_UI) :- 
%&\ref{p-l-attr-030}&        (h&$_{ui}i$&,h&$_{ui}i'$&) isin l`sigma => {h&$_{ui_i}$&,h&$_{ui_i}'$&} <<= h&$_{ui}s$& end  
%\endRSLatex 
\bp
\kw{type}\\
\ref{p-l-attr-000}\ \ L$\Sigma$ {\EQ} H\_UI\kw{-set}\\
\kw{axiom}\\
\ref{p-l-attr-000}\ \ {\ALL} l$\sigma$:L$\Sigma${\RDOT}\kw{card} l$\sigma${\EQ}2\\
\ref{p-l-attr-000}\ \ {\ALL} l:L {\RDOT} obs\_L$\Sigma$(l) {\ISIN} obs\_L$\Omega$(l)\\
\kw{type}\\
\ref{p-l-attr-010}\ \ L$\Omega$ {\EQ} L$\Sigma$\kw{-set}\\
\ref{p-link-traffic}\ \ L\_Traffic\\
\ref{p-link-traffic}\ \ L\_Traffic {\EQ} (A\_UI{\BAR}B\_UI) {\MARROW} ($\mathbb{T}${\TIMES}(H\_UI{\TIMES}Frac{\TIMES}H\_UI))$^{\ast}$\ \ \ \ \ \ \\
\ref{p-link-traffic}\ \ Frac {\EQ} \kw{Real}, \kw{axiom} frac:Fract {\RDOT} 0{\LT}frac{\LT}1\\
\kw{value} \\
\ref{p-l-attr-000}\ \ attr\_L$\Sigma$: L {\RIGHTARROW} L$\Sigma$\\
\ref{p-l-attr-010}\ \ attr\_L$\Omega$: L {\RIGHTARROW} L$\Omega$\\
\ref{p-link-traffic}\ \ attr\_L\_Traffic: : {\RIGHTARROW} L\_Traffic\\
\kw{axiom}\\
\ref{p-link-traffic}\ \ {\ALL} lt:L\_Traffic,ui:(A\_UI{\BAR}B\_UI){\RDOT}ui {\ISIN} \kw{dom} ht {\DBLRIGHTARROW} time\_ordered(ht(ui))\\
\ref{p-l-attr-030}\ \ {\ALL} l:L {\RDOT} l {\ISIN} $ls$ {\DBLRIGHTARROW} \\
\ref{p-l-attr-030}\ \ \ \ \ \kw{let} l$\sigma$ {\EQ} attr\_L$\Sigma$(l) \kw{in} {\ALL} (h$_{ui}i$,h$_{ui}i'$):(H\_UI{\TIMES}K\_UI) {\RDOT} \\
\ref{p-l-attr-030}\ \ \ \ \ \ \ \ (h$_{ui}i$,h$_{ui}i'$) {\ISIN} l$\sigma$ {\DBLRIGHTARROW} {\LBRACE}h$_{ui_i}$,h$_{ui_i}'${\RBRACE} {\SUBSETEQ} h$_{ui}s$ \kw{end}\ \ 
\ep
\pos{\emcii}{}

\bbb{Bus Company Attributes}\HHHH
\begynd
\pind Bus companies operate a number of lines that service passenger
      transport along routes of the road net. Each line being serviced
      by a number of buses.
\afslut
\begin{enumerate}\setei
\item \label{p-bc-attr-011} Bus companies create, maintain, revise and
                          distribute \nyl [to the public (not modeled
                          here), and to buses] \nyl
                          bus time tables, not further defined.
\savei\end{enumerate}
%\mnewfoil
%\RSLatex
%type
%&\ref{p-bc-attr-011}&   BusTimTbl 
%value   
%&\ref{p-bc-attr-011}&   attr_BusTimTbl: BC -> BusTimTbl
%\endRSLatex 
\bp
\kw{type}\\
\ref{p-bc-attr-011}\ \ \ BusTimTbl \\
\kw{value}\ \ \ \\
\ref{p-bc-attr-011}\ \ \ attr\_BusTimTbl: BC {\RIGHTARROW} BusTimTbl
\ep

\noindent
\begynd
\pind There are two notions of time at play here:
\begynd
\pind the indefinite ``real'' or ``actual'' time; and
\pind the definite calendar, hour, minute and second time designation occurring
      in some textual form in, e.g., time tables.
\afslut
\afslut

\nbbb{Bus Attributes} We show just a few attributes.

\begin{enumerate}\setei
\item \label{p-b-attr-000} Buses run routes, according to their line
                         number, \textsf{ln:LN}, in the 
\item \label{p-b-attr-010} bus time table, \textsf{btt:BusTimTbl}  obtained from their bus
                         company, and and keep, as inert attributes,
                         their segment of that time table.
\item \label{p-b-attr-020} Buses occupy positions on the road
                         net:
\begin{enumerate}
\item \label{p-b-attr-030} either \sfsl{at a hub} identified by some
                         \textsf{h\_ui}, %\tidxi{B: atHub::H\_UI}{b-attr-030} 
\item \label{p-b-attr-040} or \sfsl{on a link}, some \sfsl{fraction},
                         \textsf{f:Fract}, %\tidxi{B: Fract\EQ\kw{Real}}{b-attr-040}
                         down an \sfsl{identified link, l\_ui}, from one of
                         its \sfsl{identified connecting hub}s, \textsf{fh\_ui}, in the 
                         direction of the other \sfsl{identified hub},
                         \textsf{th\_ui}.
\end{enumerate}
\item \label{p-b-attr-050} Et cetera.
\savei\end{enumerate}
\mnewfoil

%\RSLatex
%type
%&\ref{p-b-attr-000}&    LN
%&\ref{p-b-attr-010}&    BusTimTbl 
%&\ref{p-b-attr-020}&    BPos   == atHub | onLink
%&\ref{p-b-attr-030}&  &\,&atHub    &\,&:: h_ui:H_UI
%&\ref{p-b-attr-040}&  &\,&onLink    :: fh_ui:H_UI><l_ui:L_UI><frac:Fract><th_ui:H_UI
%&\ref{p-b-attr-040}&  &\,&Fract     = Real, axiom frac:Fract :- 0<frac<1
%&\ref{p-b-attr-050}&    ...
%value   
%&\ref{p-b-attr-010}&    attr_BusTimTbl: B -> BusTimTbl
%&\ref{p-b-attr-020}&    attr_BPos: B -> BPos
%\endRSLatex 
\bp
\kw{type}\\
\ref{p-b-attr-000}\ \ \ \ LN\\
\ref{p-b-attr-010}\ \ \ \ BusTimTbl \\
\ref{p-b-attr-020}\ \ \ \ BPos\ \ \ {\EQ}{\EQ} atHub {\BAR} onLink\\
\ref{p-b-attr-030}\ \ \,atHub\ \ \ \ \,:: h\_ui:H\_UI\\
\ref{p-b-attr-040}\ \ \,onLink\ \ \ \ :: fh\_ui:H\_UI{\TIMES}l\_ui:L\_UI{\TIMES}frac:Fract{\TIMES}th\_ui:H\_UI\\
\ref{p-b-attr-040}\ \ \,Fract\ \ \ \ \ {\EQ} \kw{Real}, \kw{axiom} frac:Fract {\RDOT} 0{\LT}frac{\LT}1\\
\ref{p-b-attr-050}\ \ \ \ {\DOTDOTDOT}\\
\kw{value}\ \ \ \\
\ref{p-b-attr-010}\ \ \ \ attr\_BusTimTbl: B {\RIGHTARROW} BusTimTbl\\
\ref{p-b-attr-020}\ \ \ \ attr\_BPos: B {\RIGHTARROW} BPos
\ep

\nbbb{Private Automobile Attributes} 

\begynd
\pind We illustrate but a few attributes:
\afslut
\begin{enumerate}\setei
\item \label{p-a-attr-000} Automobiles have static number plate
                         registration numbers.
\item \label{p-a-attr-005} Automobiles have dynamic positions on the road net:
\begin{enumerate}
\item[] [\ref{p-b-attr-030}] either \sfsl{at a hub} identified by some \textsf{h\_ui}, 
\item[] [\ref{p-b-attr-040}] or \sfsl{on a link}, some \sfsl{fraction, frac:Fract} down an
                         \sfsl{identified link, l\_ui}, from one of
                         its \sfsl{identified connecting hub}s, \textsf{fh\_ui}, in the 
                         direction of the other \sfsl{identified hub},
                         \textsf{th\_ui}. 
\end{enumerate}
\savei\end{enumerate}
\mnewfoil
%\RSLatex
%type
%&\ref{p-a-attr-000}&   RegNo& \atidxi{A: RegNo [static]}{a-attr-000}&
%&\ref{p-a-attr-005}&   APos   == atHub | onLink &\atidxi{A: APos{\EQ}{\EQ}atHub\protect{\BAR}onLink [programmable]}{a-attr-005}&
%&\ref{p-b-attr-030}&  atHub    &\,&:: h_ui:H_UI &\tidxi{A: atHub::H\_UI}{b-attr-030}&
%&\ref{p-b-attr-040}&  onLink    :: fh_ui:H_UI >< l_ui:L_UI >< frac:Fract >< th_ui:H_UI&\tidxi{A: onLink::H\_UI{\TIMES}L\_UI{\TIMES}Fract{\TIMES}H\_UI}{b-attr-040}&
%&\ref{p-b-attr-040}&  Fract     = Real, axiom frac:Fract :- 0<frac<1 &\tidxi{A: Frac{\EQ}\kw{Real}}{b-attr-040}&
%value  
%&\ref{p-a-attr-000}&   attr_RegNo: A -> RegNo&\afidxi{A: attr\_RegNo}{a-attr-000}& 
%&\ref{p-a-attr-005}&   attr_APos: A -> APos&\afidxi{A: attr\_APos}{a-attr-005}& 
%\endRSLatex 
\bp
\kw{type}\\
\ref{p-a-attr-000}\ \ \ RegNo \atidxi{A: RegNo [static]}{a-attr-000}\\
\ref{p-a-attr-005}\ \ \ APos\ \ \ {\EQ}{\EQ} atHub {\BAR} onLink \atidxi{A: APos{\EQ}{\EQ}atHub\protect{\BAR}onLink [programmable]}{a-attr-005}\\
\ref{p-b-attr-030}\ \ atHub\ \ \ \ \,:: h\_ui:H\_UI \tidxi{A: atHub::H\_UI}{b-attr-030}\\
\ref{p-b-attr-040}\ \ onLink\ \ \ \ :: fh\_ui:H\_UI {\TIMES} l\_ui:L\_UI {\TIMES} frac:Fract {\TIMES} th\_ui:H\_UI\tidxi{A: onLink::H\_UI{\TIMES}L\_UI{\TIMES}Fract{\TIMES}H\_UI}{b-attr-040}\\
\ref{p-b-attr-040}\ \ Fract\ \ \ \ \ {\EQ} \kw{Real}, \kw{axiom} frac:Fract {\RDOT} 0{\LT}frac{\LT}1 \tidxi{A: Frac{\EQ}\kw{Real}}{b-attr-040}\\
\kw{value}\ \ \\
\ref{p-a-attr-000}\ \ \ attr\_RegNo: A {\RIGHTARROW} RegNo\afidxi{A: attr\_RegNo}{a-attr-000} \\
\ref{p-a-attr-005}\ \ \ attr\_APos: A {\RIGHTARROW} APos\afidxi{A: attr\_APos}{a-attr-005} 
\ep
\mnewfoil
\noindent
\begynd
\pind Obvious attributes that are not illustrated are those of
\begynd
\pind velocity and acceleration, 
\pind forward or backward movement, 
\pind turning right, left or going straight, 
\pind etc.
\afslut
\afslut
\mnewfoil
\noindent
\begynd
\pind The \sfsl{acceleration, deceleration, even velocity,} or \sfsl{turning 
      right, turning left, moving straight}, or \sfsl{forward} or
      \sfsl{backward} are seen as \sfsl{command actions}. 
\begynd
\pind As such they denote actions by the automobile ---
\pind such as \textsf{pressing the accelerator}, or \textsf{lifting
      accelerator pressure} or 
      \sfsl{braking}, or \sfsl{turning the wheel} in one direction or
      another, etc.
\pind As actions they have a kind of counterpart in the
      \textsf{vel}ocity, the \textsf{acc}eleration, etc. attributes.
\afslut
\afslut
\pos{\emcii}{}
\mnewfoil
\begynd
\pind Observe that bus companies each have their own distinct
      \sfsl{bus time table}, and that these are modeled as
      \sfsl{programmable}, Item\,\vref{p-bc-attr-011}, page
      \pageref{p-bc-attr-011}.  
\pind Observe then that buses each have their own distinct \sfsl{bus
      time table}, and that these are model-led as \sfsl{inert},
      Item\,\vref{p-b-attr-010}, page \pageref{p-b-attr-010}.
\dbeat{\pind In Items\,\ref{p-p-c00}--\psref{p-c50}
      we shall see how the buses communicate with
      their respective bus companies in order for the buses to obtain
      the \sfsl{programmed} bus time tables ``in lieu'' of their
      \sfsl{inert} one\,!}
\mnewfoil
\pind In Items\,\psref{hub-traffic} and \psref{link-traffic}, 
      we illustrated an aspect of domain 
      analysis \& description that may seem, and at least some decades
      ago would have seemed, strange: namely that if we can think,
      hence speak, about it, then we can model it ``as a fact'' in the
      domain. The case in point is that we include among hub and link
      attributes their histories of the timed whereabouts of buses and
      automobiles.\footnotemark
\afslut
}\footnotetext{In this day and age of road cameras and
        satellite surveillance these traffic recordings may not appear
        so strange: We now know, at least in principle, of
        technologies that can record approximations to the hub and
        link traffic attributes.} 
\pos{\psno}{\mnewfoil}

\nbbb{Intentionality}\label{p-chap2.Intentionality}

\begin{enumerate}\setei
\item \label{p-pull020} Seen from the point of view of an automobile
  there is its own traffic history, \textsf{A\_Hist},
  which is a (time ordered) sequence of timed automobile's positions; 
\item \label{p-pull030} seen from the point of view of a hub
  there is its own traffic history, \textsf{H\_Traffic} Item\,\psref{hub-traffic},
  which is a (time ordered) sequence of timed maps from automobile
  identities into automobile positions; and 
\item \label{p-pull040} seen from the point of view of a link
  there is its own traffic history, \textsf{L\_Traffic} Item\,\psref{link-traffic},
  which is a (time ordered) sequence of timed maps from automobile
  identities into automobile positions. 
  \savei\end{enumerate}
%\pos{\psno}{\mnewfoil}
\begynd
\pind The \sfsl{intentional ``pull''} of these manifestations is this:
\afslut
\begin{enumerate}\setei
\item \label{p-pull050} The union, i.e. proper merge of all automobile
  traffic histories, \textsf{AllATH}, must now be identical to the
  same  proper merge of all hub, \textsf{AllHTH}, 
  and all link traffic histories, \textsf{AllLTH}.
\savei\end{enumerate}\sf%
\pos{\psno}{\mnewfoil}
%\RSLatex
%type
%&\ref{p-pull020}&    A_Hi  = (&$\mathbb{T}$& >< APos)-list &\atidxi{A: A\_Hi}{pull020}&       
%&\ref{p-hub-traffic}&    H_Trf = A_UI -m-> (&$\mathbb{TIME}$& >< APos)-list&\atidxi{H: H\_Trf [programmable]}{hub-traffic}&
%&\ref{p-link-traffic}&    L_Trf = A_UI-m-> (&$\mathbb{TIME}$&><APos)-list &\atidxi{L: L\_Trf [programmable]}{link-traffic}& 
%&\ref{p-pull050}&    AllATH=&$\mathbb{TIME}$&-m->(AUI-m->APos) 
%&\ref{p-pull050}&    AllHTH=&$\mathbb{TIME}$&-m->(AUI-m->APos) 
%&\ref{p-pull050}&    AllLTH&\,&=&$\mathbb{TIME}$&-m->(AUI-m->APos) 
%axiom
%&\ref{p-pull050}&    let allA=mrg_AllATH({(a,attr_A_Hi(a))|a:A:-a isin &$as$&}),
%&\ref{p-pull050}&        allH=mrg_AllHTH({attr_H_Trf(h)|h:H:-h isin &$hs$&}),
%&\ref{p-pull050}&        allL&\,&=mrg_AllLTH({attr_L_Trf(l)|l:L:-h isin &$ls$&}) in
%&\ref{p-pull050}&    allA = mrg_HLT(allH,allL) end
%\endRSLatex 
\bp
\kw{type}\\
\ref{p-pull020}\ \ \ \ A\_Hi\ \ {\EQ} ($\mathbb{T}$ {\TIMES} APos)$^{\ast}$ \atidxi{A: A\_Hi}{pull020}\ \ \ \ \ \ \ \\
\ref{p-hub-traffic}\ \ \ \ H\_Trf {\EQ} A\_UI {\MARROW} ($\mathbb{TIME}$ {\TIMES} APos)$^{\ast}$\atidxi{H: H\_Trf [programmable]}{hub-traffic}\\
\ref{p-link-traffic}\ \ \ \ L\_Trf {\EQ} A\_UI{\MARROW} ($\mathbb{TIME}${\TIMES}APos)$^{\ast}$ \atidxi{L: L\_Trf [programmable]}{link-traffic} \\
\ref{p-pull050}\ \ \ \ AllATH{\EQ}$\mathbb{TIME}${\MARROW}(AUI{\MARROW}APos) \\
\ref{p-pull050}\ \ \ \ AllHTH{\EQ}$\mathbb{TIME}${\MARROW}(AUI{\MARROW}APos) \\
\ref{p-pull050}\ \ \ \ AllLTH\,{\EQ}$\mathbb{TIME}${\MARROW}(AUI{\MARROW}APos) \\
\kw{axiom}\\
\ref{p-pull050}\ \ \ \ \kw{let} allA{\EQ}mrg\_AllATH({\LBRACE}(a,attr\_A\_Hi(a)){\BAR}a:A{\RDOT}a {\ISIN} $as${\RBRACE}),\\
\ref{p-pull050}\ \ \ \ \ \ \ \ allH{\EQ}mrg\_AllHTH({\LBRACE}attr\_H\_Trf(h){\BAR}h:H{\RDOT}h {\ISIN} $hs${\RBRACE}),\\
\ref{p-pull050}\ \ \ \ \ \ \ \ allL\,{\EQ}mrg\_AllLTH({\LBRACE}attr\_L\_Trf(l){\BAR}l:L{\RDOT}h {\ISIN} $ls${\RBRACE}) \kw{in}\\
\ref{p-pull050}\ \ \ \ allA {\EQ} mrg\_HLT(allH,allL) \kw{end}
\ep

\mnewfoil
\noindent%
\normalsize\HHHH\rm
\begynd%
\pind We leave the definition of the \pos{four}{} \textsf{merge} functions
      to the \pos{reader}{listener}\,!
\begynd
\pind We endow 
\begynd
\pind each automobile with its history of timed positions and
\pind each hub and link with  their histories of timed automobile positions.
\afslut 
\pind These histories  are facts\,!
\pind They are not something that is laboriously recorded, \nyl 
      where such recordings may be imprecise or cumbersome\footnote{\LLLL or
      thought technologically in-feasible -- at least some decades
      ago!}.
\pind The facts are there, so we can (but may not necessarily) \nyl talk
      about these histories as facts.
\mnewfoil
\pind It is in that sense that the purpose (\texttt{`transport'})
\begynd
\pind for which man let automobiles, hubs and link be made 
\pind with their \texttt{`transport'} intent
\pind are subject to an \sfsl{intentional ``pull''.}
\afslut
\afslut
\pind \sfsl{It can be no other way:  if automobiles ``record''
      their history, then hubs and links must together ``record''
      identically the same history\,!}.
\afslut
\normalsize\HHHH\rm

\mnewfoil
   
\vspace{2mm}
 
\noindent
\sort{Intentional Pull -- General Transport:}{
\begynd
\pind These are examples of human intents:
\begynd
\pind they create \sfsl{roads} and \sfsl{automobiles} \nyl  with the
intent of \sfsl{transport}, 
\pind they create \sfsl{houses} \nyl  with the intents of
\sfsl{living, offices, 
      production}, etc., and
\pind they create \sfsl{pipelines} \nyl with the intent of  \sfsl{oil}
      or \sfsl{gas transport} \eod\ \ \ 
\afslut
\afslut
}


\nbbbbb{Perdurants}

\noindent
\begynd
\pind In this section we transcendentally ``morph'' \nyl \sort{parts} into \sort{behaviours.}
\pind We analyse that notion and its constituent notions of
\begynd
\pind \sort{actors},
\pind \sort{channels} and \sort{communication},
\pind \sort{actions} and
\pind \sort{events}.
\afslut 
\afslut

\begynd
\pind The main transcendental deduction of this chapter
\begynd
\pind is that of associating 
\pind with each part
\pind a behaviour.
\afslut
\pind This section shows the details of that association.
\mnewfoil
\pind Perdurants are understood in terms of
\begynd 
\pind a notion of \pcindextermi{state} and 
\pind a notion of \pcindextermi{time}.
\afslut
\afslut
\label{p-kapitel6.eoi}

\mnewfoil

\pos{\vspace{2mm}}{}
 
\noindent
\sort{State Values versus State Variables:}{
\begynd
\pind Item \vref{p-srares-090} expresses the \bbcolor{value} of all parts of a
      road transport system:
\afslut
%\RSLatex
%&\ref{p-srares-090}.&  &$ps$&:(UoB|H|L|BC|B|A)-set is &$rts$&union&$hls$&union&$bcs$&union&$bs$&union&$as$&.
%\endRSLatex 
\bp
\ref{p-srares-090}.\ \ $ps$:(UoB{\BAR}H{\BAR}L{\BAR}BC{\BAR}B{\BAR}A)\kw{-set} {\IS} $rts${\UNION}$hls${\UNION}$bcs${\UNION}$bs${\UNION}$as$.
\ep
\begin{enumerate}\setei
\item \label{p-variable-000} We now introduce the set of variables, one
  for each part value of the domain being modeled.
\savei\end{enumerate}
%\RSLatex
%&\ref{p-variable-000}.& { variable &$vp$&:(UoB|H|L|BC|B|A) | &$vp$&:(UoB|H|L|BC|B|A) :- &$vp$&isin&$ps$& }
%\endRSLatex 
\bp
\ref{p-variable-000}. {\LBRACE} \kw{variable} $vp$:(UoB{\BAR}H{\BAR}L{\BAR}BC{\BAR}B{\BAR}A) {\BAR} $vp$:(UoB{\BAR}H{\BAR}L{\BAR}BC{\BAR}B{\BAR}A) {\RDOT} $vp${\ISIN}$ps$ {\RBRACE}
\ep
\vspace*{-3mm}
}



\mnewfoil

\pos{\vspace{2mm}}{}
 
\noindent
\sort{Buses and Bus Companies}{ %
\begynd%
\pind A bus company is like a ``root'' for its fleet of ``sibling'' buses.
\pind But a bus company may cease to exist without the buses therefore
      necessarily also ceasing to exist.
\pind They may continue to operate, probably illegally, without,
      possibly. \nyl a valid bus driving certificate.
\pind Or they may be passed on to either private owners or to other bus companies.
\pind We use this example as a reason for not endowing a ``block
      structure'' concept on behaviours.
\afslut
}

\nbbbb{Channels and Communication}\label{p-mono.Channels}\label{p-kap6.Channels}

\bbb{Channel Message Types}

\begynd
\pind We ascribe types to the messages offered on channels. 
\begin{enumerate}\setei
\item \label{p-chmsg-000} Hubs and links communicate, both ways, with
                        one another, over channels, \texttt{hl\_ch},
                        whose indexes are determined by their mereologies.
\item \label{p-chmsg-010} Hubs send one kind of messages, links another.
\item \label{p-chmsg-020} Bus companies offer timed bus time tables to
                        buses, one way.
\item \label{p-chmsg-030} Buses and automobiles offer their current,
                        timed positions to the road element, hub or
                        link they are on, one way. 
\savei\end{enumerate}\footsize
\afslut

\LLLL %\pos{\psno}{\mnewfoil}

%\RSLatex
%type
%&\ref{p-chmsg-010}&  H_L_Msg, L_H_Msg &\ctidxi{H\_L\_Msg}{chmsg-000}\ctidxi{L\_H\_Msg}{chmsg-000}&
%&\ref{p-chmsg-000}&  HL_Msg = H_L_Msg | L_F_Msg &\ctidxi{HL\_Msg{\EQ}H\_L\_Msg\protect{\BAR}L\_F\_Msg}{chmsg-000}&
%&\ref{p-chmsg-020}&  BC_B_Msg = T >< BusTimTbl &\ctidxi{BC\_B\_Msg{\EQ}(T{\TIMES}BusTimTbl)}{chmsg-010}&
%&\ref{p-chmsg-030}&  V_R_Msg = T >< (BPos|APos) &\ctidxi{V\_R\_Msg{\EQ}(T{\TIMES}(BPos\protect{\BAR}APos))}{chmsg-020}&
%\endRSLatex 
\bp
\kw{type}\\
\ref{p-chmsg-010}\ \ H\_L\_Msg, L\_H\_Msg \ctidxi{H\_L\_Msg}{chmsg-000}\ctidxi{L\_H\_Msg}{chmsg-000}\\
\ref{p-chmsg-000}\ \ HL\_Msg {\EQ} H\_L\_Msg {\BAR} L\_F\_Msg \ctidxi{HL\_Msg{\EQ}H\_L\_Msg\protect{\BAR}L\_F\_Msg}{chmsg-000}\\
\ref{p-chmsg-020}\ \ BC\_B\_Msg {\EQ} T {\TIMES} BusTimTbl \ctidxi{BC\_B\_Msg{\EQ}(T{\TIMES}BusTimTbl)}{chmsg-010}\\
\ref{p-chmsg-030}\ \ V\_R\_Msg {\EQ} T {\TIMES} (BPos{\BAR}APos) \ctidxi{V\_R\_Msg{\EQ}(T{\TIMES}(BPos\protect{\BAR}APos))}{chmsg-020}
\ep
\smallish

\pos{\psno}{\mnewfoil}

\nbbb{Channel Declarations}
\begin{enumerate}\setei
\item \label{p-ch-dcl-00} This justifies the channel declaration which
                        is calculated to be:
                        \cidxi{hl\_ch[i,j]:HL\_Msg}{ch-dcl-00} 
\savei\end{enumerate}\footsize\LLLL\HHHH
%\RSLatex 
%channel
%&\ref{p-ch-dcl-00}&   { hl_ch[h_ui,l_ui]:H_L_Msg | h_ui:H_UI,l_ui:L_UI:-i isin &$h_{ui}s$&/\j isin &$lh_{ui}m$&(h_ui) }
%&\ref{p-ch-dcl-00}&   union
%&\ref{p-ch-dcl-00}&   { hl_ch[h_ui,l_ui]:L_H_Msg | h_ui:H_UI,l_ui:L_UI:-l_ui isin &$l_{ui}s$&/\i isin &$lh_{ui}m$&(l_ui) }
%\endRSLatex 
\bp
\kw{channel}\\
\ref{p-ch-dcl-00}\ \ \ {\LBRACE} hl\_ch{\LBRACKET}h\_ui,l\_ui{\RBRACKET}:H\_L\_Msg {\BAR} h\_ui:H\_UI,l\_ui:L\_UI{\RDOT}i {\ISIN} $h_{ui}s${\WEDGE}j {\ISIN} $lh_{ui}m$(h\_ui) {\RBRACE}\\
\ref{p-ch-dcl-00}\ \ \ {\UNION}\\
\ref{p-ch-dcl-00}\ \ \ {\LBRACE} hl\_ch{\LBRACKET}h\_ui,l\_ui{\RBRACKET}:L\_H\_Msg {\BAR} h\_ui:H\_UI,l\_ui:L\_UI{\RDOT}l\_ui {\ISIN} $l_{ui}s${\WEDGE}i {\ISIN} $lh_{ui}m$(l\_ui) {\RBRACE}
\ep
\smallish
\pos{\psno}{\mnewfoil}

\noindent
\begynd
\pind We shall argue for bus company-to-bus channels
      based on the mereologies of those parts. 
\begynd
\pind Bus companies need communicate to all its buses, but not the
      buses of other bus companies.
\pind Buses of a bus company need communicate to their bus company,
      but not to other bus companies. 
\afslut
\afslut
\begin{enumerate}\setei
\item \label{p-ch-dcl-10} This justifies the channel declaration which
                        is calculated to be:
                        \cidxi{bc\_b\_ch[i,j]:BC\_B\_Msg}{ch-dcl-10} 
\savei\end{enumerate}\footsize\LLLL%\HHHH
%\RSLatex
%channel
%&\ref{p-ch-dcl-10}&   { bc_b_ch[bc_ui,b_ui] | bc_ui:BC_UI, b_ui:B_UI :- bc_ui isin &$bc_{ui}s$& /\ b_ui isin &$b_{ui}s$& }: BC_B_Msg
%\endRSLatex 
\bp
\kw{channel}\\
\ref{p-ch-dcl-10}\ \ \ {\LBRACE} bc\_b\_ch{\LBRACKET}bc\_ui,b\_ui{\RBRACKET} {\BAR} bc\_ui:BC\_UI, b\_ui:B\_UI {\RDOT} bc\_ui {\ISIN} $bc_{ui}s$ {\WEDGE} b\_ui {\ISIN} $b_{ui}s$ {\RBRACE}: BC\_B\_Msg
\ep
\smallish
\pos{\psno}{\mnewfoil}

\noindent
\begynd
\pind We shall argue for vehicle to road element channels
      based on the mereologies of those parts. 
\begynd
\pind Buses and automobiles need communicate to
\begynd
\pind all hubs and
\pind all links.
\afslut
\afslut
\afslut
\begin{enumerate}\setei
\item \label{p-ch-dcl-20} This justifies the channel declaration
                        which is calculated to be:
                        \cidxi{v\_r\_ch[i,j]:V\_R\_Msg}{ch-dcl-20}   
\savei\end{enumerate}\footsize\HHHH\LLLL                  
%\RSLatex
%channel
%&\ref{p-ch-dcl-20}&   { v_r_ch[v_ui,r_ui] | v_ui:V_UI,r_ui:R_UI :- v_&ui&isin&\,$v_{ui}s$&/\r_&ui&isin&\,$r_{ui}s$& }: V_R_Msg 
%\endRSLatex 
\bp
\kw{channel}\\
\ref{p-ch-dcl-20}\ \ \ {\LBRACE} v\_r\_ch{\LBRACKET}v\_ui,r\_ui{\RBRACKET} {\BAR} v\_ui:V\_UI,r\_ui:R\_UI {\RDOT} v\_ui{\ISIN}\,$v_{ui}s${\WEDGE}r\_ui{\ISIN}\,$r_{ui}s$ {\RBRACE}: V\_R\_Msg 
\ep
\normalsize\rm\HHHH
\label{p-mono.Channels.n}

\nbbbb{Behaviours}

\bbb{Road Transport Behaviour Signatures}{
\noindent
\begynd
\pind We first decide on names of behaviours.
\begynd
\pind In the translation schemas
\pind we gave schematic names to behaviours 
\pind of the form $\mathcal{M}{_{P}}$.
\afslut
\pind We now assign mnemonic names: 
\begynd
\pind from part names to names of
      transcendentally interpreted behaviours
\pind and then we assign signatures to these behaviours.
\afslut
\afslut
\pos{\psno}{\mnewfoil}

\nbb{Hub Behaviour Signature}

\begin{enumerate}\setei
\item \label{p-process-000} \textsf{hub${_{h_{ui}}}$}:
\begin{enumerate}
\item \label{p-process-002} there is the usual ``triplet'' of arguments:
                          unique identifier, mereology and static
                          attributes; 
\item \label{p-process-004} then there are the programmable attributes;
\item \label{p-process-006} and finally there are the input/output
                          channel references: 
                          first those allowing communication between
                          hub and link behaviours, 
\item \label{p-process-008} and then those  allowing communication between
                          hub and vehicle (bus and automobile) behaviours.
\end{enumerate}
\savei\end{enumerate}

\pos{\psno}{\mnewfoil}
\footsize

\bidx{hub{$_{h_{ui}}$}}{process-000}
\dbeat{\bidxi{hub{$_{h_{ui}}$}: {\protect{h\_ui:H\_UI{\TIMES}(vuis,luis,{\_}):H{\_}Mer{\TIMES}H$\Omega$}}\\
{\RIGHTARROW} (H\protect{$\Sigma$}{\TIMES}H\_Traffic)\\ 
{\RIGHTARROW} \kw{in}
\protect{\LBRACE}ba\_r\_ch{\LBRACKET}h\_ui,v\_ui{\RBRACKET}\protect{\BAR}v\_ui:V\_UI\,{\RDOT}\,v\_ui\protect{\ISIN}vuis\protect{\RBRACE}  
{\RIGHTARROW} \kw{in,out}
  \protect{\LBRACE}h\_l\_ch{\LBRACKET}h\_ui,l\_ui{\RBRACKET}\protect{\BAR}l\_ui:L\_UI{\RDOT}l\_ui{\ISIN}luis\protect{\RBRACE}
\kw{Unit}}{process-000}}\HHHH
%\RSLatex
%value
%&\ref{p-process-000}&   hub&$_{h_{ui}}$&: 
%&\ref{p-process-002}&   h_ui:H_UI><(vuis,luis,_):H_Mer><H`Omega
%&\ref{p-process-004}&     -> (H`Sigma><H_Traffic)
%&\ref{p-process-006}&     -> in,out { h_l_ch[h_ui,l_ui] | l_ui:L_UI:-l_ui isin luis }
%&\ref{p-process-008}&           { ba_r_ch[h_ui,v_ui] | v_ui:V_UI:-&v\_ui&isin&vuis& } Unit
%&\ref{p-process-002}&      pre: vuis = &$v_{ui}s$& /\ luis = &$l_{ui}s$&
%\endRSLatex 
\bp
\kw{value}\\
\ref{p-process-000}\ \ \ hub$_{h_{ui}}$: \\
\ref{p-process-002}\ \ \ h\_ui:H\_UI{\TIMES}(vuis,luis,{\UNDERLINE}):H\_Mer{\TIMES}H$\Omega$\\
\ref{p-process-004}\ \ \ \ \ {\RIGHTARROW} (H$\Sigma${\TIMES}H\_Traffic)\\
\ref{p-process-006}\ \ \ \ \ {\RIGHTARROW} \kw{in},\kw{out} {\LBRACE} h\_l\_ch{\LBRACKET}h\_ui,l\_ui{\RBRACKET} {\BAR} l\_ui:L\_UI{\RDOT}l\_ui {\ISIN} luis {\RBRACE}\\
\ref{p-process-008}\ \ \ \ \ \ \ \ \ \ \ {\LBRACE} ba\_r\_ch{\LBRACKET}h\_ui,v\_ui{\RBRACKET} {\BAR} v\_ui:V\_UI{\RDOT}v\_ui{\ISIN}vuis {\RBRACE} \kw{Unit}\\
\ref{p-process-002}\ \ \ \ \ \ \kw{pre}: vuis {\EQ} $v_{ui}s$ {\WEDGE} luis {\EQ} $l_{ui}s$
\ep
\pos{\psno}{\mnewfoil}

\nbb{Link Behaviour Signature}

\begin{enumerate}\setei
\item \label{p-process-010} \textsf{link${_{l_{ui}}}$}:
\begin{enumerate}
\item \label{p-process-012} there is the usual ``triplet'' of arguments:
                          unique identifier, mereology and static
                          attributes;
\item \label{p-process-014} then there are the programmable attributes;
\item \label{p-process-016} and finally there are the input/output channel references:
                          first those allowing communication between
                          hub and link behaviours, 
\item \label{p-process-018} and then those  allowing communication between
                          link and vehicle (bus and automobile) behaviours.
\end{enumerate}
\savei\end{enumerate}\footsize\sf\HHHH
\pos{\psno}{\mnewfoil}
\bidx{link{$_{l_{ui}}$}}{process-010}
\dbeat{\bidxi{link{$_{l_{ui}}$}: {\protect{l\_ui:L\_UI{\TIMES}(vuis,huis,{\_}):L{\_}Mer{\TIMES}L$\Omega$}}\\
{\RIGHTARROW} (L\protect{$\Sigma$}{\TIMES}L\_Traffic)\\ 
{\RIGHTARROW} \kw{in}
\protect{\LBRACE}ba\_r\_ch{\LBRACKET}l\_ui,v\_ui{\RBRACKET}\protect{\BAR}v\_ui:V\_UI\,{\RDOT}\,v\_ui\protect{\ISIN}vuis\protect{\RBRACE}  
{\RIGHTARROW} \kw{in,out}
  \protect{\LBRACE}h\_l\_ch{\LBRACKET}h\_ui,l\_ui{\RBRACKET}\protect{\BAR}h\_ui:H\_UI:h\_ui{\ISIN}huis\protect{\RBRACE}
\kw{Unit}}{process-010}} 
%\RSLatex
%value
%&\ref{p-process-010}&   link&$_{l_{ui}}$&:
%&\ref{p-process-012}&   l_ui:L_UI><(vuis,huis,_):L_Mer><L`Omega
%&\ref{p-process-014}&     -> (L`Sigma><L_Traffic)
%&\ref{p-process-016}&     -> in,out { h_l_ch[h_ui,l_ui] | h_ui:H_UI:h_ui isin huis }
%&\ref{p-process-018}&           { ba_r_ch[l_ui,v_ui] | v_ui:(B_UI|A_UI):-&v\_ui&isin&vuis& } Unit
%&\ref{p-process-012}&     pre: vuis = &$v_{ui}s$& /\ huis = &$h_{ui}s$&
%\endRSLatex 
\bp
\kw{value}\\
\ref{p-process-010}\ \ \ link$_{l_{ui}}$:\\
\ref{p-process-012}\ \ \ l\_ui:L\_UI{\TIMES}(vuis,huis,{\UNDERLINE}):L\_Mer{\TIMES}L$\Omega$\\
\ref{p-process-014}\ \ \ \ \ {\RIGHTARROW} (L$\Sigma${\TIMES}L\_Traffic)\\
\ref{p-process-016}\ \ \ \ \ {\RIGHTARROW} \kw{in},\kw{out} {\LBRACE} h\_l\_ch{\LBRACKET}h\_ui,l\_ui{\RBRACKET} {\BAR} h\_ui:H\_UI:h\_ui {\ISIN} huis {\RBRACE}\\
\ref{p-process-018}\ \ \ \ \ \ \ \ \ \ \ {\LBRACE} ba\_r\_ch{\LBRACKET}l\_ui,v\_ui{\RBRACKET} {\BAR} v\_ui:(B\_UI{\BAR}A\_UI){\RDOT}v\_ui{\ISIN}vuis {\RBRACE} \kw{Unit}\\
\ref{p-process-012}\ \ \ \ \ \kw{pre}: vuis {\EQ} $v_{ui}s$ {\WEDGE} huis {\EQ} $h_{ui}s$
\ep
\pos{\psno}{\mnewfoil}

\nbb{Bus Company Behaviour Signature}

\begin{enumerate}\setei
\item \label{p-process-020} \textsf{bus\_company$_{bc_{ui}}$}:
\begin{enumerate}
\item \label{p-process-022} there is here just a ``doublet''  of
                          arguments: unique identifier and mereology;
\item \label{p-process-024} then there is the one programmable attribute;
\item \label{p-process-026} and finally there are the input/output
                          channel references allowing communication between
                          the bus company and buses.
\end{enumerate}
\savei\end{enumerate}\footsize\sf\HHHH
\pos{\psno}{\mnewfoil}
\bidx{bus\_company{$_{bc_{ui}}$}}{process-020}
\dbeat{\bidxi{bus\_company{$_{bc_{ui}}$}: {\protect{bc\_ui:BC{\_}UI{\TIMES}(\_,\_,buis):BC{\_}Mer}}\\
{\RIGHTARROW} \kw{in} attr\_T\_ch,
\kw{out} \protect{\LBRACE}bc\_b\_ch{\LBRACKET}bc\_ui,b\_ui{\RBRACKET}\protect{\BAR}b\_ui:B\_UI\,{\RDOT}\,b\_ui\protect{\ISIN}{buis}\protect{\RBRACE} \kw{Unit}}{process-020}}
%\RSLatex
%value
%&\ref{p-process-020}&   bus_company&$_{bc_{ui}}$&: 
%&\ref{p-process-022}&   bc_ui:BC_UI><(_,_,buis):BC_Mer
%&\ref{p-process-024}&     -> BusTimTbl
%&\ref{p-process-026}&        in,out {bc_b_ch[bc_ui,b_ui]|b_ui:B_UI:-&b\_ui&isin&buis&} Unit
%&\ref{p-process-022}&     pre: buis = &$b_{ui}s$& /\ huis = &$h_{ui}s$&
%\endRSLatex 
\bp
\kw{value}\\
\ref{p-process-020}\ \ \ bus\_company$_{bc_{ui}}$: \\
\ref{p-process-022}\ \ \ bc\_ui:BC\_UI{\TIMES}({\UNDERLINE},{\UNDERLINE},buis):BC\_Mer\\
\ref{p-process-024}\ \ \ \ \ {\RIGHTARROW} BusTimTbl\\
\ref{p-process-026}\ \ \ \ \ \ \ \ \kw{in},\kw{out} {\LBRACE}bc\_b\_ch{\LBRACKET}bc\_ui,b\_ui{\RBRACKET}{\BAR}b\_ui:B\_UI{\RDOT}b\_ui{\ISIN}buis{\RBRACE} \kw{Unit}\\
\ref{p-process-022}\ \ \ \ \ \kw{pre}: buis {\EQ} $b_{ui}s$ {\WEDGE} huis {\EQ} $h_{ui}s$
\ep
\pos{\psno}{\mnewfoil}

\nbb{Bus Behaviour Signature}

\begin{enumerate}\setei
\item \label{p-process-030} \textsf{bus$_{b_{ui}}$}:
\begin{enumerate}
\item \label{p-process-032} there is here just a ``doublet''  of
                          arguments: unique identifier and mereology;
\item \label{p-process-034} then there are the programmable attributes;
\item \label{p-process-036} and finally there are the input/output
                          channel references: first the input/output
                          allowing communication between 
                          the bus company and buses,
\item \label{p-process-038} and the  input/output allowing communication between
                          the bus and the hub and link behaviours.
\end{enumerate}
\savei\end{enumerate}\footsize\sf\HHHH
\pos{\psno}{\mnewfoil}
\bidx{bus{$_{b_{ui}}$}}{process-030}
\dbeat{\bidxi{bus{$_{b_{ui}}$}: {\protect{b\_ui:B{\_}UI{\TIMES}(bc\_ui,\_,ruis):B{\_}Mer}}\\
{\RIGHTARROW} bpos:BPos
{\RIGHTARROW}
\kw{out} \protect{\LBRACE}ba\_r\_ch{\LBRACKET}b\_ui,r\_ui{\RBRACKET}\protect{\BAR}r\_ui:R\_UI\,{\RDOT}\,r\_ui\protect{\ISIN}{ruis}\protect{\RBRACE} \kw{Unit}}{process-030}}
%\RSLatex
%value
%&\ref{p-process-030}&   bus&$_{b_{ui}}$&:
%&\ref{p-process-032}&   b_ui:B_UI><(bc_ui,_,ruis):B_Mer
%&\ref{p-process-034}&     -> (LN >< BTT >< BPOS) 
%&\ref{p-process-036}&     -> out bc_b_ch[bc_ui,b_ui], 
%&\ref{p-process-038}&           {ba_r_ch[r_ui,b_ui]|r_ui:(H_UI|L_UI):-&ui&isin&$v_{ui}s$&} Unit
%&\ref{p-process-032}&     pre: ruis = &$r_{ui}s$& /\ bc_ui isin &$bc_{ui}s$&
%\endRSLatex 
\bp
\kw{value}\\
\ref{p-process-030}\ \ \ bus$_{b_{ui}}$:\\
\ref{p-process-032}\ \ \ b\_ui:B\_UI{\TIMES}(bc\_ui,{\UNDERLINE},ruis):B\_Mer\\
\ref{p-process-034}\ \ \ \ \ {\RIGHTARROW} (LN {\TIMES} BTT {\TIMES} BPOS) \\
\ref{p-process-036}\ \ \ \ \ {\RIGHTARROW} \kw{out} bc\_b\_ch{\LBRACKET}bc\_ui,b\_ui{\RBRACKET}, \\
\ref{p-process-038}\ \ \ \ \ \ \ \ \ \ \ {\LBRACE}ba\_r\_ch{\LBRACKET}r\_ui,b\_ui{\RBRACKET}{\BAR}r\_ui:(H\_UI{\BAR}L\_UI){\RDOT}ui{\ISIN}$v_{ui}s${\RBRACE} \kw{Unit}\\
\ref{p-process-032}\ \ \ \ \ \kw{pre}: ruis {\EQ} $r_{ui}s$ {\WEDGE} bc\_ui {\ISIN} $bc_{ui}s$
\ep
\pos{\psno}{\mnewfoil}

\nbb{Automobile Behaviour Signature}

\begin{enumerate}\setei              
\item \label{p-process-040} \textsf{automobile$_{a_{ui}}$}:
\begin{enumerate}
\item \label{p-process-042} there is the usual ``triplet'' of arguments:
                          unique identifier, mereology and static
                          attributes;
\item \label{p-process-044} then there is the one programmable attribute;
\item \label{p-process-046} and finally there are the input/output
                          channel references allowing communication between
                          the automobile and the hub and link behaviours.
\end{enumerate}
\savei\end{enumerate}\footsize\sf\HHHH
\pos{\psno}{\mnewfoil}
\bidx{automobile{$_{a_{ui}}$}}{process-040}
\dbeat{\bidxi{automobile{$_{a_{ui}}$}: {\protect{a\_ui:A{\_}UI{\TIMES}(\_,\_,ruis):A{\_}Mer{\TIMES}RegNo}}\\
{\RIGHTARROW} apos:APos
{\RIGHTARROW}
\kw{out} \protect{\LBRACE}ba\_r\_ch{\LBRACKET}a\_ui,r\_ui{\RBRACKET}\protect{\BAR}r\_ui:R\_UI\,{\RDOT}\,r\_ui\protect{\ISIN}{ruis}\protect{\RBRACE} \kw{Unit}}{process-040}}
%\RSLatex
%value
%&\ref{p-process-040}&   automobile&$_{a_{ui}}$&: 
%&\ref{p-process-042}&   a_ui:A_UI><(_,_,ruis):A_Mer><rn:RegNo
%&\ref{p-process-044}&     -> apos:APos
%&\ref{p-process-046}&        in,out {ba_r_ch[a_ui,r_ui]|r_ui:(H_UI|L_UI):-&r\_ui&isin&ruis&} Unit
%&\ref{p-process-042}&     pre: ruis = &$r_{ui}s$& /\ a_ui isin &$a_{ui}s$ \eox&
%\endRSLatex
\bp
\kw{value}\\
\ref{p-process-040}\ \ \ automobile$_{a_{ui}}$: \\
\ref{p-process-042}\ \ \ a\_ui:A\_UI{\TIMES}({\UNDERLINE},{\UNDERLINE},ruis):A\_Mer{\TIMES}rn:RegNo\\
\ref{p-process-044}\ \ \ \ \ {\RIGHTARROW} apos:APos\\
\ref{p-process-046}\ \ \ \ \ \ \ \ \kw{in},\kw{out} {\LBRACE}ba\_r\_ch{\LBRACKET}a\_ui,r\_ui{\RBRACKET}{\BAR}r\_ui:(H\_UI{\BAR}L\_UI){\RDOT}r\_ui{\ISIN}ruis{\RBRACE} \kw{Unit}\\
\ref{p-process-042}\ \ \ \ \ \kw{pre}: ruis {\EQ} $r_{ui}s$ {\WEDGE} a\_ui {\ISIN} $a_{ui}s$ \eox
\ep
\normalsize\rm
\vspace*{1mm}
}\normalsize\rm\HHHH
%%  LocalWords:  behaviours ui mereology vuis luis Mer ba pre huis bc
%%  LocalWords:  buis attr BusTimTbl ruis bpos BPos BTT RegNo apos rn
%%  LocalWords:  APos Behaviour schemas hl lh oad ompany utomobile bm

\index{defind}{Behaviour!Signatures|)}
\label{p-kap6-end-of-Signatures}

\nbbb{Behaviour Definitions}

\begynd
\pind We only illustrate automobile, hub and link behaviours.
\afslut

\bb{Automobile Behaviour at a Hub}{ \label{p-x.Automobile
    Behaviour (at a hub)}\label{p-x.Automobile Behaviour} 
\noindent
\begynd
\pind We define the behaviours in a different order than the treatment 
      of their signatures.
\pind We ``split'' definition of the \textsf{automobile} behaviour
\begynd
\pind into the behaviour of \textsf{automobile}s when positioned at
      a hub, and
\pind into the behaviour \textsf{automobile}s when positioned at
      on a link.
\pind In both cases the behaviours include the ``idling'' of the
      automobile, i.e., its ``not moving'', standing still.
\afslut
\afslut
\pos{\normalsize}{\LLLL}\rm
\mnewfoil
\begin{enumerate}\setei
\item \label{p-habff1} We abstract automobile
                    behaviour \textsf{at} a 
                    \textsf{Hub} (\textsf{hui}). 
\item \label{p-habff2} The vehicle remains at that hub, ``idling'', 
\item \label{p-habff3} informing
                    the hub behaviour,
\item \label{p-habff4} or, internally non-deterministically, 
\begin{enumerate} 
\item \label{p-habff5} moves
                    onto a link, \textsf{tli}, whose ``next'' 
                    hub, identified by \textsf{th\_ui}, is obtained from
                    the mereology of the link identified by \textsf{tl\_ui};  
\item \label{p-habff6} informs the hub it is leaving and the link it is
                    entering of its initial link position,
\item \label{p-habff7} whereupon the vehicle resumes the vehicle
                    behaviour positioned at the very beginning
                    (\textsf{0}) of that link,
\end{enumerate}
\item \label{p-habff8} or, again internally non-deterministically, 
\item \label{p-habff9} the vehicle ``disappears --- off the radar''~!
\savei\end{enumerate}
\pos{\normalsize}{\HHHH}\rm
\mnewfoil
%\RSLatex
%&\ref{p-habff1}&  automobile&$_{a_{ui}}$&(a_ui,({},(ruis,vuis),{}),rn)
%&\ref{p-habff1}&             (apos:atH(fl_ui,h_ui,tl_ui)) is
%&\ref{p-habff2}&      (ba_r_ch[a_ui,h_ui] ! (&\recordtime&,atH(fl_ui,h_ui,tl_ui));
%&\ref{p-habff3}&       automobile&$_{a_{ui}}$&(a_ui,({},(ruis,vuis),{}),rn)(apos))
%&\ref{p-habff4}&     |^|
%&\ref{p-habff5}&     (let ({fh_ui,th_ui},ruis&$'$&)=mereo_L(&$\wp$&(tl_ui)) in
%&\ref{p-habff5}&           &\kw{assert:}& fh_ui=h_ui /\ ruis=ruis&$'$&
%&\ref{p-habff1}&       let onl = (tl_ui,h_ui,0,th_ui) in
%&\ref{p-habff6}&      (ba_r_ch[a_ui,h_ui] ! (&\recordtime&,onL(onl)) ||
%&\ref{p-habff6}&       ba_r_ch[a_ui,tl_ui] ! (&\recordtime&,onL(onl))) ;
%&\ref{p-habff7}&       automobile&$_{a_{ui}}$&(a_ui,({},(ruis,vuis),{}),rn)
%&\ref{p-habff7}&              (onL(onl)) end end)
%&\ref{p-habff8}&     |^|
%&\ref{p-habff9}&        stop
%\endRSLatex 
\bp
\ref{p-habff1}\ \ automobile$_{a_{ui}}$(a\_ui,({\LBRACE}{\RBRACE},(ruis,vuis),{\LBRACE}{\RBRACE}),rn)\\
\ref{p-habff1}\ \ \ \ \ \ \ \ \ \ \ \ \ (apos:atH(fl\_ui,h\_ui,tl\_ui)) {\IS}\\
\ref{p-habff2}\ \ \ \ \ \ (ba\_r\_ch{\LBRACKET}a\_ui,h\_ui{\RBRACKET} ! (\recordtime,atH(fl\_ui,h\_ui,tl\_ui));\\
\ref{p-habff3}\ \ \ \ \ \ \ automobile$_{a_{ui}}$(a\_ui,({\LBRACE}{\RBRACE},(ruis,vuis),{\LBRACE}{\RBRACE}),rn)(apos))\\
\ref{p-habff4}\ \ \ \ \ {\NONDETCHOICE}\\
\ref{p-habff5}\ \ \ \ \ (\kw{let} ({\LBRACE}fh\_ui,th\_ui{\RBRACE},ruis$'$){\EQ}mereo\_L($\wp$(tl\_ui)) \kw{in}\\
\ref{p-habff5}\ \ \ \ \ \ \ \ \ \ \ \kw{assert:} fh\_ui{\EQ}h\_ui {\WEDGE} ruis{\EQ}ruis$'$\\
\ref{p-habff1}\ \ \ \ \ \ \ \kw{let} onl {\EQ} (tl\_ui,h\_ui,0,th\_ui) \kw{in}\\
\ref{p-habff6}\ \ \ \ \ \ (ba\_r\_ch{\LBRACKET}a\_ui,h\_ui{\RBRACKET} ! (\recordtime,onL(onl)) {\PARL}\\
\ref{p-habff6}\ \ \ \ \ \ \ ba\_r\_ch{\LBRACKET}a\_ui,tl\_ui{\RBRACKET} ! (\recordtime,onL(onl))) ;\\
\ref{p-habff7}\ \ \ \ \ \ \ automobile$_{a_{ui}}$(a\_ui,({\LBRACE}{\RBRACE},(ruis,vuis),{\LBRACE}{\RBRACE}),rn)\\
\ref{p-habff7}\ \ \ \ \ \ \ \ \ \ \ \ \ \ (onL(onl)) \kw{end} \kw{end})\\
\ref{p-habff8}\ \ \ \ \ {\NONDETCHOICE}\\
\ref{p-habff9}\ \ \ \ \ \ \ \ \kw{stop}
\ep
}\rm\HHHH


\nbb{Automobile Behaviour On a Link}
{
\bmcii%\pos{\begin{multicols}{2}}{}
\begin{enumerate}\setei
\item \label{p-abff0} We abstract automobile
                    behaviour \textsf{on} a \textsf{Link}. 
\begin{enumerate}
\item \label{p-abff1} Internally non-deterministically,  either 
\begin{enumerate}
\item \label{p-abff3} the automobile remains, ``idling'', i.e., not moving,
                    on the link,
\item \label{p-abff2} however, first informing the link of its position,
\end{enumerate}
\item \label{p-abff4} or
\begin{enumerate}
\item \label{p-abff5} \kw{if} if the automobile's position on the link
                    \sfsl{has not 
                    yet reached the hub},  \kw{then}
\begin{enumerate}
\item \label{p-abff7} then the automobile moves an arbitrary
                    small, positive \kw{Real}-valued \sfsl{increment}
                    along the link 
\item \label{p-abff8} informing the hub of this, 
\item \label{p-abff9} while resuming being an automobile ate the new
                    position, or 
\end{enumerate}
\pos{\psno}{\mnewfoil}
\item \label{p-abffa} \kw{else},
\begin{enumerate}
\item \label{p-abffb} while obtaining a ``next link'' from the
                    mereology of the hub (where that next link could
                    very well be the same as the link the vehicle is
                    about to leave),
\item \label{p-abffc} the vehicle informs both the link and the imminent hub
                    that it is now at that
                    hub, identified by \textsf{th\_ui},
\item \label{p-abffd} whereupon the vehicle resumes the vehicle
                    behaviour positioned at that hub;
\end{enumerate} 
\end{enumerate} 
\item \label{p-abffe} or
\item \label{p-abfff} the vehicle ``disappears --- off the radar''~!
\end{enumerate} 
\savei\end{enumerate}

\mnewfoil\LLLL
%\RSLatex
%&\ref{p-abff0}&   automobile&$_{a_{ui}}$&(a_ui,({},ruis,{}),rno)
%&\ref{p-abff0}&                      (vp:onL(fh_ui,l_ui,f,th_ui)) is
%&\ref{p-abff2}&   (ba_r_ch[thui,aui]!atH(lui,thui,nxt_lui) ;
%&\ref{p-abff3}&     automobile&$_{a_{ui}}$&(a_ui,({},ruis,{}),rno)(vp))  
%&\ref{p-abff4}&     |^|      
%&\ref{p-abff5}&   (if not_yet_at_hub(f)     
%&\ref{p-abff5}&      then  
%&\ref{p-abff7}&       (let incr = increment(f) in
%&\ref{p-habff1}&               let onl = (tl_ui,h_ui,incr,th_ui) in
%&\ref{p-abff8}&         ba-r_ch[l_ui,a_ui] ! onL(onl) ;    
%&\ref{p-abff9}&         automobile&$_{a_{ui}}$&(a_ui,({},ruis,{}),rno)   
%&\ref{p-abff9}&                          (onL(onl))  
%&\ref{p-abff5}&          end end)  
%&\ref{p-abffa}&      else 
%&\ref{p-abffb}&       (let nxt_lui:L_UI:-nxt_lui isin mereo_H(&$\wp$&(th_ui)) in
%&\ref{p-abffc}&        ba_r_ch[thui,aui]!atH(l_ui,th_ui,nxt_lui) ;
%&\ref{p-abffd}&        automobile&$_{a_{ui}}$&(a_ui,({},ruis,{}),rno)
%&\ref{p-abffd}&                         (atH(l_ui,th_ui,nxt_lui)) end)       
%&\ref{p-abff5}&     end)
%&\ref{p-abffe}&     |^|
%&\ref{p-abfff}&        stop
%&\ref{p-abff7}&   increment: Fract -> Fract   
%\endRSLatex 
\bp
\ref{p-abff0}\ \ \ automobile$_{a_{ui}}$(a\_ui,({\LBRACE}{\RBRACE},ruis,{\LBRACE}{\RBRACE}),rno)\\
\ref{p-abff0}\ \ \ \ \ \ \ \ \ \ \ \ \ \ \ \ \ \ \ \ \ \ (vp:onL(fh\_ui,l\_ui,f,th\_ui)) {\IS}\\
\ref{p-abff2}\ \ \ (ba\_r\_ch{\LBRACKET}thui,aui{\RBRACKET}!atH(lui,thui,nxt\_lui) ;\\
\ref{p-abff3}\ \ \ \ \ automobile$_{a_{ui}}$(a\_ui,({\LBRACE}{\RBRACE},ruis,{\LBRACE}{\RBRACE}),rno)(vp))\ \ \\
\ref{p-abff4}\ \ \ \ \ {\NONDETCHOICE}\ \ \ \ \ \ \\
\ref{p-abff5}\ \ \ (\kw{if} not\_yet\_at\_hub(f)\ \ \ \ \ \\
\ref{p-abff5}\ \ \ \ \ \ \kw{then}\ \ \\
\ref{p-abff7}\ \ \ \ \ \ \ (\kw{let} incr {\EQ} increment(f) \kw{in}\\
\ref{p-habff1}\ \ \ \ \ \ \ \ \ \ \ \ \ \ \ \kw{let} onl {\EQ} (tl\_ui,h\_ui,incr,th\_ui) \kw{in}\\
\ref{p-abff8}\ \ \ \ \ \ \ \ \ ba{\MINUS}r\_ch{\LBRACKET}l\_ui,a\_ui{\RBRACKET} ! onL(onl) ;\ \ \ \ \\
\ref{p-abff9}\ \ \ \ \ \ \ \ \ automobile$_{a_{ui}}$(a\_ui,({\LBRACE}{\RBRACE},ruis,{\LBRACE}{\RBRACE}),rno)\ \ \ \\
\ref{p-abff9}\ \ \ \ \ \ \ \ \ \ \ \ \ \ \ \ \ \ \ \ \ \ \ \ \ \ (onL(onl))\ \ \\
\ref{p-abff5}\ \ \ \ \ \ \ \ \ \ \kw{end} \kw{end})\ \ \\
\ref{p-abffa}\ \ \ \ \ \ \kw{else} \\
\ref{p-abffb}\ \ \ \ \ \ \ (\kw{let} nxt\_lui:L\_UI{\RDOT}nxt\_lui {\ISIN} mereo\_H($\wp$(th\_ui)) \kw{in}\\
\ref{p-abffc}\ \ \ \ \ \ \ \ ba\_r\_ch{\LBRACKET}thui,aui{\RBRACKET}!atH(l\_ui,th\_ui,nxt\_lui) ;\\
\ref{p-abffd}\ \ \ \ \ \ \ \ automobile$_{a_{ui}}$(a\_ui,({\LBRACE}{\RBRACE},ruis,{\LBRACE}{\RBRACE}),rno)\\
\ref{p-abffd}\ \ \ \ \ \ \ \ \ \ \ \ \ \ \ \ \ \ \ \ \ \ \ \ \ (atH(l\_ui,th\_ui,nxt\_lui)) \kw{end})\ \ \ \ \ \ \ \\
\ref{p-abff5}\ \ \ \ \ \kw{end})\\
\ref{p-abffe}\ \ \ \ \ {\NONDETCHOICE}\\
\ref{p-abfff}\ \ \ \ \ \ \ \ \kw{stop}\\
\ref{p-abff7}\ \ \ increment: Fract {\RIGHTARROW} Fract\ \ \ 
\ep
\emcii

\nbb{Hub Behaviour}

\begin{enumerate}\setei
\item \label{p-p-h04} The hub behaviour 
\begin{enumerate}
\item \label{p-p-h05} non-deterministically, externally  offers 
\item \label{p-p-h06} to accept timed vehicle positions ---
\item \label{p-p-h07} which will be at the hub, from some vehicle,
                    \textsf{v\_ui}.  
\item \label{p-p-h10} The timed vehicle hub position is appended to the
                    front of that vehicle's entry in the hub's traffic
                    table; 
\item \label{p-p-h15} whereupon the hub proceeds as a hub behaviour with 
                    the updated hub traffic table.
\item \label{p-p-h20} The hub behaviour offers to accept from any vehicle.
\item \label{p-p-h2x} A \kw{post} condition expresses what is really a
                    \kw{proof obligation}: that the \textsf{hub
                    traffic, ht$'$} satisfies the \kw{axiom} of the
                    endurant hub traffic attribute Item\,\psref{hub-traffic}. 
\end{enumerate}
\savei\end{enumerate}
%\RSLatex
%value
%&\ref{p-p-h04}&  hub&$_{h_{ui}}$&(h_ui,({..},(luis,vuis)),h`omega)(h`sigma,ht) is
%&\ref{p-p-h05}&      |=|
%&\ref{p-p-h06}&          { let m = ba_r_ch[h_ui,v_ui] ? in  
%&\ref{p-p-h07}&                 &\kw{assert:}& m=(_,atHub(_,h_ui,_))
%&\ref{p-p-h10}&             let ht&$'$& = ht !! [h_ui +> <.m.>^ht(h_ui)] in
%&\ref{p-p-h15}&             hub&$_{h_{ui}}$&(h_ui,({..},(luis,vuis)),(h`omega))(h`sigma,ht&$'$&)
%&\ref{p-p-h20}&           | v_ui:V_UI:-v_&ui&isin&vuis& end end }
%&\ref{p-p-h2x}&     post: all v_ui:V_UI:-v_ui isin dom ht&$'$&=>time_ordered(ht&$'$&(v_ui)) 
%\endRSLatex 
\bp
\kw{value}\\
\ref{p-p-h04}\ \ hub$_{h_{ui}}$(h\_ui,({\it },(luis,vuis)),h$\omega$)(h$\sigma$,ht) {\IS}\\
\ref{p-p-h05}\ \ \ \ \ \ {\DETCHOICE}\\
\ref{p-p-h06}\ \ \ \ \ \ \ \ \ \ {\LBRACE} \kw{let} m {\EQ} ba\_r\_ch{\LBRACKET}h\_ui,v\_ui{\RBRACKET} ? \kw{in}\ \ \\
\ref{p-p-h07}\ \ \ \ \ \ \ \ \ \ \ \ \ \ \ \ \ \kw{assert:} m{\EQ}({\UNDERLINE},atHub({\UNDERLINE},h\_ui,{\UNDERLINE}))\\
\ref{p-p-h10}\ \ \ \ \ \ \ \ \ \ \ \ \ \kw{let} ht$'$ {\EQ} ht {\DAGGER} {\LBRACKET}h\_ui {\MAPSTO} {\LANGLE}m{\RANGLE}{\CONCAT}ht(h\_ui){\RBRACKET} \kw{in}\\
\ref{p-p-h15}\ \ \ \ \ \ \ \ \ \ \ \ \ hub$_{h_{ui}}$(h\_ui,({\it },(luis,vuis)),(h$\omega$))(h$\sigma$,ht$'$)\\
\ref{p-p-h20}\ \ \ \ \ \ \ \ \ \ \ {\BAR} v\_ui:V\_UI{\RDOT}v\_ui{\ISIN}vuis \kw{end} \kw{end} {\RBRACE}\\
\ref{p-p-h2x}\ \ \ \ \ \kw{post}: {\ALL} v\_ui:V\_UI{\RDOT}v\_ui {\ISIN} \kw{dom} ht$'${\DBLRIGHTARROW}time\_ordered(ht$'$(v\_ui)) 
\ep
\emcii

\nbb{Link Behaviour}


\begin{enumerate}\setei
\item \label{p-p-l00} The link behaviour non-deterministically, externally
                    offers 
\item \label{p-p-l10} to accept timed vehicle positions ---
\item \label{p-p-l20} which will be on the link, from some vehicle,
                    \textsf{v\_ui}.  
\item \label{p-p-l30} The timed vehicle link position is appended to the
                    front of that vehicle's entry in the link's traffic
                    table; 
\item \label{p-p-l40} whereupon the link proceeds as a link behaviour with 
                    the updated link traffic table.
\item \label{p-p-l50} The link behaviour offers to accept from any vehicle.
\item \label{p-p-l51} A \kw{post} condition expresses what is really a
                    \kw{proof obligation}: that the \textsf{link
                    traffic, lt$'$} satisfies the \kw{axiom} of the
                    endurant link traffic attribute Item\,\psref{link-traffic}. 
\savei\end{enumerate}
%\RSLatex
%&\ref{p-p-l00}&   link&$_{l_{ui}}$&(l_ui,(_,(huis,vuis),_),l`omega)(l`sigma,lt) is
%&\ref{p-p-l00}&      |=| 
%&\ref{p-p-l10}&          { let m = ba_r_ch[l_ui,v_ui] ? in  
%&\ref{p-p-l20}&                  &\kw{assert:}& m=(_,onLink(_,l_ui,_,_))
%&\ref{p-p-l30}&             let lt&$'$& = lt !! [l_ui +> <.m.>^lt(l_ui)] in
%&\ref{p-p-l40}&             link&$_{l_{ui}}$&(l_ui,(huis,vuis),h`omega)(h`sigma,lt&$'$&)
%&\ref{p-p-l50}&          | v_ui:V_UI:-v_&ui&isin&vuis& end end }
%&\ref{p-p-l51}&      post: all v_ui:V_UI:-v_ui isin dom lt&$'$&=>time_ordered(lt&$'$&(v_ui))           
%\endRSLatex 
\bp
\ref{p-p-l00}\ \ \ link$_{l_{ui}}$(l\_ui,({\UNDERLINE},(huis,vuis),{\UNDERLINE}),l$\omega$)(l$\sigma$,lt) {\IS}\\
\ref{p-p-l00}\ \ \ \ \ \ {\DETCHOICE} \\
\ref{p-p-l10}\ \ \ \ \ \ \ \ \ \ {\LBRACE} \kw{let} m {\EQ} ba\_r\_ch{\LBRACKET}l\_ui,v\_ui{\RBRACKET} ? \kw{in}\ \ \\
\ref{p-p-l20}\ \ \ \ \ \ \ \ \ \ \ \ \ \ \ \ \ \ \kw{assert:} m{\EQ}({\UNDERLINE},onLink({\UNDERLINE},l\_ui,{\UNDERLINE},{\UNDERLINE}))\\
\ref{p-p-l30}\ \ \ \ \ \ \ \ \ \ \ \ \ \kw{let} lt$'$ {\EQ} lt {\DAGGER} {\LBRACKET}l\_ui {\MAPSTO} {\LANGLE}m{\RANGLE}{\CONCAT}lt(l\_ui){\RBRACKET} \kw{in}\\
\ref{p-p-l40}\ \ \ \ \ \ \ \ \ \ \ \ \ link$_{l_{ui}}$(l\_ui,(huis,vuis),h$\omega$)(h$\sigma$,lt$'$)\\
\ref{p-p-l50}\ \ \ \ \ \ \ \ \ \ {\BAR} v\_ui:V\_UI{\RDOT}v\_ui{\ISIN}vuis \kw{end} \kw{end} {\RBRACE}\\
\ref{p-p-l51}\ \ \ \ \ \ \kw{post}: {\ALL} v\_ui:V\_UI{\RDOT}v\_ui {\ISIN} \kw{dom} lt$'${\DBLRIGHTARROW}time\_ordered(lt$'$(v\_ui))\ \ \ \ \ \ \ \ \ \ \ 
\ep

\newcommand{\isminus}{-3mm}


\nbbbbb{System Initialisation}\HHHH\label{p-ch.Running Systems}

\bbbb{Initial States}
%\RSLatex
%value
%   &$hs$&:H-set is is obs_sH(obs_SH(obs_RN(&$rts$&)))   
%   &$ls$&:L-set is is obs_sL(obs_SL(obs_RN(&$rts$&)))
%   &$bcs$&:BC-set is obs_BCs(obs_SBC(obs_FV(obs_RN(&$rts$&))))   
%   &$bs$&:B-set is union{obs_Bs(bc)|bc:BC:-bc isin &$bcs$&}   
%   &$as$&:A-set is obs_BCs(obs_SBC(obs_FV(obs_RN(&$rts$&))))
%\endRSLatex
\bp
\kw{value}\\
\>\ $hs$:H\kw{-set} {\IS} {\IS} obs\_sH(obs\_SH(obs\_RN($rts$)))\ \ \ \\
\>\ $ls$:L\kw{-set} {\IS} {\IS} obs\_sL(obs\_SL(obs\_RN($rts$)))\\
\>\ $bcs$:BC\kw{-set} {\IS} obs\_BCs(obs\_SBC(obs\_FV(obs\_RN($rts$))))\ \ \ \\
\>\ $bs$:B\kw{-set} {\IS} {\UNION}{\LBRACE}obs\_Bs(bc){\BAR}bc:BC{\RDOT}bc {\ISIN} $bcs${\RBRACE}\ \ \ \\
\>\ $as$:A\kw{-set} {\IS} obs\_BCs(obs\_SBC(obs\_FV(obs\_RN($rts$))))
\ep

\nbbbb{Initialisation}\HHHH\LLLL

\begynd
\pind We are reaching the end of this domain modeling example. 
\begynd
\pind Behind us there are narratives and formalisations. 
\pind Based on these we now express 
\begynd
\pind  the signature and 
\pind the body of the definition
\afslut
\pind of a \sfsl{``system build and execute''} function.
\afslut
\afslut
\begin{enumerate}\setei
\item \label{p-ars-00y} The system to be initialised is
\begin{enumerate}
\item \label{p-ars-00x} the parallel
                      compositions ({\PARL}) of 
\item \label{p-ars-010}  the distributed parallel composition
                      ({\PARL}{\LBRACE}...{\BAR}...{\RBRACE}) 
                      of all hub behaviours,
\item \label{p-ars-020}  the distributed parallel composition
                      ({\PARL}{\LBRACE}...{\BAR}...{\RBRACE}) 
                      of all link behaviours,
\item \label{p-ars-030}  the distributed parallel composition
                      ({\PARL}{\LBRACE}...{\BAR}...{\RBRACE}) 
                      of all bus company behaviours,
\item \label{p-ars-040}  the distributed parallel composition
                      ({\PARL}{\LBRACE}...{\BAR}...{\RBRACE}) 
                      of all  bus behaviours, and
\item \label{p-ars-050}  the distributed parallel composition
                      ({\PARL}{\LBRACE}...{\BAR}...{\RBRACE}) 
                      of all automobile behaviours.
\end{enumerate} 
\savei\end{enumerate}
\mnewfoil\LLLL\pos{}{\vspace*{-18mm}}
%\RSLatex
%value
%&\ref{p-ars-00y}&   initial_system: Unit -> Unit &\syfidxi{initial\_system: \kw{Unit} {\RIGHTARROW} \kw{Unit}}{ars-00y}&
%&\ref{p-ars-00y}&   initial_system() is
%&\ref{p-ars-010}&      || { hub&$_{h_{ui}}$&(h_ui,me,h`omega)(htrf,h`sigma) 
%&\ref{p-ars-010}&         | h:H:-h isin &$hs$&, h_ui:H_UI:-h_ui=uid_H(h), me:HMetL:-me=mereo_H(h),
%&\ref{p-ars-010}&           htrf:H_Traffic:-htrf=attr_H_Traffic_H(h),
%&\ref{p-ars-010}&           h`omega:H`Omega:-h`omega=attr_H`Omega(h), h`sigma:H`Sigma:-h`sigma=attr_H`Sigma(h)/\h`sigma isin h`omega }       
%&\ref{p-ars-00x}&      || 
%&\ref{p-ars-020}&      || { link&$_{l_{ui}}$&(l_ui,me,l`omega)(ltrf,l`sigma)
%&\ref{p-ars-020}&           l:L:-l isin &$ls$&, l_ui:L_UI:-l_ui=uid_L(l), me:LMet:-me=mereo_L(l),
%&\ref{p-ars-020}&           ltrf:L_Traffic:-ltrf=attr_L_Traffic_H(l),
%&\ref{p-ars-020}&           l`omega:L`Omega:-l`omega=attr_L`Omega(l), l`sigma:L`Sigma:-l`sigma=attr_L`Sigma(l)/\l`sigma isin l`omega }   
%&\ref{p-ars-00x}&      ||  
%&\ref{p-ars-030}&      || { bus_company&$_{bc_{ui}}$&(bcui,me)(btt)
%&\ref{p-ars-030}&           bc:BC:-bc isin &$bcs$&, bc_ui:BC_UI:-bc_ui=uid_BC(bc), me:BCMet:-me=mereo_BC(bc),
%&\ref{p-ars-030}&           btt:BusTimTbl:-btt=attr_BusTimTbl(bc) }  
%&\ref{p-ars-00x}&      ||  
%&\ref{p-ars-040}&      || { bus&$_{b_{ui}}$&(b_ui,me)(ln,btt,bpos) 
%&\ref{p-ars-040}&           b:B:-b isin &$bs$&, b_ui:B_UI:-b_ui=uid_B(b), me:BMet:-me=mereo_B(b), ln:LN:pln=attr_LN(b),  
%&\ref{p-ars-040}&           btt:BusTimTbl:-btt=attr_BusTimTbl(b), bpos:BPos:-bpos=attr_BPos(b) }  
%&\ref{p-ars-00x}&   &\,&   ||  
%&\ref{p-ars-050}&   &\,\,&   || { automobile&$_{a_{ui}}$&(a_ui,me,rn)(apos) 
%&\ref{p-ars-050}&   &\,&        a:A:-a isin &$as$&, a_ui:A_UI:-a_ui=uid_A(a), me:AMet:-me=mereo_A(a), 
%&\ref{p-ars-050}&   &\,&        rn:RegNo:-rno=attr_RegNo(a), apos:APos:-apos=attr_APos(a) }  &\eox&  
%\endRSLatex 
\bp
\kw{value}\\
\ref{p-ars-00y}\ \ \ initial\_system: \kw{Unit} {\RIGHTARROW} \kw{Unit} \syfidxi{initial\_system: \kw{Unit} {\RIGHTARROW} \kw{Unit}}{ars-00y}\\
\ref{p-ars-00y}\ \ \ initial\_system() {\IS}\\
\ref{p-ars-010}\ \ \ \ \ \ {\PARL} {\LBRACE} hub$_{h_{ui}}$(h\_ui,me,h$\omega$)(htrf,h$\sigma$) \\
\ref{p-ars-010}\ \ \ \ \ \ \ \ \ {\BAR} h:H{\RDOT}h {\ISIN} $hs$, h\_ui:H\_UI{\RDOT}h\_ui{\EQ}uid\_H(h), me:HMetL{\RDOT}me{\EQ}mereo\_H(h),\\
\ref{p-ars-010}\ \ \ \ \ \ \ \ \ \ \ htrf:H\_Traffic{\RDOT}htrf{\EQ}attr\_H\_Traffic\_H(h),\\
\ref{p-ars-010}\ \ \ \ \ \ \ \ \ \ \ h$\omega$:H$\Omega${\RDOT}h$\omega${\EQ}attr\_H$\Omega$(h), h$\sigma$:H$\Sigma${\RDOT}h$\sigma${\EQ}attr\_H$\Sigma$(h){\WEDGE}h$\sigma$ {\ISIN} h$\omega$ {\RBRACE}\ \ \ \ \ \ \ \\
\ref{p-ars-00x}\ \ \ \ \ \ {\PARL} \\
\ref{p-ars-020}\ \ \ \ \ \ {\PARL} {\LBRACE} link$_{l_{ui}}$(l\_ui,me,l$\omega$)(ltrf,l$\sigma$)\\
\ref{p-ars-020}\ \ \ \ \ \ \ \ \ \ \ l:L{\RDOT}l {\ISIN} $ls$, l\_ui:L\_UI{\RDOT}l\_ui{\EQ}uid\_L(l), me:LMet{\RDOT}me{\EQ}mereo\_L(l),\\
\ref{p-ars-020}\ \ \ \ \ \ \ \ \ \ \ ltrf:L\_Traffic{\RDOT}ltrf{\EQ}attr\_L\_Traffic\_H(l),\\
\ref{p-ars-020}\ \ \ \ \ \ \ \ \ \ \ l$\omega$:L$\Omega${\RDOT}l$\omega${\EQ}attr\_L$\Omega$(l), l$\sigma$:L$\Sigma${\RDOT}l$\sigma${\EQ}attr\_L$\Sigma$(l){\WEDGE}l$\sigma$ {\ISIN} l$\omega$ {\RBRACE}\ \ \ \\
\ref{p-ars-00x}\ \ \ \ \ \ {\PARL}\ \ \\
\ref{p-ars-030}\ \ \ \ \ \ {\PARL} {\LBRACE} bus\_company$_{bc_{ui}}$(bcui,me)(btt)\\
\ref{p-ars-030}\ \ \ \ \ \ \ \ \ \ \ bc:BC{\RDOT}bc {\ISIN} $bcs$, bc\_ui:BC\_UI{\RDOT}bc\_ui{\EQ}uid\_BC(bc), me:BCMet{\RDOT}me{\EQ}mereo\_BC(bc),\\
\ref{p-ars-030}\ \ \ \ \ \ \ \ \ \ \ btt:BusTimTbl{\RDOT}btt{\EQ}attr\_BusTimTbl(bc) {\RBRACE}\ \ \\
\ref{p-ars-00x}\ \ \ \ \ \ {\PARL}\ \ \\
\ref{p-ars-040}\ \ \ \ \ \ {\PARL} {\LBRACE} bus$_{b_{ui}}$(b\_ui,me)(ln,btt,bpos) \\
\ref{p-ars-040}\ \ \ \ \ \ \ \ \ \ \ b:B{\RDOT}b {\ISIN} $bs$, b\_ui:B\_UI{\RDOT}b\_ui{\EQ}uid\_B(b), me:BMet{\RDOT}me{\EQ}mereo\_B(b), ln:LN:pln{\EQ}attr\_LN(b),\ \ \\
\ref{p-ars-040}\ \ \ \ \ \ \ \ \ \ \ btt:BusTimTbl{\RDOT}btt{\EQ}attr\_BusTimTbl(b), bpos:BPos{\RDOT}bpos{\EQ}attr\_BPos(b) {\RBRACE}\ \ \\
\ref{p-ars-00x}\ \ \ \,\ \ \ {\PARL}\ \ \\
\ref{p-ars-050}\ \ \ \,\,\ \ \ {\PARL} {\LBRACE} automobile$_{a_{ui}}$(a\_ui,me,rn)(apos) \\
\ref{p-ars-050}\ \ \ \,\ \ \ \ \ \ \ \ a:A{\RDOT}a {\ISIN} $as$, a\_ui:A\_UI{\RDOT}a\_ui{\EQ}uid\_A(a), me:AMet{\RDOT}me{\EQ}mereo\_A(a), \\
\ref{p-ars-050}\ \ \ \,\ \ \ \ \ \ \ \ rn:RegNo{\RDOT}rno{\EQ}attr\_RegNo(a), apos:APos{\RDOT}apos{\EQ}attr\_APos(a) {\RBRACE}\ \ \eox\ \ 
\ep
}\normalsize\HHHH\rm

\label{p-ch.Running Systems.n}%%  LocalWords:  ehicle bcb ijective bbc
%%  LocalWords:  mereo dom mereologies Invariance invariants cetera
%%  LocalWords:  rom VPos Bool tvpl cul de Frac frac Fract lt ln btt
%%  LocalWords:  ocity eleration Intentionality AllATH AllHTH AllLTH
%%  LocalWords:  Trf AUI allA mrg allH allL HLT Perdurants analyse vp
%%  LocalWords:  behaviour Msg chmsg dcl defind hui deterministically
%%  LocalWords:  tli tl atH onl onL rno thui aui lui nxt incr ars
%%  LocalWords:  Initialisation formalisations initialised htrf HMetL
%%  LocalWords:  ltrf LMet bcui BCMet BMet AMet vehic

\label{p-ch:Road Transport.n}