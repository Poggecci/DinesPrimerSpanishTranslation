
%%%%%%%%%%%%%%%%%%%%%%%%%%%%%%%%%%%%%%%%%%%%%%%%%%%%%%%%%%%%%%%%%%%%%%%
%%%% Dear ShaoFa - this doc. is redady for Your translation ! %%%%%%%%%
%%%% Dear ShaoFa - five typos corrected, 15 April 23 %%%%%%%%%%%%%%%%%%
%%%% Dear ShaoFa - typos corrected,      26 December 23 %%%%%%%%%%%%%%%
%%%% Dear ShaoFa - typos corrected,      04 January 24  %%%%%%%%%%%%%%%
%%%%%%%%%%%%%%%%%%%%%%%%%%%%%%%%%%%%%%%%%%%%%%%%%%%%%%%%%%%%%%%%%%%%%%%

\nbbbbbb{Introduction}\label{chap:Introduction}\label{chap:Introduction.1}
\minitoc


\begin{flushright}\label{intro:The Triptych Dogma}
\sort{The Triptych Dogma}\\[4mm]
\addcontentsline{toc}{subsection}{\bbcolor{The Triptych Dogma}}
\sf In order to \textsl{specify} \bmcolor{{$\mathcal{S}$}oftware},\\
\sf we must understand its requirements.\\[1mm]
\sf In order to \textsl{prescribe} \bmcolor{{$\mathcal{R}$}equirements},\\
\sf we must understand the \bmcolor{$\mathcal{D}$omain}.\\[1mm]
\sf So we must \bbcolor{study, analyse} and \bbcolor{describe} domains.\\
\end{flushright}\rm%
\index{pdefind}{The Triptych Dogma}\index{pdefind}{Triptych Dogma, The}\index{pdefind}{Dogma, The Triptych}
\mnewfoil

\boiteepaisseavecuntitre{$\mathcal{D},\mathcal{S}\models\mathcal{R}$}
\noindent
\begynd
\pind In proofs of correctness ($\models$)
\begynd
\pind of $\mathcal{S}$oftware
\pind with respect to $\mathcal{R}$equirements
\pind assumptions are often stated \ysfchg{with respect to } the $\mathcal{D}$omain.
\afslut
\pind So this, therefore, alone justifies our focus on domains.
\afslut
\endboiteepaisseavecuntitre

\mnewfoil

\noindent
\begynd
\pind \ysfchgii{This } \primer\ is 
\begynd
\pind both a significantly reduced version \nyl of the scientific monograph \cite{BjornerMonograph2020} 
\pind and a revision and, notably, simplification, of some of its findings.  
\afslut
\afslut

\nbbbbb{Why This Primer\,?}\label{sec:Why This Primer}

\begynd
\pind This \primer\ is intended as a \sort{textbook}.
\pind The courses that \ysfchg{we } have in mind are, in the
\sort{lectures}\ysfchg{, } \sort{to focus} on
      Chapters\,\ref{chapter:Domains}--\ref{chapter:Perdurants},
      i.e., Pages\,\pageref{chapter:Domains}--\pageref{chapter:Perdurants.n}.
\pind The \sort{serious students}, whether just readers or actual,
      physical course lecture attendants, are expected to study
      Chapters\,\ref{chap:Introduction}--\ref{chapter:Philosophy} as
      well as Chapter\,\ref{chapter:Closing} and the Bibliography
      (Chapter\,\ref{primer.bib}) and the appendices on their own\,! 
\afslut

\begynd
\pind  \ysfchgii{This } \primer\ is about how to \sfsl{analyse \& describe} man-made
      domains (including their possible interaction with nature).
      \pind We emphasize the ampersand: `\&'.\footnote{By not writing `and', but `\&', we shall emphasize that in
      ${A\&B}$ we are dealing with \sort{one} concept which consists of
      both $A$ and $B$ ``tightly interacting''.}
\pind We justify competency in \sfsl{Domain Science \& Engineering}
      for two reasons.
\begynd
\pind (i) For reasons of proper \sfsl{engineering} software
      development --  as indicated by the above \sort{Triptych Dogma}. In 
      possible proofs of software properties references are made, not
      only to the software code itself and the requirements, but also
      to the domain, the latter in the form of \sfsl{assumptions about
      the domain}. In our mind no software development project ought
      be undertaken unless it more-or-less starts with a proper domain
      engineering phase. And
\pind (ii) for reasons of \sfsl{scientifically} understanding our own
      everyday practical world: financial institutions, the transport
      industry (road, rail and air traffic, shipping), feeder systems
      (such as oil, gas, water and other such pipeline systems), etc. 
\afslut
\afslut

\nbbbbb{Structure}\label{sec:Structure}

\begynd
\pind The \primer, beyond the present chapter, has, syntactically
speaking, three elements:
\afslut

\begin{enumerate}
\item \sort{Chapter\,\ref{chap2.tex.Philosophy}} covers the \sfsl{philosophy} of
      \texttt{Kai S{\o}rlander}
      \cite{kaisorlander1994,kaisorlander1997,kaisorlander2002,kaisorlander2016,kaisorlander2022}.

      Yes, a major contribution of \cite{BjornerMonograph2020} and
      this \primer\ is to justify important domain concepts by
      their sheer inevitability in any world description.
      
\item \sort{Chapters\,\ref{chapter:Domains}--\ref{chapter:Perdurants}} presents \sfsl{the methodology of
    domain engineering.} It is split into four chapters for practical
    and pragmatic
    reasons. Chapter\,\ref{chapter:Domains} gives a ``capsule
    introduction'' into Chapters\,\,\ref{primer-extq.1}--\ref{chapter:Perdurants}.
  
\item \sort{Chapters\,\ref{chapter:Closing}--\ref{primer.bib}}
  and \sort{Appendices\,\ref{Chapter:Road Transport}--\ref{primer.indexes}} cover
  such things as `closing remarks' (\sort{\ref{chapter:Closing}}), a `bibliography' (\sort{\ref{primer.bib}}), a `Road
  Transport' example (\sort{\ref{Chapter:Road Transport}}), a `Pipeline System' example (\sort{\ref{appendix:Pipelines}}), an `\texttt{RSL}
  formal specification language' primer (\sort{\ref{RSL-intro}}), and `Indexes' to definitions, concepts, etc. (\sort{\ref{primer.indexes}}).
\end{enumerate}

\nbbbbb{Prerequisite Skills}\label{Prerequisite Skills}

\begynd
\pind The \pos{reader}{course student} is expected to possess the
      following skills:
\afslut
\begin{itemize}
\item To be reasonably versed in \sort{discrete mathematics:} mathematical logic
  and set theory.
\item To have had, even if only a  fleeting, acquaintance with abstract
  specifications in the style of
  \texttt{VDM} \citevdm,
  \texttt{Z} \citez,
  \texttt{CafeObj} \citecafeobj, 
  \texttt{Maude} \cite{maude-primer,maude-manual}, or the like -- and thus
  to enjoy abstractions\footnote{Some say: \sfsl{``Mathematics is the
  Science of Abstractions''}\,! Others say that both \sfsl{``Mathematics and
  Physics are Abstractions of Reality''}.}.
\item To have reasonable experience with \sort{functional programming} a
      la \texttt{Standard ML} or \textsf{F}
      \cite{MilnerTofte,Harper,MRHansen+HRischel} respectively
      \cite{Hansen+Rischel} -- or similar such language.
    \item To have  reasonable experience with \sort{\texttt{CSP}}
      \cite{Hoa78,Hoare85,Hoare85+2004,Roscoe97,Schneider99}.
\end{itemize}
\noindent
\begynd
\pind The \pos{reader}{course student} is further expected to possess
      the following mindset:
\afslut
\begin{itemize}
\item To basically consider software as\ \sort{mathematical
      objects}. That is: as quantities about which one can (and must) reason logically.
  \item To \sort{think and ``act'' abstractly}. An essence of
  abstraction is expressed in the next section.

\item To \sort{act responsibly}\footnote{It is, today, \todaytime, 
    very fashionable to propagate messages of `ethics' to programmers
    -- without even touching upon issues such as \sfsl{``have You
    understood Your application domain thoroughly\,?''}, or
    \sfsl{``have You reasoned about adequacy of your requirements\,?''},
    or \sfsl{``have You model-checked, proved and formally tested your
    specifications (descriptions and prescriptions) and Your
    code\,?''}, etc.}, that is to make sure that You have 
    indeed understood Your domain, that You  have indeed reasoned
    about adequacy of your requirements, and  You have indeed
    model-checked, proved and formally tested Your  
    specifications.
\end{itemize}

\nbbbbb{Abstraction}\label{primer:Abstraction}

%\input{abstraction}

\QUOTATION{Conception, my boy, fundamental brain-work, \\ 
           is what makes the difference in all art\ysf{.}}%
          {D.G.\ Rossetti\footnote{Dante Gabrielli Rosetti,
          1828--1882, English poet, illustrator, painter and
          translator.}: letter to T.\ H.\ Hall Caine\footnote{T.\ H.\ Hall Caine,
          1853--1931, British novelist, dramatist, short story
          writer, poet and critic.}}  

\pos{\psno}{\mnewfoil}

\noindent
\pos{}{Quoting \sort{C.A.R.\,Hoare:}}

\noindent
\begynd
\pind {\sl Abstraction is a tool,
\begynd
\pind used by the human mind, 
\pind and to be applied in the process of describing \nyl 
          (understanding) complex phenomena.
\afslut
}

\pind {\sl Abstraction is the most powerful such tool  \nyl
available to the human intellect.}

\pind {\sl Science proceeds by simplifying reality. }
\begynd
\pind {\sl The first step in simplification is abstraction.}
\pind {\sl Abstraction (in the context of science) \nyl  
          means leaving out of account all those empirical \nyl
          data which do not fit the particular, conceptual 
          framework \nyl within which science at the moment
          happens to be working.}
\afslut
\afslut

\pos{\psno}{\mnewfoil}
\begynd
\pind {\sl Abstraction (in the process of specification)}
\begynd
\pind {\sl arises from a conscious
          decision
\pind  to advocate certain 
          desired objects, situations and processes
\pind  as being fundamental; 
\pind by exposing, 
          in a first, or higher, level of
          description, their similarities 
\pind  and \ysf{--} at that level -- ignoring
          possible differences.}
\afslut
\afslut

  \ \hfill     \pos{ [From the opening paragraphs of}{} 
           \cite[C.A.R.\ Hoare\ysf{,} \textsl{Notes on Data
           Structuring}]{Hoa72a}\pos{\ysf{.}]}{}
       %}{}

\nbbbbb{Software Engineering}\label{sec:Software Engineering}
\bbbb{Domain Science \& Engineering}

\begynd
\pind This \primer\ covers only the \sfsl{application domain} of software development.
\pind There are two things to say about that.
\begynd
\pind One is that facets of \sfsl{requirements},
\begynd
\pind essential ones, is covered in \cite[Chapter 8]{BjornerMonograph2020},
\pind general ones in \cite[Software Engineering, III, Part V]{TheSEBook3};
\afslut
\pind the other is that the pursuit of developing domain models \nyl
      is not just for the sake of software development, \nyl
      but also for the sake of just understanding the man-made world
      around us.
\afslut
\pind Domain science and engineering can thus be pursued in-and-by itself.
\afslut

\nbbbb{Software Engineering}

\begynd
\pind In 2006 \ysfchg{these books were } published: \cite{TheSEBook1,TheSEBook2,TheSEBook3}:
\afslut

\begin{center}
\epsfig{file=SE1.eps,width=4cm}\ \
\epsfig{file=SE2.eps,width=4cm}\ \
\epsfig{file=SE3.eps,width=4cm}
\end{center}

\nbbb{Domain Engineering: 2016--2022}\label{Domain Engineering 2016 2022}

\begynd
\pind The first inklings of the domain science and engineering of
      \cite{BjornerMonograph2020} appeared in \cite[2010]{Kiev:2010ptI,Kiev:2010ptII}.
\pind More-or-less ``final'' ideas were published, first in \cite[2017]{BjornerFAoC2015MDAAD},
      then in \cite[March 2019]{BjornerTOSEM2018}.
\pind The book \cite{BjornerMonograph2020} with updates in this
      \primer, then constitutes the most recent status of our work in domain
      science \& engineering. 

\pind \cite[Software Engineering, III, Part V]{TheSEBook3} does not
      cover the \sfsl{Domain Engineering} material covered in
      \cite[Chapter 8: Domain Facets]{BjornerMonograph2020}.
\begynd
\pind That latter was researched \cite{dines:facs:2008} and developed between the appearance
      of \cite{TheSEBook3} and, obviously, \cite{BjornerMonograph2020}.
\afslut
\afslut

\begynd
\pind Part V of \cite{TheSEBook3}, except for Chapters\,17--18 is
      still relevant.
\pind Chapters\,17--18 of \cite{TheSEBook3} are now to be replaced in
      any study by Chapters\,4--7 of \cite{BjornerMonograph2020} \sort{or}
      this \primer\,!
\afslut

\nbbb{Requirements Engineering}

\begynd
\pind This \primer\ does not show You how to proceed into software
      development according to the \sort{Triptych Dogma}.
\begynd
\pind This is strongly hinted at in \cite[Chapter 9]{BjornerMonograph2020}.
\pind (That chapter is an adaptation of \cite[May 2008]{dines:ugo65:2008}.)
\pind Our approach to \sfsl{requirements engineering} is rather
      different from that of both \cite[A. van Laamswerde]{Lamsweerde}
      and \cite[M.\ A.\ Jackson]{Jackson2010Facs} -- to cite 
      two relevant works.
\pind It is, \ysfchg{we } strongly think, commensurate with these works.
\pind  \ysfchg{We } wish that someone could take up this line of research:
\begynd
\pind making more precise, perhaps more formal, the ideas of
      \sfsl{projection, intialisation, determination, extension} and \sfsl{fitting};
\pind and comparing, perhaps unifying our approach with that of
      Lamsweerde and Jackson.
\afslut
\afslut
\afslut

\nbbb{Software Design}

\begynd
\pind For the software design phase, after requirements engineering,
\pind we, of course, recommend \cite[\sfsl{Software Engineering}
      \ysfchgv{V}ols.\,1--2]{TheSEBook1,TheSEBook2} 
\afslut

\nbbbbb{The Structuring of The Text}\label{The Structuring of The Text}

\begynd
\pind The \pos{reader}{student} will find that this text consists of
      ``diverse'' kinds of usually small paragraphs of texts:
\begynd
\pind \bbcolor{definition}s -- properly numbered and labeled;
\pind \bbcolor{example}s -- properly numbered and labeled;
\pind \bbcolor{analysis predicate, function,} and
      \bbcolor{description prompt} ``formalisations'';
\pind \bbcolor{method} principle, procedure, technique and tool paragraphs;
\pind -- all of these delineated by closing {\dbsquare}s;
\pind -- with short, usually one or two small paragraphs of
      introductory or otherwise explaining texts.
\afslut
\pind All of \ysfchg{these are } \textsf{``brought to You in living
  colours''\,!}\footnote{\LLLL -- as the \textsf{NBC} Television
  Network programmmes would ``proudly'' announce in \ysf{t}he 1960s\,!}
\pind So be prepared:
\begynd
\pind Study such paragraphs: paragraph-by-paragraph.
\pind Each form\ysfchg{s } a separate ``whole''.
\afslut
\afslut


\nbbbbb{Self-Study}\label{sec:Used for Self-Study}

\begynd
\pind This \primer\ is primarily intended to support actual, physical lectures.
\pind For self-study by B.Sc.\ and M.Sc.\ students and practicing
      novice software engineers  we recommend to use this \primer\ in
      connection with its ``origin'' \cite{BjornerMonograph2020}.
\pind For  self-study by Ph.D.\ students and graduated computer
      scientist we recommend going directly to the source:
      \cite{BjornerMonograph2020}.   
\afslut

\nbbbbb{Two Examples}\label{primer:Two Examples}

\begynd
\pind There are around 80 examples, scattered all over the first 120 pages.
\pind In addition we bring two larger examples:
\afslut
\begin{itemize}
\item \textsf{Road Transport}, Appendix\,\ref{Chapter:Road
        Transport}, pages\,\pageref{Chapter:Road Transport}--\pageref{p-ch:Road Transport.n}, 
\item \textsf{Pipelines},
      Appendix\,\ref{appendix:Pipelines}, pages\,\pageref{appendix:Pipelines}--\pageref{appendix:Pipelines.n}. 
\end{itemize}

\dbeat{%%%%%%%%%%%%%%%%%%%%%%%%%%%%%%%%%%%%%%%%%%%%%%%%%%%%%%%%%%%%%%%%%%%%%%%%%%%%%%%%%%%%%%%%%%%%%%
\nbbbbb{Lectures}\label{sec:Use in Lectures}

\begynd
\pind This \primer\ was developed during the summer of 2022.
\pind The author was to give a set of seven double lectures \nyl at the Technical
      University of Vienna\footnote{The author ``woke up'' to what
        software engineering and computer science is really about in
        Vienna, in 1973--1975 -- at the \sort{IBM Vienna Labor}, with
        colleagues like \sfsl{Hans Beki{\v{c}}, Cliff Jones, Peter Lucas} and \sfsl{Kurt Walk}. It
        had to be Vienna, the city of the \sfsl{Wiener Kreis}
        \texttt{en.\-wiki\-pedia.\-org/\-wi\-ki/\-Vienna\_\-Circle}, \sfsl{Ludwig
          Wittgenstein}
        \texttt{en.\-wiki\-pedia.\-org/\-wi\-ki/\-Lud\-wig\_\-Witt\-gen\-stein}
        and \sfsl{Sir Karl
          Popper}
        \texttt{en.\-wiki\-pedia.\-org/\-wi\-ki/\-Karl\_\-Popper}.}
      October
        24 -- November 4, 2022.
      \pind These lectures were: 
\begin{itemize}
\item Week 1:
\begin{description}
\item[Lecture Day 1:] Chapter\,\ref{chapter:Domains}:
\begin{itemize}
\item \sort{Domains} -- lecture ``doubles'' as a Faculty Seminar\,!
\end{itemize} and Appendix\,A: 
\begin{itemize}
\item \sort{An Example, I}
\end{itemize}
\item[Lecture Day 2:] Chapter\,\,\ref{primer-extq.1}:
\begin{itemize}
\item \sort{Endurants: External Qualities}
\end{itemize}
  
\item[Lecture Day 3:] Sections\,\,\ref{chap4.Internal
    Qualities}--\ref{chap4.Mereology}:
\begin{itemize}
\item \sort{Unique Identifiers}
\item \sort{Mereology}
\end{itemize}

\item[Lecture Day 4:] Sections\,\,\ref{chap4.Attributes}--\ref{A
    Domain Discovery Process, II}:
\begin{itemize}
\item \sort{Attributes},
\item \sort{Intentional Pull},
\item \sort{A Domain Discovery Process}
\end{itemize}
\end{description}
\item Week 2:
\begin{description}
\item[Lecture Day 5:]  Appendix\,A: 
\begin{itemize}
\item \sort{An Example, II}  Further on \sort{Road Transport}, and
\end{itemize}
 Chapter\,\,\ref{chapter:Perdurants}, Sect.\,\ref{A General Analysis of Part  Behaviours}:
\begin{itemize}
\item \sort{A General Analysis of Part Behaviours}
\end{itemize}
\item[Lecture Day 6:] Chapter\,\,\ref{chapter:Perdurants},
  Sects.\,\ref{Domain Channel Description}--\ref{Behaviour Definition Bodies}: 
\begin{itemize}
\item \sort{Channels} and
\item \sort{Behaviours Definitions: Signatures and Bodies}
\end{itemize}
\item[Lecture Day 7:] Chapter\,\,\ref{chapter:Perdurants},
  Sects.\,\ref{Domain Behaviour Initialisation}--\ref{A Domain Discovery Process, III}:
\begin{itemize}
\item \sort{Domain Behaviour Initialisation}, 
\item \sort{A Domain Discovery Process},
\end{itemize}
\end{description}
\end{itemize}
\afslut
}%%%%%%%%%%%%%%%%%%%%%%%%%%%%%%%%%%%%%%%%%%%%%%%%%%%%%%%%%%%%%%%%%%%%%%%%%%%%%%%

\nbbbbb{Relation to \cite{BjornerMonograph2020}}

\begynd
\pind This \primer\ is based on \cite[Nov.\,2021]{BjornerMonograph2020}.
\pind Chapter\,\ref{chapter:Philosophy} is a complete rewrite of
      \cite[Chapter\,2]{BjornerMonograph2020}.  
\pind Chapters\,\ref{primer-extq.1}--\ref{chapter:Perdurants} is a
      ``condensation'' of \cite[Chapters\,4--7]{BjornerMonograph2020}:
\begynd
\pind \cite[Chapter\,6]{BjornerMonograph2020} has been shortened and
      appears in this \primer\ as Sect.\,\ref{Transcendental Deductions}.
\pind From \cite[Chapter\,4]{BjornerMonograph2020} we have, in
      Chapter\,\ref{primer-extq.1}, omitted all 
      material on -- what is there referred to as \sfsl{Conjoins}.
\pind And we have further sharpened the notion of \sfsl{type names}.
\afslut
\pind We have sharpened the focus on methods: principle, procedures,
      techniques and tools.
\begynd
\pind You will find, in the \sfsl{Indexes} section,
      Sect.\,\vref{label.dadmethod}, a summary of references to these.
\pind Work is still in progress on highlighting more of the method steps.
\afslut
\pind Section\,\ref{Discrete Dynamic Domains} is new. 
\afslut

\nbbbbb{The \texttt{RAISE} Specification Language, \texttt{RSL}, and \rslplus}\label{RSL-I}

\begynd
\pind The formal notation (to go with the informal text) of this
      \primer\ is that of \texttt{RSL} \cite{RSL}, the \sort{R}AISE
      \sort{S}pecification \sort{L}anguage, where \texttt{RAISE}
      stand\ysfchg{s }
      for \sort{R}igorous \sort{A}pproach to \sort{I}ndustrial
      \sort{S}oftware \sort{E}ngineering \cite{RaiseMethod}. 
\pind Other formal notations could be used instead.
\pind Replacement examples could be \texttt{VDM} \citevdm , \texttt{Z}
      \citez, or \texttt{Alloy} \citealloy.
\pind We are more using the \texttt{RAISE} specification language,
      \texttt{RSL} than using the method.
\pind And we are using it in two ways:
\begin{itemize}
\item Informally, to present and explain the domain analysis \&
      description methods of this \primer, and
\item formally, to present domain descriptions. 
\end{itemize}
\pind The informal \texttt{RSL} is an extended version,
\rslplus.\footnote{See Appendix Sect.\,\vref{chap2.tex.rsltext}.}
\pind The two ways are otherwise not related.
\pind One could use another specification language \ysfchg{either } for the
      informal or \ysfchg{for } the formal aspects.
\afslut

\nbbbbb{Closing}

\begynd
\pind The purpose of this introduction is to place the present \primer\
\pind in the context of \ysfchgv{Dines Bj{\o}rner's } other books
      \cite{TheSEBook1,TheSEBook2,TheSEBook3} on software development 
\pind and possible lectures and self\ysf{-}study.
\afslut

\label{chap:Introduction.n}

%%  LocalWords:  analyse defind VDM CafeObj CSP Kai rlander RSL Ph wi
%%  LocalWords:  summarises Laamswerde intialisation Lamsweerde Beki
%%  LocalWords:  Kreis pedia ki Lud Endurants Mereology Behaviours
%%  LocalWords:  Behaviour Initialisation formalisations colours AISE
%%  LocalWords:  programmmes dadmethod pecification anguage igorous 
%%  LocalWords:  pproach ndustrial oftware ngineering Gabrielli Caine
%%  LocalWords:  Rosetti Hoare equirements omain pdefind wrt
