\nbbbbbb{Introducción}\label{chap:Introduction}\label{chap:Introduction.1}
\minitoc

\begin{flushright}\label{intro:The Triptych Dogma}
\sort{El Dogma del Tríptico}\\[4mm]
\addcontentsline{toc}{subsection}{\bbcolor{El Dogma del Tríptico}}
\sf Para \textsl{especificar} \bmcolor{{$\mathcal{S}$}oftware},\\
\sf debemos entender sus requisitos.\\[1mm]
\sf Para \textsl{prescribir} \bmcolor{{$\mathcal{R}$}equisitos},\\
\sf debemos entender el \bmcolor{$\mathcal{D}$ominio}.\\[1mm]
\sf Así que debemos \bbcolor{estudiar, analizar} y \bbcolor{describir} dominios.\\
\end{flushright}\rm%
\index{pdefind}{El Dogma del Tríptico}\index{pdefind}{Dogma del Tríptico, El}\index{pdefind}{Dogma, El Tríptico}
\mnewfoil

\boiteepaisseavecuntitre{$\mathcal{D},\mathcal{S}\models\mathcal{R}$}
\noindent
\begynd
\pind En pruebas de exactitud ($\models$)
\begynd
\pind de $\mathcal{S}$oftware
\pind con respecto a $\mathcal{R}$equisitos,
\pind las suposiciones a menudo se establecen \ysfchg{con respecto al} $\mathcal{D}$ominio.
\afslut
\pind Esto por sí solo justifica nuestro enfoque en los dominios.
\afslut
\endboiteepaisseavecuntitre

\mnewfoil

\noindent
\begynd
\pind \ysfchgii{Este} \primer\ es
\begynd
\pind tanto una versión significativamente reducida \nyl de la monografía científica \cite{BjornerMonograph2020} 
\pind como una revisión y, notablemente, simplificación, de algunos de sus hallazgos.
\afslut
\afslut

\nbbbbb{¿Porqué este manual?}\label{sec:Why This Primer}

\begynd
\pind Este \primer\ está pensado como un \sort{libro de texto}.
\pind Los cursos que \ysfchg{tenemos} en mente, durante sus \sort{conferencias}, se deben \sort{enfocar} en los
      capítulos\,\ref{chapter:Domains}--\ref{chapter:Perdurants},
      es decir, páginas\,\pageref{chapter:Domains}--\pageref{chapter:Perdurants.n}.
\pind ¡Se espera que los \sort{estudiantes aplicados}, ya sean solo lectores o matriculados del curso, estudien
      los capítulos\,\ref{chap:Introduction}--\ref{chapter:Philosophy} así
      como el capítulo\,\ref{chapter:Closing}, la bibliografía
      (capítulo\,\ref{primer.bib}) y los apéndices por su propia cuenta\,!
\afslut

\begynd
\pind  \ysfchgii{Este} \primer\ trata sobre cómo \sfsl{analizar \& describir} dominios artificiales
      (incluyendo su posible interacción con la naturaleza).
      \pind Enfatizamos el ampersand: `\&'.\footnote{Al no escribir `y', sino `\&', enfatizamos que en
      ${A\&B}$ estamos tratando con un \sort{único} concepto que consiste en
      tanto $A$ como $B$ ``interactuando cercanamente''.}
\pind Justificamos la competencia en \sfsl{Ciencia e Ingeniería de Dominios}
      por dos razones.
\begynd
\pind (i) Por motivos de un \sfsl{desarrollo de software siguiendo buenas prácticas de ingeniería} -- como se indica por el \sort{Dogma del Tríptico} mencionado anteriormente. "En posibles pruebas de propiedades de software, se hacen referencias, no solamente al software en sí y a los requisitos sino también al dominio en forma de \sfsl{suposiciones sobre este mismo}.
      En nuestra opinión, ningún proyecto de desarrollo de software debería
      emprenderse a menos que comience más o menos con una etapa adecuada de ingeniería
      de dominio. Y
      \pind (ii) por motivos de comprender \sfsl{científicamente} nuestro propio
      mundo práctico cotidiano: instituciones financieras, la industria del transporte
      (tráfico por carretera, ferroviario y aéreo, envío), sistemas de alimentación
      (como sistemas de tuberías de petróleo, gas, agua y otros), etc.
\afslut
\afslut


\nbbbbb{Estructura}\label{sec:Structure}

\begynd
\pind El \primer, más allá del presente capítulo, tiene, sintácticamente
hablando, tres elementos:
\afslut

\begin{enumerate}
\item \sort{El capítulo\,\ref{chap2.tex.Philosophy}} cubre la \sfsl{filosofía} de
      \texttt{Kai S{\o}rlander}
      \cite{kaisorlander1994,kaisorlander1997,kaisorlander2002,kaisorlander2016,kaisorlander2022}.

      Sí, una contribución importante de \cite{BjornerMonograph2020} y
      este \primer\ es justificar conceptos importantes del dominio por
      su pura inevitabilidad en cualquier descripción del mundo.

\item \sort{Los capítulos\,\ref{chapter:Domains}--\ref{chapter:Perdurants}} presentan \sfsl{la metodología de
    la ingeniería de dominios.} Esta metodología se divide en cuatro capítulos por razones prácticas
    y pragmáticas. El capítulo\,\ref{chapter:Domains} ofrece una ``introducción
    independiente'' a los capítulos\,\,\ref{primer-extq.1}--\ref{chapter:Perdurants}.
  
\item \sort{Los capítulos\,\ref{chapter:Closing}--\ref{primer.bib}}
  y \sort{los apéndices\,\ref{Chapter:Road Transport}--\ref{primer.indexes}} cubren
  temas como `comentarios finales' (\sort{\ref{chapter:Closing}}), una `bibliografía' (\sort{\ref{primer.bib}}), un ejemplo de `Transporte
  por Carretera' (\sort{\ref{Chapter:Road Transport}}), un ejemplo de `Sistema de Tuberías' (\sort{\ref{appendix:Pipelines}}), un manual para `\texttt{RSL}' (\sort{\ref{RSL-intro}}) un lenguaje formal de especificación  e `Índices' a definiciones, conceptos, etc. (\sort{\ref{primer.indexes}}).
\end{enumerate}

\nbbbbb{Destrezas Prerrequisitas}\label{Prerequisite Skills}

\begynd
\pind Se espera que el \pos{lector}{estudiante del curso} posea las
      siguientes destrezas:
\afslut
\begin{itemize}
\item Tener un conocimiento razonable en \sort{matemáticas discretas:} lógica matemática
  y teoría de conjuntos.
\item Haber tenido, aunque sea solo un conocimiento fugaz, de especificaciones abstractas al estilo de
  \texttt{VDM} \citevdm,
  \texttt{Z} \citez,
  \texttt{CafeObj} \citecafeobj, 
  \texttt{Maude} \cite{maude-primer,maude-manual}, o similares -- y así
  disfrutar de las abstracciones\footnote{Algunos dicen: \sfsl{``Las matemáticas son la
  ciencia de las abstracciones''}\,! Otros dicen que tanto ¡\sfsl{``Las matemáticas como
  la física son abstracciones de la realidad''}.}.
\item Tener una experiencia razonable con \sort{programación funcional} como \texttt{Standard ML} o \textsf{F}
      \cite{MilnerTofte,Harper,MRHansen+HRischel} respectivamente
      \cite{Hansen+Rischel} -- o lenguajes similares.
    \item Tener experiencia razonable con \sort{\texttt{CSP}}
      \cite{Hoa78,Hoare85,Hoare85+2004,Roscoe97,Schneider99}.
\end{itemize}
\noindent
\begynd
\pind Se espera además que el \pos{lector}{estudiante del curso} posea
      la siguiente mentalidad:
\afslut
\begin{itemize}
\item Considerar básicamente software como \sort{objetos
      matemáticos}. Es decir, como cantidades sobre las cuales se puede (y se debe) razonar lógicamente.
  \item \sort{Pensar y ``actuar'' abstractamente}. La esencia de
  la abstracción se expresa en la siguiente sección.

\item \sort{Actuar responsablemente}\footnote{Hoy en día, \todaytime, 
    está muy de moda propagar mensajes de `ética' a los programadores
    -- sin siquiera tocar temas como \sfsl{``¿Has comprendido
    completamente tu dominio de aplicación\,?''}, o
    \sfsl{``¿Has razonado sobre la adecuación de tus requisitos\,?''},
    o \sfsl{``¿Has verificado mediante modelos, demostrado y probado
    formalmente tus especificaciones (descripciones y prescripciones) y tu
    código\,?''}, etc.}, es decir, asegurarse de que 
    realmente has comprendido tu dominio, que realmente has razonado
    sobre la adecuación de tus requisitos, y que realmente has
    verificado mediante modelos, demostrado y probado formalmente tus
    especificaciones.
\end{itemize}
\nbbbbb{Abstracción}\label{primer:Abstraction}


\QUOTATION{La concepción, muchacho, el trabajo cerebral fundamental, \\
           es lo que marca la diferencia en toda arte\ysf{.}}
          {D.G.\ Rossetti\footnote{Dante Gabrielli Rosetti,
          1828--1882, poeta, ilustrador, pintor y traductor inglés.}: carta a T.\ H.\ Hall Caine\footnote{T.\ H.\ Hall Caine,
          1853--1931, novelista, dramaturgo, cuentista, poeta y crítico británico.}}  

\pos{\psno}{\mnewfoil}

\noindent
\pos{}{Citando a \sort{C.A.R.\,Hoare:}}

\noindent
\begynd
\pind {\sl La abstracción es una herramienta utilizada por la mente humana
\begynd
\pind y debe aplicarse en el proceso de describir \nyl 
          (entender) fenómenos complejos.
\afslut
}

\pind {\sl La abstracción es la herramienta más poderosa \nyl
disponible al intelecto humano.}

\pind {\sl La ciencia avanza simplificando la realidad. }
\begynd
\pind {\sl El primer paso de la simplificación es la abstracción.}
\pind {\sl La abstracción (en el contexto de la ciencia) \nyl  
          significa dejar al lado todos aquellos datos empíricos \nyl
          que no encajan en el marco conceptual particular \nyl 
          en el que la ciencia está trabajando en ese momento.}
\afslut
\afslut

\pos{\psno}{\mnewfoil}
\begynd
\pind {\sl La abstracción (en el proceso de especificación)}
\begynd
\pind {\sl surge de una decisión consciente
\pind  de promover ciertos 
          objetos, situaciones y procesos deseados
\pind  como fundamentales;
\pind exponiendo, 
          en un primer nivel, o nivel superior de
          descripción, sus similitudes 
\pind  e \ysf{--} en ese nivel -- ignorando
          posibles diferencias.}
\afslut
\afslut

  \ \hfill     \pos{ [De los párrafos iniciales de}{} 
           \cite[C.A.R.\ Hoare\ysf{,} \textsl{Notas sobre la Estructuración de Datos}]{Hoa72a}\pos{\ysf{.}]}{}

\nbbbbb{Ingeniería de Software}\label{sec:Software Engineering}
\bbbb{Ciencia \& Ingeniería de Dominios}
       
\begynd
\pind Este \primer\ cubre solo el \sfsl{dominio de aplicación} del desarrollo de software.
\pind Hay dos cosas que decir al respecto.
\begynd
\pind Una es que los aspectos esenciales de \sfsl{requisitos},
\begynd
\pind se cubren en \cite[Capítulo 8]{BjornerMonograph2020} y
\pind los generales en \cite[Ingeniería de Software, III, Parte V]{TheSEBook3};
\afslut
\pind la otra es que la búsqueda de desarrollar modelos de dominio \nyl
        no es solo por el bien del desarrollo de software, \nyl
        sino también para entender el mundo creado por la humanidad
        que nos rodea.
\afslut
\pind La ciencia e ingeniería de dominios pueden, por lo tanto, ser perseguidas por sí mismas.
\afslut

\nbbbb{Ingeniería de Software}

\begynd
\pind En el 2006, \ysfchg{estos libros fueron} publicados: \cite{TheSEBook1,TheSEBook2,TheSEBook3}:
\afslut

\begin{center}
\epsfig{file=SE1.eps,width=4cm}\ \
\epsfig{file=SE2.eps,width=4cm}\ \
\epsfig{file=SE3.eps,width=4cm}
\end{center}

\nbbb{Ingeniería de Dominios: 2016--2022}\label{Domain Engineering 2016 2022}
\begynd
\pind Las primeras nociones de la ciencia e ingeniería de dominios de
      \cite{BjornerMonograph2020} aparecieron en \cite[2010]{Kiev:2010ptI,Kiev:2010ptII}.
\pind Las ideas más o menos ``finales'' se publicaron primero en \cite[2017]{BjornerFAoC2015MDAAD} y 
      luego en \cite[marzo de 2019]{BjornerTOSEM2018}.
\pind El libro \cite{BjornerMonograph2020}, junto a actualizaciones en este
      manual, constituye entonces el estado más reciente de nuestro trabajo en ciencia
      \& ingeniería de dominios.

\pind \cite[Ingeniería de Software, III, parte V]{TheSEBook3} no
      cubre el material de \sfsl{Ingeniería de Dominios} discutido en el
      \cite[capítulo 8: Facetas del Dominio]{BjornerMonograph2020}.
\begynd
\pind Este último fue investigado \cite{dines:facs:2008} y desarrollado entre la publicación
      de \cite{TheSEBook3} y, obviamente, \cite{BjornerMonograph2020}.
\afslut
\afslut

\begynd
\pind La parte V de \cite{TheSEBook3}, excepto por los capítulos\,17--18 sigue siendo
      relevante.
\pind ¡Los capítulos\,17--18 de \cite{TheSEBook3} ahora deben ser reemplazados en
      cualquier estudio por los capítulos\,4--7 de \cite{BjornerMonograph2020} \sort{o}
      este manual\,!
\afslut

\nbbb{Ingeniería de Requisitos}

\begynd
\pind Este manual no muestra cómo proceder al desarrollo de software según el \sort{Dogma del Tríptico}.
\begynd
\pind Esto se insinúa fuertemente en \cite[capítulo 9]{BjornerMonograph2020}.
\pind (Ese capítulo es una adaptación de \cite[mayo de 2008]{dines:ugo65:2008}.)
\pind Nuestro enfoque en la \sfsl{ingeniería de requisitos} es bastante
      diferente del de \cite[A. van Laamswerde]{Lamsweerde} y
      \cite[M.\ A.\ Jackson]{Jackson2010Facs}, por citar dos obras
      relevantes.
\pind Creemos \ysfchg{firmemente} que es congruente con estos trabajos.
\pind Deseamos \ysfchg{firmemente} que alguien pueda retomar esta línea de investigación:
\begynd
\pind haciendo más precisas, quizás más formales, las ideas de
      \sfsl{proyección, inicialización, determinación, extensión} y \sfsl{ajuste};
\pind y comparando, quizás unificando nuestro enfoque con el de
      Lamsweerde y Jackson.
\afslut
\afslut
\afslut
\nbbb{Diseño de Software}

\begynd
\pind Para la fase de diseño de software, después de la ingeniería de requisitos,
\pind recomendamos, por supuesto, \cite[\sfsl{Ingeniería de Software}
      \ysfchgv{v}ols.\,1--2]{TheSEBook1,TheSEBook2} 
\afslut

\nbbbbb{La Estructuración del Texto}\label{The Structuring of The Text}

\begynd
\pind El \pos{lector}{estudiante} encontrará que este texto consiste en
      tipos ``diversos'' de párrafos de texto, usualmente pequeños:
\begynd
\pind \bbcolor{definiciones} -- debidamente numeradas y etiquetadas;
\pind \bbcolor{ejemplos} -- debidamente numerados y etiquetados;
\pind ``formalizaciones'' para \bbcolor{análisis de predicado}, \bbcolor{funciones} e 
      \bbcolor{indicaciones de descripción} ';
\pind párrafos sobre \bbcolor{principios de métodos}, procedimientos, técnicas y herramientas;
\pind -- todos estos delineados por {\dbsquare}s de cierre;
\pind -- con textos introductorios o explicativos breves, usualmente uno o dos párrafos pequeños.
\afslut
\pind ¡Todo esto \ysfchg{se presenta } \textsf{``a usted en colores
  vivos''\,!}\footnote{\LLLL -- ¡como los programas de la cadena de televisión \textsf{NBC}
  anunciarían ``orgullosamente'' en \ysf{la }década de los 1960\,!}
\pind Así que prepárese:
\begynd
\pind Estudie estos párrafos: párrafo por párrafo.
\pind Cada uno \ysfchg{forma } un ``entero'' separado.
\afslut
\afslut


\nbbbbb{Autoestudio}\label{sec:Used for Self-Study}

\begynd
\pind Este \primer\ es predominantemente para uso junto a conferencias presenciales.
\pind Para el autoestudio de estudiantes de B.Sc.\ y M.Sc.\ e
      ingenieros de software novatos, recomendamos utilizar este manual en
      junto a su ``origen'' \cite{BjornerMonograph2020}.
\pind Para el autoestudio de estudiantes de doctorado y científicos
      informáticos graduados, recomendamos ir directamente a la fuente:
      \cite{BjornerMonograph2020}.   
\afslut

\nbbbbb{Dos Ejemplos}\label{primer:Two Examples}

\begynd
\pind Hay alrededor de 80 ejemplos dispersos en las primeras 120 páginas.
\pind Además, presentamos dos ejemplos más grandes:
\afslut
\begin{itemize}
\item \textsf{Transporte por Carretera}, Apéndice\,\ref{Chapter:Road
        Transport}, páginas\,\pageref{Chapter:Road Transport}--\pageref{p-ch:Road Transport.n}, 
\item \textsf{Tuberías},
      apéndice\,\ref{appendix:Pipelines}, páginas\,\pageref{appendix:Pipelines}--\pageref{appendix:Pipelines.n}. 
\end{itemize}

\dbeat{%%%%%%%%%%%%%%%%%%%%%%%%%%%%%%%%%%%%%%%%%%%%%%%%%%%%%%%%%%%%%%%%%%%%%%%%%%%%%%%%%%%%%%%%%%%%%%
\nbbbbb{Conferencias}\label{sec:Use in Lectures}

\begynd
\pind Este \primer\ fue desarrollado durante el verano del 2022.
\pind El autor iba a dar un conjunto de siete conferencias dobles \nyl en la Universidad Técnica de Viena\footnote{El autor ``despertó'' a lo que realmente es la ingeniería de software y la ciencia de computación en Viena,
 en 1973--1975, en el \sort{IBM Vienna Labor}, con colegas como \sfsl{Hans Beki{\v{c}}, Cliff Jones, Peter Lucas} y \sfsl{Kurt Walk}. Tenía que ser Viena, la ciudad del \sfsl{Wiener Kreis} \texttt{en.\-wiki\-pedia.\-org/\-wi\-ki/\-Vienna\_\-Circle}, \sfsl{Ludwig Wittgenstein} \texttt{en.\-wiki\-pedia.\-org/\-wi\-ki/\-Lud\-wig\_\-Witt\-gen\-stein} y \sfsl{Sir Karl Popper} \texttt{en.\-wiki\-pedia.\-org/\-wi\-ki/\-Karl\_\-Popper}.}
      del 24 de octubre al 4 de noviembre de 2022.
      \pind Estas conferencias fueron: 
\begin{itemize}
\item Semana 1:
\begin{description}
\item[Primer día de conferencia:] capítulo\,\ref{chapter:Domains}:
\begin{itemize}
\item \sort{Dominios} -- ¡la conferencia ``dobla'' como un seminario de facultad\,!
\end{itemize} y apéndice\,A: 
\begin{itemize}
\item \sort{Un ejemplo, I}
\end{itemize}
\item[Segundo día de conferencia:] capítulo\,\,\ref{primer-extq.1}:
\begin{itemize} 
\item \sort{Entes Resistentes: Cualidades Externas}
\end{itemize}   
  
\item[Tercer día de conferencia:] secciones\,\,\ref{chap4.Internal
    Qualities}--\ref{chap4.Mereology}:
\begin{itemize}
\item \sort{Identificadores únicos}
\item \sort{Mereología}
\end{itemize}

\item[Cuarto día de conferencia:] secciones\,\,\ref{chap4.Attributes}--\ref{A
    Domain Discovery Process, II}:
\begin{itemize}
\item \sort{Atributos},
\item \sort{Atracción Intencional},
\item \sort{Un proceso para el descubrimiento del dominio}
\end{itemize}
\end{description}
\item Semana 2:
\begin{description}
\item[Quinto día de conferencia:]  apéndice\,A: 
\begin{itemize}
\item \sort{Un ejemplo, II}  Más sobre \sort{transporte por carretera}, y
\end{itemize}
 capítulo\,\,\ref{chapter:Perdurants}, sección\,\ref{A General Analysis of Part  Behaviours}:
\begin{itemize}
\item \sort{Un análisis general de los comportamientos de las partes}
\end{itemize}
\item[Sexto día de conferencia:] capítulo\,\,\ref{chapter:Perdurants},
  secciones\,\ref{Domain Channel Description}--\ref{Behaviour Definition Bodies}: 
\begin{itemize}
\item \sort{Canales} y
\item \sort{definiciones de comportamientos: firmas y cuerpos}
\end{itemize}
\item[Séptimo día de conferencia:] capítulo\,\,\ref{chapter:Perdurants},
  secciones\,\ref{Domain Behaviour Initialisation}--\ref{A Domain Discovery Process, III}:
\begin{itemize}
\item \sort{Inicialización del comportamiento del dominio}, 
\item \sort{Un proceso para el descubrimiento del dominio}, % Repeated from lecture day 4, is this a mistake?
\end{itemize}
\end{description}
\end{itemize}
\afslut
}%%%%%%%%%%%%%%%%%%%%%%%%%%%%%%%%%%%%%%%%%%%%%%%%%%%%%%%%%%%%%%%%%%%%%%%%%%%%%%%
\nbbbbb{Relación a \cite{BjornerMonograph2020}}

\begynd
\pind Este \primer\ está basado en \cite[nov.\,2021]{BjornerMonograph2020}.
\pind El capítulo\,\ref{chapter:Philosophy} es una reescritura completa del
      \cite[capítulo\,2]{BjornerMonograph2020}.  
\pind Los capítulos\,\ref{primer-extq.1}--\ref{chapter:Perdurants} son una
      ``condensación'' de \cite[capítulos\,4--7]{BjornerMonograph2020}:
\begynd
\pind \cite[Capítulo\,6]{BjornerMonograph2020} ha sido abreviado y
      aparece en este manual como sect.\,\ref{Transcendental Deductions}.
\pind Del \cite[capítulo\,4]{BjornerMonograph2020}, en el
      capítulo\,\ref{primer-extq.1}, hemos omitido todo 
      el material sobre lo que allí se refiere como \sfsl{Conjuntos} 
\pind y hemos refinado aún más la noción de los \sfsl{nombres de tipos}.
\afslut
\pind Hemos refinado el enfoque en métodos: principios, procedimientos,
      técnicas y herramientas.
\begynd
\pind Encontrarás, en la sección de \sfsl{índices},
      sect.\,\vref{label.dadmethod}, un resumen de las referencias a estos.
\pind El trabajo para resaltar más los pasos del método sigue en progreso.
\afslut
\pind La sección\,\ref{Discrete Dynamic Domains} es nueva. 
\afslut

\nbbbbb{El lenguaje de especificación \texttt{RAISE}, \texttt{RSL}, y \rslplus}\label{RSL-I}

\begynd
\pind La notación formal (que acompaña al texto informal) de este
     \primer\ es la de \texttt{RSL} \cite{RSL} ("\sort{R}AISE \sort{S}pecification \sort{L}anguage"), el \sort{L}enguaje de
      \sort{E}specificación de \sort{RAISE}, donde \texttt{RAISE} ("\sort{R}igorous \sort{A}pproach to \sort{I}ndustrial \sort{S}oftware \sort{E}ngineering")
      significa \ysfchg{E}nfoque \ysfchg{R}iguroso para la
      \ysfchg{I}ngeniería de \ysfchg{S}oftware \ysfchg{I}ndustrial \cite{RaiseMethod}. 
\pind Se pueden utilizar otras notaciones formales en su lugar.
\pind Ejemplos alternativos podrían ser \texttt{VDM} \citevdm, \texttt{Z}
      \citez, o \texttt{Alloy} \citealloy.
\pind Estamos utilizando más el lenguaje de especificación \texttt{RAISE},
      \texttt{RSL} que el método.
\pind Lo estamos usando de dos maneras:
\begin{itemize}
\item informalmente, para presentar y explicar los métodos de análisis \&
      descripción de dominios en este \primer\ y
\item formalmente, para presentar descripciones de dominios. 
\end{itemize}
\pind El \texttt{RSL} informal es una versión extendida,
\rslplus.\footnote{Ver apéndice sect.\,\vref{chap2.tex.rsltext}.}
\pind Estas dos maneras no están relacionadas de otra forma.
\pind Se podría utilizar otro lenguaje de especificación \ysfchg{ya sea} para los
      aspectos informales o \ysfchg{para} los formales.
\afslut

\nbbbbb{Conclusión}

\begynd
\pind El propósito de esta introducción es ubicar el presente \primer\
\pind en el contexto de \ysfchgv{otros libros de Dines Bj{\o}rner } 
      \cite{TheSEBook1,TheSEBook2,TheSEBook3} sobre desarrollo de software 
\pind , conferencias y auto\ysf{-}estudio.
\afslut

\label{chap:Introduction.n}

%%  LocalWords:  analyse defind VDM CafeObj CSP Kai rlander RSL Ph wi
%%  LocalWords:  summarises Laamswerde intialisation Lamsweerde Beki
%%  LocalWords:  Kreis pedia ki Lud Endurants Mereology Behaviours
%%  LocalWords:  Behaviour Initialisation formalisations colours AISE
%%  LocalWords:  programmmes dadmethod pecification anguage igorous 
%%  LocalWords:  pproach ndustrial oftware ngineering Gabrielli Caine
%%  LocalWords:  Rosetti Hoare equirements omain pdefind wrt
