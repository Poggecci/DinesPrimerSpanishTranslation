

%%%%%%%%%%%%%%%%%%%%%%%%%%%%%%%%%%%%%%%%%%%%%%%%%%%%%%%%%%%%%%%%%%%%%%%
%%%% Released for translation 20 April 2023  %%%%%%%%%%%%%%%%%%%%%%%%%%
%%%%%%%%%%%%%%%%%%%%%%%%%%%%%%%%%%%%%%%%%%%%%%%%%%%%%%%%%%%%%%%%%%%%%%%

\nbbbbbb{\HHHH Dominios}\label{chapter:Domains}\label{chap2.tex.Preview}
\pos{\minitoc}{}
\pos{\label{lect1:label1}}{}

\pos{
\noindent
\begynd
\pind Est\pos{e}{a} \pos{capítulo}{sección} es informal.
\pind Aquí introducimos conceptos importantes de dominios.
\pind L\pos{o}{a}s siguientes \pos{capítulos}{secciones} serán más técnic\pos{o}{a}s.
\begynd
\pind Ell\pos{o}{a}s definirán la mayoría de los conceptos de dominio cubiertos en est\pos{e}{a}
      \pos{capítulo}{sección} adecuadamente.
\afslut
\afslut
}{}

\bbbbb{Definición de Dominio}\HHHH

\pos{Repetimos la definición del concepto de dominios dada por primera vez en la
página\,\pageref{1stdd}.}{}

\bookdefn{Dominio}{ \domaindefinition}

\pos{%
\noindent
\ysf{Excluimos de nuestro tratamiento de dominios cuestiones de asuntos biológicos y % "tratamiento" feels off here
psicológicos.}}{}

\mnewfoil

\monoexample{Dominios}{Algunos ejemplos, más o menos autoexplicativos:
\begin{itemize}
\item \bbcolor{Ríos --} con sus fuentes naturales, deltas, afluentes,
  cascadas, etc., y sus presas, puertos, esclusas, etc. construidos por el hombre --
  y su transporte de materiales (barcos, etc.) \cite{BjornerRiversCanals2021};
\item \bbcolor{Redes viales --} con segmentos de calles e intersecciones, semáforos
  y automóviles -- y el flujo de estos;
\item \bbcolor{Tuberías --} con sus pozos, tuberías, válvulas, bombas, bifurcaciones,
  uniones y pozos y el flujo de fluidos \cite{2013pipe}; y
\item \bbcolor{Terminales de contenedores --} con sus barcos portacontenedores,
  contenedores, grúas, camiones, etc. -- y el movimiento de todos estos \cite{BjornerContainer2018}\dbsquare
\end{itemize}}

\noindent
\mnewfoil%\LLLL
\begynd
\pind La definición se basa en la comprensión de los términos \nyl
\pos{}{\vspace*{-10mm}\begin{multicols}{2}}
      \sfsl{`racionalmente describible'}, \nyl \sfsl{`dinámica discreta'}, \nyl \sfsl{`asistido por humanos'}, \nyl \sfsl{`sólido'} y \nyl \sfsl{`fluido'}.
\pos{}{\end{multicols}}
 Los dos últimos se explicarán más adelante.
\pind Por \bbcolor{\sfsl{racionalmente describible}} queremos decir que lo que se describe \nyl
      puede ser entendido, incluido razonado, de manera racional, \nyl es decir,
      de manera lógica -- \ysf{en otras palabras  \bbcolor{\sfsl{lógicamente tratable}}.}
\pind Por \bbcolor{\sfsl{dinámica discreta}} implicamos que básicamente descartamos
      aquellos fenómenos de dominio que tienen
      propiedades \ysfchg{que} son continuas con
      respecto a su comportamiento \pos{dinámico,}{}
      en el tiempo.
\pind Por \bbcolor{\sfsl{asistido por humanos}} queremos decir que los dominios -- que estamos
      interesados en modelar -- tienen, como propiedad importante, que
      poseen entidades creadas por la humanidad.
\afslut

\pos{%%%%%%%%%%%%%%%%%%%%%%%%%%%%%%%%%%%%%%%%%%%%%%%%%%%%%
\begynd
\pind Este manual presenta un \sfsl{método}\pconindex{method}, sus
      \sfsl{principios}\pconindex{principle}, 
      \sfsl{procedimientos}\pconindex{procedure}, 
      \sfsl{técnicas}\pconindex{technique} 
      y 
      \sfsl{herramientas}\pconindex{tool}, 
      para \sfsl{analizar} \&\footnote{Usamos aquí el ampersand, `\&',
      como en $A\&B$, para enfatizar que estamos tratando $A$ y $B$ como
      un solo concepto.} \sfsl{describir} dominios.
\afslut
}{}%%%%%%%%%%%%%%%%%%%%%%%%%%%%%%%%%%%%%%%%%%%%%%%%%%%%%%%%

\nbbbbb{Fenómenos y Entidades}\HHHH

\begynd
\pind \primerdefn{Fenómenos}{ {\pdefindex{phenomenon} Por un
\sfsl{fenómeno} entenderemos un hecho que se observa que existe o sucede\dbsquare}}

\noindent
\begynd
\pind Algunos fenómenos son racionalmente describibles -- en gran medida o en su totalidad -- otros no.
\afslut

\noindent
\begynd
\pind Algunos fenómenos son racionalmente describibles -- en gran medida o en su totalidad -- otros no.
\afslut

\noindent
\pind \primerdefn{Entidades}{ Por una entidad \pdefindex{entity} \ysfchg{}
      entenderemos un fenómeno más o menos racionalmente describible\dbsquare}

\noindent
\pind \newestxample{Fenómenos y Entidades}{Algunos, pero no necesariamente
  todos los aspectos de un río pueden ser racionalmente descritos, por lo tanto, pueden ser considerados
  entidades. De manera similar, muchos aspectos de una
  red vial pueden ser racionalmente descritos, por lo tanto, serán considerados entidades\dbsquare}
\afslut

\nbbbbb{Endurantes y Perdurantes}

\bbbb{Endurantes}

\begynd
%%%%%%%
\pind \bookdefn{Endurantes}{ Los endurantes son \pdefindex{endurant}\ 
      aquellas cantidades en dominios que podemos
      observar (ver y tocar), en \sfsl{espacio}, como entidades ``completas'' en cualquier punto en \sfsl{tiempo} -- entidades ``materiales'' que
      persisten\ysfchg{}, perduran\ysfchg{}\dbsquare}%%%%
      
\noindent
\monoexample{Endurantes}{ Ejemplos de endurantes son: un segmento de calle [enlace], una intersección de calles [conexión], un
      automóvil\dbsquare}
      
\noindent
\pind Los endurantes de dominio, cuando eventualmente se modelan en software, típicamente
      se convierten en datos. Por lo tanto, el análisis cuidadoso de los endurantes de dominio es un
      requisito previo para la concepción y análisis cuidadosos de
      estructuras de datos para software, incluidas las bases de datos.
\afslut
%%%%%%%
    
\nbbbb{Perdurantes}

\begynd
\pind \newestpdefn{Perdurantes}{ Los perdurantes son \pdefindex{perdurant}\   
      aquellas cantidades de dominios de las cuales solo existe un
      fragmento, en \sfsl{espacio}, si los miramos o tocamos en
      cualquier instantánea en
      \sfsl{tiempo}\dbsquare}%%%%
      
\noindent
\newestxample{Perdurante}{ Un automóvil en movimiento es un ejemplo de un perdurante\dbsquare}
        
\noindent
\pind Los perdurantes de dominio, cuando eventualmente se modelan en software, típicamente
      se convierten en procesos. Por lo tanto, el análisis cuidadoso de los perdurantes de dominio es un
      requisito previo para la concepción y análisis cuidadosos de
      funciones (procedimientos).
\afslut
%%%%%%%
    
\nbbbbb{Cualidades Externas e Internas de Endurantes }

\bbbb{Cualidades Externas}

\newestpdefn{Cualidades Externas}{ \pdefindex{external quality}\
     Las cualidades externas de los endurantes de un dominio manifiesto 
\begynd
\pind son, en un sentido simplificador, aquellas que podemos
\begynd
\pind \ysf{ver,}
\pind tocar y
\pind tener extensión espacial.
\afslut
\pind Ellas, por así decirlo, toman forma.
\afslut}

\mnewfoil

\noindent
\newestxample{Cualidades Externas}{ Un ejemplo de cualidades externas de
  un dominio\ysfchg{} es:
\begynd
\pind el Cartesiano\footnote{\LLLL Cartesiano
      según el filósofo, matemático y científico francés Ren{\'e}
      Descartes (1596--1650)} 
\begynd
\pind de conjuntos de intersecciones de calles atómicas sólidas,
\pind de conjuntos de segmentos de calles atómicos sólidos y 
\pind de conjuntos de automóviles sólidos
\afslut de un sistema de transporte por carretera
\pind  donde el
\pos{}{\begin{multicols}{4}}   
\begynd
\pind Cartesiano, \pind conjuntos, 
      \pind atómico, y \pind sólido
\afslut
\pos{}{\end{multicols}} reflejan cualidades externas\dbsquare
\afslut}
%%%%%%%
\nbbb{Endurantes discretos o sólidos}

\newestpdefn{Endurantes discretos o sólidos}{ \pdefindex{discrete!endurant}\pdefindex{endurant!discrete}%%%% 
      \pdefindex{solid!endurant}\pdefindex{endurant!solid} Por un endurante
      \sfsl{sólido} \nyl [o \sfsl{discreto}]  entenderemos un endurante
\begynd
\pind que es separado, individual o distinto en forma o concepto,
\pind o, reformulando: tiene `cuerpo' [o magnitud]
      de tres dimensiones: longitud, anchura y profundidad
      \cite[vol.\,II, pág.\,2046]{OED}\dbsquare
\afslut}

\noindent
%\mnewfoil
\newestxample{Endurantes sólidos}{ Ejemplos de endurantes sólidos son
\begynd
\pind los
\pos{}{\begin{multicols}{4}}
\begynd
\pind pozos, 
\pind tuberías,
\pind válvulas,
\pind bombas, 
\pind bifurcaciones, 
\pind uniones y 
\pind desagües
\afslut
\pos{}{\end{multicols}} de tuberías\ysfchg{}. 
\pind{} [Estas unidades pueden, sin embargo, y generalmente lo harán, contener fluidos, por ejemplo, petróleo, gas
  o agua]\dbsquare
\afslut}
\mnewfoil

\noindent
\begynd
\pind Principalmente estaremos analizando y describiendo endurantes sólidos.

\pind Como veremos, en la siguiente \pos{capítulo}{sección},
\begynd
\pind analizamos y describimos endurantes de sólidos como
\pind  ya sea partes 
\pind o especies vivas: animales y humanos.
\afslut
\pind Principalmente nos ocuparemos de las partes.
\begynd
\pind Es decir, solo, como: ``de paso'', 
\pind por \ysfchg{el} motivo de completitud,
\pind mencionaremos especies vivas\,!
\afslut
\afslut
%%%%%%%
    
\nbbb{Fluidos}

\begynd
\pind \newestpdefn{Endurantes fluidos}{ \pdefindex{fluid!endurant}\pdefindex{endurant!fluid}\ 
\begynd
\pind Por un \sfsl{endurante fluido} entenderemos un endurante \nyl que es 
\begynd
\pind prolongado, sin interrupción, \nyl en una serie o patrón ininterrumpido;
\pind o, reformulando: una sustancia (líquido,
  gas o plasma) que tiene la propiedad de fluir, consistiendo en
  partículas que se mueven entre sí \cite[vol.\,I, pág.\,774]{OED} \dbsquare
\afslut
\afslut}
%%%%%%%
\afslut

\noindent
\newestxample{Endurantes fluidos}{ Ejemplos de endurantes fluidos son: % Sh
  \pos{}{\begin{multicols}{3}}
\begynd
\pind agua, \pind petróleo, \pind gas,
  \pind aire comprimido, \pind humo\dbsquare
  \afslut
\pos{}{\end{multicols}}}

\mnewfoil

\noindent
\begynd
\pind Los fluidos son de otro modo
\begynd
\pind líquidos, o
\pind gaseosos, o
\pind plasmáticos, o
\pind granulares\footnote{\label{fn.granular} Esta es una decisión puramente pragmática. ``Por supuesto'' la arena, la grava, el suelo, etc., no son
  fluidos, pero para nuestros propósitos de modelado es conveniente
  ``compartimentarlos'' como fluidos\,!}, o como
\pind productos vegetales, por ejemplo, caña de azúcar picada, trillada o procesada de alguna otra manera\footnote{Ver nota al pie \vref{fn.granular}.},
\pind etcétera.
\afslut
\pind Los endurantes fluidos serán analizados y descritos \nyl en relación con
      los endurantes de sólidos, es decir, sus ``contenedores''.
\afslut
\nbbb{Partes}
  
\begynd
\pind \newestpdefn{Partes}{ \pdefindex{parts}\
\begynd
\pind Los sólidos no vivos son lo que llamaremos partes\dbsquare
\afslut}

\noindent
\pind Las partes son los ``caballos de batalla'' de los dominios hechos por el hombre.  % Unsure how to translate workhorse idiom
\pind Es decir, nos ocuparemos principalmente \nyl del análisis y
      descripción de endurantes en partes.
\afslut
%%%%%%%

\noindent
\newestxample{Partes}{ El ejemplo anterior de sólidos
  también era un ejemplo de partes\dbsquare}

\noindent
\begynd
\pind Distinguimos entre partes atómicas y compuestas.
\afslut

\nbb{Partes atómicas}
  
\bookdefn{Parte atómica, I}{ \label{pd-atomic-parts}
\begynd
\pind Por una \sfsl{parte atómica} entenderemos una parte
\begynd
\pind que el analista de dominio considera indivisible
\pind en el sentido de no ser \ysfchg{divisible} de manera significativa,
\pind para los propósitos del dominio en consideración,
\pind es decir, que no consiste significativamente en subpartes\eod
\afslut
\afslut
}

\monoexample{Partes atómicas}{ Ejemplos de partes atómicas son:
\begynd
\pind un centro, es decir, una intersección de calles;
\pind un enlace, es decir, el tramo de carretera entre dos centros vecinos; y
\pind un automóvil \dbsquare
\afslut}

\nbb{Partes compuestas}
  
\begynd
\pind Nosotros, pragmáticamente, distinguimos entre
\begynd
\pind partes orientadas al producto cartesiano, y 
\pind partes orientadas a conjuntos.
\afslut
\pind Si están orientadas al producto cartesiano, consisten en dos o más
      endurantes (sólidos o fluidos) claramente diferenciados \ysfchg{. }
\pind Si están orientadas a conjuntos, consisten en un número indefinido de \nyl cero,
      una o más partes.
\afslut

\bookdefn{Parte compuesta, I}{ %
\begynd
\pind \pdindextermii{Partes}{compuestas} son aquellas que son
\begynd
\pind ya sea \ysf{cartesiano-}
\pind o son partes orientadas a conjuntos
\afslut
\pind  \eod
\afslut
}

\monoexample{Partes compuestas}{ Un ejemplo de partes compuestas es:
\begynd
\pind \ysfchg{(i) } una red vial que consiste en un conjunto de centros,
  es decir, intersecciones de calles o ``fin de calles'',  y
\pind  \ysfchg{(ii) } un conjunto de
  enlaces, es decir, segmentos de calles (sin centros contenidos),
\afslut
 es un compuesto cartesiano\ysfchg{. }
\begynd
\pind  \ysfchg{(iii) } \ysfchg{Cada conjunto } de centros y \ysfchg{cada conjunto } de enlaces
\afslut son compuestos de conjuntos\dbsquare}

\nbbbb{Un apunte: una ontología superior}\label{An Upper Ontology}

\begynd
\pind Hemos sido razonablemente cuidadosos
\begynd  
\pind en solo introducir y establecer
      definiciones informales 
\pind de fenómenos y algunas clases de estos.
\afslut
\pind En el próximo capítulo, en cierto sentido, ``repetiremos'' la cobertura de
      estos fenómenos.
\begynd
\pind Pero \ysf{entonces} de una manera más analítica.
\pind La figura\,\vref{onto.fig0} tiene la intención de indicar esto.
\afslut
\afslut

\mnewfoil

\noindent
\hDBfigure{onto}{\pos{10.5}{11.7}cm}{{Una} ontología superior}{onto.fig0}

\mnewfoil

\noindent
\begynd
\pind Hasta ahora solo hemos tocado 
\begynd
\pind el cuadro de líneas discontinuas etiquetado como `cualidades externas' 
\pind del cuadro de líneas discontinuas etiquetado como `Endurantes' de la fig.\,\ref{onto.fig0}.
\afslut
\pind En el capítulo \ref{primer-extq.1} trataremos las cualidades externas
      con más profundidad ---
\begynd
\pind de manera más sistemática: analíticamente y descriptivamente.
\afslut
\afslut


\nbbbb{Cualidades Internas}
     
\newestpdefn{Cualidades internas}{ Las cualidades internas son \pdefindex{internal quality}
\begynd
\pind aquellas propiedades [de los endurantes]
\pind que no ocupan \sfsl{espacio}
\pind pero que pueden ser medidas o mencionadas\dbsquare
\afslut}%%%%
      
\newestxample{Cualidades internas}{ Ejemplos de cualidades internas son %%%%
\begynd
\pind la identidad única de una parte, 
\pind la relación de una parte con otras
      partes, y 
\pind los atributos de los endurantes como temperatura, longitud, color\dbsquare
\afslut} 

  
\nbbb{Identidad única}
    
\begynd
\pind \newestpdefn{Identidad única}{ Una identidad única es \pdefindexii{unique}{identity}\  
\begynd
\pind una propiedad inmaterial
\pind que distingue a dos sólidos \sfsl{espacialmente}
      distintos\dbsquare
\afslut}
\afslut

\newestxample{Identidades únicas}{ Cada 
\begynd
\pind centro en una red de carreteras es identificado de manera única,
\pind así como cada enlace 
\pind y cada automóvil\dbsquare
\afslut}
%%%%%%%

\nbbb{Mereología}
\begynd 
\pind \newestpdefn{Mereología, I}{ La mereología es \pdefindex{mereology}\  
      una teoría de las relaciones de parte-idad [de endurantes]: % doubts about part-hood
\begynd 
\pind de las relaciones de una parte [endurante] con un todo
\pind y de las
      relaciones de las partes [endurantes] con otras partes [endurantes] dentro de ese
      todo\dbsquare
\afslut}%%%%
\afslut
      
\newestxample{Mereología}{ Ejemplos de mereologías son %%%%
\begynd 
\pind que un enlace está topológicamente \sl conectado \rm a \nyl
      exactamente dos centros específicos,
\pind que un centro está \sl conectado \rm a \nyl cero, uno o
      más enlaces específicos,
\pind y que enlaces y centros están \sl abiertos \rm a \nyl
      subconjuntos específicos de automóviles\dbsquare
\afslut}
%%%%%%%

\nbbb{Atributos}
\newestpdefn{Atributos}{ Los atributos son \pdefindex{attribute}\
      propiedades de los endurantes
\begynd 
\pind que no son \sfsl{espacialmente} observables, 
\pind pero que pueden ser
      físicamente
\pind (electrónicamente, químicamente u otros modos)
\pind medidos o pueden
      ser mencionados objetivamente\dbsquare
\afslut}%%%%
      
\newestxample{Atributos}{ Ejemplos de atributos son:
\begynd 
\pind los enlaces \ysfchg{ } tienen longitudes, y,
\pind que en cualquier momento,
\pind cero, uno o más automóviles están ocupando \ysfchg{cada
enlace}\ysf{\footnote{Oh sí, por supuesto, es espacialmente observable
    que un enlace tiene una longitud, pero la medición, digamos \sfsl{123
      metros} no lo es; y el número de automóviles en el enlace tampoco es
    espacialmente observable.}}\dbsquare
\afslut}
%%%%%%%
\label{pre-primer.Prompts}
\nbbbbb{Indicaciones}\label{primer.Prompts} % still unsure about translating prompts


\bbbb{Indicaciones de Análisis}

\begynd 
\pind \newestpdefn{Indicaciones de análisis}{ Una indicación de análisis es \pdefindex{analysis!prompt}\
\begynd 
\pind un predicado o una función 
\pind que puede ser planteada por humanos como \ysf{segmentos}
      de un dominio.
\pind Observando el dominio, el analista puede entonces
\pind actuar sobre la combinación de la indicación particular 
\pind (ya sea un
      predicado o una función \nyl y luego cuál de estos en particular es)
\pind  así ``aplicándola'' a un fenómeno del dominio
\pind  y obteniendo, en las
      mentes de los humanos,
\pind  ya sea un valor de verdad u otra forma de valor\dbsquare
\afslut}
\afslut
      
\nbbb{Predicado de análisis}
\begynd
\pind \newestpdefn{Predicados de análisis}{ \pdefindex{analysis!predicate}\ 
\begynd
\pind Un predicado de análisis es una indicación de análisis
\pind que produce un valor de verdad\dbsquare
\afslut}
\afslut
      
\newestxample{Predicados de análisis}{ Ejemplos generales de predicados de
  análisis son:
\begynd
\pind \sfsl{``¿puede un fenóme\ysfchg{no} observable ser descrito racionalmente''}, \nyl es decir, una entidad,
\pind \sfsl{``¿es una entidad un sólido o un fluido''},
\pind \sfsl{``¿es un endurante sólido una parte o una especie viviente''}\dbsquare
\afslut}
%%%%%%%
    
\nbbb{Función de análisis}

\newestpdefn{Función de análisis}{ Una función de análisis es \pdefindex{analysis!function}%%%
\begynd
\pind una indicación de análisis que produce algún \texttt{\rsltxt}\dbsquare
\afslut} 
      
\newestxample{Funciones de análisis}{ Dos ejemplos de funciones de análisis son:
\begynd
\pind una produce los endurantes de una parte cartesiana \nyl y sus respectivos nombres de tipo,
\pind otra produce el conjunto de partes de un conjunto de partes \nyl y su tipo común\dbsquare
\afslut}
%%%%%%%
  
\nbbbb{Indicación de descripción}

\begynd  
\pind \newestpdefn{Indicación de descripción}{ Una indicación de descripción es \pdefindex{description!prompt}
\begynd  
\pind una función que puede ser planteada por humanos
\pind quienes pueden entonces actuar sobre ella:
\begynd  
\pind \ysf{{[el humano]}} ``aplicándola'' a un fenóme\ysfchg{no} del dominio y
\pind \ysf{{[el humano]}} ``obteniendo'', es decir, escribiendo,
      narrativas y \rsltxt{}s formales \nyl describiendo 
      lo que se está observando \ysf{[por ese humano]}\dbsquare
\afslut
\afslut}
\afslut

\mnewfoil

\newestxample{Indicaciones de descripción}{ Las indicaciones de descripción resultan en
\begynd
\pind \rsltxt{}s que describen, por ejemplo, un
\begynd
\pind (i) \ysfchg{un} endurante cartesiano o
\pind (ii) su identificador único \ysf{o}
\pind (iii) su mereología o
\pind (iv) sus atributos \ysf{u}
\pind(iv) otros\dbsquare
\afslut
\afslut}
%%%%%%%

\nbbbbb{Conceptos perdurantes}\label{Perdurant Concepts}

\bbbb{``Transformación'' de partes a comportamientos}

\begynd
\pind Como ya se indicó, vamos a
\begynd
\pind deducir trascendentalmente
\pind comportamientos (perdurantes) de
\pind aquellas partes (endurantes)
\begynd
\pind que nosotros, como analistas y describidores de dominio,
\pind hemos dotado con los tres tipos de cualidades internas:
\pind identificadores únicos, mereologías y atributos.
\afslut
\afslut
\pind El capítulo\,\ref{chap6.tex.1}, mostrará cómo.
\afslut

\nbbbb{Estado}
\newestpdefn{Estado, I}{ \pdefindex{state}\ 
\begynd
\pind  Un estado es cualquier conjunto de
      las partes de un dominio\dbsquare
\afslut}
%%%%%%%
\monoexample{Un estado del sistema de carreteras}{ El analista y describidor del dominio
  puede, \nyl \ysf{} decidir que un estado del sistema vial consiste en
\begynd
\pind el agregado de la red de carreteras (de centros y enlaces)\footnote{\LLLL El agregado de la red de carreteras,
  en su forma perdurante, puede ``modelar'' el \sfsl{Departamento
  de Caminos} de algún país, provincia o ciudad.},
\pind todos los centros y todos los enlaces, y
\pind el agregado de automóviles (de todos los automóviles)\footnote{\LLLL
  El agregado de automóviles,
  en su forma perdurante, puede ``modelar'' el \sfsl{Departamento
  de Vehículos} de algún país, provincia o ciudad.}, y
\pind todos los automóviles individuales\dbsquare
\afslut}

\nbbbb{Actores}
\newestpdefn{Actores}{ \pdefindex{actor}\ 
\begynd
\pind Un actor es cualquier cosa
      que pueda iniciar una acción, un evento o un comportamiento\dbsquare
\afslut}
%%%%%%%
 
\bbb{Acción}
   
\newestpdefn{Acciones}{ \pdefindex{action}\
\begynd
\pind  Una acción es una
      función que puede cambiar deliberadamente un estado\dbsquare
\afslut}
\monoexample{Acciones de la red de carreteras}{ Estas son algunas acciones de la red de carreteras:
\begynd
\pind La inserción de un nuevo centro o la eliminación de uno existente; o
\pind la inserción de un nuevo enlace o la eliminación de uno existente;
\afslut}

%%%%%%%

\nbbb{Evento}

\newestpdefn{Eventos}{ \pdefindex{event}\
\begynd
\pind  Un evento es una función
      que cambia un estado subrepticiamente\dbsquare
\afslut}
\monoexample{Eventos de la red de carreteras}{ Estos son algunos eventos de la red de carreteras:
\begynd
\pind El bloqueo de un enlace debido a un deslizamiento de tierra;
\pind el fallo de una señal de tráfico de un centro debido a un corte de energía;
\pind el bloqueo de un enlace debido a un accidente automovilístico.
\afslut}

%%%%%%%
    
\nbbb{Comportamiento}

\newestpdefn{Comportamientos}{ \pdefindex{behaviour}\
\begynd
\pind Un comportamiento es un conjunto 
\pind de secuencias de 
\pind acciones, eventos y comportamientos\dbsquare
\afslut}
%%%%%%%
\mnewfoil\LLll
\monoexample{Tráfico en la red vial}{ %
\begynd
\pind El tráfico en la red vial puede ser visto como un comportamiento
\begynd
\pind \ysfchg{(i) } de todos los comportamientos de los automóviles,
\begynd
\pind donde cada comportamiento del automóvil se ve como una secuencia de \nyl arranque, parada,
      giro a la derecha, giro a la izquierda, etc., acciones;
\afslut
\pind \ysfchg{(ii) } de todos los comportamientos de los enlaces
\begynd
\pind donde cada comportamiento del enlace se ve \nyl como un conjunto de secuencias
(i.e., comportamientos) de ``seguir'' el
\begynd
\pind entrada al enlace, salida del enlace y movimiento
\afslut de automóviles en el enlace;
\afslut
\pind \ysfchg{(iii) } de todos los comportamientos de los centros (etc.);
\pind \ysfchg{(iv) } del comportamiento del agregado de carreteras, \nyl
      viz.\ \sfsl{El Departamento de Carreteras}, y 
\pind \ysfchg{(v) } del comportamiento del agregado de automóviles, \nyl viz,\
      \sfsl{El Departamento de Vehículos}.
\afslut
\afslut}\HHHH


\nbbbb{Canal}

\begynd
\pind \newestpdefn{Canal}{ \pdefindex{channel}\
\begynd
\pind Un canal es cualquier cosa
\pind que permite la sincronización y comunicación
\pind de valores
\pind entre dos comportamientos\dbsquare
\afslut}
\afslut

\mnewfoil

\noindent
\begynd
\pind Utilizaremos el concepto \texttt{CSP} de Tony Hoare \citecsp\
\begynd
\pind para expresar la sincronización y comunicación \nyl de valores
      entre comportamientos \brcolor{\textsl{i}} y \brcolor{\textsl{j}}.
\pind Por lo tanto, la declaración del comportamiento \brcolor{\textsl{i}} \nyl
      \bbcolor{\textsl{ch[\brcolor{j}]\,\,!\,\,value}} \nyl
      \ysfchg{indica} que
      el comportamiento \brcolor{\textsl{i}} \nyl  ofrece, ``salidas'':
      \bbcolor{!}, \bbcolor{\textsl{value}} \nyl  al comportamiento
      indicado por \bbcolor{\textsl{\brcolor{j}}}. 
\pind Y el comportamiento \brcolor{\textsf{\ysfchgii{j}}} expresa \nyl  \bbcolor{\textsl{ch[\brcolor{\ysfchgii{i}}]\,\,?}} \nyl 
      que está dispuesto a aceptar \nyl  ``entrada de \&
      sincronizarse con'' el comportamiento \brcolor{\textsl{i}}, \bbcolor{?}, \nyl cualquier \bbcolor{\textsl{valor}}.
\afslut
\afslut
%%%%%%%

\nbbbbb{Análisis y descripción del dominio}

\bbbb{Análisis del dominio}

\newestpdefn{Análisis del dominio}{ El análisis del dominio \pdefindex{domain!analysis}\pdefindex{analysis!domain}\
\begynd
\pind es el acto de estudiar un dominio
\pind así como el resultado de ese estudio
\pind en forma de declaraciones \bmcolor{informales}\dbsquare
\afslut}

\bbbb{Descripción del dominio}

\newestpdefn{Descripción del dominio}{ La descripción del dominio \pdefindex{domain!description}\pdefindex{description!domain}\ 
\begynd
\pind es el acto de describir un dominio 
\pind así como el resultado de ese acto
\pind en la
      forma de narrativas \bmcolor{y} \bmcolor{\rsltxt formal}\dbsquare
\afslut}

\nbbbbb{Cierre}

\begynd
\pind Est\pos{e}{a} \pos{capítulo}{lectura} ha introducido \nyl los conceptos principales de los dominios que usaremos
      para \nyl (analizarlos y describirlos).\footnote{\LLLL Hemos
      omitido el tratamiento de \sfsl{especies vivas: plantas} y
      \sfsl{animales} -- estos últimos incluyendo \sfsl{humanos}. Serán
      incluidos en el próximo capítulo\,!}
\pind L\pos{o}{a}s próxim\pos{o}{a}s \pos{tres capítulos}{lecturas} tratarán ahora sistemáticamente \nyl el
      análisis y la descripción de los dominios.
\begynd
\pind Ese tratamiento toma concepto por concepto y
\begynd
\pind proporciona definiciones adecuadas e
\pind introduce indicaciones de análisis y descripción apropiadas;
\pind una por una, en una progresión casi pedante,
\pind ¡por lo tanto, quizás ``lenta''!
\afslut
\afslut
\mnewfoil
\pind El \pos{lector}{estudiante} puede ser excusado
\begynd
\pind si de vez en cuando pierde de vista ``su camino''.
\afslut
\pind Por lo tanto, el capítulo actual.
\begynd
\pind Para mostrar ``el camino'':
\pind que, por ejemplo, \nyl cuando tratamos cualidades endurantes externas,
\pind aún existen las cualidades endurantes internas,
\pind y que todo conduce a los perdurantes:
\begynd
\pind actores,
\pind acciones,
\pind eventos y
\pind comportamientos.
\afslut
\afslut
\afslut

\label{chap2.tex.Preview.n}
%%  LocalWords:  artefactual analysing Endurants endurant Perdurants
%%  LocalWords:  perdurant endurants colour Mereology mereology RSL
%%  LocalWords:  behaviour Behaviours behaviours harbours modelled de
  %%  LocalWords:  synchronisation perdurants philosophee Ren analyse
%%  LocalWords:  plasmatic compartmentalise fn et cetera analysed CSP
%%  LocalWords:  descriptionally analyser Hoare's synchronise ly wo
%%  LocalWords:  analysers mereologies modelling conveyage phenomen
%%  LocalWords:  neighbouring
