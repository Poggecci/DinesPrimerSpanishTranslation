

%%%%%%%%%%%%%%%%%%%%%%%%%%%%%%%%%%%%%%%%%%%%%%%%%%%%%%%%%%%%%%%%%%%%%%%
%%%% Released for translation 20 April 2023  %%%%%%%%%%%%%%%%%%%%%%%%%%
%%%%%%%%%%%%%%%%%%%%%%%%%%%%%%%%%%%%%%%%%%%%%%%%%%%%%%%%%%%%%%%%%%%%%%%

\nbbbbbb{\HHHH Dominios}\label{chapter:Domains}\label{chap2.tex.Preview}
\pos{\minitoc}{}
\pos{\label{lect1:label1}}{}

\pos{
\noindent
\begynd
\pind Est\pos{e}{a} \pos{capítulo}{sección} es informal.
\pind Aquí introducimos conceptos importantes de dominios.
\pind L\pos{o}{a}s siguientes \pos{capítulos}{secciones} serán más técnic\pos{o}{a}s.
\begynd
\pind Ell\pos{o}{a}s definirán la mayoría de los conceptos de dominio cubiertos en est\pos{e}{a}
      \pos{capítulo}{sección} adecuadamente.
\afslut
\afslut
}{}

\bbbbb{Definición de Dominio}\HHHH

\pos{Repetimos la definición del concepto de dominios dada por primera vez en la
página\,\pageref{1stdd}.}{}

\bookdefn{Dominio}{ \domaindefinition}

\pos{%
\noindent
\ysf{Excluimos de nuestro tratamiento de dominios cuestiones de asuntos biológicos y % "tratamiento" feels off here
psicológicos.}}{}

\mnewfoil

\monoexample{Dominios}{Algunos ejemplos, más o menos autoexplicativos:
\begin{itemize}
\item \bbcolor{Ríos --} con sus fuentes naturales, deltas, afluentes,
  cascadas, etc., y sus presas, puertos, esclusas, etc. construidos por el hombre --
  y su transporte de materiales (barcos, etc.) \cite{BjornerRiversCanals2021};
\item \bbcolor{Redes viales --} con segmentos de calles e intersecciones, semáforos
  y automóviles -- y el flujo de estos;
\item \bbcolor{Tuberías --} con sus pozos, tuberías, válvulas, bombas, bifurcaciones,
  uniones y pozos y el flujo de fluidos \cite{2013pipe}; y
\item \bbcolor{Terminales de contenedores --} con sus barcos portacontenedores,
  contenedores, grúas, camiones, etc. -- y el movimiento de todos estos \cite{BjornerContainer2018}\dbsquare
\end{itemize}}

\noindent
\mnewfoil%\LLLL
\begynd
\pind La definición se basa en la comprensión de los términos \nyl
\pos{}{\vspace*{-10mm}\begin{multicols}{2}}
      \sfsl{`racionalmente describible'}, \nyl \sfsl{`dinámica discreta'}, \nyl \sfsl{`asistido por humanos'}, \nyl \sfsl{`sólido'} y \nyl \sfsl{`fluido'}.
\pos{}{\end{multicols}}
 Los dos últimos se explicarán más adelante.
\pind Por \bbcolor{\sfsl{racionalmente describible}} queremos decir que lo que se describe \nyl
      puede ser entendido, incluido razonado, de manera racional, \nyl es decir,
      de manera lógica -- \ysf{en otras palabras  \bbcolor{\sfsl{lógicamente tratable}}.}
\pind Por \bbcolor{\sfsl{dinámica discreta}} implicamos que básicamente descartamos
      aquellos fenómenos de dominio que tienen
      propiedades \ysfchg{que} son continuas con
      respecto a su comportamiento \pos{dinámico,}{}
      en el tiempo.
\pind Por \bbcolor{\sfsl{asistido por humanos}} queremos decir que los dominios -- que estamos
      interesados en modelar -- tienen, como propiedad importante, que
      poseen entidades creadas por la humanidad.
\afslut

\pos{%%%%%%%%%%%%%%%%%%%%%%%%%%%%%%%%%%%%%%%%%%%%%%%%%%%%%
\begynd
\pind Este manual presenta un \sfsl{método}\pconindex{method}, sus
      \sfsl{principios}\pconindex{principle}, 
      \sfsl{procedimientos}\pconindex{procedure}, 
      \sfsl{técnicas}\pconindex{technique} 
      y 
      \sfsl{herramientas}\pconindex{tool}, 
      para \sfsl{analizar} \&\footnote{Usamos aquí el ampersand, `\&',
      como en $A\&B$, para enfatizar que estamos tratando $A$ y $B$ como
      un solo concepto.} \sfsl{describir} dominios.
\afslut
}{}%%%%%%%%%%%%%%%%%%%%%%%%%%%%%%%%%%%%%%%%%%%%%%%%%%%%%%%%

\nbbbbb{Fenómenos y Entidades}\HHHH

\begynd
\pind \primerdefn{Fenómenos}{ {\pdefindex{phenomenon} Por un
\sfsl{fenómeno} entenderemos un hecho que se observa que existe o sucede\dbsquare}}

\noindent
\begynd
\pind Algunos fenómenos son racionalmente describibles -- en gran medida o en su totalidad -- otros no.
\afslut

\noindent
\begynd
\pind Algunos fenómenos son racionalmente describibles -- en gran medida o en su totalidad -- otros no.
\afslut

\noindent
\pind \primerdefn{Entidades}{ Por una entidad \pdefindex{entity} \ysfchg{}
      entenderemos un fenómeno más o menos racionalmente describible\dbsquare}

\noindent
\pind \newestxample{Fenómenos y Entidades}{Algunos, pero no necesariamente
  todos los aspectos de un río pueden ser racionalmente descritos, por lo tanto, pueden ser considerados
  entidades. De manera similar, muchos aspectos de una
  red vial pueden ser racionalmente descritos, por lo tanto, serán considerados entidades\dbsquare}
\afslut

\nbbbbb{Endurantes y Perdurantes}

\bbbb{Endurantes}

\begynd
%%%%%%%
\pind \bookdefn{Endurantes}{ Los endurantes son \pdefindex{endurant}\ 
      aquellas cantidades en dominios que podemos
      observar (ver y tocar), en \sfsl{espacio}, como entidades ``completas'' en cualquier punto en \sfsl{tiempo} -- entidades ``materiales'' que
      persisten\ysfchg{}, perduran\ysfchg{}\dbsquare}%%%%
      
\noindent
\monoexample{Endurantes}{ Ejemplos de endurantes son: un segmento de calle [enlace], una intersección de calles [conexión], un
      automóvil\dbsquare}
      
\noindent
\pind Los endurantes de dominio, cuando eventualmente se modelan en software, típicamente
      se convierten en datos. Por lo tanto, el análisis cuidadoso de los endurantes de dominio es un
      requisito previo para la concepción y análisis cuidadosos de
      estructuras de datos para software, incluidas las bases de datos.
\afslut
%%%%%%%
    
\nbbbb{Perdurantes}

\begynd
\pind \newestpdefn{Perdurantes}{ Los perdurantes son \pdefindex{perdurant}\   
      aquellas cantidades de dominios de las cuales solo existe un
      fragmento, en \sfsl{espacio}, si los miramos o tocamos en
      cualquier instantánea en
      \sfsl{tiempo}\dbsquare}%%%%
      
\noindent
\newestxample{Perdurante}{ Un automóvil en movimiento es un ejemplo de un perdurante\dbsquare}
        
\noindent
\pind Los perdurantes de dominio, cuando eventualmente se modelan en software, típicamente
      se convierten en procesos. Por lo tanto, el análisis cuidadoso de los perdurantes de dominio es un
      requisito previo para la concepción y análisis cuidadosos de
      funciones (procedimientos).
\afslut
%%%%%%%
    
\nbbbbb{Cualidades Externas e Internas de Endurantes }

\bbbb{Cualidades Externas}

\newestpdefn{Cualidades Externas}{ \pdefindex{external quality}\
     Las cualidades externas de los endurantes de un dominio manifiesto 
\begynd
\pind son, en un sentido simplificador, aquellas que podemos
\begynd
\pind \ysf{ver,}
\pind tocar y
\pind tener extensión espacial.
\afslut
\pind Ellas, por así decirlo, toman forma.
\afslut}

\mnewfoil

\noindent
\newestxample{Cualidades Externas}{ Un ejemplo de cualidades externas de
  un dominio\ysfchg{} es:
\begynd
\pind el Cartesiano\footnote{\LLLL Cartesiano
      según el filósofo, matemático y científico francés Ren{\'e}
      Descartes (1596--1650)} 
\begynd
\pind de conjuntos de intersecciones de calles atómicas sólidas,
\pind de conjuntos de segmentos de calles atómicos sólidos y 
\pind de conjuntos de automóviles sólidos
\afslut de un sistema de transporte por carretera
\pind  donde el
\pos{}{\begin{multicols}{4}}   
\begynd
\pind Cartesiano, \pind conjuntos, 
      \pind atómico, y \pind sólido
\afslut
\pos{}{\end{multicols}} reflejan cualidades externas\dbsquare
\afslut}
%%%%%%%
\nbbb{Endurantes discretos o sólidos}

\newestpdefn{Endurantes discretos o sólidos}{ \pdefindex{discrete!endurant}\pdefindex{endurant!discrete}%%%% 
      \pdefindex{solid!endurant}\pdefindex{endurant!solid} Por un endurante
      \sfsl{sólido} \nyl [o \sfsl{discreto}]  entenderemos un endurante
\begynd
\pind que es separado, individual o distinto en forma o concepto,
\pind o, reformulando: tiene `cuerpo' [o magnitud]
      de tres dimensiones: longitud, anchura y profundidad
      \cite[vol.\,II, pág.\,2046]{OED}\dbsquare
\afslut}

\noindent
%\mnewfoil
\newestxample{Endurantes sólidos}{ Ejemplos de endurantes sólidos son
\begynd
\pind los
\pos{}{\begin{multicols}{4}}
\begynd
\pind pozos, 
\pind tuberías,
\pind válvulas,
\pind bombas, 
\pind bifurcaciones, 
\pind uniones y 
\pind desagües
\afslut
\pos{}{\end{multicols}} de tuberías\ysfchg{}. 
\pind{} [Estas unidades pueden, sin embargo, y generalmente lo harán, contener fluidos, por ejemplo, petróleo, gas
  o agua]\dbsquare
\afslut}
\mnewfoil

\noindent
\begynd
\pind Principalmente estaremos analizando y describiendo endurantes sólidos.

\pind Como veremos, en la siguiente \pos{capítulo}{sección},
\begynd
\pind analizamos y describimos endurantes de sólidos como
\pind  ya sea partes 
\pind o especies vivas: animales y humanos.
\afslut
\pind Principalmente nos ocuparemos de las partes.
\begynd
\pind Es decir, solo, como: ``de paso'', 
\pind por \ysfchg{el} motivo de completitud,
\pind mencionaremos especies vivas\,!
\afslut
\afslut
%%%%%%%
    
\nbbb{Fluidos}

\begynd
\pind \newestpdefn{Endurantes fluidos}{ \pdefindex{fluid!endurant}\pdefindex{endurant!fluid}\ 
\begynd
\pind Por un \sfsl{endurante fluido} entenderemos un endurante \nyl que es 
\begynd
\pind prolongado, sin interrupción, \nyl en una serie o patrón ininterrumpido;
\pind o, reformulando: una sustancia (líquido,
  gas o plasma) que tiene la propiedad de fluir, consistiendo en
  partículas que se mueven entre sí \cite[vol.\,I, pág.\,774]{OED} \dbsquare
\afslut
\afslut}
%%%%%%%
\afslut

\noindent
\newestxample{Endurantes fluidos}{ Ejemplos de endurantes fluidos son: % Sh
  \pos{}{\begin{multicols}{3}}
\begynd
\pind agua, \pind petróleo, \pind gas,
  \pind aire comprimido, \pind humo\dbsquare
  \afslut
\pos{}{\end{multicols}}}

\mnewfoil

\noindent
\begynd
\pind Los fluidos son de otro modo
\begynd
\pind líquidos, o
\pind gaseosos, o
\pind plasmáticos, o
\pind granulares\footnote{\label{fn.granular} Esta es una decisión puramente pragmática. ``Por supuesto'' la arena, la grava, el suelo, etc., no son
  fluidos, pero para nuestros propósitos de modelado es conveniente
  ``compartimentarlos'' como fluidos\,!}, o como
\pind productos vegetales, por ejemplo, caña de azúcar picada, trillada o procesada de alguna otra manera\footnote{Ver nota al pie \vref{fn.granular}.},
\pind etcétera.
\afslut
\pind Los endurantes fluidos serán analizados y descritos \nyl en relación con
      los endurantes de sólidos, es decir, sus ``contenedores''.
\afslut
\nbbb{Partes}
  
\begynd
\pind \newestpdefn{Partes}{ \pdefindex{parts}\
\begynd
\pind Los sólidos no vivos son lo que llamaremos partes\dbsquare
\afslut}

\noindent
\pind Las partes son los ``caballos de batalla'' de los dominios hechos por el hombre.  % Unsure how to translate workhorse idiom
\pind Es decir, nos ocuparemos principalmente \nyl del análisis y
      descripción de endurantes en partes.
\afslut
%%%%%%%

\noindent
\newestxample{Partes}{ El ejemplo anterior de sólidos
  también era un ejemplo de partes\dbsquare}

\noindent
\begynd
\pind Distinguimos entre partes atómicas y compuestas.
\afslut

\nbb{Partes atómicas}
  
\bookdefn{Parte atómica, I}{ \label{pd-atomic-parts}
\begynd
\pind Por una \sfsl{parte atómica} entenderemos una parte
\begynd
\pind que el analista de dominio considera indivisible
\pind en el sentido de no ser \ysfchg{divisible} de manera significativa,
\pind para los propósitos del dominio en consideración,
\pind es decir, que no consiste significativamente en subpartes\eod
\afslut
\afslut
}

\monoexample{Partes atómicas}{ Ejemplos de partes atómicas son:
\begynd
\pind un centro, es decir, una intersección de calles;
\pind un enlace, es decir, el tramo de carretera entre dos centros vecinos; y
\pind un automóvil \dbsquare
\afslut}

\nbb{Partes compuestas}
  
\begynd
\pind Nosotros, pragmáticamente, distinguimos entre
\begynd
\pind partes orientadas al producto cartesiano, y 
\pind partes orientadas a conjuntos.
\afslut
\pind Si están orientadas al producto cartesiano, consisten en dos o más
      endurantes (sólidos o fluidos) claramente diferenciados \ysfchg{. }
\pind Si están orientadas a conjuntos, consisten en un número indefinido de \nyl cero,
      una o más partes.
\afslut

\bookdefn{Parte compuesta, I}{ %
\begynd
\pind \pdindextermii{Partes}{compuestas} son aquellas que son
\begynd
\pind ya sea \ysf{cartesiano-}
\pind o son partes orientadas a conjuntos
\afslut
\pind  \eod
\afslut
}

\monoexample{Partes compuestas}{ Un ejemplo de partes compuestas es:
\begynd
\pind \ysfchg{(i) } una red vial que consiste en un conjunto de centros,
  es decir, intersecciones de calles o ``fin de calles'',  y
\pind  \ysfchg{(ii) } un conjunto de
  enlaces, es decir, segmentos de calles (sin centros contenidos),
\afslut
 es un compuesto cartesiano\ysfchg{. }
\begynd
\pind  \ysfchg{(iii) } \ysfchg{Cada conjunto } de centros y \ysfchg{cada conjunto } de enlaces
\afslut son compuestos de conjuntos\dbsquare}

\nbbbb{Un apunte: una ontología superior}\label{An Upper Ontology}

\begynd
\pind Hemos sido razonablemente cuidadosos
\begynd  
\pind en solo introducir y establecer
      definiciones informales 
\pind de fenómenos y algunas clases de estos.
\afslut
\pind En el próximo capítulo, en cierto sentido, ``repetiremos'' la cobertura de
      estos fenómenos.
\begynd
\pind Pero \ysf{entonces} de una manera más analítica.
\pind La figura\,\vref{onto.fig0} tiene la intención de indicar esto.
\afslut
\afslut

\mnewfoil

\noindent
\hDBfigure{onto}{\pos{10.5}{11.7}cm}{{Una} ontología superior}{onto.fig0}

\mnewfoil

\noindent
\begynd
\pind Hasta ahora solo hemos tocado 
\begynd
\pind el cuadro de líneas discontinuas etiquetado como `cualidades externas' 
\pind del cuadro de líneas discontinuas etiquetado como `Endurantes' de la fig.\,\ref{onto.fig0}.
\afslut
\pind En el capítulo \ref{primer-extq.1} trataremos las cualidades externas
      con más profundidad ---
\begynd
\pind de manera más sistemática: analíticamente y descriptivamente.
\afslut
\afslut


\nbbbb{Cualidades Internas}
     
\newestpdefn{Cualidades internas}{ Las cualidades internas son \pdefindex{internal quality}
\begynd
\pind aquellas propiedades [de los endurantes]
\pind que no ocupan \sfsl{espacio}
\pind pero que pueden ser medidas o mencionadas\dbsquare
\afslut}%%%%
      
\newestxample{Cualidades internas}{ Ejemplos de cualidades internas son %%%%
\begynd
\pind la identidad única de una parte, 
\pind la relación de una parte con otras
      partes, y 
\pind los atributos de los endurantes como temperatura, longitud, color\dbsquare
\afslut} 

  
\nbbb{Identidad única}
    
\begynd
\pind \newestpdefn{Identidad única}{ Una identidad única es \pdefindexii{unique}{identity}\  
\begynd
\pind una propiedad inmaterial
\pind que distingue a dos sólidos \sfsl{espacialmente}
      distintos\dbsquare
\afslut}
\afslut

\newestxample{Identidades únicas}{ Cada 
\begynd
\pind centro en una red de carreteras es identificado de manera única,
\pind así como cada enlace 
\pind y cada automóvil\dbsquare
\afslut}
%%%%%%%

\nbbb{Mereología}
\begynd 
\pind \newestpdefn{Mereología, I}{ La mereología es \pdefindex{mereology}\  
      una teoría de las relaciones de parte-idad [de endurantes]: % doubts about part-hood
\begynd 
\pind de las relaciones de una parte [endurante] con un todo
\pind y de las
      relaciones de las partes [endurantes] con otras partes [endurantes] dentro de ese
      todo\dbsquare
\afslut}%%%%
\afslut
      
\newestxample{Mereología}{ Ejemplos de mereologías son %%%%
\begynd 
\pind que un enlace está topológicamente \sl conectado \rm a \nyl
      exactamente dos centros específicos,
\pind que un centro está \sl conectado \rm a \nyl cero, uno o
      más enlaces específicos,
\pind y que enlaces y centros están \sl abiertos \rm a \nyl
      subconjuntos específicos de automóviles\dbsquare
\afslut}
%%%%%%%

\nbbb{Atributos}
\newestpdefn{Atributos}{ Los atributos son \pdefindex{attribute}\
      propiedades de los endurantes
\begynd 
\pind que no son \sfsl{espacialmente} observables, 
\pind pero que pueden ser
      físicamente
\pind (electrónicamente, químicamente u otros modos)
\pind medidos o pueden
      ser mencionados objetivamente\dbsquare
\afslut}%%%%
      
\newestxample{Atributos}{ Ejemplos de atributos son:
\begynd 
\pind los enlaces \ysfchg{ } tienen longitudes, y,
\pind que en cualquier momento,
\pind cero, uno o más automóviles están ocupando \ysfchg{cada
enlace}\ysf{\footnote{Oh sí, por supuesto, es espacialmente observable
    que un enlace tiene una longitud, pero la medición, digamos \sfsl{123
      metros} no lo es; y el número de automóviles en el enlace tampoco es
    espacialmente observable.}}\dbsquare
\afslut}
%%%%%%%
\label{pre-primer.Prompts}
\nbbbbb{Indicaciones}\label{primer.Prompts} % still unsure about translating prompts


\bbbb{Analysis Prompts}

\begynd 
\pind \newestpdefn{Analysis Prompt}{ An analysis prompt is \pdefindex{analysis!prompt}\
\begynd 
\pind a predicate or a function 
\pind that may be posed by humans to \ysf{segments}
      of a domain. 
\pind Observing the domain the analyser  may then
\pind act upon the combination of the particular prompt  
\pind (whether a
      predicate or a function, \nyl and then what particular one of these
      it is) 
\pind  thus ``applying'' it to a domain phenomen\ysfchg{on}, 
\pind  and yielding, in the
      minds of the humans, 
\pind  either a truth value or some other form of value\dbsquare
\afslut}
\afslut
      
\nbbb{Analysis Predicate}
\begynd
\pind \newestpdefn{Analysis predicates}{ \pdefindex{analysis!predicate}\ 
\begynd
\pind An analysis predicate is an analysis prompt
\pind which yields a truth value\dbsquare
\afslut}
\afslut
      
\newestxample{Analysis Predicates}{ General examples of analysis
  predicates are:
\begynd
\pind \sfsl{``can an observable phenomen\ysfchg{on }be rationally described''}, \nyl i.e., an entity, 
\pind \sfsl{``is an entity a solid or a fluid''},
\pind \sfsl{``is a solid endurant a part or a living species''}\dbsquare
\afslut}
%%%%%%%
    
\nbbb{Analysis Function}

\newestpdefn{Analysis function}{ An analysis function is \pdefindex{analysis!function}%%%
\begynd
\pind an analysis prompt which yields some \texttt{\rsltxt}\dbsquare
\afslut} 
      
\newestxample{Analysis Functions}{ Two examples of analysis functions are:
\begynd
\pind one yields the endurants of a Cartesian part \nyl and their respective sort names,
\pind another yields the set of \ysfchg{} parts of a part set \nyl and their common type\dbsquare
\afslut}
%%%%%%%
  
\nbbbb{Description Prompt}

\begynd  
\pind \newestpdefn{Description Prompt}{ A description prompt is \pdefindex{description!prompt}
\begynd  
\pind a function that may be posed by humans  
\pind who may then act  upon it:   
\begynd  
\pind \ysf{{[the human]}} ``applying'' it to a domain phenomen\ysfchg{on}, and 
\pind \ysf{{[the human]}} ``yielding'', i.e., writing down, \ysf{a}
narrative and formal {\rsltxt}s \nyl describing 
      what is being observed \ysf{[by that human]}\dbsquare
\afslut
\afslut}
\afslut

\mnewfoil

\newestxample{Description Prompts}{ Description prompts result
\begynd
\pind in {\rsltxt}s describing for example a 
\begynd
\pind (i) \ysfchg{a } Cartesian endurant, or
\pind (ii) its unique identifier, \ysf{or}
\pind (iii) its mereology, or
\pind (iv) its attributes, \ysf{or}
\pind(iv) other\dbsquare
\afslut
\afslut}
%%%%%%%

\nbbbbb{Perdurant Concepts}\label{Perdurant Concepts}

\bbbb{``Morphing'' Parts into Behaviours}

\begynd
\pind As already indicated we shall
\begynd
\pind transcendentally deduce
\pind (perdurant) behaviours from
\pind those (endurant) parts
\begynd
\pind which we, as domain analysers cum describers,
\pind have endowed with all three kinds of internal qualities:
\pind unique identifiers, mereologies and attributes.
\afslut
\afslut
\pind Chapter\,\ref{chap6.tex.1}, will show how.
\afslut

\nbbbb{State}
\newestpdefn{State, I}{ \pdefindex{state}\ 
\begynd
\pind  A state is any set of
      the parts of a domain\dbsquare
\afslut}
%%%%%%%

\monoexample{A Road System State}{ The domain analyser cum describer
  may, \nyl \ysf{} decide that a road system state consists of
\begynd
\pind the road net aggregate (of hubs and links)\footnote{\LLLL The road net
  aggregate, in its perdurant form, may ``model'' the \sfsl{Department
  of Roads} of some country, province, or town.},
\pind all the hubs, and all the links, and
\pind the automobile aggregate (of all the automobiles)\footnote{\LLLL
  The  automobile \ysfchg{}
  aggregate, in its perdurant form, may ``model'' the \sfsl{Department
  of Vehicles} of some country, province, or town.}, and
\pind all the individual automobiles\dbsquare
\afslut}

\nbbbb{Actors}
\newestpdefn{Actors}{ \pdefindex{actor}\ 
\begynd
\pind An actor is anything
      that can initiate an action, an event or a behaviour\dbsquare
\afslut}
%%%%%%%
 
\bbb{Action}
   
\newestpdefn{Actions}{ \pdefindex{action}\
\begynd
\pind  An action is a
      function that can purposefully change\ysfchg{ } a state\dbsquare
\afslut}
\monoexample{Road Net Actions}{ These are some road net actions:
\begynd
\pind The insertion of a new or removal of an existing hub; or
\pind the insertion of a new, or removal of an existing link;
\afslut}

%%%%%%%

\nbbb{Event}

\newestpdefn{Events}{ \pdefindex{event}\
\begynd
\pind  An event is a function
      that surreptitiously changes a state\dbsquare
\afslut}
\monoexample{Road Net Events}{ These are some road net events:
\begynd
\pind The blocking of a link due to a mud slide;
\pind the failing of a hub traffic signal due to power outage;
\pind the blocking of a link due to an automobile accident.
\afslut}

%%%%%%%
    
\nbbb{Behaviour}

\newestpdefn{Behaviours}{ \pdefindex{behaviour}\
\begynd
\pind A behaviour is a set 
\pind of sequences of 
\pind actions, events and behaviours\dbsquare
\afslut}
%%%%%%%
\mnewfoil\LLll
\monoexample{Road Net Traffic}{ %
\begynd
\pind Road net traffic can be seen as a behaviour
\begynd
\pind \ysfchg{(i) } of all the behaviours of automobiles,
\begynd
\pind where each automobile behaviour is seen as sequence of \nyl start, stop,
      turn right, turn left, etc., actions;
\afslut
\pind \ysfchg{(ii) } of all the behaviours of links
\begynd
\pind where each link behaviour is seen \nyl as a set of sequences
(i.e., behaviours) of ``following'' the 
\begynd
\pind link entering, link leaving, and movement
\afslut of automobiles on the link;
\afslut
\pind \ysfchg{(iii) } of all the behaviours of hubs (etc.);
\pind \ysfchg{(iv) } of the behaviour of the aggregate of roads, \nyl
      viz.\ \sfsl{The Department of Roads}, and 
\pind \ysfchg{(v) } of the behaviour of the aggregate of automobiles, \nyl viz,\
      \sfsl{The Department of Vehicles}.
\afslut
\afslut}\HHHH


\nbbbb{Channel}

\begynd
\pind \newestpdefn{Channel}{ \pdefindex{channel}\
\begynd
\pind A channel is anything 
\pind that allows synchronisation and communication
\pind of values
\pind between two behaviours\dbsquare
\afslut}
\afslut

\mnewfoil

\noindent
\begynd
\pind We shall use Tony Hoare's \texttt{CSP} concept \citecsp\
\begynd
\pind to express synchronisation and communication \nyl of values
      between behaviours \brcolor{\textsl{i}} and \brcolor{\textsl{j}}.
\pind Hence the behaviour \brcolor{\textsl{i}} statement \nyl
      \bbcolor{\textsl{ch[\brcolor{j}]\,\,!\,\,value}} \nyl
      \ysfchg{states } that
      behaviour \brcolor{\textsl{i}} \nyl  offers, ``outputs'':
      \bbcolor{!}, \bbcolor{\textsl{value}} \nyl  to \ysfchg{the
      behaviour } indicated by \bbcolor{\textsl{\brcolor{j}}}. 
\pind And behaviour \brcolor{\textsf{\ysfchgii{j}}} expresses \nyl  \bbcolor{\textsl{ch[\brcolor{\ysfchgii{i}}]\,\,?}} \nyl 
      that  it is willing to accept \nyl  ``input from \&
      synchronise with'' behaviour \brcolor{\textsl{i}}, \bbcolor{?}, \nyl any \bbcolor{\textsl{value}}.
\afslut
\afslut
%%%%%%%

\nbbbbb{Domain Analysis \& Description}

\bbbb{Domain Analysis}

\newestpdefn{Domain Analysis}{ Domain analysis \pdefindex{domain!analysis}\pdefindex{analysis!domain}\
\begynd
\pind is the act of studying a domain
\pind as well as the result of that study
\pind in the form of \bmcolor{informal} statements\dbsquare
\afslut}

\bbbb{Domain Description}

\newestpdefn{Domain Description}{ Domain description \pdefindex{domain!description}\pdefindex{description!domain}\ 
\begynd
\pind is the act of describing a domain 
\pind as well as the result of that act
\pind in the
      form of  \bmcolor{narratives} and \bmcolor{formal \rsltxt}\dbsquare
\afslut}

\nbbbbb{Closing}

\begynd
\pind This \pos{chapter}{lecture} has introduced \nyl the main concepts of domains such as
      we shall treat \nyl  (analyse and describe) domains.\footnote{\LLLL We
      have omitted treatment of \sfsl{living species: plants} and
      \sfsl{animals} -- the latter including \sfsl{humans}. They will
      be treated in the next chapter\,!}
\pind The next \pos{three chapters}{lectures} shall now systematically treat \nyl  the
      analysis and description of domains.
\begynd
\pind That treatment takes concept by concept and 
\begynd
\pind provides proper definitions and
\pind introduces appropriate analysis and description prompts;
\pind one-by-one, in an almost pedantic, 
\pind hence perhaps ``slow'' progression\,!
\afslut
\afslut
\mnewfoil
\pind The \pos{reader}{student} may be excused
\begynd
\pind if they, now-and-then, lose sight of ``their way''.
\afslut
\pind Hence the present chapter.
\begynd
\pind To show ``the way'':
\pind that, for example, \nyl when we treat external endurant qualities,
\pind there are still the internal endurant qualities,
\pind and that the whole thing leads of to perdurants:
\begynd
\pind actors,
\pind actions,
\pind events and
\pind behaviours.
\afslut
\afslut
\afslut

\label{chap2.tex.Preview.n}
%%  LocalWords:  artefactual analysing Endurants endurant Perdurants
%%  LocalWords:  perdurant endurants colour Mereology mereology RSL
%%  LocalWords:  behaviour Behaviours behaviours harbours modelled de
  %%  LocalWords:  synchronisation perdurants philosophee Ren analyse
%%  LocalWords:  plasmatic compartmentalise fn et cetera analysed CSP
%%  LocalWords:  descriptionally analyser Hoare's synchronise ly wo
%%  LocalWords:  analysers mereologies modelling conveyage phenomen
%%  LocalWords:  neighbouring
