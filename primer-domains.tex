

%%%%%%%%%%%%%%%%%%%%%%%%%%%%%%%%%%%%%%%%%%%%%%%%%%%%%%%%%%%%%%%%%%%%%%%
%%%% Released for translation 20 April 2023  %%%%%%%%%%%%%%%%%%%%%%%%%%
%%%%%%%%%%%%%%%%%%%%%%%%%%%%%%%%%%%%%%%%%%%%%%%%%%%%%%%%%%%%%%%%%%%%%%%

\nbbbbbb{\HHHH Domains}\label{chapter:Domains}\label{chap2.tex.Preview}
\pos{\minitoc}{}
\pos{\label{lect1:label1}}{}

\pos{
\noindent
\begynd
\pind This \pos{chapter}{section} is informal.
\pind Here we introduce You to  important  concepts of domains.
\pind Subsequent  \pos{chapter}{section}s will be more technical.
\begynd
\pind They will define most of the domain concepts of this
      \pos{chapter}{section} properly.
\afslut
\afslut
}{}

\bbbbb{Domain Definition}\HHHH

\pos{We repeat the definition of the concept of domains as first given on
Page\,\pageref{1stdd}.}{}

\bookdefn{Domain}{ \domaindefinition}

\pos{%
\noindent
\ysf{We exclude from our treatment of domains issues of biological and
psychological matters.}}{}

\mnewfoil

\monoexample{Domains}{A few, more-or-less self-explanatory examples:
\begin{itemize}
\item \bbcolor{Rivers --} with their natural sources, deltas, tributaries,
  waterfalls, etc., and their man-made dams, harbours, locks, etc. --
  and their conveyage of materials (ships etc.) \cite{BjornerRiversCanals2021};
\item \bbcolor{Road nets --} with street segments and intersections, traffic
  lights and automobiles -- and the flow of these;
\item \bbcolor{Pipelines --} with their wells, pipes, valves, pumps, forks,
  joins and wells and the flow of fluids \cite{2013pipe}; and
\item \bbcolor{Container terminals --} with their container vessels, containers,
  cranes, trucks, etc. -- and the movement of all of these \cite{BjornerContainer2018}\dbsquare
\end{itemize}}

\noindent
\mnewfoil%\LLLL
\begynd
\pind The definition relies on the understanding of the terms \nyl
\pos{}{\vspace*{-10mm}\begin{multicols}{2}}
      \sfsl{`rationally describable'}, \nyl \sfsl{`discrete dynamics'}, \nyl \sfsl{`human
      assisted'}, \nyl \sfsl{`solid'} and \nyl \sfsl{`fluid'}.
\pos{}{\end{multicols}}
 The last two will be explained later.
\pind By \bbcolor{\sfsl{rationally describable}} we mean that what is described \nyl
      can be understood, including reasoned about, in a rational, \nyl  that is,
      logical manner -- \ysf{in other words  \bbcolor{\sfsl{logically tractable}}.}
\pind By \bbcolor{\sfsl{discrete dynamics}} we imply that we shall basically
      rule out such domain phenomena which have
      properties \ysfchg{that } are continuous with
      respect to their time-wise\pos{, i.e., dynamic,}{} behaviour.
\pind By \bbcolor{\sfsl{human-assisted}} we mean that the domains -- that we are
      interested in modelling -- have, as an important property, that
      they possess man-made entities.
\afslut

\pos{%%%%%%%%%%%%%%%%%%%%%%%%%%%%%%%%%%%%%%%%%%%%%%%%%%%%%
\begynd
\pind This primer presents a \sfsl{method}\pconindex{method}, its
      \sfsl{principles}\pconindex{principle}, 
      \sfsl{procedures}\pconindex{procedure}, 
      \sfsl{techniques}\pconindex{technique} 
      and 
      \sfsl{tools}\pconindex{tool}, 
      for \sfsl{analysing} \&\footnote{We use here the ampersand, `\&',
      as in $A\&B$, to emphasize that we are treating $A$ and $B$ as
      one concept.} \sfsl{describing} domains.
\afslut
}{}%%%%%%%%%%%%%%%%%%%%%%%%%%%%%%%%%%%%%%%%%%%%%%%%%%%%%%%%

\nbbbbb{Phenomena and Entities}\HHHH

\begynd
\pind \primerdefn{Phenomena}{ {\pdefindex{phenomenon} By a
\sfsl{phenomenon} we shall understand a fact that is
     observed to exist or happen\dbsquare}}

\noindent
\begynd
\pind Some phenomena are rationally describable -- to a large or full
degree -- others are not.
\afslut

\noindent
\pind \primerdefn{Entities}{ By an entity \pdefindex{entity} \ysfchg{}
      we shall understand a more-or-less rationally describable phenomenon\dbsquare}

\noindent
\pind \newestxample{Phenomena and Entities}{Some, but not necessarily
  all aspects of a river can
  be rationally described, hence can be still be considered
  entities. Similarly, many aspects of a
  road net can be rationally described, hence will be considered entities\dbsquare}
\afslut

\nbbbbb{Endurants and Perdurants}

\bbbb{Endurants}

\begynd
%%%%%%%
\pind \bookdefn{Endurants}{ Endurants are \pdefindex{endurant}\ 
      those quantities of domains that we can
      observe (see and touch), in \sfsl{space}, as ``complete'' entities at no matter
      which point in \sfsl{time} -- ``material'' entities that
      persist\ysfchg{}, endure\ysfchg{}\dbsquare}%%%%
      
\noindent
\monoexample{Endurants}{ Examples of endurants are: a street segment [link], a street intersection [hub], an
      automobile\dbsquare}
      
\noindent
\pind Domain endurants, when eventually modelled in software, typically
      become data. Hence the careful analysis of domain endurants is a
      prerequisite for subsequent careful conception and analyses of
      data structures for software, including data bases.
\afslut
%%%%%%%
    
\nbbbb{Perdurants}

\begynd
\pind \newestpdefn{Perdurants}{ Perdurants are \pdefindex{perdurant}\   
      those quantities of domains for which only a
      fragment exists, in \sfsl{space}, if we look at or touch them at
      any given snapshot in
      \sfsl{time}\dbsquare}%%%%
      
\noindent
\newestxample{Perdurant}{ A moving automobile is an example of a perdurant\dbsquare}
        
\noindent
\pind Domain perdurants, when eventually modelled in software, typically
      become processes. Hence the careful analysis of domain perdurants is a
      prerequisite for subsequent careful conception and analyses of
      functions (procedures).
\afslut
%%%%%%%
    
\nbbbbb{External and Internal Endurant Qualities}

\bbbb{External Qualities}

\newestpdefn{External Qualities}{ \pdefindex{external quality}\
     External qualities of endurants of a manifest domain 
\begynd
\pind are, in a simplifying sense, those we can
\begynd
\pind \ysf{see,}
\pind touch and
\pind have spatial extent.
\afslut
\pind They, so to speak, take form.
\afslut}

\mnewfoil

\noindent
\newestxample{External Qualities}{ An example of external qualities of
  a domain\ysfchg{ } is:
\begynd
\pind the Cartesian\footnote{\LLLL Cartesian
      after the French philosopher, mathematician, scientist Ren{\'e}
      Descartes (1596--1650)} 
\begynd
\pind of sets of solid atomic street intersections, and
\pind of sets  of solid atomic street segments, and 
\pind of sets  of solid automobiles
\afslut of a road transport system
\pind  where the
\pos{}{\begin{multicols}{4}}   
\begynd
\pind Cartesian, \pind sets, 
      \pind atomic, and \pind solid
\afslut
\pos{}{\end{multicols}} reflect external qualities\dbsquare
\afslut}
%%%%%%%
    
\nbbb{Discrete or Solid Endurants}

\newestpdefn{Discrete or Solid
      Endurants}{ \pdefindex{discrete!endurant}\pdefindex{endurant!discrete}%%%% 
      \pdefindex{solid!endurant}\pdefindex{endurant!solid} By a
      \sfsl{solid} \nyl [or \sfsl{discrete}] endurant we shall understand an endurant
\begynd
\pind which is  separate,  individual or distinct in form or concept,
\pind or, rephrasing: have `body'  [or magnitude]
      of three-dimensions: length, breadth and depth
      \cite[Vol.\,II, pg.\,2046]{OED}\dbsquare
\afslut}

\noindent
%\mnewfoil
\newestxample{Solid Endurants}{ Examples of sol\ysfchg{i}d endurants are
\begynd
\pind the
\pos{}{\begin{multicols}{4}}
\begynd
\pind wells, 
\pind pipes,
\pind valves,
\pind pumps, 
\pind forks, 
\pind joins and 
\pind sinks
\afslut
\pos{}{\end{multicols}} of pipelines\ysfchg{}. 
\pind{} [These
  units may, however, and usually will, contain fluids, e.g., oil, gas
  or water]\dbsquare
\afslut}
\mnewfoil

\noindent
\begynd
\pind We shall mostly be analysing and describing solid endurants.

\pind As we shall see, in the next \pos{chapter}{section},
\begynd
\pind we analyse and describe solid endurants as
\pind  either parts 
\pind or living species: animals and humans.
\afslut
\pind We shall mostly be concerned with parts.
\begynd
\pind That is, we shall just, as: ``in passing'', 
\pind for \ysfchg{the } sake of completeness,
\pind mention living species\,!
\afslut
\afslut
%%%%%%%
    
\nbbb{Fluids}

\begynd
\pind \newestpdefn{Fluid
  Endurants}{ \pdefindex{fluid!endurant}\pdefindex{endurant!fluid}\ 
\begynd
\pind By a \sfsl{fluid endurant} we shall understand  an endurant \nyl which is 
\begynd
\pind prolonged, without interruption, \nyl in an unbroken series or pattern;
\pind or, rephrasing:  a substance (liquid,
  gas or plasma) having the property of flowing, consisting of
  particles that move  among themselves \cite[Vol.\,I, pg.\,774]{OED} \dbsquare
\afslut
\afslut}
%%%%%%%
\afslut

\noindent
\newestxample{Fluid Endurants}{ Examples of fluid endurants are:
  \pos{}{\begin{multicols}{3}}
\begynd
\pind water, \pind oil, \pind gas,
  \pind compressed air, \pind smoke\dbsquare
  \afslut
\pos{}{\end{multicols}}}


\mnewfoil

\noindent
\begynd
\pind Fluids are otherwise
\begynd
\pind liquid, or
\pind gaseous, or
\pind plasmatic, or
\pind granular\footnote{\label{fn.granular} This is a purely
  pragmatic decision. ``Of course'' sand, gravel, soil, etc., are not
  fluids, but for our modelling purposes it is convenient to
  ``compartmentalise'' them as fluids\,!}, or
\pind plant products, i.e., chopped sugar cane, threshed, or
otherwise\footnote{See footnote \vref{fn.granular}.},
\pind et cetera.
\afslut
\pind Fluid endurants will be analysed and described \nyl in relation to
      solid endurants, viz.\ their ``containers''.
\afslut

\nbbb{Parts}
  
\begynd
\pind \newestpdefn{Parts}{ \pdefindex{parts}\
\begynd
\pind The non-living species solids are what we shall call parts\dbsquare
\afslut}

\noindent
\pind Parts are the ``work-horses'' of man-made domains.
\pind That is, we shall mostly be concerned  \nyl with the analysis and
      description of endurants into parts.
\afslut
%%%%%%%

\noindent
\newestxample{Parts}{ The previous example of solids
  was also an example of parts\dbsquare}

\noindent
\begynd
\pind We distinguish between atomic and compound parts.
\afslut

\nbb{Atomic Parts}
  
\bookdefn{Atomic Part, I}{ \label{pd-atomic-parts}
\begynd
\pind By an \sfsl{atomic part} we shall understand a part
\begynd
\pind which the domain analyser considers to be indivisible
\pind in the sense of not meaningfully \ysfchg{divisible},
\pind for the purposes of the domain under consideration,
\pind that is, to not meaningfully consist of sub-parts\eod
\afslut
\afslut
}

\monoexample{Atomic Parts}{ Examples of atomic parts are:
\begynd
\pind a hub, i.e., a street intersection;
\pind a link, i.e., the stretch of road between two neighbouring hubs; and
\pind an automobile \dbsquare
\afslut}

\nbb{Compound Parts}
  
\begynd
\pind We, pragmatically, distinguish between
\begynd
\pind Cartesian-product-, and 
\pind set-
\afslut oriented parts.
\pind If Cartesian-oriented, to consist of two or more
      distinctly sort-named endurants (solids or fluids)\ysfchg{. }
\pind If set-oriented, to consist of an indefinite number of  \nyl zero,
      one or more parts.
\afslut

\bookdefn{Compound Part, I}{ %
\begynd
\pind \pdindextermii{Compound}{part}s are those which are
\begynd
\pind either \ysf{Cartesian-}
\pind or are set-
\afslut
\pind oriented parts  \eod
\afslut
}

\monoexample{Compound Parts}{ An example of \ysfchg{} compound parts
  is: 
\begynd
\pind \ysfchg{(i) } a road net consisting of a set of hubs,
  i.e., street intersections or ``end-of-streets'',  and
\pind  \ysfchg{(ii) } a set of
  links, i.e., street segments (with no contained hubs),
\afslut
 is a Cartesian compound\ysfchg{. }
\begynd
\pind  \ysfchg{(iii) } \ysfchg{Each set } of hubs and \ysfchg{each set } of links
\afslut are part set compounds\dbsquare}

\nbbbb{An Aside: An Upper Ontology}\label{An Upper Ontology}

\begynd
\pind We have been reasonably careful
\begynd  
\pind to just introduce and state
      informal definitions 
\pind of phenomena and some classes thereof.
\afslut
\pind In the next chapter we shall, in a sense, ``repeat'' coverage of
      these phenomena.
\begynd
\pind But \ysf{then} in a more analytic manner.
\pind Figure\,\vref{onto.fig0} is intended to indicate this.
\afslut
\afslut

\mnewfoil

\noindent
\hDBfigure{onto}{\pos{10.5}{11.7}cm}{{An} Upper Ontology}{onto.fig0}

\mnewfoil

\noindent
\begynd
\pind So far we have only touched upon 
\begynd
\pind the `External Qualities' labeled, dotted-dashed box 
\pind of the `Endurants'-labeled dashed box of Fig.\,\ref{onto.fig0}.
\afslut
\pind In Chapter \ref{primer-extq.1} we shall treat external qualities
      in more depth ---
\begynd
\pind more systematically: analytically and descriptionally.
\afslut
\afslut


\nbbbb{Internal Qualities}
     
\newestpdefn{Internal Qualities}{ Internal qualities are \pdefindex{internal quality}
\begynd
\pind those properties [of endurants]
\pind that do not occupy \sfsl{space}
\pind but can be measured or spoken about\dbsquare
\afslut}%%%%
      
\newestxample{Internal qualities}{ Examples of internal qualities are %%%%
\begynd
\pind the unique identity of a part, 
\pind the relation of \ysfchg{a } part to other
      parts, and 
\pind the endurant attributes such as temperature, length, colour\dbsquare
\afslut} 

  
\nbbb{Unique identity}
    
\begynd
\pind \newestpdefn{Unique Identity}{ A unique identity is \pdefindexii{unique}{identity}\  
\begynd
\pind an immaterial property
\pind that distinguishes any two \sfsl{spatially}
      distinct solids\dbsquare
\afslut}
\afslut

\newestxample{Unique Identities}{ Each 
\begynd
\pind hub in a road net is unique\ysf{ly} identified, 
\pind so is each link  
\pind and \ysfchg{each } automobile\dbsquare
\afslut}
%%%%%%%

\nbbb{Mereology}
\begynd 
\pind \newestpdefn{Mereology, I}{ Mereology is \pdefindex{mereology}\  
      a theory of [endurant] part-hood relations: 
\begynd 
\pind of the relations of an [endurant]
      part\ysfchg{ } to a whole 
\pind and the
      relations of [endurant] parts to [endurant] parts within that
      whole\dbsquare
\afslut}%%%%
\afslut
      
\newestxample{Mereology}{ Examples of mereologies are %%%%
\begynd 
\pind that a link is topologically \sl connected \rm to \nyl
      exactly two specific hubs, 
\pind that \ysfchg{a hub is } \sl connected \rm to \nyl zero, one or
      more specific links, 
\pind and that links and hubs are   \sl  open  \rm to \nyl
      specific subsets of automobiles\dbsquare
\afslut}
%%%%%%%

\nbbb{Attribute}
\newestpdefn{Attributes}{ Attributes are \pdefindex{attribute}\
      properties of  endurants
\begynd 
\pind that are not \sfsl{spatially} observable, 
\pind but can be either
      physically 
\pind (electronically, chemically, or otherwise) 
\pind measured or can
      be objectively spoken about\dbsquare
\afslut}%%%%
      
\newestxample{Attributes}{ Examples of attributes are: 
\begynd 
\pind links \ysfchg{ } have lengths, and, 
\pind that at any one time,
\pind zero, one or more automobiles are occupying \ysfchg{each
link}\ysf{\footnote{Oh yes, it is, of course, spatially observable
    that a link has a length, but the measurement, say \sfsl{123
      meters} is not; and the number of cars on the link is also not
    spatially observable.}}\dbsquare
\afslut}
%%%%%%%
\label{pre-primer.Prompts}
\nbbbbb{Prompts}\label{primer.Prompts}

\bbbb{Analysis Prompts}

\begynd 
\pind \newestpdefn{Analysis Prompt}{ An analysis prompt is \pdefindex{analysis!prompt}\
\begynd 
\pind a predicate or a function 
\pind that may be posed by humans to \ysf{segments}
      of a domain. 
\pind Observing the domain the analyser  may then
\pind act upon the combination of the particular prompt  
\pind (whether a
      predicate or a function, \nyl and then what particular one of these
      it is) 
\pind  thus ``applying'' it to a domain phenomen\ysfchg{on}, 
\pind  and yielding, in the
      minds of the humans, 
\pind  either a truth value or some other form of value\dbsquare
\afslut}
\afslut
      
\nbbb{Analysis Predicate}
\begynd
\pind \newestpdefn{Analysis predicates}{ \pdefindex{analysis!predicate}\ 
\begynd
\pind An analysis predicate is an analysis prompt
\pind which yields a truth value\dbsquare
\afslut}
\afslut
      
\newestxample{Analysis Predicates}{ General examples of analysis
  predicates are:
\begynd
\pind \sfsl{``can an observable phenomen\ysfchg{on }be rationally described''}, \nyl i.e., an entity, 
\pind \sfsl{``is an entity a solid or a fluid''},
\pind \sfsl{``is a solid endurant a part or a living species''}\dbsquare
\afslut}
%%%%%%%
    
\nbbb{Analysis Function}

\newestpdefn{Analysis function}{ An analysis function is \pdefindex{analysis!function}%%%
\begynd
\pind an analysis prompt which yields some \texttt{\rsltxt}\dbsquare
\afslut} 
      
\newestxample{Analysis Functions}{ Two examples of analysis functions are:
\begynd
\pind one yields the endurants of a Cartesian part \nyl and their respective sort names,
\pind another yields the set of \ysfchg{} parts of a part set \nyl and their common type\dbsquare
\afslut}
%%%%%%%
  
\nbbbb{Description Prompt}

\begynd  
\pind \newestpdefn{Description Prompt}{ A description prompt is \pdefindex{description!prompt}
\begynd  
\pind a function that may be posed by humans  
\pind who may then act  upon it:   
\begynd  
\pind \ysf{{[the human]}} ``applying'' it to a domain phenomen\ysfchg{on}, and 
\pind \ysf{{[the human]}} ``yielding'', i.e., writing down, \ysf{a}
narrative and formal {\rsltxt}s \nyl describing 
      what is being observed \ysf{[by that human]}\dbsquare
\afslut
\afslut}
\afslut

\mnewfoil

\newestxample{Description Prompts}{ Description prompts result
\begynd
\pind in {\rsltxt}s describing for example a 
\begynd
\pind (i) \ysfchg{a } Cartesian endurant, or
\pind (ii) its unique identifier, \ysf{or}
\pind (iii) its mereology, or
\pind (iv) its attributes, \ysf{or}
\pind(iv) other\dbsquare
\afslut
\afslut}
%%%%%%%

\nbbbbb{Perdurant Concepts}\label{Perdurant Concepts}

\bbbb{``Morphing'' Parts into Behaviours}

\begynd
\pind As already indicated we shall
\begynd
\pind transcendentally deduce
\pind (perdurant) behaviours from
\pind those (endurant) parts
\begynd
\pind which we, as domain analysers cum describers,
\pind have endowed with all three kinds of internal qualities:
\pind unique identifiers, mereologies and attributes.
\afslut
\afslut
\pind Chapter\,\ref{chap6.tex.1}, will show how.
\afslut

\nbbbb{State}
\newestpdefn{State, I}{ \pdefindex{state}\ 
\begynd
\pind  A state is any set of
      the parts of a domain\dbsquare
\afslut}
%%%%%%%

\monoexample{A Road System State}{ The domain analyser cum describer
  may, \nyl \ysf{} decide that a road system state consists of
\begynd
\pind the road net aggregate (of hubs and links)\footnote{\LLLL The road net
  aggregate, in its perdurant form, may ``model'' the \sfsl{Department
  of Roads} of some country, province, or town.},
\pind all the hubs, and all the links, and
\pind the automobile aggregate (of all the automobiles)\footnote{\LLLL
  The  automobile \ysfchg{}
  aggregate, in its perdurant form, may ``model'' the \sfsl{Department
  of Vehicles} of some country, province, or town.}, and
\pind all the individual automobiles\dbsquare
\afslut}

\nbbbb{Actors}
\newestpdefn{Actors}{ \pdefindex{actor}\ 
\begynd
\pind An actor is anything
      that can initiate an action, an event or a behaviour\dbsquare
\afslut}
%%%%%%%
 
\bbb{Action}
   
\newestpdefn{Actions}{ \pdefindex{action}\
\begynd
\pind  An action is a
      function that can purposefully change\ysfchg{ } a state\dbsquare
\afslut}
\monoexample{Road Net Actions}{ These are some road net actions:
\begynd
\pind The insertion of a new or removal of an existing hub; or
\pind the insertion of a new, or removal of an existing link;
\afslut}

%%%%%%%

\nbbb{Event}

\newestpdefn{Events}{ \pdefindex{event}\
\begynd
\pind  An event is a function
      that surreptitiously changes a state\dbsquare
\afslut}
\monoexample{Road Net Events}{ These are some road net events:
\begynd
\pind The blocking of a link due to a mud slide;
\pind the failing of a hub traffic signal due to power outage;
\pind the blocking of a link due to an automobile accident.
\afslut}

%%%%%%%
    
\nbbb{Behaviour}

\newestpdefn{Behaviours}{ \pdefindex{behaviour}\
\begynd
\pind A behaviour is a set 
\pind of sequences of 
\pind actions, events and behaviours\dbsquare
\afslut}
%%%%%%%
\mnewfoil\LLll
\monoexample{Road Net Traffic}{ %
\begynd
\pind Road net traffic can be seen as a behaviour
\begynd
\pind \ysfchg{(i) } of all the behaviours of automobiles,
\begynd
\pind where each automobile behaviour is seen as sequence of \nyl start, stop,
      turn right, turn left, etc., actions;
\afslut
\pind \ysfchg{(ii) } of all the behaviours of links
\begynd
\pind where each link behaviour is seen \nyl as a set of sequences
(i.e., behaviours) of ``following'' the 
\begynd
\pind link entering, link leaving, and movement
\afslut of automobiles on the link;
\afslut
\pind \ysfchg{(iii) } of all the behaviours of hubs (etc.);
\pind \ysfchg{(iv) } of the behaviour of the aggregate of roads, \nyl
      viz.\ \sfsl{The Department of Roads}, and 
\pind \ysfchg{(v) } of the behaviour of the aggregate of automobiles, \nyl viz,\
      \sfsl{The Department of Vehicles}.
\afslut
\afslut}\HHHH


\nbbbb{Channel}

\begynd
\pind \newestpdefn{Channel}{ \pdefindex{channel}\
\begynd
\pind A channel is anything 
\pind that allows synchronisation and communication
\pind of values
\pind between two behaviours\dbsquare
\afslut}
\afslut

\mnewfoil

\noindent
\begynd
\pind We shall use Tony Hoare's \texttt{CSP} concept \citecsp\
\begynd
\pind to express synchronisation and communication \nyl of values
      between behaviours \brcolor{\textsl{i}} and \brcolor{\textsl{j}}.
\pind Hence the behaviour \brcolor{\textsl{i}} statement \nyl
      \bbcolor{\textsl{ch[\brcolor{j}]\,\,!\,\,value}} \nyl
      \ysfchg{states } that
      behaviour \brcolor{\textsl{i}} \nyl  offers, ``outputs'':
      \bbcolor{!}, \bbcolor{\textsl{value}} \nyl  to \ysfchg{the
      behaviour } indicated by \bbcolor{\textsl{\brcolor{j}}}. 
\pind And behaviour \brcolor{\textsf{\ysfchgii{j}}} expresses \nyl  \bbcolor{\textsl{ch[\brcolor{\ysfchgii{i}}]\,\,?}} \nyl 
      that  it is willing to accept \nyl  ``input from \&
      synchronise with'' behaviour \brcolor{\textsl{i}}, \bbcolor{?}, \nyl any \bbcolor{\textsl{value}}.
\afslut
\afslut
%%%%%%%

\nbbbbb{Domain Analysis \& Description}

\bbbb{Domain Analysis}

\newestpdefn{Domain Analysis}{ Domain analysis \pdefindex{domain!analysis}\pdefindex{analysis!domain}\
\begynd
\pind is the act of studying a domain
\pind as well as the result of that study
\pind in the form of \bmcolor{informal} statements\dbsquare
\afslut}

\bbbb{Domain Description}

\newestpdefn{Domain Description}{ Domain description \pdefindex{domain!description}\pdefindex{description!domain}\ 
\begynd
\pind is the act of describing a domain 
\pind as well as the result of that act
\pind in the
      form of  \bmcolor{narratives} and \bmcolor{formal \rsltxt}\dbsquare
\afslut}

\nbbbbb{Closing}

\begynd
\pind This \pos{chapter}{lecture} has introduced \nyl the main concepts of domains such as
      we shall treat \nyl  (analyse and describe) domains.\footnote{\LLLL We
      have omitted treatment of \sfsl{living species: plants} and
      \sfsl{animals} -- the latter including \sfsl{humans}. They will
      be treated in the next chapter\,!}
\pind The next \pos{three chapters}{lectures} shall now systematically treat \nyl  the
      analysis and description of domains.
\begynd
\pind That treatment takes concept by concept and 
\begynd
\pind provides proper definitions and
\pind introduces appropriate analysis and description prompts;
\pind one-by-one, in an almost pedantic, 
\pind hence perhaps ``slow'' progression\,!
\afslut
\afslut
\mnewfoil
\pind The \pos{reader}{student} may be excused
\begynd
\pind if they, now-and-then, lose sight of ``their way''.
\afslut
\pind Hence the present chapter.
\begynd
\pind To show ``the way'':
\pind that, for example, \nyl when we treat external endurant qualities,
\pind there are still the internal endurant qualities,
\pind and that the whole thing leads of to perdurants:
\begynd
\pind actors,
\pind actions,
\pind events and
\pind behaviours.
\afslut
\afslut
\afslut

\label{chap2.tex.Preview.n}
%%  LocalWords:  artefactual analysing Endurants endurant Perdurants
%%  LocalWords:  perdurant endurants colour Mereology mereology RSL
%%  LocalWords:  behaviour Behaviours behaviours harbours modelled de
  %%  LocalWords:  synchronisation perdurants philosophee Ren analyse
%%  LocalWords:  plasmatic compartmentalise fn et cetera analysed CSP
%%  LocalWords:  descriptionally analyser Hoare's synchronise ly wo
%%  LocalWords:  analysers mereologies modelling conveyage phenomen
%%  LocalWords:  neighbouring
