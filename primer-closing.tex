
\nbbbbbb{Closing}\label{chap7.tex.1}\label{chapter:Closing}
\pos{\minitoc}{}

\bbbbb{What has been Achieved and Not Achieved\,?}

\bbbb{What has been Achieved\,?}

\begynd
\pind An initial phase of software development has been suggested, one
      that is also independent of whether software is indeed intended:
      that of domain engineering.
\begynd
\pind A calculus of domain inquiry and a calculus of domain
      description ha\ysfchg{ve } been put forward --- 
\pind calculi that are presently focused on domain endurants.
\afslut
\afslut

\nbbbb{What has Not been Achieved\,?}

The next section elucidates on both what has been, and what has not
been achieved. 

\bbbbb{Related Issues}

\begynd
\pind A number of issues related to domain modelling need be briefly addressed.
\afslut

\nbbbb{Axioms, Well-formedness and Proof Obligations}\LLLL

\begynd 
\pind \pos{The reader}{Someone in the audience} may have noticed \nyl
      that this \primer\ hardly mentions the notion of verification,
\begynd
\pind yet domain descriptions, \nyl as possibly any specification related
      to software, 
\pind may require some form of verification.
\afslut
\pind Yet this \primer\ appears to skirt the issue.
\pind Indeed, we have, regrettably, omitted the issue.
\pind So we must refer t\ysfchg{he } \pos{reader}{student} to relevant literature.
\pind We cannot, \today, point to any definitive book on the topic.
\pind The field is under intense research.
\pind Instead we refer to
\begynd 
\pind  such diverse papers as:\cite{CPAIOR:Nikolaj:2021,AdvCafeOBJ2022}
\pind  as well as the seminal book \cite{PatrickCousot2021}.
\afslut
\afslut

\mnewfoil
  
\begynd
\pind In \ysfchg{\sfsl{Endurant Description Prompt}}
      \vref{ddp:calc-Cartesian-parts} and  we mention the
      concept of proof obligation. They are also mentioned in
      \ysfchg{\sfsl{Attribute  Description Prompt}} \vref{ddp:observe-attributes}. 
\pind In numerous other places we mention the concept of \sfsl{axiom}:
      \vref{apuid000} (uniqueness of unique identifiers),
      \vref{observe-mereology} (mereology),
      \vref{xtime-020} (property of time),
\pind And in some places we mention  the concept of
      \sfsl{well-formedness,} \ysfchg{for example}, Sect.\,\vref{wff-event-histories},
\afslut
 
\mnewfoil
   
\begin{itemize}
\item \bbcolor{Axioms} express properties of endurants, whether
      external or internal qualities, that hold\ysfchg{ } -- as were they laws of
      the domain.
\item \bbcolor{Well-formedness} predicates are defined where  external
      or internal qualities of endurants are defined by concrete types
      in such ways as to warrant such predicates.  
\item \bbcolor{Proof obligations} are usually warranted where distinct
      sort definitions need be separated.
\end{itemize}
  
\nbbbb{From Programming Language Semantics to Domain Models}\label{primer:From Programming Language Semantics to Domain Models}

\begynd
\pind In 1973--1974, at the \sfsl{IBM Vienna Laboratory}, we,
\begynd
\pind Peter Lucas, Hans Beki{\v{c}}, Cliff Jones, Wolfgang Henhapl and
      \ysfchgiii{Dines Bj{\o}rner }
\pind researched \& developed a formal description of the
      \texttt{PL/I} programming language \cite{db74:r:Bek74}.
\afslut
\pind In 1979--1984, at the \sfsl{Dansk Datamatik Center, DDC}, \nyl under \ysfchgiii{Dines Bj{\o}rner's }
      leadership and with invaluable help from \ysfchgiii{his } colleague, Dr.\ Hans
      Bruun, \nyl and based on  \ysfchgiii{Dines Bj{\o}rner's } MSc.\ lectures,
\begynd
\pind seven M.Sc.\ students%
\footnote{%
          J{\o}rgen Bundgaard,
          Ole Dommergaard,
          Peter L. Haff,
          Hans Henrik L{\o}vengreen,
          Jan Storbank Petersen,
          S{\o}ren Prehn,
          Lennart Schulz}
developed formal descriptions of (and later full compilers for) the
     programming languages \texttt{CHILL} \cite{vdm:Haff} and \texttt{Ada}, the latter
          under the informed management of Dr.\ Ole N.\ Oest \cite{e:db:Bj80f}.
\afslut
\afslut

\mnewfoil\LLLL

\begynd
\pind In a domain model we describe essential nouns and verbs of the
      ``language spoken'' by practitioners of the domain.
\begynd
\pind The ``extension'' from the language ``spoken by programmers'' to
      that ``spoken by domain practitioners'' should be obvious.
\afslut
\afslut

\begynd
\pind In both cases, the descriptions, for realistic programming
      languages and for realistic domains, are not trivial.
\begynd
\pind They are sizable.
\pind The \texttt{PL/I, CHILL} and \texttt{Ada} descriptions
      span from a hundred pages to several hundred pages,!
\afslut
\pind Similarly, their implementation, in terms of interpreters and
      compilers, took many man-years.
\begynd
\pind For the \sfsl{DDC} \texttt{Ada Compiler} \cite{Haff87,Clem84} it took 44 man-years\,!
\afslut
\afslut

\begynd
\pind From a description of realistic facets of a domain \nyl one can
      develop a number of more-or-less distinct requirements,
\pind and from these one can develop computing systems software \nyl
      and we can expect similar size efforts.
\afslut

\nbbbb{Domain Specific Languages}

\begynd
\pind A domain specific language, generically referred to as a \texttt{DSL},
\begynd
\pind is a language whose basic syntactic elements directly reflect
      endurants and perdurants of a specific domain.
\pind \sfsl{Actulus}, a language in which to express calculations of
      actuarian character \cite{actulus2015}, is a DSL. 
\afslut
      
\pind The semantics of a \texttt{DSL}, obviously, must relate to a
      model for the domain in question. 
\pind In fact, we advice, that \texttt{DSL}s be developed from the
      basis of relevant domain models.
\afslut
  

\nbbbb{The \texttt{RAISE} Specification Language, \texttt{RSL}}\label{RSL-II}

\begynd
\pos{\pind We refer to Sect.\,\vref{RSL-I}.}{}
\pind So we have used \texttt{RSL} in two ways in this \primer:
\begynd
\pind (i)  informally, to explain the domain analysis \& description
           method -- in \rslplus, and
\pind (ii) formally, to present [fragments of] specific domain
           specifications. 
\afslut
\pind The latter always in enumerated examples.\footnote{\LLLL These
  are: Examples\,%
  \pos{}{\begin{multicols}{5}}\LLll\noindent%
\vref{A Road Transport System Domain: Cartesians},\nyl
      \vref{Road Transport System: Sets of Hubs, Links and Automobiles},  \nyl 
      \vref{Alternative Rail Units}, \nyl
      \vref{Constants and States},  \nyl
      \vref{Unique Identifiers},    \nyl     
      \vref{Unique Road Transport System Identifiers},  \nyl     
      \vref{Uniqueness of Road Net Identifiers}, \nyl  
      \vref{Rail Net Unique Identifiers},  \nyl      
      \vref{Mereology of a Road Net},  \nyl
      \vref{Invariance of Road Nets},  \nyl     
      \vref{Rail Net Mereology},    \nyl
      \vref{Road Net Attributes},     \nyl  
      \vref{Invariance of Road Net Traffic States}, \nyl    
      \vref{Road Transport -- Further Attributes}, \nyl  
      \vref{Intentional Pull -- Road Transport},  \nyl   
      \vref{Automobile Behaviour},     \nyl
      \vref{The Road Transport System Initialisation}, \nyl    
      \vref{Road Net Administrator},     \nyl  
      \vref{Road Net Development: Hub Insertion}, \nyl      
      \vref{Road Net Development: Link Insertion}, \nyl
      \vref{Road Net Development: Hub Removal}, and  \nyl    
      \vref{Hub Adjustments}
\pos{}{\end{multicols}}}
\pos{\pind Appendices\,\ref{Chapter:Road
       Transport}--\ref{appendix:Pipelines} both exemplify 
       formal uses of 
       \texttt{RSL}. 
\pind All the functions listed in Index
       Sects.\,\ref{label.apps}--\ref{label.dps}  and their
       explication are using the informal \rslplus.}{}
\afslut 

\nbbbb{R{\^o}le of Algorithms}\label{Role of Algorithms}

\begynd
\pind In all of the function formulation of domain phenomena, in this
\primer, You have 
      not seen a single, interesting algorithm\,!\footnote{\LLLL
        Algorithm: a process or set of rules to be followed in
        calculations or other problem-solving operations, especially
        by a computer.} 
\begynd
\pind We need not apologize for that.
\pind There is a reason.
\begynd
\pind The reason is that we almost only describe properties.
\pind To that end we make use of classical mathematical notions such as
      set comprehension, for example: \textsf{{\LBRACE}
      a {\BAR} a:A {\RDOT} $\mathcal{P}$(a) {\RBRACE}}.
\pind The search for a\ysfchg{n } appropriate $a$ such that $\mathcal{P}(a)$
      holds is often what requires, often beautiful algorithms.
\afslut
\pind We refer to \cite[\sfsl{Knuth and Harel}]{Knuth,Harel87b}.
\afslut
\mnewfoil
\pind The need for clever algorithms, usually, first arise when
      designing software. 
\begynd
\pind Not in requirements engineering
      (cf.\,Sect.\,\vref{primer:Requirements Engineering}), but in
      software design. 
\pind Then requirements prescriptions,
\begynd
\pind also usually expressed in terms of set, list or map comprehension, 
\pind or corresponding quantifications,
\afslut
\pind need efficient implementations;
\pind hence clever algorithms.
\afslut
\afslut

\nbbbb{CSP versus PDEs}\label{CSP versus PDEs}

\begynd
\pind To model the behaviour of discrete dynamic domains, such as are
      the main focus of this \primer, we use the \texttt{CSP} process
      concept \citecsp.
\pind To model the behaviour of continuous dynamic domains, which we
      really have not, we suggest that You use  methods of
      analysis, to wit: \sfsl{[Partial] Differential Equations, PDEs}.
\pind Perhaps also some \sfsl{Fuzzy Logic} \cite{Zadeh1965Fuzzy,FuzzyTextBook2014}.   
\pind That is: We see this as the ``dividing line'' between discrete
      and continuous dynamic systems modelling: \sfsl{CSP versus DPEs}.
\pos{\pind Appendix\,\ref{appendix:Pipelines},
      pages\,\pageref{appendix:Pipelines}--\pageref{appendix:Pipelines.n},
      puts forward a domain whose continuous dynamics need be
      formalised, for example using PDEs \cite{CourantAaC}.}{}
\pind Mathematical modelling such as based on \sfsl{Adaptive Control
      Theory} \cite{aastroem89}, \sfsl{Stochastic Control Theory}
      \cite{Karlin+Taylor1998} or maybe \sfsl{Fuzzy Control} \cite{Michel-etal-2010},
      like algorithmics, first be required 
      as possible techniques when issues of correct continuous
      dynamics and optimisation arise, as when implementing certain requirements. 
\afslut

\nbbbb{Domain Facets}\label{primer:Domain Facets}

\begynd
\pind There are other, additional methodological domain modelling steps.
\pind In \cite[Chapter\,8, Pages 205--240]{BjornerMonograph2020} we
      cover the notion of \sfsl{domain facets}.
\begynd
\pind By a domain facet we shall understand one amongst a finite set
      of generic ways of analysing a domain:
\begynd
\pind a view of the domain,
\pind such that the different facets 
\pind cover conceptually different views, 
\pind and such that these views together
\pind cover the domain. 
\afslut
\afslut
\pind \pos{We here}{As examples of domain facets we} list:
\begin{multicols}{2}
\begin{itemize}
\item intrinsics,
\item support technologies,
\item rules \& regulations, including
\begin{itemize}
\item scripts,
\item license languages,
\end{itemize}
\item management \& organisation, and
\item human behaviour.
\end{itemize}
\end{multicols}\noindent as such facets.
\pos{\pind The referenced chapter (\cite[Chapter\,8, Pages
      205--240]{BjornerMonograph2020}) is traditional, programming
      methodological, in the sense that there is no [semi-]formal
      calculi involved, as in this primer's
      Chapters\,\ref{primer-extq.1}--\ref{chap4.tex.1}\ysfchgii{. }
\pind \ysfchg{We } could wish for that\,!}{}
\afslut

\nbbbb{Requirements Engineering}\label{primer:Requirements Engineering}

\begynd
\pind Domain modelling, to repeat, can be pursued for two different,
      but related, reasons.
\begynd
\pind (i) \ysfchg{S}imply, without any concern for, or idea of possible
          software, in order to ``just'' understand a domain, or
\pind (ii) for reasons of subsequent software development.
\afslut
\pind In the later case a step of \sfsl{requirements engineering} need be pursued.
\begynd
\pind \cite[Chapter\,9, Pages\,243--298]{BjornerMonograph2020} covers
      a notion of \sfsl{requirements engineering}.
\pind In that chapter we show three stages of requirements
      development:
\begynd
\pind ($\alpha$) \sfsl{domain requirements},
\pind ($\beta$) \sfsl{interface requirements}, and 
\pind ($\gamma$) \sfsl{machine requirements}.
\afslut
\mnewfoil
\pind But first a definition of the term \sfsl{`machine'}.

\lastchg{%%%%%%%%%%%%%%%%%%%%%%%%%%%%%%%%%%%%%%%%%%%%%%%%%%%%%%%%%%%%%
\bookdefn{Machine}{
\begynd
\pind By \sfsl{machine} we shall understand\index{pdefind}{machine}
\begynd
\pind a, or the, combination of hardware and software 
\pind that is the target for, or result of 
\pind the required computing systems development. 
\afslut 
\afslut
}

\bookdefn{Requirement}{
\pind By a \sfsl{requirement\ysfchg{ }} we shall understand (cf., IEEE Standard
      610.12):\index{pdefind}{requirements}
\begynd
\pind  \sfsl{``A condition or capability
\pind  needed by a user
\pind  to solve a problem
\pind  or achieve an objective.''}\dbsquare\
\afslut
\afslut
}
\mnewfoil

\bookdefn{Domain Requirements}{
\pind By \ysfchg{ } \sfsl{domain requirements} we shall understand
      \index{pdefind}{domain!requirements}\index{pdefind}{requirements!domain} 
\begynd
\pind those requirements 
\pind which can be expressed 
\pind s{\^o}lely using terms of the domain\dbsquare\
\afslut
}

\bookdefn{Interface Requirements}{
\pind By  \ysfchg{ } \sfsl{interface requirements} we shall understand
      \index{pdefind}{interface!requirements}\index{pdefind}{requirements!interface} 
\begynd
\pind those requirements
\pind which can be expressed
\pind only using technical terms of both the domain and the machine\dbsquare\
\afslut
}

\bookdefn{Machine Requirements}{
\pind By  \ysfchg{ } \sfsl{machine requirements} we shall understand
      \index{pdefind}{machine!requirements}\index{pdefind}{requirements!machine} 
\begynd
\pind those requirements which, in principle, 
\pind can be expressed 
\pind s{\^o}lely using terms of the machine\dbsquare\
\afslut
}
}%%%%%%%%%%%%%%%%%%%%%%%%%%%%%%%%%%%%%%%%%%%%%%%%%%%

\noindent
\pind The  \sfsl{domain requirements} stage of requirements
      development
\begynd
\pind starts with a basis in the domain engineering's domain description.
\pind It is, so-to-speak, a first step in the development of a
      requirements prescription.\footnote{\LLLL The ``passage'' from
      domain description to requirements 
      prescription marks a transcendental deduction. Domain descriptions
      designate that which  is being described. Requirements 
      prescriptions designate what is intended to be implemented by computing. Please note
      the distinction: At the end of the development of a domain description we have just that:
      a domain description. At the beginning of the development of a
      requirements prescription we consider the domain description to be
      the initial requirements prescription: Thus, 
      seemingly bewildering, in one instance a document is considered a domain description,
      in the next instance, without that document having been textually changed, it is now
      considered a requirements prescription.
      The transition from domain description to requirements prescription also marks a
      transition from ``no-design mode'' description to ``design-mode'' prescription.}
\mnewfoil
\pind From there follows, according to
      \cite[Chapter\,9]{BjornerMonograph2020} a number of (five)
      steps:
      
\begynd
\bookdefn{Projection}{\index{pdefind}{requirements!projection}
      By projection is meant a subset of the
      domain description, one which projects out all those endurants:
      parts and fluids, as well as perdurants: actions, events and
      behaviours that the stake-holders do not wish represented or
      relied upon by the machine\dbsquare\
    }
    
\bookdefn{Instantiation}{\index{pdefind}{requirements!instantiation} By
      instantiation we mean a refinement of the partial domain
      requirements prescription (resulting from the projection step)
      in which the refinements aim at rendering more concrete, more
      specific the endurants: parts and fluids, as well as the
      perdurants: actions, events and behaviours of the domain
      requirements prescription\dbsquare\
}
      \mnewfoil
   
\bookdefn{Determination}{\index{pdefind}{requirements!determination} By
      determination we mean a refinement of the partial domain
      requirements prescription, resulting from the instantiation
      step, in which the refinements aim at rendering less
      non-determinate, more determinate the endurants: parts and
      fluids, as well as the perdurants: functions, events and
      behaviours of the partial domain requirements
      prescription\dbsquare\
}
      
\bookdefn{Extension}{\index{pdefind}{requirements!extension}
      By  extension we understand the introduction of endurants
      and perdurants that were not feasible in the original domain,
      but for which, with computing and communication, and with new,
      emerging technologies, for example, sensors, actuators and
      satellites, there is the possibility of feasible
      implementations, hence the requirements, that what is introduced
      becomes part of the unfolding requirements
      prescription\dbsquare\
    }
    
\mnewfoil  
\bookdefn{Fitting}{\index{pdefind}{requirements!fitting} By
      requirements fitting we mean a harmonisation of two or more
      domain requirements that have overlapping (shared) not always
      consistent parts and which results in $n$ partial domain
      requirement, and \ysfchg{$m$ } shared domain requirement, that ``fit
      into'' two or more of the partial domain requirements\dbsquare\
}
\afslut
      
\noindent
\pind \cite[Chapter\,9]{BjornerMonograph2020} then goes on to outline
      interface and machine requirements steps.
      \afslut
      
\pind So domain engineering is a sound basis, we claim, for software
      development.

\pind How that basis harmonises with the approaches taken by
      \sfsl{{Axel van} Lamsweerde} 
      \cite{Lamsweerde} and \sfsl{Michael A. Jackson}
      \cite{Jackson2010Facs} is, really, a worthwhile study    
      in-and-by itself\,!
\afslut
\afslut

\nbbbb{Possible [PhD] Research Topics}\label{Possible Research Topics}

\begynd
\pind \ysfchg{We } list here a number of possible (PhD) research topics:
\begin{enumerate}
\item \label{rp:Intentional Pull} \sort{Intentional Pull:} This topic is not treated to the depth
      it deserves in this \primer. Try think of intentional pulls in
      several domains: (i) \sfsl{money flow in financial institutions}
      (while domain modelling a fair selection of such: banks, credit
      card companies, brokers, stock exchanges \cite{db:tse:2010},
      etc.); (ii) \sfsl{railway systems} (study, for example,
      \cite{db04:railway,db97:r:UNUIIST:93,train-ifip-wcc2004-db1,train-ifip-wcc2004-mp,%
            train-ifip-wcc2004-db2,db02-amore-ros,db02-amore-maint,db03:ifac-cts2003,dines-idpt-02}); 
      and (iii) \sfsl{container terminals} (see \cite{BjornerContainer2018}). 
\mnewfoil
\item \label{rp:Discrete vs Continuous}  \sort{Discrete vs. Continuous
      Endurants and Perdurants:} Take 
      the example of (oil, gas, water) \sfsl{pipelines}. \pos{See Appendix\,\ref{appendix:Pipelines}.}{}
      Try model the dynamic flow of liquid in pipes, valves, pumps,
      etc., that is ``mix'', as may be expected, differential
      equations with \texttt{RSL} formulas. Some have tried. No real
      progress seems attained. Se however
      \cite{rTiMo2018,BigrTiMo2021}.
      The pipeline example should illustrate the use of monitorable
      attributes, their ``reading'' and their ``biddable updates''.

      The challenge here is threefold: (i) first the PDE etc.\ modelling of
      the flow for each kind of unit, including curved pipe
      units; \pos{}{and} (ii) then for their composition -- for a specific layout\pos{,
      for example that hinted at in Fig.\,\vref{anoipilisy}; and (iii) finally
      for the infinite collection of pipeline systems such as defined
      by the ``abstract syntax'' of
      Appendix.\,\ref{appendix:Pipelines} Item\,\vref{p-e--1}
      (including its wellformedness).}{.} 
\mnewfoil
\item \label{rp:Towards a Calculus of Perdurants}  \sort{Towards a
      Calculus of Perdurants:} This \primer\ has 
      unveiled the beginnings of a \sfsl{Calculus of Endurants}. (Yet, its
      real ``calculus-orientation'' has yet to emerge: its laws, etc.)
      Sect.\,\ref{Behaviour Definition Bodies}  hints at what
      \ysfchg{we } have in
      mind. A systematic analysis which aims at uncovering a fixed number
      of behaviour patterns such as sketched in Sect.\,\ref{Behaviour Definition Bodies}. 
\mnewfoil
\item \label{rp:Human Interaction}  \sort{Modelling Human
      Interaction:} The ``running example''\pos{, 
      summarised in Appendix\,\ref{p-ch:Road Transport},}{} illustrated a
      road net ``populated'' with automobiles driving ``hither \&
      dither''. The current \primer\ has not treated the interaction
      between humans and man-made artifacts, like, for example,
      drivers and their automobiles. You are to model, for example,
      such human actions as starting an automobile, accelerating,
      braking\ysfchg{, } turning left, turning right, and stopping. Doing so
      You will have to try out, experiment with the r{\^{o}}les of
      monitorable, including biddable automobile attributes. An aim,
      besides such a domain model, is to research method issues of
      modelling human interaction. Please disregard modelling issues
      of sentiments, feelings, etc. 
\mnewfoil
\item \label{rp:Transcendental Deduction}  \sort{Transcendental
      Deduction:} In the philosophy of Kai 
      S{\o}rlander \pos{such as, for example, explained in
      Chapter\,\ref{chap2.tex.Philosophy},}{} transcendental deduction is
      appealed to repeatably. In this \primer, as in
      \cite{BjornerMonograph2020}, transcendental deduction is
      appealed to only once\,! Maybe research into possible calculi
      for perdurants, cf.\,\textsf{Research Challenge}\,\ref{rp:Towards
      a Calculus of Perdurants}, might yield some more examples of transcendental deductions. 
\mnewfoil
  \item \label{rp:Formal Models} \sort{Formal Models of Domain
      Modelling Calculi:} In \cite{bjorner-jaist-2014} \ysfchg{an
      attempt is made at } a
      first formal model of the domain analysis \& description calculi.
      With  \cite{BjornerMonograph2020} and, especially, this \primer\
      as a background, perhaps a more thorough attempt should be made to
      bring the model of \cite{bjorner-jaist-2014} up-to-date and complete\,! 
\mnewfoil
\item \label{rp:filosofi} \sort{Kai S{\o}rlander's Philosophy:} \pos{We
    refer to Chapter\,\ref{chapter:Philosophy}.}{{[Kai S{\o}rlander's
    Philosophy is treated in a special chapter of the notes fr
    these lectures.]}} It is here strongly
    suggested that this research project be based on
    \cite{kaisorlander2022}, Kai S{\o}rlander's most recent
    book.\footnote{All of S{\o}rlander's books
    \cite[1994--2022]{kaisorlander1994,kaisorlander1997,kaisorlander2002,kaisorlander2016,kaisorlander2022} 
    are in Danish -- so the researcher would either be able to read
    Danish, or, more preferably to \ysfchgv{us}, to have a suitable (German,
    English, French, ...) translation at hand.} 
    The challenge, in a sense, has two elements: (i) the
    identification of S{\o}rlander's use of \textsf{transcendental
    deduction}: painstakingly identifying \sort{all} it uses,
    analysing each of these, studying whether one can characterise
    these uses into more than one common kind of deduction, or whether
    one might claim \sfsl{``classes of deductions''}, not necessarily
    disjoint, but perhaps structured in some kind of taxonomy; and
    (ii) the analysis of this report's presentation of S{\o}rlander's
    metaphysics.
\end{enumerate}
\afslut

%%\nbbbb{What Next\,?}

%%\tbw

\eysf

\nbbbbb{Closing Remarks}

\bbbb{Endurants versus Perdurants}

\begynd
\pind The number of concepts pertaining to endurants versus the number
      of concepts pertaining to perdurants appears to signify something\,!
\begynd
\pind The number of concept definitions that relate to endurants is around 80.
\pind Those of perdurant concepts is around 10\,!
\pind How can that large difference be understood\,?
\afslut
\afslut

\nbbbb{Domain Science \& Engineering}

\begynd
\pind The present primer represents, at the moment, a long line of development.
\pind As mentioned in Sect.\,\vref{Domain Engineering 2016 2022} this grew out of a series of works:
\cite{TheSEBook3,Kiev:2010ptI,Kiev:2010ptII,BjornerFAoC2015MDAAD,BjornerTOSEM2018,BjornerMonograph2020}. 
\begynd
\pind The first inklings --- in \ysfchg{our } work on 
\pind what is now the \sfsl{Domain
      Science \& Engineering} of this \primer --- \nyl appeared in
      \cite[1995-1996]{UNUIIST:37,UNUIIST:46,db:UNUIIST:48,UNUIIST:58,UNUIIST:59}. 
\pind The \sfsl{UN University's International Institute for Software
      Technology,} \texttt{UNU/IIST} \ysfchg{ } conducted several domain engineering-based
      research \& development projects, most of them under the
      leaderships of (the late) S{\o}ren Prehn and Chris W.\ George
      \cite{DBLP:conf/ac/George07x}. 
\pind \cite[2008]{dines:facs:2008} touched upon the concept of 
      \sfsl{Domain Facets}, not covered in this \primer, but in
      \cite[2021]{BjornerMonograph2020}. 
\pind Two papers \cite[2010]{Kiev:2010ptI,Kiev:2010ptII} suggested
      reasonably relevant properties of domain descriptions.
\pind It was not until \cite[2017]{BjornerFAoC2015MDAAD} that the analysis
      \& description calculi of this \primer\ emerged, and were
      refined in \cite[2019]{BjornerTOSEM2018}. 
\afslut

                                %\summerchg
{%%%%%%%%%%%%%%%%%%
\pind The iteration from \cite{TheSEBook3} via
      \cite{Kiev:2010ptI,Kiev:2010ptII,BjornerFAoC2015MDAAD,BjornerTOSEM2018,BjornerMonograph2020}
      to the present primer 
      reminds \ysfchgiii{us } of the French author \textsl{Anatole France's} story of the
      history of the  mankind.\footnote{%%
On acceding to the throne of Persia, a young king assembled all the
academicians of his realm and charged them with writing a detailed
history of mankind, that he may learn from it to become a wise ruler. 

The wise men deliberated and returned after twenty years with twelve
camels, each carrying five hundred volumes. But the king could not
find the time to read so many volumes, and tasked them with reducing
the number of volumes ``to the brevity of human existence''. 

The academicians worked for another twenty years and returned with
fifteen hundred volumes. But the king said, ``I am getting old and
cannot read all these volumes''. 

The academicians returned after ten years with five hundred volumes
but the king asked them to shorten it further so that he could learn,
before dying, human history. 

After five years, a lone academician carrying a single volume arrived
at the palace. ``Hurry up''\ysfchgii{, } an officer told him, ``the king is
dying''. The king looked at the academician and said, ``So I shall die
without knowing human history''. 

``Sir'', replied the academician, ``\ysfchgii{I } can summarize it for you in three
words: – they are born, they suffer, they die.'' [\texttt{http://profzeki.blogspot.com/2012/05/anatole-france-and-reductionism.html}]
}%%%%%%%%%%%%%%%%%%%%%%%%%%%%    
\afslut
\afslut
\afslut

\nbbbbb{Acknowledgments}

\begynd
\pind In \cite[\sfsl{Preface/Acknowledgments,
      Page\,xiv}]{BjornerMonograph2020} \ysfchg{the first author } acknowledged the very many
      who, over \ysfchg{his } professional life, has inspired \ysfchg{him}.
\pind In ``rewriting'' this primer from \cite{BjornerMonograph2020}
       \ysfchg{the first author has}, again, attempted to ``capture'' Kai S{\o}rlander's
      Philosophy\pos{, cf.\,Chapter\,\ref{chapter:Philosophy}}{}. And
      again \ysfchg{we }
      wish\ysfchg{ } to deeply acknowledge that work and, hence, \sort{Kai S{\o}rlander}.
\pind Here  \ysfchg{the first author}, additionally,wishes to acknowledge, with pleasure, \sort{Laura
      Kovacs}, TU Wien. Laura invited  \ysfchg{him } to lecture, in the fall of
      2022\footnote{\LLLL Well, an invitation for Covid-19 year 2021 had to be
      postponed\,!}, at TU Wien. 
      This \primer\ is the result of that invitation.
\pind Drs.\ Mikhail Chupilko\footnote{\LLLL ISP/RAS: Inst\pos{itute}{.} of Systems
      Programming, The Russian Acad\pos{emy}{.} of Sci\pos{ences}{.}, Moscow} and Yang
      ShaoFa\footnote{\LLLL IoS/CAS: Inst\pos{itute}{.} of Software, The Chinese Acad\pos{emy}{.} of
      Sci\pos{ences}{.}, Beijing} are currently translating this primer into
      Russian, respectively Chinese. \ysfchg{The first author } acknowledge\ysfchg{s}, with many thanks, their
      ongoing comments.
\pind \ysfchg{The work of Dr.\.Yang ShaoFa has in this respect been of
      such help that the first author decided, in September 2023, to
      list him as co-author\,!}
\afslut

\label{chap7.tex.n}


%%  LocalWords:  modelling Endurants Perdurants RSL monitorable les
%%  LocalWords:  behaviour summarised computable Kovacs Wien Covid le
%%  LocalWords:  Kai rlander perdurants rlander's analysing amongst
%%  LocalWords:  characterise intrinsics organisation Harel CSP DPEs
%%  LocalWords:  pdefind lely endurants behaviours instantiation Axel
%%  LocalWords:  harmonisation harmonises Lamsweerde quantifications
%%  LocalWords:  PDEs formalised algorithmics optimisation Beki Dansk
%%  LocalWords:  Henhapl Datamatik Bruun MSc rgen Bundgaard Haff ren
%%  LocalWords:  Dommergaard vengreen Storbank Prehn Lennart DDC DSL
%%  LocalWords:  Actulus actuarian Cartesians Mereology Invariance sa
%%  LocalWords:  Initialisation PDE anoipilisy wellformedness UNU ddp
%%  LocalWords:  IIST formedness endurant calc apuid mereology xtime
%%  LocalWords:  wff perdurant Drs Chupilko RAS ShaoFa IoS CAS Oest
%%  LocalWords:  itute Acad emy ences ve
