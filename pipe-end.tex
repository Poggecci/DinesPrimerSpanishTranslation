
\nbbbbb{Endurantes: Cualidades Externas}\label{pipe-ex1}\tehran{Endurants}{pipe:Edurants}\HHHH 

Seguimos la ontología de la Fig.\,\ref{anoipilisy}, el cuadro de líneas discontinuas de la izquierda
etiquetado como \sfsl{Cualidades Externas}.

\hDBfigure{onto}{\pos{10.8}{10}cm}{Ontología Superior}{anoipilisy}

\mnewfoil

\nbbbb{Partes}\tehran{Parts}{pipe:Parts}\HHHH

\DBfigure{opls-edin}{\pos{5}{10}cm}{Ejemplo de un sistema de tuberías}{nabucco}

\mnewfoil

\tehrantutorial{\begin{multicols}{2}}{}\HHHH
\begin{enumerate}\setei
\item \label{p-e-00} Un sistema de tuberías contiene un conjunto de unidades de tuberías y un monitor del sistema de tuberías.
\item \label{p-e--1} La buena formación de un sistema de tuberías
  depende de su mereología\pos{ (cf.\ sec.\,\ref{PLS Mereology})}{}
  y el enrutamiento de sus tuberías\pos{ (cf.\ sec.\,\ref{Wellformed Pipes})}{}. 
\item \label{p-e-01} Una unidad de tubería puede ser un pozo, una tubería, una bomba,
  una válvula, una bifurcación, una unión, una placa\footnote{\LLLL Una unidad de \sfsl{placa} es una
  placa de acero generalmente circular y plana utilizada para ``comenzar'' o ``terminar'' un
  segmento de tubería.}, o una unidad de desagüe.
\item \label{p-e-02} Consideramos todas estas unidades como
  distinguibles, es decir, el conjunto de pozos, el conjunto de tuberías, etc., el conjunto
  de desagües son disjuntos.
\savei\end{enumerate} 
\tehrantutorial{\end{multicols}}{}

\mnewfoil\LLLL
\tehrantutorial{\begin{multicols}{2}\footnotesize}{\LLLL}
\bp
\kw{type}\\
\ref{p-e-00}.\ \ PLS{\PRIM}, U, M \iptye{PLS$'$}{p-e-00}\iptye{U}{p-e-00}\iptye{M}{p-e-00}\\
\ref{p-e--1}.\ \ PLS {\EQ} {\LBRACE}{\BAR} pls:PLS{\PRIM}{\RDOT}wf\_PLS(pls) {\BAR}{\RBRACE} \iptye{PLS}{p-e--1} \\
\kw{value}\\
\ref{p-e--1}.\ \ wf\_PLS: PLS {\RIGHTARROW} \kw{Bool} \ipwf{wf\_PLS}{p-e--1}\\
\ref{p-e--1}.\ \ wf\_PLS(pls) {\IS}\\
\ref{p-e--1}.\ \ wf\_Mereology(pls){\WEDGE}wf\_Routes(pls){\WEDGE}wf\_Metrics(pls)\footnotemark\\
\ref{p-e-00}.\ \ obs\_Us: PLS {\RIGHTARROW} U\kw{-set} \ipob{obs\_Us}{p-e-00}\\
\ref{p-e-00}.\ \ obs\_M: PLS {\RIGHTARROW} M \ipob{obs\_M}{p-e-00}\\
\kw{type}\\
\ref{p-e-01}.\ \ U {\EQ} We {\BAR} Pi {\BAR} Pu {\BAR} Va {\BAR} Fo {\BAR} Jo {\BAR} Pl {\BAR} Si\iptye{U}{p-e-01}\\
\ref{p-e-02}.\ \ We :: Well\iptye{We}{p-e-02}\\
\ref{p-e-02}.\ \ Pi :: Pipe\iptye{Pi}{p-e-02} \\
\ref{p-e-02}.\ \ Pu :: Pump\iptye{Pu}{p-e-02} \\
\ref{p-e-02}.\ \ Va :: Valv\iptye{Va}{p-e-02}\\
\ref{p-e-02}.\ \ Fo :: Fork\iptye{Fo}{p-e-02}\\
\ref{p-e-02}.\ \ Jo :: Join\iptye{Jo}{p-e-02}\\
\ref{p-e-02}.\ \ Pl :: Plate\iptye{Pl}{p-e-02}\\
\ref{p-e-02}.\ \ Si :: Sink\iptye{Si}{p-e-02}
\ep
\tehrantutorial{\end{multicols}}{}
\footnotetext{\LLLL%
  \textsf{wf\_Mereology}, 
  \textsf{wf\_Routes} y
  \textsf{wf\_Metrics} se explicarán en las secciones\,%
  \vref{pipe:wfMereology},
  \vref{pipe:wfRoutes} y
  \vref{pipe:Wellformed Unit Metrics}.}


\nbbbb{Un Estado Endurante}

\begin{enumerate}\setei
\item \label{end-state-000} Para un sistema de tuberías dado % unsure how to make this flow well. The original english sentence is rather confusing.
\item \label{end-state-010} ejemplificamos un estado endurante $\sigma$
\item \label{end-state-020} compuesto del sistema de tuberías dado y
                            todas sus unidades manifiestas, es decir, sin placas.
\savei\end{enumerate}

\bp
\kw{value}\\
\ref{end-state-000}.\ \ \ pls:PLS\ \ \ipva{pls}{end-state-000}\\
\kw{variable}\\
\ref{end-state-010}.\ \ \ $\sigma$ :{\EQ} collect\_state(pls)\ \ \ipst{$\sigma$}{end-state-010} \\
\kw{value} \\
\ref{end-state-020}.\ \ \ collect\_state: PLS\ \ \ \ipfu{collect\_state}{end-state-020}\\
\ref{end-state-020}.\ \ \ collect\_state(pls) {\IS} {\LBRACE}pls{\RBRACE}\,{\UNION}\,obs\_Us(pls) {\SETMINUS} Pl
\ep

\nbbbbb{Endurantes: Cualidades Internas}

\begynd
\pind Seguimos la ontología de la fig.\,\vref{anoipilisy}, \nyl las
líneas verticales y horizontales de la izquierda.
\afslut


\bbbb{Identificación Única}

\begin{enumerate}\setei
\item \label{p-e-03z} El sistema de tuberías, como tal,
\item \label{p-e-03x} tiene un identificador único, distinto de sus
                      identificadores de unidades de tuberías. 
\item \label{p-e-03y} Cada unidad de tubería se distingue de manera única por
                      su identificador de unidad. 
\item \label{p-e-03zz} Existe un estado de todos los identificadores únicos.
\savei\end{enumerate}
\mnewfoil
%\RSLatex
%   type
%&\ref{p-e-03x}.&      PLSI &\iptyu{PLSI}{p-e-03x}&
%&\ref{p-e-03y}.&      UI &\iptyu{UI}{p-e-03y}&
%   value
%&\ref{p-e-03z}.&      pls:PLS
%&\ref{p-e-03x}.&      uid_PLS: PLS -> PLSI&\ipob{uid\_PLS}{p-e-03x}&
%&\ref{p-e-03y}.&      uid_U: U -> UI&\ipob{uid\_U}{p-e-03y}&
%   variable
%&\ref{p-e-03zz}.&      `sigma&$_{uid}$& := { uid_PLS(pls) } union xtr_UIs(pls) &\ipst{$\sigma_{uid}$}{p-e-03z}&
%   axiom &\ipax{Unique Identification}{p-e-03y}&
%&\ref{p-e-03y}.&      all u,u':U:-{u,u'}<<=obs_Us(pls)=>(u~=u'=>uid_UI(u)~=uid_UI(u'))
%&\ref{p-e-03y}.&    /\  uid_PLS(pls) ~isin {uid_UI(u)|u:U:-u isin obs_Us(pls)}
%\endRSLatex 
\bp
\>\ \kw{type}\\
\ref{p-e-03x}.\ \ \ \ \ \ PLSI \iptyu{PLSI}{p-e-03x}\\
\ref{p-e-03y}.\ \ \ \ \ \ UI \iptyu{UI}{p-e-03y}\\
\>\ \kw{value}\\
\ref{p-e-03z}.\ \ \ \ \ \ pls:PLS\\
\ref{p-e-03x}.\ \ \ \ \ \ uid\_PLS: PLS {\RIGHTARROW} PLSI\ipob{uid\_PLS}{p-e-03x}\\
\ref{p-e-03y}.\ \ \ \ \ \ uid\_U: U {\RIGHTARROW} UI\ipob{uid\_U}{p-e-03y}\\
\>\ \kw{variable}\\
\ref{p-e-03zz}.\ \ \ \ \ \ $\sigma$$_{uid}$ :{\EQ} {\LBRACE} uid\_PLS(pls) {\RBRACE} {\UNION} xtr\_UIs(pls) \ipst{$\sigma_{uid}$}{p-e-03z}\\
\>\ \kw{axiom} \ipax{Unique Identification}{p-e-03y}\\
\ref{p-e-03y}.\ \ \ \ \ \ {\ALL} u,u{\PRIM}:U{\RDOT}{\LBRACE}u,u{\PRIM}{\RBRACE}{\SUBSETEQ}obs\_Us(pls){\DBLRIGHTARROW}(u{\NOTEQ}u{\PRIM}{\DBLRIGHTARROW}uid\_UI(u){\NOTEQ}uid\_UI(u{\PRIM}))\\
\ref{p-e-03y}.\ \ \ \ {\WEDGE}\ \ uid\_PLS(pls) {\NOTISIN} {\LBRACE}uid\_UI(u){\BAR}u:U{\RDOT}u {\ISIN} obs\_Us(pls){\RBRACE}
\ep

\mnewfoil
\begin{enumerate}\setei
\item \label{p-e-03a} De un sistema de tuberías uno puede observar 
el conjunto de todos los identificadores únicos de unidades.
\savei\end{enumerate}


%\RSLatex
%   value
%&\ref{p-e-03a}.&   xtr_UIs: PLS -> UI-set&\ipfu{xtr\_UIs}{p-e-03a}&
%&\ref{p-e-03a}.&   xtr_UIs(pls) is {uid_UI(u)|u:U:-u isin obs_Us(pls)}
%\endRSLatex 
\bp
\>\ \kw{value}\\
\ref{p-e-03a}.\ \ \ xtr\_UIs: PLS {\RIGHTARROW} UI\kw{-set}\ipfu{xtr\_UIs}{p-e-03a}\\
\ref{p-e-03a}.\ \ \ xtr\_UIs(pls) {\IS} {\LBRACE}uid\_UI(u){\BAR}u:U{\RDOT}u {\ISIN} obs\_Us(pls){\RBRACE}
\ep

\begin{enumerate}\setei
\item \label{p-e-03b} Podemos demostrar que el número de identificadores
                      únicos de unidades es igual al número de unidades de ese sistema.
\savei\end{enumerate}


%\RSLatex
%   &\kw{theorem:}\ipth{Unique Endurants}{p-e-03b}&
%&\ref{p-e-03b}.&     all pls:PLS:-card obs_Us(pl)=card xtr_UIs(pls)
%\endRSLatex 
\bp
\>\ \kw{theorem:}\ipth{Endurantes Únicos}{p-e-03b}\\
\ref{p-e-03b}.\ \ \ \ \ {\ALL} pls:PLS{\RDOT}\kw{card} obs\_Us(pl){\EQ}\kw{card} xtr\_UIs(pls)
\ep

\nbbbb{Mereología}\label{PLS Mereology}\LLLL\HHHH

\bbb{Mereología PLS}

\item \label{pls-mer-00}La mereología de un sistema de tuberías es el 
  conjunto de identificadores únicos de todas las unidades del sistema.
\begin{enumerate}\setei
\savei\end{enumerate}

%\RSLatex
%type
%&\ref{pls-mer-00}.&   PLS_Mer = UI-set&\ipty{PLS\_Mer}{pls-mer-00}&
%value
%&\ref{pls-mer-00}.&   mereo_PLS: PLS -> PLS_Mer&\ipob{mereo\_PLS}{pls-mer-00}&
%axiom&\ipty{Wellformed Mereologies}{pls-mer-00}&
%&\ref{pls-mer-00}.&   all uis:PLS_Mer :- uis = card xtr_UIs(pls)
%\endRSLatex 
\bp
\kw{type}\\
\ref{pls-mer-00}.\ \ \ PLS\_Mer {\EQ} UI\kw{-set}\ipty{PLS\_Mer}{pls-mer-00}\\
\kw{value}\\
\ref{pls-mer-00}.\ \ \ mereo\_PLS: PLS {\RIGHTARROW} PLS\_Mer\ipob{mereo\_PLS}{pls-mer-00}\\
\kw{axiom}\ipty{Wellformed Mereologies}{pls-mer-00}\\
\ref{pls-mer-00}.\ \ \ {\ALL} uis:PLS\_Mer {\RDOT} uis {\EQ} \kw{card} xtr\_UIs(pls)
\ep

\nbbb{Unit Mereologies}\HHHH\LLLL

\begin{enumerate}\setei
\item \label{p-e-11}   Each unit is connected to zero, one or two
                       other existing input units and  zero, one or two
                       other existing output units as follows:
\begin{enumerate}
\item \label{p-e-04}   A well unit is connected to exactly one output
                       unit (and, hence, has no ``input'').
\item \label{p-e-05}   A pipe unit is connected to exactly one input unit
                       and one output unit.
\item \label{p-e-06}   A pump unit is connected to exactly one input unit
                       and one output unit.
\item \label{p-e-07}   A valve is connected to exactly one input unit
                       and one output unit.
\item \label{p-e-08}   A fork is connected to exactly one input unit
                       and two distinct output units.
\item \label{p-e-09}   A join is connected to exactly two distinct input units
                       and one output unit.
\item \label{p-e-85}   A plate is connected to exactly one unit.
\item \label{p-e-10}   A sink is connected to exactly one input unit
                       (and, hence, has no ``output''). 
\end{enumerate}
\savei\end{enumerate}

\mnewfoil\LLLL

%\RSLatex
%   type
%&\ref{p-e-11}.&      MER = UI-set >< UI-set&\iptym{MER}{p-e-11}&
%   value
%&\ref{p-e-11}.&      mereo_U: U -> MER &\ipob{mereo\_U}{p-e-11}&
%   axiom
%&\ref{p-e-11}.&       wf_Mereology: PLS -> Bool &\label{pipe:wfMereology}\ipwf{wf\_Mereology}{p-e-11}&
%&\ref{p-e-11}.&       wf_Mereology(pls) is
%&\ref{p-e-11}.&          all u:U:-u isin obs_Us(pls)=> 
%&\ref{p-e-11}.&             let (iuis,ouis) = mereo_U(u) in iuis union ouis <<= xtr_UIs(pls) /\
%&\ref{p-e-11}.&                 case (u,(card uius,card ouis)) of
%&\ref{p-e-04}.&                     (mk_We(we),(0,1)) -> true,
%&\ref{p-e-05}.&                     (mk_Pi(pi),(1,1)) -> true,
%&\ref{p-e-06}.&                     (mk_Pu(pu),(1,1)) -> true,
%&\ref{p-e-07}.&                     (mk_Va(va),(1,1)) -> true,
%&\ref{p-e-08}.&                     (mk_Fo(fo),(1,1)) -> true,
%&\ref{p-e-09}.&                     (mk_Jo(jo),(1,1)) -> true,
%&\ref{p-e-09}.&                     (mk_Pl(pl),(0,1)) -> true, &``begin''&
%&\ref{p-e-09}.&                     (mk_Pl(pl),(1,0)) -> true, &``end''&
%&\ref{p-e-10}.&                     (mk_Si(si),(1,1)) -> true,
%&\ref{p-e-11}.&                     _ -> false end end
%\endRSLatex 
\bp
\>\ \kw{type}\\
\ref{p-e-11}.\ \ \ \ \ \ MER {\EQ} UI\kw{-set} {\TIMES} UI\kw{-set}\iptym{MER}{p-e-11}\\
\>\ \kw{value}\\
\ref{p-e-11}.\ \ \ \ \ \ mereo\_U: U {\RIGHTARROW} MER \ipob{mereo\_U}{p-e-11}\\
\>\ \kw{axiom}\\
\ref{p-e-11}.\ \ \ \ \ \ \ wf\_Mereology: PLS {\RIGHTARROW} \kw{Bool} \label{pipe:wfMereology}\ipwf{wf\_Mereology}{p-e-11}\\
\ref{p-e-11}.\ \ \ \ \ \ \ wf\_Mereology(pls) {\IS}\\
\ref{p-e-11}.\ \ \ \ \ \ \ \ \ \ {\ALL} u:U{\RDOT}u {\ISIN} obs\_Us(pls){\DBLRIGHTARROW} \\
\ref{p-e-11}.\ \ \ \ \ \ \ \ \ \ \ \ \ \kw{let} (iuis,ouis) {\EQ} mereo\_U(u) \kw{in} iuis {\UNION} ouis {\SUBSETEQ} xtr\_UIs(pls) {\WEDGE}\\
\ref{p-e-11}.\ \ \ \ \ \ \ \ \ \ \ \ \ \ \ \ \ \kw{case} (u,(\kw{card} uius,\kw{card} ouis)) \kw{of}\\
\ref{p-e-04}.\ \ \ \ \ \ \ \ \ \ \ \ \ \ \ \ \ \ \ \ \ (mk\_We(we),(0,1)) {\RIGHTARROW} \kw{true},\\
\ref{p-e-05}.\ \ \ \ \ \ \ \ \ \ \ \ \ \ \ \ \ \ \ \ \ (mk\_Pi(pi),(1,1)) {\RIGHTARROW} \kw{true},\\
\ref{p-e-06}.\ \ \ \ \ \ \ \ \ \ \ \ \ \ \ \ \ \ \ \ \ (mk\_Pu(pu),(1,1)) {\RIGHTARROW} \kw{true},\\
\ref{p-e-07}.\ \ \ \ \ \ \ \ \ \ \ \ \ \ \ \ \ \ \ \ \ (mk\_Va(va),(1,1)) {\RIGHTARROW} \kw{true},\\
\ref{p-e-08}.\ \ \ \ \ \ \ \ \ \ \ \ \ \ \ \ \ \ \ \ \ (mk\_Fo(fo),(1,1)) {\RIGHTARROW} \kw{true},\\
\ref{p-e-09}.\ \ \ \ \ \ \ \ \ \ \ \ \ \ \ \ \ \ \ \ \ (mk\_Jo(jo),(1,1)) {\RIGHTARROW} \kw{true},\\
\ref{p-e-09}.\ \ \ \ \ \ \ \ \ \ \ \ \ \ \ \ \ \ \ \ \ (mk\_Pl(pl),(0,1)) {\RIGHTARROW} \kw{true}, ``begin''\\
\ref{p-e-09}.\ \ \ \ \ \ \ \ \ \ \ \ \ \ \ \ \ \ \ \ \ (mk\_Pl(pl),(1,0)) {\RIGHTARROW} \kw{true}, ``end''\\
\ref{p-e-10}.\ \ \ \ \ \ \ \ \ \ \ \ \ \ \ \ \ \ \ \ \ (mk\_Si(si),(1,1)) {\RIGHTARROW} \kw{true},\\
\ref{p-e-11}.\ \ \ \ \ \ \ \ \ \ \ \ \ \ \ \ \ \ \ \ \ {\UNDERLINE} {\RIGHTARROW} \kw{false} \kw{end} \kw{end}
\ep

\nbbbb{Pipeline Concepts, I}

\bbb{Pipe Routes}\label{primer-pipe}\HHHH

\begin{enumerate}\setei
\item \label{p-e-12} A route (of a pipeline system) is a sequence of
                     connected units (of the pipeline system). 
\item \label{p-e-13} A route descriptor is a sequence of unit
                     identifiers and the connected units of a route
                     (of a pipeline system).
\savei\end{enumerate}

%\RSLatex
%   type
%&\ref{p-e-12}.&      R' = U-inflist &\iptyo{R$'$}{p-e-12}&
%&\ref{p-e-12}.&      R = {| r:Route':-wf_Route(r) |} &\iptyo{R}{p-e-12}&
%&\ref{p-e-13}.&      RD = UI-inflist &\iptyo{RD}{p-e-13}&
%   axiom
%&\ref{p-e-13}.&      all rd:RD :- exists r:R:-rd=descriptor(r) &\ipax{Route Describability}{p-e-13}&
%   value
%&\ref{p-e-13}.&      descriptor: R -> RD &\ipfu{descriptor}{p-e-13}&
%&\ref{p-e-13}.&      descriptor(r) is <.uid_UI(r[i])|i:Nat:-1<=i<=len r.>   
%\endRSLatex 
\bp
\>\ \kw{type}\\
\ref{p-e-12}.\ \ \ \ \ \ R{\PRIM} {\EQ} U$^{\omega}$ \iptyo{R$'$}{p-e-12}\\
\ref{p-e-12}.\ \ \ \ \ \ R {\EQ} {\LBRACE}{\BAR} r:Route{\PRIM}{\RDOT}wf\_Route(r) {\BAR}{\RBRACE} \iptyo{R}{p-e-12}\\
\ref{p-e-13}.\ \ \ \ \ \ RD {\EQ} UI$^{\omega}$ \iptyo{RD}{p-e-13}\\
\>\ \kw{axiom}\\
\ref{p-e-13}.\ \ \ \ \ \ {\ALL} rd:RD {\RDOT} {\EXISTS} r:R{\RDOT}rd{\EQ}descriptor(r) \ipax{Route Describability}{p-e-13}\\
\>\ \kw{value}\\
\ref{p-e-13}.\ \ \ \ \ \ descriptor: R {\RIGHTARROW} RD \ipfu{descriptor}{p-e-13}\\
\ref{p-e-13}.\ \ \ \ \ \ descriptor(r) {\IS} {\LANGLE}uid\_UI(r{\LBRACKET}i{\RBRACKET}){\BAR}i:\kw{Nat}{\RDOT}1{\LEQ}i{\LEQ}\kw{len} r{\RANGLE}\ \ \ 
\ep

\mnewfoil\LLLL\HHHH

\begin{enumerate}\setei
\item \label{p-e-20x}   Two units are adjacent if the output unit
  identifiers of one shares a unique unit identifier with the input
  identifiers of the other.
\savei\end{enumerate}

%\RSLatex
%   value 
%&\ref{p-e-20x}.&      adjacent: U >< U -> Bool &\ipfu{adjacent}{p-e-20x}&
%&\ref{p-e-20x}.&      adjacent(u,u') is let (,ouis)=mereo_U(u),(iuis,)=mereo_U(u') in ouis inter iuis ~= {} end
%\endRSLatex 
\bp
\>\ \kw{value} \\
\ref{p-e-20x}.\ \ \ \ \ \ adjacent: U {\TIMES} U {\RIGHTARROW} \kw{Bool} \ipfu{adjacent}{p-e-20x}\\
\ref{p-e-20x}.\ \ \ \ \ \ adjacent(u,u{\PRIM}) {\IS} \kw{let} (,ouis){\EQ}mereo\_U(u),(iuis,){\EQ}mereo\_U(u{\PRIM}) \kw{in} ouis {\INTER} iuis {\NOTEQ} {\LBRACE}{\RBRACE} \kw{end}
\ep

\LLLL

\begin{enumerate}\setei
\item \label{p-e-20a}   Given a pipeline system, $pls$, one can identify the
                  (possibly infinite) set of (possibly infinite) routes
                  of that pipeline system.
\begin{enumerate}
\item \label{p-e-20}  The empty sequence, {\LANGLE}{\RANGLE}, is a
                      route of $pls$.  
\item \label{p-e-21}  Let $u, u'$ be any units of $pls$, such that
                      an output unit identifier of $u$ is the same as
                      an input unit identifier of $u'$ then
                      {\LANGLE}$u,u'${\RANGLE} is a route of $pls$.
\item \label{p-e-22}  If $r$ and $r'$ are routes of $pls$ such that
                      the last element of $r$ is the same as the first
                      element of $r'$, then $r${\CONCAT}\kw{tl}$r'$ is
                      a route of $pls$.  
\item \label{p-e-23}  No sequence of units is a route unless it
                      follows from a finite (or an infinite) number of
                      applications of the basis and induction clauses
                      of Items~\ref{p-e-20}--\ref{p-e-22}.
\end{enumerate}
\savei\end{enumerate}

\mnewfoil\LLLL\HHHH

%\RSLatex 
%   value
%&\ref{p-e-20a}.&      Routes: PLS -> RD-infset&\ipfu{Routes}{p-e-20a}&
%&\ref{p-e-20a}.&      Routes(pls) is 
%&\ref{p-e-20}.&          let rs = <..> union 
%&\ref{p-e-21}.&                   {<.uid_UI(u),uid_UI(u').>|u,u':U:-{u,u'}<<=obs_Us(pls) /\ adjacent(u,u')}
%&\ref{p-e-22}.&                 union {r^tl r'|r,r':R:-{r,r'}<<=rs}
%&\ref{p-e-23}.&          in rs end
%\endRSLatex 
\bp
\>\ \kw{value}\\
\ref{p-e-20a}.\ \ \ \ \ \ Routes: PLS {\RIGHTARROW} RD\kw{-infset}\ipfu{Routes}{p-e-20a}\\
\ref{p-e-20a}.\ \ \ \ \ \ Routes(pls) {\IS} \\
\ref{p-e-20}.\ \ \ \ \ \ \ \ \ \ \kw{let} rs {\EQ} {\LANGLE}{\RANGLE} {\UNION} \\
\ref{p-e-21}.\ \ \ \ \ \ \ \ \ \ \ \ \ \ \ \ \ \ \ {\LBRACE}{\LANGLE}uid\_UI(u),uid\_UI(u{\PRIM}){\RANGLE}{\BAR}u,u{\PRIM}:U{\RDOT}{\LBRACE}u,u{\PRIM}{\RBRACE}{\SUBSETEQ}obs\_Us(pls) {\WEDGE} adjacent(u,u{\PRIM}){\RBRACE}\\
\ref{p-e-22}.\ \ \ \ \ \ \ \ \ \ \ \ \ \ \ \ \ {\UNION} {\LBRACE}r{\CONCAT}\kw{tl} r{\PRIM}{\BAR}r,r{\PRIM}:R{\RDOT}{\LBRACE}r,r{\PRIM}{\RBRACE}{\SUBSETEQ}rs{\RBRACE}\\
\ref{p-e-23}.\ \ \ \ \ \ \ \ \ \ \kw{in} rs \kw{end}
\ep

\bbb{Well-formed Routes}\label{Wellformed Pipes}

\begin{enumerate}\setei
\item \label{p-e-14a}  A route is acyclic if no two route positions
  reveal the same unique unit identifier.
\savei\end{enumerate}
%\RSLatex
%   value
%&\ref{p-e-14a}.&   is_acyclic_Route: R -> Bool&\ippr{is\_acyclic\_Route}{p-e-14a}&
%&\ref{p-e-14a}.&   is_acyclic_Route(r) is ~exists i,j:Nat:-{i,j}<<=inds r /\ i~=j /\ r[i]=r[j]
%\endRSLatex 
\bp
\>\ \kw{value}\\
\ref{p-e-14a}.\ \ \ is\_acyclic\_Route: R {\RIGHTARROW} \kw{Bool}\ippr{is\_acyclic\_Route}{p-e-14a}\\
\ref{p-e-14a}.\ \ \ is\_acyclic\_Route(r) {\IS} {\SIM}{\EXISTS} i,j:\kw{Nat}{\RDOT}{\LBRACE}i,j{\RBRACE}{\SUBSETEQ}\kw{inds} r {\WEDGE} i{\NOTEQ}j {\WEDGE} r{\LBRACKET}i{\RBRACKET}{\EQ}r{\LBRACKET}j{\RBRACKET}
\ep

\mnewfoil

\begin{enumerate}\setei
\item \label{p-e-14b}  A pipeline system is well-formed if none of its
                      routes are circular (and all of its routes
                      embedded in well-to-sink routes). 
\savei\end{enumerate}
%\RSLatex
%   value
%&\ref{p-e-14b}.&   wf_Routes: PLS -> Bool&\ipwf{wf\_Routes}{p-e-14b}&
%&\ref{p-e-14b}.&   wf_Routes(pls) is  &\label{pipe:wfRoutes}&
%&\ref{p-e-14b}.&       non_circular(pls) /\ are_embedded_Routes(pls)
%
%&\ref{p-e-14b}.&   is_non_circular_PLS: PLS -> Bool&\ipwf{is\_non\_circular\_PLS}{p-e-14b}&
%&\ref{p-e-14b}.&   is_non_circular_PLS(pls) is 
%&\ref{p-e-14b}.&       all r:R:-r isin routes(p)/\acyclic_Route(r)
%\endRSLatex 
\bp
\>\ \kw{value}\\
\ref{p-e-14b}.\ \ \ wf\_Routes: PLS {\RIGHTARROW} \kw{Bool}\ipwf{wf\_Routes}{p-e-14b}\\
\ref{p-e-14b}.\ \ \ wf\_Routes(pls) {\IS}\ \ \label{pipe:wfRoutes}\\
\ref{p-e-14b}.\ \ \ \ \ \ \ non\_circular(pls) {\WEDGE} are\_embedded\_Routes(pls)\\
\\
\ref{p-e-14b}.\ \ \ is\_non\_circular\_PLS: PLS {\RIGHTARROW} \kw{Bool}\ipwf{is\_non\_circular\_PLS}{p-e-14b}\\
\ref{p-e-14b}.\ \ \ is\_non\_circular\_PLS(pls) {\IS} \\
\ref{p-e-14b}.\ \ \ \ \ \ \ {\ALL} r:R{\RDOT}r {\ISIN} routes(p){\WEDGE}acyclic\_Route(r)
\ep

\mnewfoil

\begin{enumerate}\setei
\item \label{p-e-19}  We define well-formedness in terms of
  well-to-sink routes, i.e., routes which start with a well unit and
  end with a sink unit.
\savei\end{enumerate}

%\RSLatex
%   value   
%&\ref{p-e-19}.&   well_to_sink_Routes: PLS -> R-set&\ipfu{well\_to\_sink\_Routes}{p-e-19}&
%&\ref{p-e-19}.&   well_to_sink_Routes(pls) is
%&\ref{p-e-19}.&      let rs = Routes(pls) in
%&\ref{p-e-19}.&      {r|r:R:-r isin rs /\ is_We(r[1]) /\ is_Si(r[len r])} end
%\endRSLatex 
\bp
\>\ \kw{value}\ \ \ \\
\ref{p-e-19}.\ \ \ well\_to\_sink\_Routes: PLS {\RIGHTARROW} R\kw{-set}\ipfu{well\_to\_sink\_Routes}{p-e-19}\\
\ref{p-e-19}.\ \ \ well\_to\_sink\_Routes(pls) {\IS}\\
\ref{p-e-19}.\ \ \ \ \ \ \kw{let} rs {\EQ} Routes(pls) \kw{in}\\
\ref{p-e-19}.\ \ \ \ \ \ {\LBRACE}r{\BAR}r:R{\RDOT}r {\ISIN} rs {\WEDGE} is\_We(r{\LBRACKET}1{\RBRACKET}) {\WEDGE} is\_Si(r{\LBRACKET}\kw{len} r{\RBRACKET}){\RBRACE} \kw{end}
\ep

\mnewfoil

\begin{enumerate}\setei
\item \label{p-e-18} A pipeline system is well-formed
      if all of its routes are embedded in well-to-sink routes.
\savei\end{enumerate}

%\RSLatex
%&\ref{p-e-18}.&   are_embedded_Routes: PLS -> Bool&\ippr{are\_embedded\_Routes}{p-e-18}&
%&\ref{p-e-18}.&   are_embedded_Routes(pls) is   
%&\ref{p-e-18}.&       let wsrs = well_to_sink_Routes(pls) in
%&\ref{p-e-18}.&       all r:R :- r isin Routes(pls) =>
%&\ref{p-e-18}.&           exists r':R,i,j:Nat :- 
%&\ref{p-e-18}.&              r' isin wsrs /\ {i,j}<<=inds r'/\i<=j /\ r = <.r'[k]|k:Nat:-i<=k<=j.> end
%\endRSLatex 
\bp
\ref{p-e-18}.\ \ \ are\_embedded\_Routes: PLS {\RIGHTARROW} \kw{Bool}\ippr{are\_embedded\_Routes}{p-e-18}\\
\ref{p-e-18}.\ \ \ are\_embedded\_Routes(pls) {\IS}\ \ \ \\
\ref{p-e-18}.\ \ \ \ \ \ \ \kw{let} wsrs {\EQ} well\_to\_sink\_Routes(pls) \kw{in}\\
\ref{p-e-18}.\ \ \ \ \ \ \ {\ALL} r:R {\RDOT} r {\ISIN} Routes(pls) {\DBLRIGHTARROW}\\
\ref{p-e-18}.\ \ \ \ \ \ \ \ \ \ \ {\EXISTS} r{\PRIM}:R,i,j:\kw{Nat} {\RDOT} \\
\ref{p-e-18}.\ \ \ \ \ \ \ \ \ \ \ \ \ \ r{\PRIM} {\ISIN} wsrs {\WEDGE} {\LBRACE}i,j{\RBRACE}{\SUBSETEQ}\kw{inds} r{\PRIM}{\WEDGE}i{\LEQ}j {\WEDGE} r {\EQ} {\LANGLE}r{\PRIM}{\LBRACKET}k{\RBRACKET}{\BAR}k:\kw{Nat}{\RDOT}i{\LEQ}k{\LEQ}j{\RANGLE} \kw{end}
\ep

\bbb{Embedded Routes}

\begin{enumerate}\setei
\item \label{p-e-33a} For every route we can define the set of all its
                      embedded routes. 
\savei\end{enumerate} 

%\RSLatex
%   value
%&\ref{p-e-33a}.&  embedded_Routes: R -> R-set&\ipfu{embedded\_Routes}{p-e-33a}&
%&\ref{p-e-33a}.&  embedded_Routes(r) is {<.r[k]|k:Nat:-i<=k<=j.> | i,j:Nat:- i {i,j}<<=inds(r) /\ i<=j}  
%\endRSLatex 
\bp
\>\ \kw{value}\\
\ref{p-e-33a}.\ \ embedded\_Routes: R {\RIGHTARROW} R\kw{-set}\ipfu{embedded\_Routes}{p-e-33a}\\
\ref{p-e-33a}.\ \ embedded\_Routes(r) {\IS} {\LBRACE}{\LANGLE}r{\LBRACKET}k{\RBRACKET}{\BAR}k:\kw{Nat}{\RDOT}i{\LEQ}k{\LEQ}j{\RANGLE} {\BAR} i,j:\kw{Nat}{\RDOT} i {\LBRACE}i,j{\RBRACE}{\SUBSETEQ}\kw{inds}(r) {\WEDGE} i{\LEQ}j{\RBRACE}\ \ 
\ep

\nbbb{A Theorem}

\begin{enumerate}\setei
\item \label{p-e-33b} The following theorem is conjectured:
\begin{enumerate}
\item \label{p-e-30}  the set of all routes  (of the pipeline system)
\item \label{p-e-31}  is the set of all well-to-sink routes (of a
                      pipeline system) and 
\item \label{p-e-32}  all their embedded routes
\end{enumerate}
\savei\end{enumerate}

%\RSLatex
%   &\kw{theorem:}\ipth{Routes of a PLS}{p-e-33b}&
%&\ref{p-e-33b}.&   all pls:PLS :-
%&\ref{p-e-33b}.&   let rs = Routes(pls),
%&\ref{p-e-33b}.&       wsrs = well_to_sink_Routes(pls) in
%&\ref{p-e-30}.&   rs = 
%&\ref{p-e-31}.&        wsrs union
%&\ref{p-e-32}.&        union {{r'|r':R :- r' isin is_embedded_Routes(r'')} | r'':R :- r'' isin wsrs}
%&\ref{p-e-33a}.&   end
%\endRSLatex 
\bp
\>\ \kw{theorem:}\ipth{Routes of a PLS}{p-e-33b}\\
\ref{p-e-33b}.\ \ \ {\ALL} pls:PLS {\RDOT}\\
\ref{p-e-33b}.\ \ \ \kw{let} rs {\EQ} Routes(pls),\\
\ref{p-e-33b}.\ \ \ \ \ \ \ wsrs {\EQ} well\_to\_sink\_Routes(pls) \kw{in}\\
\ref{p-e-30}.\ \ \ rs {\EQ} \\
\ref{p-e-31}.\ \ \ \ \ \ \ \ wsrs {\UNION}\\
\ref{p-e-32}.\ \ \ \ \ \ \ \ {\UNION} {\LBRACE}{\LBRACE}r{\PRIM}{\BAR}r{\PRIM}:R {\RDOT} r{\PRIM} {\ISIN} is\_embedded\_Routes(r{\PRIM}{\PRIM}){\RBRACE} {\BAR} r{\PRIM}{\PRIM}:R {\RDOT} r{\PRIM}{\PRIM} {\ISIN} wsrs{\RBRACE}\\
\ref{p-e-33a}.\ \ \ \kw{end}
\ep

\nbbb{Fluids}

\begin{enumerate}\setei
\item \label{p-e-34} The only fluid of concern to pipelines is the
                     gas\footnote{\LLLL Gaseous materials include: air, gas,
                     etc.} or liquid\footnote{\LLLL Liquid materials include
                     water, oil, etc.} which the pipes
                     transport\footnote{\LLLL The description of this
                     document is relevant only to gas or oil pipelines.}.
\savei\end{enumerate}

%\RSLatex
%   type
%&\ref{p-e-34}.&      GoL [ = M ] &\iptye{GoL}{p-e-34}&
%   value
%&\ref{p-e-34}.&      obs_GoL: U -> GoL&\ipob{obs\_GoL}{p-e-34}&  
%\endRSLatex 
\bp
\>\ \kw{type}\\
\ref{p-e-34}.\ \ \ \ \ \ GoL {\LBRACKET} {\EQ} M {\RBRACKET} \iptye{GoL}{p-e-34}\\
\>\ \kw{value}\\
\ref{p-e-34}.\ \ \ \ \ \ obs\_GoL: U {\RIGHTARROW} GoL\ipob{obs\_GoL}{p-e-34}\ \ 
\ep

\nbbbb{Attributes}\LLLL

\bbb{Unit Flow Attributes}\HHHH


\begin{enumerate}\setei
\item A number of attribute types characterise units:
\begin{enumerate}
\item \label{pa01x} estimated current well capacity (barrels of oil, etc.),
\item \label{pa03x} pump height (a static attribute), 
\item \label{pa04x} current pump status (not pumping, pumping; a programmable attribute),
\item \label{pa05x} current valve status (closed, open; a programmable attribute) and
\item \label{pa06x} flow (barrels/second, a biddable attribute).
\end{enumerate}
\savei\end{enumerate}

\mnewfoil\LLLL\HHHH%&$_{\mathcall{L}}$&

%\begin{multicols}{2}
%\RSLatex
%   type
%&\ref{pa01x}.&      WellCap &\iptya{WellCap}{pa01x}&
%&\ref{pa03x}.&      Pump_Height &\iptya{Pump\_Height}{pa03x}&
%&\ref{pa04x}.&      Pump_State == {|&\sort{not\_pumping}&,&\sort{pumping}&|} &\iptya{Pump\_State}{pa04x}&
%&\ref{pa05x}.&      Valve_State == {|&\sort{closed}&,&\sort{open}&|} &\iptya{Valve\_State}{pa05x}&
%&\ref{pa06x}.&      Flow &\iptya{Flow}{pa06x}&
%\endRSLatex 
\bp
\>\ \kw{type}\\
\ref{pa01x}.\ \ \ \ \ \ WellCap \iptya{WellCap}{pa01x}\\
\ref{pa03x}.\ \ \ \ \ \ Pump\_Height \iptya{Pump\_Height}{pa03x}\\
\ref{pa04x}.\ \ \ \ \ \ Pump\_State {\EQ}{\EQ} {\LBRACE}{\BAR}\sort{not\_pumping},\sort{pumping}{\BAR}{\RBRACE} \iptya{Pump\_State}{pa04x}\\
\ref{pa05x}.\ \ \ \ \ \ Valve\_State {\EQ}{\EQ} {\LBRACE}{\BAR}\sort{closed},\sort{open}{\BAR}{\RBRACE} \iptya{Valve\_State}{pa05x}\\
\ref{pa06x}.\ \ \ \ \ \ Flow \iptya{Flow}{pa06x}
\ep
\begin{enumerate}\setei
\item \label{pa05y1} Flows can be added and subtracted,
\item \label{pa05y2} added distributively  and
\item \label{pa05z} flows can be compared.
\savei\end{enumerate}

%\RSLatex
%   value
%&\ref{pa05y1}.&      &$\oplus,\ominus$&: Flow><Flow -> Flow &\ipop{$\oplus$}{pa05y1}\ipop{$\ominus$}{pa05y1}&
%&\ref{pa05y2}.&      &$\oplus$&: Flow-set -> Flow  &\ipop{$\oplus$}{pa05y2}&
%&\ref{pa05z}.&      <,<=,=,~=,>=,>: Flow >< Flow -> Bool
%\endRSLatex 
\bp
\>\ \kw{value}\\
\ref{pa05y1}.\ \ \ \ \ \ $\oplus,\ominus$: Flow{\TIMES}Flow {\RIGHTARROW} Flow \ipop{$\oplus$}{pa05y1}\ipop{$\ominus$}{pa05y1}\\
\ref{pa05y2}.\ \ \ \ \ \ $\oplus$: Flow\kw{-set} {\RIGHTARROW} Flow\ \ \ipop{$\oplus$}{pa05y2}\\
\ref{pa05z}.\ \ \ \ \ \ {\LT},{\LEQ},{\EQ},{\NOTEQ},{\GEQ},{\GT}: Flow {\TIMES} Flow {\RIGHTARROW} \kw{Bool}
\ep
\ipop{\LT}{pa05z}%
\ipop{\LEQ}{pa05z}%
\ipop{\EQ}{pa05z}%
\ipop{\NOTEQ}{pa05z}%
\ipop{\GEQ}{pa05z}%
\ipop{\GT}{pa05z}
\mnewfoil
\begin{enumerate}\setei
\item \label{pa00} Properties of pipeline units include
%%%\begin{multicols}{2}
\begin{enumerate}
\item \label{pa01} estimated current well capacity (barrels of oil,
  etc.) [a biddable attribute], 
\item \label{pa02} pipe length [a static attribute],
\item \label{pa03} current pump height [a biddable attribute], 
\item \label{pa04} current valve open/close status [a programmable attribute],
\item \label{pa05} current [$\mathcal{L}$aminar] in-flow at unit input
                   [a monitorable attribute],
\item \label{pa06} current $\mathcal{L}$aminar] in-flow leak at unit input
                   [a monitorable attribute],
\item \label{pa07} maximum [$\mathcal{L}$aminar] guaranteed in-flow leak at unit input
  [a static attribute],
\mnewfoil
\item \label{pa08} current [$\mathcal{L}$aminar] leak unit interior
                   [a monitorable attribute],
\item \label{pa09} current [$\mathcal{L}$aminar] flow in unit interior
                   [a monitorable attribute],
\item \label{pa10} maximum $\mathcal{L}$aminar] guaranteed flow in unit interior
                   [a monitorable attribute],
\item \label{pa11} current [$\mathcal{L}$aminar] out-flow at unit output
                   [a monitorable attribute],
\item \label{pa12} current [$\mathcal{L}$aminar] out-flow leak at unit output
                   [a monitorable attribute] and
\item \label{pa13} maximum guaranteed $\mathcal{L}$aminar out-flow leak at unit output
                   [a static attribute.
\end{enumerate}
%%%\end{multicols}
\savei\end{enumerate}
\mnewfoil
\begin{multicols}{2}\LLLL
%\RSLatex
%type 
%&\ref{pa05}&  In_Flow = Flow&\iptya{In\_Flow}{pa05}&
%&\ref{pa06}&  In_Leak = Flow&\iptya{In\_Leak}{pa06}&
%&\ref{pa07}&  Max_In_Leak = Flow&\iptya{Max\_In\_Leak}{pa07}&
%&\ref{pa08}&  Body_Flow = Flow&\iptya{Body\_Flow}{pa08}&
%&\ref{pa09}&   Body_Leak = Flow&\iptya{Body\_Leak}{pa09}&
%&\ref{pa10}&   Max_Flow = Flow&\iptya{Max\_Flow}{pa10}&
%&\ref{pa11}&  Out_Flow = Flow&\iptya{Out\_Flow}{pa11}&
%&\ref{pa12}&   Out_Leak = Flow&\iptya{Out\_Leak}{pa12}& 
%&\ref{pa13}&  Max_Out_Leak = Flow&\iptya{Max\_Out\_Leak}{pa13}&
%value
%&\ref{pa01}&  attr_WellCap: We -> WellCap    
%&\ref{pa02}&  attr_LEN: Pi -> LEN   
%&\ref{pa03}&  attr_Height: Pu -> Height
%&\ref{pa04}&  attr_ValSta: Va -> VaSta
%&\ref{pa05}&  attr_In_Flow: U -> UI -> Flow&\ipob{attr\_In\_Flow}{pa05}&
%&\ref{pa06}&  attr_In_Leak: U -> UI -> Flow &\ipob{attr\_In\_Leak}{pa06}&  
%&\ref{pa07}&  attr_Max_In_Leak: U -> UI -> Flow &\ipob{attr\_Max\_In\_Leak}{pa07}&  
%&\ref{pa08}&  attr_Body_Flow: U -> Flow &\ipob{attr\_Body\_Flow}{pa08}&    
%&\ref{pa09}&   attr_Body_Leak: U -> Flow  &\ipob{attr\_Body\_Leak}{pa09}&
%&\ref{pa10}&   attr_Max_Flow: U -> Flow  &\ipob{attr\_Max\_Flow}{pa10}&
%&\ref{pa11}&  attr_Out_Flow: U -> UI -> Flow  &\ipob{attr\_Out\_Flow}{pa11}&
%&\ref{pa12}&   attr_Out_Leak: U -> UI -> Flow &\ipob{attr\_Out\_Leak}{pa12}& 
%&\ref{pa13}&  attr_Max_Out_Leak: U -> UI -> Flow&\ipob{attr\_Max\_Out\_Leak}{pa13}&
%\endRSLatex 
\bp
\kw{type} \\
\ref{pa05}\ \ In\_Flow {\EQ} Flow\iptya{In\_Flow}{pa05}\\
\ref{pa06}\ \ In\_Leak {\EQ} Flow\iptya{In\_Leak}{pa06}\\
\ref{pa07}\ \ Max\_In\_Leak {\EQ} Flow\iptya{Max\_In\_Leak}{pa07}\\
\ref{pa08}\ \ Body\_Flow {\EQ} Flow\iptya{Body\_Flow}{pa08}\\
\ref{pa09}\ \ \ Body\_Leak {\EQ} Flow\iptya{Body\_Leak}{pa09}\\
\ref{pa10}\ \ \ Max\_Flow {\EQ} Flow\iptya{Max\_Flow}{pa10}\\
\ref{pa11}\ \ Out\_Flow {\EQ} Flow\iptya{Out\_Flow}{pa11}\\
\ref{pa12}\ \ \ Out\_Leak {\EQ} Flow\iptya{Out\_Leak}{pa12} \\
\ref{pa13}\ \ Max\_Out\_Leak {\EQ} Flow\iptya{Max\_Out\_Leak}{pa13}\\
\kw{value}\\
\ref{pa01}\ \ attr\_WellCap: We {\RIGHTARROW} WellCap\ \ \ \ \\
\ref{pa02}\ \ attr\_LEN: Pi {\RIGHTARROW} LEN\ \ \ \\
\ref{pa03}\ \ attr\_Height: Pu {\RIGHTARROW} Height\\
\ref{pa04}\ \ attr\_ValSta: Va {\RIGHTARROW} VaSta\\
\ref{pa05}\ \ attr\_In\_Flow: U {\RIGHTARROW} UI {\RIGHTARROW} Flow\ipob{attr\_In\_Flow}{pa05}\\
\ref{pa06}\ \ attr\_In\_Leak: U {\RIGHTARROW} UI {\RIGHTARROW} Flow \ipob{attr\_In\_Leak}{pa06}\ \ \\
\ref{pa07}\ \ attr\_Max\_In\_Leak: U {\RIGHTARROW} UI {\RIGHTARROW} Flow \ipob{attr\_Max\_In\_Leak}{pa07}\ \ \\
\ref{pa08}\ \ attr\_Body\_Flow: U {\RIGHTARROW} Flow \ipob{attr\_Body\_Flow}{pa08}\ \ \ \ \\
\ref{pa09}\ \ \ attr\_Body\_Leak: U {\RIGHTARROW} Flow\ \ \ipob{attr\_Body\_Leak}{pa09}\\
\ref{pa10}\ \ \ attr\_Max\_Flow: U {\RIGHTARROW} Flow\ \ \ipob{attr\_Max\_Flow}{pa10}\\
\ref{pa11}\ \ attr\_Out\_Flow: U {\RIGHTARROW} UI {\RIGHTARROW} Flow\ \ \ipob{attr\_Out\_Flow}{pa11}\\
\ref{pa12}\ \ \ attr\_Out\_Leak: U {\RIGHTARROW} UI {\RIGHTARROW} Flow \ipob{attr\_Out\_Leak}{pa12} \\
\ref{pa13}\ \ attr\_Max\_Out\_Leak: U {\RIGHTARROW} UI {\RIGHTARROW} Flow\ipob{attr\_Max\_Out\_Leak}{pa13}
\ep
\end{multicols}

\mnewfoil\HHHH
\begin{enumerate}\setei
\item \label{pa13xy} Summarising we can define a two notions of flow:
\begin{enumerate}
\item \label{pa13x} static and
\item \label{pa13xz} monitorable.
\savei\end{enumerate}
\savei\end{enumerate}\footnotesize\small\LLLL
%\RSLatex   
%type
%&\ref{pa13x}&  Sta_Flows = Max_In_Leak><In_Max_Flow>Max_Out_Leak&\iptya{Sta\_Flows}{pa13x}&
%&\ref{pa13xz}&  Mon_Flows = In_Flow><In_Leak><Body_Flow><Body_Leak><Out_Flow><Out_Leak&\iptya{Mon\_Flows}{pa13xz}&
%\endRSLatex 
\bp
\kw{type}\\
\ref{pa13x}\ \ Sta\_Flows {\EQ} Max\_In\_Leak{\TIMES}In\_Max\_Flow{\GT}Max\_Out\_Leak\iptya{Sta\_Flows}{pa13x}\\
\ref{pa13xz}\ \ Mon\_Flows {\EQ} In\_Flow{\TIMES}In\_Leak{\TIMES}Body\_Flow{\TIMES}Body\_Leak{\TIMES}Out\_Flow{\TIMES}Out\_Leak\iptya{Mon\_Flows}{pa13xz}
\ep
\normalsize
\mnewfoil

\nbbb{Unit Metrics}


\label{wf-pls-metrics}
\begynd
\pind Pipelines are laid out in the terrain.
\begynd
\pind Units have length and diameters.
\pind Units are positioned in space: have altitude, longitude and
      latitude positions of its one, two or three
      connection PoinTs\footnote{1 for \sfsl{well}s, \sfsl{plate}s and \sfsl{sink}s; 2 for \sfsl{pipe}s,
      \sfsl{pump}s and \sfsl{valve}s; 1+2 for \sfsl{fork}s, 2+1 for \sfsl{join}s.}. 
\afslut
\afslut

%\begin{multicols}{2}
\begin{enumerate}\setei
\item \label{metrics-0000} length (a static attribute),
\item \label{metrics-0100} diameter (a static attribute) and
\item \label{metrics-0200} position (a static attribute).
\savei\end{enumerate}
                                %\end{multicols}
\normalsize\mnewfoil\HHHH

%\begin{multicols}{2}
%\RSLatex
%type
%&\ref{metrics-0000}.&   LEN&\iptya{LEN}{metrics-0000}&
%&\ref{metrics-0100}.&   &$\bigcirc$\iptya{$\bigcirc$}{metrics-0100}&
%&\ref{metrics-0200}.&   POS == mk_One(pt:PT) | mk_Two(ipt:PT,opt:PT)&\iptya{POS}{metrics-0200}&
%&\ref{metrics-0200}.&             | mk_OneTwo(ipt:PT,opts:(lpt:PT,rpt:PT)) 
%&\ref{metrics-0200}.&             | mk_TwoOne(ipts:(lpt:PT,rpt:PT),opt:PT)  
%&\ref{metrics-0200}.&   PT = Alt >< Lon >< Lat  &\iptya{PT}{metrics-0200}&
%&\ref{metrics-0200}.&   Alt, Lon, Lat = ...&\iptya{Alt}{metrics-0200}\iptya{Lon}{metrics-0200}\iptya{Lat}{metrics-0200}&
%value
%&\ref{metrics-0000}.&   attr_LEN: U -> LEN&\ipob{attr\_LEN}{metrics-0000}&
%&\ref{metrics-0100}.&   attr_&$\bigcirc$&: U -> &$\bigcirc$\ipob{attr\_$\bigcirc$}{metrics-0100}&
%&\ref{metrics-0200}.&   attr_POS: U -> POS&\ipob{attr\_POS}{metrics-0200}&
%\endRSLatex 
\bp
\kw{type}\\
\ref{metrics-0000}.\ \ \ LEN\iptya{LEN}{metrics-0000}\\
\ref{metrics-0100}.\ \ \ $\bigcirc$\iptya{$\bigcirc$}{metrics-0100}\\
\ref{metrics-0200}.\ \ \ POS {\EQ}{\EQ} mk\_One(pt:PT) {\BAR} mk\_Two(ipt:PT,opt:PT)\iptya{POS}{metrics-0200}\\
\ref{metrics-0200}.\ \ \ \ \ \ \ \ \ \ \ \ \ {\BAR} mk\_OneTwo(ipt:PT,opts:(lpt:PT,rpt:PT)) \\
\ref{metrics-0200}.\ \ \ \ \ \ \ \ \ \ \ \ \ {\BAR} mk\_TwoOne(ipts:(lpt:PT,rpt:PT),opt:PT)\ \ \\
\ref{metrics-0200}.\ \ \ PT {\EQ} Alt {\TIMES} Lon {\TIMES} Lat\ \ \iptya{PT}{metrics-0200}\\
\ref{metrics-0200}.\ \ \ Alt, Lon, Lat {\EQ} {\DOTDOTDOT}\iptya{Alt}{metrics-0200}\iptya{Lon}{metrics-0200}\iptya{Lat}{metrics-0200}\\
\kw{value}\\
\ref{metrics-0000}.\ \ \ attr\_LEN: U {\RIGHTARROW} LEN\ipob{attr\_LEN}{metrics-0000}\\
\ref{metrics-0100}.\ \ \ attr\_$\bigcirc$: U {\RIGHTARROW} $\bigcirc$\ipob{attr\_$\bigcirc$}{metrics-0100}\\
\ref{metrics-0200}.\ \ \ attr\_POS: U {\RIGHTARROW} POS\ipob{attr\_POS}{metrics-0200}
\ep
%\end{multicols}
\mnewfoil\normalsize\HHHH
\noindent
\begynd
\pind We can summarise the metric attributes:
\afslut
\begin{enumerate}\setei
\item \label{metrics-0500} Units are subject to either of four
  (mutually exclusive) metrics:
\begin{enumerate}
\item \label{metrics-0510} Length, diameter and a one point position.  
\item \label{metrics-0520} Length, diameter and a two points position. 
\item \label{metrics-0530} Length, diameter and a one+two points position. 
\item \label{metrics-0540} Length, diameter and a two+one points position. 
\end{enumerate}
\savei\end{enumerate}
\mnewfoil\normalsize\HHHH
%\RSLatex
%type
%&\ref{metrics-0500}.&   Unit_Sta = Sta1_Metric | Sta2_Metric | Sta12_Metric | Sta21_Metric&\iptya{Unit\_Sta}{metrics-0500}&  
%&\ref{metrics-0510}&  Sta1_Metric = LEN >< &{\O}& >< mk_One(pt:PT)&\iptya{Sta1\_Metric}{metrics-0510}& 
%&\ref{metrics-0520}&  Sta2_Metric = LEN >< &{\O}& >< mk_Two(ipt:PT,opt:PT)&\iptya{Sta2\_Metric}{metrics-0520}& 
%&\ref{metrics-0530}&  Sta12_Metric = LEN >< &{\O}& >< mk_OneTwo(ipt:PT,opts:(lpt:PT,rpt:PT))&\iptya{Sta12\_Metric}{metrics-0530}& 
%&\ref{metrics-0540}&  Sta21_Metric = LEN >< &{\O}& >< mk_TwpOne(ipts:(lpt:PT,rpt:PT),opt:PT)&\iptya{Sta21\_Metric}{metrics-0540}& 
%\endRSLatex 
\bp
\kw{type}\\
\ref{metrics-0500}.\ \ \ Unit\_Sta {\EQ} Sta1\_Metric {\BAR} Sta2\_Metric {\BAR} Sta12\_Metric {\BAR} Sta21\_Metric\iptya{Unit\_Sta}{metrics-0500}\ \ \\
\ref{metrics-0510}\ \ Sta1\_Metric {\EQ} LEN {\TIMES} {\O} {\TIMES} mk\_One(pt:PT)\iptya{Sta1\_Metric}{metrics-0510} \\
\ref{metrics-0520}\ \ Sta2\_Metric {\EQ} LEN {\TIMES} {\O} {\TIMES} mk\_Two(ipt:PT,opt:PT)\iptya{Sta2\_Metric}{metrics-0520} \\
\ref{metrics-0530}\ \ Sta12\_Metric {\EQ} LEN {\TIMES} {\O} {\TIMES} mk\_OneTwo(ipt:PT,opts:(lpt:PT,rpt:PT))\iptya{Sta12\_Metric}{metrics-0530} \\
\ref{metrics-0540}\ \ Sta21\_Metric {\EQ} LEN {\TIMES} {\O} {\TIMES} mk\_TwpOne(ipts:(lpt:PT,rpt:PT),opt:PT)\iptya{Sta21\_Metric}{metrics-0540} 
\ep


\nbbb{Wellformed Unit Metrics}\label{pipe:Wellformed Unit Metrics}

\begynd
\pind The points positions of neighbouring units must ``fit'' one-another.
\afslut

\begin{enumerate}\setei
\item \label{pls-metrics} Without going into details we can define a predicate,
      \textsf{wf\_Metrics}, \nyl that applies to a pipeline system and
      yields \sort{true} \nyl iff neighbouring units must ``fit'' one-another.
\savei\end{enumerate}
    
%\RSLatex
%value
%&\ref{pls-metrics}.&  wf_Metrics: PLS -> Bool&\ipwf{wf\_Metrics}{pls-metrics}&
%&\ref{pls-metrics}.&  wf_Metrics(pls) is ... 
%\endRSLatex 
\bp
\kw{value}\\
\ref{pls-metrics}.\ \ wf\_Metrics: PLS {\RIGHTARROW} \kw{Bool}\ipwf{wf\_Metrics}{pls-metrics}\\
\ref{pls-metrics}.\ \ wf\_Metrics(pls) {\IS} {\DOTDOTDOT} 
\ep

%%  LocalWords:  PoinTs POS mk ipt OneTwo lpt rpt TwoOne ipts attr wf
%%  LocalWords:  summarise TwpOne Wellformed neighbouring iff PLS pls
%%  LocalWords:  Bool


\nbbb{Summary}

\noindent
\begynd
\pind We summarise the static, monitorable and programmable attributes
      for each manifest part of the pipeline system:
\afslut      
%\begin{multicols}{2}
%\RSLatex
%type
%   PLS_Sta = PLS_net><...
%   PLS_Mon = ...
%   PLS_Prg = PLS_`Sigma><...
%   Well_Sta = Sta1_Metric><Sta_Flows><Orig_Cap><...
%   Well_Mon = Mon_Flows><Well_Cap><...
%   Well_Prg = ... 
%   Pipe_Sta = Sta2_Metric><Sta_Flows><LEN><...
%   Pipe_Mon = Mon_Flows><In_Temp><Out_Temp><...
%   Pipe_Prg = ...  
%   Pump_Sta = Sta2_Metric><Sta_Flows><Pump_Height><...
%   Pump_Mon = Mon_Flows><...
%   Pump_Prg = Pump_State><...
%   Valve_Sta = Sta2_Metric><Sta_Flows><...
%   Valve_Mon = Mon_Flows><In_Temp><Out_Temp><...
%   Valve_Prg = Valve_State><...
%   Fork_Sta = Sta12_Metric><Sta_Flows><...
%   Fork_Mon = Mon_Flows><In_Temp><Out_Temp><...
%   Fork_Prg = ... 
%   Join_Sta = Sta21_Metric><Sta_Flows><...
%   Join_Mon = Mon_Flows><In_Temp><Out_Temp><...
%   Join_Prg = ...
%   Sink_Sta = Sta1_Metric><Sta_Flows><Max_Vol><...
%   Sink_Mon = Mon_Flows><Curr_Vol><In_Temp><Out_Temp><...
%   Sink_Prg = ... 
%\endRSLatex 
\bp
\kw{type}\\
\>\ PLS\_Sta {\EQ} PLS\_net{\TIMES}{\DOTDOTDOT}\\
\>\ PLS\_Mon {\EQ} {\DOTDOTDOT}\\
\>\ PLS\_Prg {\EQ} PLS\_$\Sigma${\TIMES}{\DOTDOTDOT}\\
\>\ Well\_Sta {\EQ} Sta1\_Metric{\TIMES}Sta\_Flows{\TIMES}Orig\_Cap{\TIMES}{\DOTDOTDOT}\\
\>\ Well\_Mon {\EQ} Mon\_Flows{\TIMES}Well\_Cap{\TIMES}{\DOTDOTDOT}\\
\>\ Well\_Prg {\EQ} {\DOTDOTDOT} \\
\>\ Pipe\_Sta {\EQ} Sta2\_Metric{\TIMES}Sta\_Flows{\TIMES}LEN{\TIMES}{\DOTDOTDOT}\\
\>\ Pipe\_Mon {\EQ} Mon\_Flows{\TIMES}In\_Temp{\TIMES}Out\_Temp{\TIMES}{\DOTDOTDOT}\\
\>\ Pipe\_Prg {\EQ} {\DOTDOTDOT}\ \ \\
\>\ Pump\_Sta {\EQ} Sta2\_Metric{\TIMES}Sta\_Flows{\TIMES}Pump\_Height{\TIMES}{\DOTDOTDOT}\\
\>\ Pump\_Mon {\EQ} Mon\_Flows{\TIMES}{\DOTDOTDOT}\\
\>\ Pump\_Prg {\EQ} Pump\_State{\TIMES}{\DOTDOTDOT}\\
\>\ Valve\_Sta {\EQ} Sta2\_Metric{\TIMES}Sta\_Flows{\TIMES}{\DOTDOTDOT}\\
\>\ Valve\_Mon {\EQ} Mon\_Flows{\TIMES}In\_Temp{\TIMES}Out\_Temp{\TIMES}{\DOTDOTDOT}\\
\>\ Valve\_Prg {\EQ} Valve\_State{\TIMES}{\DOTDOTDOT}\\
\>\ Fork\_Sta {\EQ} Sta12\_Metric{\TIMES}Sta\_Flows{\TIMES}{\DOTDOTDOT}\\
\>\ Fork\_Mon {\EQ} Mon\_Flows{\TIMES}In\_Temp{\TIMES}Out\_Temp{\TIMES}{\DOTDOTDOT}\\
\>\ Fork\_Prg {\EQ} {\DOTDOTDOT} \\
\>\ Join\_Sta {\EQ} Sta21\_Metric{\TIMES}Sta\_Flows{\TIMES}{\DOTDOTDOT}\\
\>\ Join\_Mon {\EQ} Mon\_Flows{\TIMES}In\_Temp{\TIMES}Out\_Temp{\TIMES}{\DOTDOTDOT}\\
\>\ Join\_Prg {\EQ} {\DOTDOTDOT}\\
\>\ Sink\_Sta {\EQ} Sta1\_Metric{\TIMES}Sta\_Flows{\TIMES}Max\_Vol{\TIMES}{\DOTDOTDOT}\\
\>\ Sink\_Mon {\EQ} Mon\_Flows{\TIMES}Curr\_Vol{\TIMES}In\_Temp{\TIMES}Out\_Temp{\TIMES}{\DOTDOTDOT}\\
\>\ Sink\_Prg {\EQ} {\DOTDOTDOT} 
\ep

\mnewfoil%
\normalsize\HHHH
\noindent
\begin{enumerate}\setei
\item \label{com-attr}Corresponding to the above three attribute categories
      we can define ``collective'' attribute observers:
\savei\end{enumerate}

%\small\footnotesize
%\begin{multicols}{2}
%\RSLatex
%value
%&\ref{com-attr}.&  sta_A_We: We -> Sta1_Metric><Sta_Flows><Orig_Cap><...
%&\ref{com-attr}.&  mon_A_We: We -> `eta&M&on_Flows><`eta&W&ell_Cap><`eta&I&n_Temp><`eta&O&ut_Temp><...
%&\ref{com-attr}.&  prg_A_We: We -> ...
%&\ref{com-attr}.&  sta_A_Pi: Pi -> Sta2_Metric><Sta_Flows><LEN><...
%&\ref{com-attr}.&  mon_A_Pi: Pi -> &$\mathcal{N}$&Mon_Flows><`eta&I&n_Temp><`eta&O&ut_Temp><... 
%&\ref{com-attr}.&  prg_A_Pi: Pi -> ... 
%&\ref{com-attr}.&  sta_A_Pu: Pu -> Sta2_Metric><Sta_Flows><LEN><...
%&\ref{com-attr}.&  mon_A_Pu: Pu -> &$\mathcal{N}$&Mon_Flows><`eta&I&n_Temp><`eta&O&ut_Temp><...
%&\ref{com-attr}.&  prg_A_Pu: Pu -> Pump_State><... 
%&\ref{com-attr}.&  sta_A_Va: Va -> Sta2_Metric><Sta_Flows><LEN><...
%&\ref{com-attr}.&  mon_A_Va: Va -> &$\mathcal{N}$&Mon_Flows><`eta&I&n_Temp><`eta&O&ut_Temp><...
%&\ref{com-attr}.&  prg_A_Va: Va -> Valve_State><... 
%&\ref{com-attr}.&  sta_A_Fo: Fo -> Sta12_Metric><Sta_Flows><...
%&\ref{com-attr}.&  mon_A_Fo: Fo -> &$\mathcal{N}$&Mon_Flows><`eta&I&n_Temp><`eta&O&ut_Temp><... 
%&\ref{com-attr}.&  prg_A_Fo: Fo -> ... 
%&\ref{com-attr}.&  sta_A_Jo: Jo -> Sta21_Metric><Sta_Flows><...
%&\ref{com-attr}.&  mon_A_Jo: Jo -> Mon_Flows><`eta&I&n_Temp><`eta&O&ut_Temp><...
%&\ref{com-attr}.&  prg_A_Jo: Jo -> ... 
%&\ref{com-attr}.&  sta_A_Si: Si -> Sta1_Metric><Sta_Flows><Max_Vol><...
%&\ref{com-attr}.&  mon_A_Si: Si -> &$\mathcal{N}$&Mon_Flows><`eta&I&n_Temp><`eta&O&ut_Temp><...
%&\ref{com-attr}.&  prg_A_Si: Si -> ...
%
%&\ref{com-attr}.&  &$\mathcal{N}$&Mon_Flows is (`eta&I&n_Flow,`eta&I&n_Leak,`eta&B&ody_Flow,`eta&B&ody_Leak,`eta&O&ut_Flow,`eta&O&ut_Leak)
%\endRSLatex 
\bp
\kw{value}\\
\ref{com-attr}.\ \ sta\_A\_We: We {\RIGHTARROW} Sta1\_Metric{\TIMES}Sta\_Flows{\TIMES}Orig\_Cap{\TIMES}{\DOTDOTDOT}\\
\ref{com-attr}.\ \ mon\_A\_We: We {\RIGHTARROW} $\eta$Mon\_Flows{\TIMES}$\eta$Well\_Cap{\TIMES}$\eta$In\_Temp{\TIMES}$\eta$Out\_Temp{\TIMES}{\DOTDOTDOT}\\
\ref{com-attr}.\ \ prg\_A\_We: We {\RIGHTARROW} {\DOTDOTDOT}\\
\ref{com-attr}.\ \ sta\_A\_Pi: Pi {\RIGHTARROW} Sta2\_Metric{\TIMES}Sta\_Flows{\TIMES}LEN{\TIMES}{\DOTDOTDOT}\\
\ref{com-attr}.\ \ mon\_A\_Pi: Pi {\RIGHTARROW} $\mathcal{N}$Mon\_Flows{\TIMES}$\eta$In\_Temp{\TIMES}$\eta$Out\_Temp{\TIMES}{\DOTDOTDOT} \\
\ref{com-attr}.\ \ prg\_A\_Pi: Pi {\RIGHTARROW} {\DOTDOTDOT} \\
\ref{com-attr}.\ \ sta\_A\_Pu: Pu {\RIGHTARROW} Sta2\_Metric{\TIMES}Sta\_Flows{\TIMES}LEN{\TIMES}{\DOTDOTDOT}\\
\ref{com-attr}.\ \ mon\_A\_Pu: Pu {\RIGHTARROW} $\mathcal{N}$Mon\_Flows{\TIMES}$\eta$In\_Temp{\TIMES}$\eta$Out\_Temp{\TIMES}{\DOTDOTDOT}\\
\ref{com-attr}.\ \ prg\_A\_Pu: Pu {\RIGHTARROW} Pump\_State{\TIMES}{\DOTDOTDOT} \\
\ref{com-attr}.\ \ sta\_A\_Va: Va {\RIGHTARROW} Sta2\_Metric{\TIMES}Sta\_Flows{\TIMES}LEN{\TIMES}{\DOTDOTDOT}\\
\ref{com-attr}.\ \ mon\_A\_Va: Va {\RIGHTARROW} $\mathcal{N}$Mon\_Flows{\TIMES}$\eta$In\_Temp{\TIMES}$\eta$Out\_Temp{\TIMES}{\DOTDOTDOT}\\
\ref{com-attr}.\ \ prg\_A\_Va: Va {\RIGHTARROW} Valve\_State{\TIMES}{\DOTDOTDOT} \\
\ref{com-attr}.\ \ sta\_A\_Fo: Fo {\RIGHTARROW} Sta12\_Metric{\TIMES}Sta\_Flows{\TIMES}{\DOTDOTDOT}\\
\ref{com-attr}.\ \ mon\_A\_Fo: Fo {\RIGHTARROW} $\mathcal{N}$Mon\_Flows{\TIMES}$\eta$In\_Temp{\TIMES}$\eta$Out\_Temp{\TIMES}{\DOTDOTDOT} \\
\ref{com-attr}.\ \ prg\_A\_Fo: Fo {\RIGHTARROW} {\DOTDOTDOT} \\
\ref{com-attr}.\ \ sta\_A\_Jo: Jo {\RIGHTARROW} Sta21\_Metric{\TIMES}Sta\_Flows{\TIMES}{\DOTDOTDOT}\\
\ref{com-attr}.\ \ mon\_A\_Jo: Jo {\RIGHTARROW} Mon\_Flows{\TIMES}$\eta$In\_Temp{\TIMES}$\eta$Out\_Temp{\TIMES}{\DOTDOTDOT}\\
\ref{com-attr}.\ \ prg\_A\_Jo: Jo {\RIGHTARROW} {\DOTDOTDOT} \\
\ref{com-attr}.\ \ sta\_A\_Si: Si {\RIGHTARROW} Sta1\_Metric{\TIMES}Sta\_Flows{\TIMES}Max\_Vol{\TIMES}{\DOTDOTDOT}\\
\ref{com-attr}.\ \ mon\_A\_Si: Si {\RIGHTARROW} $\mathcal{N}$Mon\_Flows{\TIMES}$\eta$In\_Temp{\TIMES}$\eta$Out\_Temp{\TIMES}{\DOTDOTDOT}\\
\ref{com-attr}.\ \ prg\_A\_Si: Si {\RIGHTARROW} {\DOTDOTDOT}\\
\\
\ref{com-attr}.\ \ $\mathcal{N}$Mon\_Flows {\IS} ($\eta$In\_Flow,$\eta$In\_Leak,$\eta$Body\_Flow,$\eta$Body\_Leak,$\eta$Out\_Flow,$\eta$Out\_Leak)
\ep
%\end{multicols}\normalsize

\noindent
\begynd
\pind Monitored flow attributes
\begynd
\pind are [to be] passed as arguments to behaviours \sfsl{by reference}
\pind so that their monitorable attribute values can be sampled.
\afslut
\afslut

\nbbb{Fluid Attributes}
\begynd
\pind Fluids, we here assume, oil, as it appears in the pipeline units
\begynd
\pind have no unique identity,
\pind have not mereology, 
\pind but does have attributes:  hydrocarbons consisting predominantly of
\begynd
\pind aliphatic, 
\pind alicyclic and 
\pind aromatic hydrocarbons.
\afslut
\pind It may also contain small amounts of 
\begynd
\pind nitrogen, 
\pind oxygen, and
\pind sulfur
\afslut compounds
\afslut
\afslut

\begin{enumerate}\setei
\item \label{fluid-0000} We shall simplify, just for illustration, crude oil fluid of
      units to have these attributes:
\begin{enumerate}
\item \label{fluid-0050} volume,
\item \label{fluid-0100} viscosity,
\item \label{fluid-0200} temperature, 
\item \label{fluid-0300} paraffin content (\%age),
\item \label{fluid-0400} naphtenes content (\%age), 
\end{enumerate}
\savei\end{enumerate}
\begin{multicols}{2}
%\RSLatex
%type
%&\ref{fluid-0000}.&   Oil
%&\ref{fluid-0050}.&   Vol
%&\ref{fluid-0100}.&   Visc   
%&\ref{fluid-0200}.&   Temp
%&\ref{fluid-0300}.&   Paraffin
%&\ref{fluid-0400}.&   Naphtene
%value
%&\ref{fluid-0100}.&   obs_Oil: U -> Oil
%&\ref{fluid-0050}.&   attr_Vol: Oil -> Vol
%&\ref{fluid-0100}.&   attr_Visc: Oil -> Visc   
%&\ref{fluid-0200}.&   attr_Temp: Oil -> Temp
%&\ref{fluid-0300}.&   attr_Paraffin: Oil -> Paraffin
%&\ref{fluid-0400}.&   attr_Naphtene: Oil -> Naphtene  
%\endRSLatex
\bp
\kw{type}\\
\ref{fluid-0000}.\ \ \ Oil\\
\ref{fluid-0050}.\ \ \ Vol\\
\ref{fluid-0100}.\ \ \ Visc\ \ \ \\
\ref{fluid-0200}.\ \ \ Temp\\
\ref{fluid-0300}.\ \ \ Paraffin\\
\ref{fluid-0400}.\ \ \ Naphtene\\
\kw{value}\\
\ref{fluid-0100}.\ \ \ obs\_Oil: U {\RIGHTARROW} Oil\\
\ref{fluid-0050}.\ \ \ attr\_Vol: Oil {\RIGHTARROW} Vol\\
\ref{fluid-0100}.\ \ \ attr\_Visc: Oil {\RIGHTARROW} Visc\ \ \ \\
\ref{fluid-0200}.\ \ \ attr\_Temp: Oil {\RIGHTARROW} Temp\\
\ref{fluid-0300}.\ \ \ attr\_Paraffin: Oil {\RIGHTARROW} Paraffin\\
\ref{fluid-0400}.\ \ \ attr\_Naphtene: Oil {\RIGHTARROW} Naphtene\ \ 
\ep
\end{multicols} 


\nbbb{Pipeline System Attributes}

\begynd
\pind The ``root'' pipeline system is a compound.
\pind In its transcendentally deduced behavioral form
\begynd
\pind it is, amongst other ``tasks'', entrusted with the monitoring
      and control of all its units.
\begynd
\pind To do so it must, as a basically static attribute
\pind possess awareness, say in the form of a net diagram
\begynd
\pind of how these units are interconnected,
\pind together with all their internal qualities, 
\pind by type and by value.
\afslut
\afslut
\pind Next we shall give a very simplified account of the possible
      pipeline system attribute.
\afslut
      
\mnewfoil
\begin{enumerate}\setei
\item \label{pls-attrs-0000} We shall make use, in this example, of
  just a simple pipeline state, \textsf{pls\_$\omega$}.
\savei\end{enumerate}

\noindent
\begynd
\pind The pipeline state, \textsf{pls\_$\omega$}, embodies all the
      information that is relevant to the monitoring and control of an
      entire pipeline system, whether static or dynamic.
\afslut
%\RSLatex
%type
%&\ref{pls-attrs-0000}.&  PLS_`Omega 
%\endRSLatex 
\bp
\kw{type}\\
\ref{pls-attrs-0000}.\ \ PLS\_$\Omega$ 
\ep

\nbbbb{Pipeline Concepts, II: Flow Laws}

\begin{enumerate}\setei
\item \label{pefl00} ``What flows in, flows out~!''. For $\mathcal{L}$aminar
                     flows: for any non-well and non-sink  unit
                     the sums of input leaks and in-flows equals the
                     sums of unit and output leaks and out-flows.
\savei\end{enumerate}
%\RSLatex
%   &\kw{Law:}&
%&\ref{pefl00}.&      all u:U\We\Si :- &\ipla{In\_Flow{\IS}Out\_Flow}{pefl00}&
%&\ref{pefl00}.&          sum_in_leaks(u) &$\oplus$& sum_in_flows(u) =
%&\ref{pefl00}.&          attr_body_Leak&$_{\mathcal{L}}$&(u) &$\oplus$& 
%&\ref{pefl00}.&          sum_out_leaks(u) &$\oplus$& sum_out_flows(u)
%\endRSLatex 
\bp
\>\ \kw{Law:}\\
\ref{pefl00}.\ \ \ \ \ \ {\ALL} u:U{\SETMINUS}We{\SETMINUS}Si {\RDOT} \ipla{In\_Flow{\IS}Out\_Flow}{pefl00}\\
\ref{pefl00}.\ \ \ \ \ \ \ \ \ \ sum\_in\_leaks(u) $\oplus$ sum\_in\_flows(u) {\EQ}\\
\ref{pefl00}.\ \ \ \ \ \ \ \ \ \ attr\_body\_Leak$_{\mathcal{L}}$(u) $\oplus$ \\
\ref{pefl00}.\ \ \ \ \ \ \ \ \ \ sum\_out\_leaks(u) $\oplus$ sum\_out\_flows(u)
\ep

\mnewfoil\LLLL

%\RSLatex
%value
%  sum_in_leaks: U -> Flow
%  sum_in_leaks(u) is let (iuis,) = mereo_U(u) in &$\oplus$& {attr_In_Leak&$_{\mathcal{L}}$&(u)(ui)|ui:UI:-ui isin iuis} end
%  sum_in_flows: U -> Flow
%  sum_in_flows(u) is let (iuis,) = mereo_U(u) in &$\oplus$& {attr_In_Flow&$_{\mathcal{L}}$&(u)(ui)|ui:UI:-ui isin iuis} end
%  sum_out_leaks: U -> Flow
%  sum_out_leaks(u) is let (,ouis) = mereo_U(u) in &$\oplus$& {attr_Out_Leak&$_{\mathcal{L}}$&(u)(ui)|ui:UI:-ui isin ouis} end
%  sum_out_flows: U -> Flow
%  sum_out_flows(u) is let (,ouis) = mereo_U(u) in &$\oplus$& {attr_Out_Leak&$_{\mathcal{L}}$&(u)(ui)|ui:UI:-ui isin ouis} end
%\endRSLatex 
\bp
\kw{value}\\
\>sum\_in\_leaks: U {\RIGHTARROW} Flow\\
\>sum\_in\_leaks(u) {\IS} \kw{let} (iuis,) {\EQ} mereo\_U(u) \kw{in} $\oplus$ {\LBRACE}attr\_In\_Leak$_{\mathcal{L}}$(u)(ui){\BAR}ui:UI{\RDOT}ui {\ISIN} iuis{\RBRACE} \kw{end}\\
\>sum\_in\_flows: U {\RIGHTARROW} Flow\\
\>sum\_in\_flows(u) {\IS} \kw{let} (iuis,) {\EQ} mereo\_U(u) \kw{in} $\oplus$ {\LBRACE}attr\_In\_Flow$_{\mathcal{L}}$(u)(ui){\BAR}ui:UI{\RDOT}ui {\ISIN} iuis{\RBRACE} \kw{end}\\
\>sum\_out\_leaks: U {\RIGHTARROW} Flow\\
\>sum\_out\_leaks(u) {\IS} \kw{let} (,ouis) {\EQ} mereo\_U(u) \kw{in} $\oplus$ {\LBRACE}attr\_Out\_Leak$_{\mathcal{L}}$(u)(ui){\BAR}ui:UI{\RDOT}ui {\ISIN} ouis{\RBRACE} \kw{end}\\
\>sum\_out\_flows: U {\RIGHTARROW} Flow\\
\>sum\_out\_flows(u) {\IS} \kw{let} (,ouis) {\EQ} mereo\_U(u) \kw{in} $\oplus$ {\LBRACE}attr\_Out\_Leak$_{\mathcal{L}}$(u)(ui){\BAR}ui:UI{\RDOT}ui {\ISIN} ouis{\RBRACE} \kw{end}
\ep

\mnewfoil

\begin{enumerate}\setei
\item \label{pefl01} ``What flows out, flows in~!''. For $\mathcal{L}$aminar
                     flows: for any adjacent pairs of  units
                     the output flow at one unit connection equals the
                     sum of adjacent unit leak and in-flow at that connection.
\savei\end{enumerate}


%\RSLatex
%   &\kw{Law:}&
%&\ref{pefl01}.&  all u,u':U:-adjacent(u,u') =>&\ipla{Out\_Flow{\IS}In\_Flow}{pefl00}&
%&\ref{pefl01}.&      let (,ouis)=mereo_U(u), (iuis',)=mereo_U(u') in
%&\ref{pefl01}.&      &\kw{assert:}& uid_U(u') isin ouis /\ uid_U(u) isin iuis '
%&\ref{pefl01}.&      attr_Out_Flow&$_{\mathcal{L}}$&(u)(uid_U(u')) = 
%&\ref{pefl01}.&      attr_In_Leak&$_{\mathcal{L}}$&(u)(uid_U(u))&$\oplus$&attr_In_Flow&$_{\mathcal{L}}$&(u')(uid_U(u)) end
%\endRSLatex 
\bp
\>\ \kw{Law:}\\
\ref{pefl01}.\ \ {\ALL} u,u{\PRIM}:U{\RDOT}adjacent(u,u{\PRIM}) {\DBLRIGHTARROW}\ipla{Out\_Flow{\IS}In\_Flow}{pefl00}\\
\ref{pefl01}.\ \ \ \ \ \ \kw{let} (,ouis){\EQ}mereo\_U(u), (iuis{\PRIM},){\EQ}mereo\_U(u{\PRIM}) \kw{in}\\
\ref{pefl01}.\ \ \ \ \ \ \kw{assert:} uid\_U(u{\PRIM}) {\ISIN} ouis {\WEDGE} uid\_U(u) {\ISIN} iuis {\PRIM}\\
\ref{pefl01}.\ \ \ \ \ \ attr\_Out\_Flow$_{\mathcal{L}}$(u)(uid\_U(u{\PRIM})) {\EQ} \\
\ref{pefl01}.\ \ \ \ \ \ attr\_In\_Leak$_{\mathcal{L}}$(u)(uid\_U(u))$\oplus$attr\_In\_Flow$_{\mathcal{L}}$(u{\PRIM})(uid\_U(u)) \kw{end}
\ep

\noindent
\pind These ``laws'' should hold for a pipeline system without plates.
\afslut

% LocalWords:  Endurants formedness mereology cf PLS pls wf Bool obs Fo Valv UI
% LocalWords:  uid xtr UIs isin mereo iuis ouis uius mk pu va fo jo si
%%  LocalWords:  organised IPM nstitute hysics athematics PLSI Mer tl
%%  LocalWords:  uis Mereologies len infset acyclic inds wsrs GoL Prg
%%  LocalWords:  characterise WellCap aminar monitorable attr ValSta
%%  LocalWords:  VaSta Summarising summarise FLows Curr amongst ui xz
%%  LocalWords:  formalisations Edurants wfMereology wfRoutes mani
%%  LocalWords:  Wellformed behaviours endurants Endurant endurant
%%  LocalWords:  ipty mer Meripob Describability sta mon prg Visc
%%  LocalWords:  aliphatic alicyclic naphtenes Naphtene pefl
