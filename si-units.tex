\nbbbb{Physics Attributes}\label{si}\label{si.1}\pos{\normalsize}{}

\begynd
\pind In this section we shall muse 
\begynd
\pind about the kind of attributes that
      are typical of natural parts,
\pind but which may also be relevant as attributes of artefacts.
\afslut

\pind Typically, when physicists write computer programs,
\begynd
\pind intended for calculating physics behaviours,
\pind they ``lump'' all of these into the \sort{type} \sort{Real},
\pind thereby hiding some important physics 'dimensions'.
\pind In this section we shall review that which is missing\,!
\afslut
\afslut

\pos{\psno}{\mnewfoil}
\begynd
\pind The subject of physical dimensions in programming languages \nyl
      is rather decisively treated in David Kennedy's 1996 PhD Thesis
      \cite{Kennedy96} ---
\pind so there really is no point in trying to cast new light on this subject
\pind other than to remind the reader of what these physical
      dimensions are all about.
\afslut

\nbbb{SI: The International System of Quantities}\label{si-1}

\begynd
\pind In physics we operate on \nyl values of attributes of manifest,
      i.e., physical phenomena. \pos{
\pind The type of some of these attributes are recorded \nyl in well
      known tables, cf.\,Tables\,\ref{table:1}--\ref{table:3}.}{}
\afslut
%\mnewfoil
\pos{Table\,\vref{table:1} shows the base units of physics.}{}

\pos{\footnotesize}{\HHHH}
\begin{table}[h]\HHHH
  \centering
\begin{tabular}{|lll|} \hline
Base quantity     &          Name   &      Type \\ \hline
length            &          meter   &     m \\
mass              &          kilogram   &  kg \\
time              &          second  &     s \\
electric current     &       ampere   &    A \\
thermodynamic temperature &  kelvin  &     K \\
amount of substance    &     mole    &     mol \\
luminous intensity     &     candela &     cd \\ \hline
\end{tabular} \caption{\HHHH Base SI Units} \label{table:1}
\end{table}
\normalsize\HHHH
\mnewfoil
\noindent
\pos{Table\,\vref{table:2} shows the units of physics derived from the base units.}{}
\pos{\footnotesize}{\HHHH\LLLL}%%%
\begin{table}[h]\LLLL\LLll
  \centering
\begin{tabular}{|llll|} \hline
         Name   &      Type & Derived Quantity & Derived Type \\ \hline
         radian & rad & angle & m/m  \\ 
         steradian & sr & solid angle & m$^2\times$m$^{-2}$ \\  
         Hertz  & Hz  & frequency & s$^{-1}$ \\
         newton & N   & force, weight   & kg{$\times$}m{$\times$}s$^{-2}$ \\
         pascal & Pa & pressure, stress & N/m$^2$ \\
         joule  & J & energy, work, heat & N$\times$m \\
         watt   & W & power, radiant flux &     J/s \\
         coulomb & C & electric charge & s$\times$A \\
volt &  V & electromotive force &  W/A (kg$\times$m$^2\times$s$^{-3}\times$A$^{-1}$) \\
farad & F & capacitance & C/V (kg$^{-1}\times$·m$^{-2}\times$s$^4\times$A$^2$) \\
ohm & $\Omega$ &  electrical resistance & V/A (kg$\times$m$^2\times$s$^{−3}\times$A$^{−2}$) \\
siemens & S & electrical conductance & A/V (kg${−1}\times$m$^{−2}\times$s$^3\times$A$^2$)\\
weber & Wb & magnetic flux &  V$\times$s  (kg$\times$m$^2\times$s$^{-2}\times$A$^{-1}$)\\
tesla   & T &  magnetic flux density   & Wb/m$^2$ (kg$\times$s$^{−2}\times$A$^{-1}$) \\
henry   & H &  inductance    &   Wb/A (kg$\times$m$^2\times$s$^{-2}\times$A$^2$) \\
degree Celsius    &     $^o$C    &   temp.\ rel.\ to 273.15 K & K \\
lumen &  lm  &   luminous flux   & cd$\times$sr   (cd) \\
lux   &  lx  & illuminance   &   lm/m$^2$   (m$^{−2}\times$cd) \\ \hline
\end{tabular}  \caption{\HHHH Derived SI Units} \label{table:2}
\end{table}
\mnewfoil 
\noindent\pos{\normalsize}{\HHHH\LLLL}
\pos{Table\,\ref{table:3} shows further units of physics derived from the base units.}{}
 \pos{\footnotesize}{\large\HHHH}
\begin{table}[h]\HHHH\LLLL
  \centering
\begin{tabular}{|lll|} \hline
Name   &  Explanation & Derived Type \\ \hline
area                               & square meter                &    m$^2$ \\
volume                             & cubic meter                 &    m$^3$ \\
speed                              & meter per second            &    m/s \\
wave number                        & reciprocal meter            &    m-1 \\
mass density                       & kilogram per cubic meter    &    kg/m$^3$ \\
specific volume                    & cubic meter per kilogram    &    m3/kg \\
current density                    & ampere per square meter     &    A/m$^2$ \\
magnetic field strength            & ampere per meter            &    A/m \\
substance concentration            & mole per cubic meter        &    mol/m$3$ \\
luminance                          & candela per square meter    &    cd/m$^2$ \\
mass fraction                      & kilogram per kilogram       &    kg/kg = 1 \\ \hline
\end{tabular}
  \caption{\HHHH Further SI Units}
  \label{table:3}
\end{table}
\mnewfoil
\noindent
\pos{\sfsl{velocity} is speed with three dimensional direction and is,
  for example, given as
\begin{itemize}
\item \sfsl{velocity}, \sf meter per second \rm with \sf direction\rm: \hfill $m/s$
\item \sfsl{acceleration},  \sf meter per second squared\rm, \hfill  $ m/s^2$
\item \sfsl{(longitude,latitude,azimuth)} measured in \sf radian\rm: \hfill  $(r,r,r)$ 
\end{itemize}}{}

\noindent
\pos{\normalsize Table\,\ref{table:4} shows standard
prefixes for SI units of measure 
and Tables\,\ref{table:6} show
fractions of SI units.}{}

\pos{\scriptsize}{\large\HHHH\LLLL}
%\pos{\begin{multicols}{2}}{}

\begin{table}[h]\HHHH\center
\begin{tabular}{|lllllll|} \hline
Prefix name   &      & deca  &  hecto &  kilo &   mega &   giga \\
Prefix symbol &      & da    &  h  &     k   &    M   &    G  \\
Factor        & 10$^0$ & 10$^1$ & 10$^2$ & 10$^3$ & 10$^6$  & 10$^9$\\ \hline
Prefix name   &      &   tera  &  peta  &  exa  &   zetta &  yotta\\
Prefix symbol &      &    T  &     P &      E &      Z &     Y\\
Factor        &&  10$^{12}$  & 10$^{15}$ & 10$^{18}$ & 10$^{21}$ & 10$^{24}$\\ \hline
\end{tabular} \ \ \
  \caption{\HHHH Standard Prefixes for SI Units of Measure}
  \label{table:4}
\end{table}

\dbeat{
\pos{\psno}{\mnewfoil}
\begin{table}[h]\LLLL\HHHH\center
\begin{tabular}{|lllllll|} \hline
Prefix name      &    &   deci  &  centi  & milli &  micro &  nano   \\
Prefix symbol     &   &   d   &    c  &     m   &    $\mu$ &       n   \\
Factor        & 10$^0$ & 10$^{-1}$ & 10$^{-2}$ & 10$^{-3}$ & 10$^{-6}$ & 
10$^{-9}$   \\ \hline
Prefix name      &    &     pico &   femto &  atto &   zepto  & yocto\\
Prefix symbol     &   &          p &      f &      a &      z    &   y\\
Factor        &  &  10$^{-12}$  & 10$^{-15}$ & 10$^{-18}$ & 10$^{-21}$ &
10$^{-24}$\\ \hline
\end{tabular}
  \caption{\HHHH Fractions}
  \label{table:5}
\end{table}
%\pos{\end{multicols}}{}
}


\normalsize
\mnewfoil
%\fest{%\begin{multicols}{2}
\begin{table}[h]\LLll\center
\begin{tabular}{|lllllll|} \hline %\hline
Prefix name   &      & deca  &  hecto &  kilo &   mega &   giga \\
Prefix symbol &      & da    &  h  &     k   &    M   &    G  \\
Factor        & 10$^0$ & 10$^1$ & 10$^2$ & 10$^3$ & 10$^6$  & 10$^9$\\ \hline
Prefix name   &      &   tera  &  peta  &  exa  &   zetta &  yotta\\
Prefix symbol &      &    T  &     P &      E &      Z &     Y\\
Factor        &&  10$^{12}$  & 10$^{15}$ & 10$^{18}$ & 10$^{21}$ & 10$^{24}$\\ \hline\hline
Prefix name      &    &   deci  &  centi  & milli &  micro &  nano   \\
Prefix symbol     &   &   d   &    c  &     m   &    $\mu$ &       n   \\
Factor        & 10$^0$ & 10$^{-1}$ & 10$^{-2}$ & 10$^{-3}$ & 10$^{-6}$ & 
10$^{-9}$   \\ \hline
Prefix name      &    &     pico &   femto &  atto &   zepto  & yocto\\
Prefix symbol     &   &          p &      f &      a &      z    &   y\\
Factor        &  &  10$^{-12}$  & 10$^{-15}$ & 10$^{-18}$ & 10$^{-21}$ &
10$^{-24}$\\ \hline
\end{tabular}
  \caption{\HHHH SI Units of Measure and Fractions}
  \label{table:6}
\end{table}
\normalsize\HHHH
\mnewfoil
\treprikker
\noindent
\begynd
\pind The point in \ysfchg{including } this material is
\begynd
\pind that when modelling, i.e., describing domains
\pind we must be extremely careful in not falling into the trap
\pind of modelling physics types, etc.,  as we do in programming --
\pind by simple \sort{Real}s.
\afslut
\pind We claim, without evidence, 
\begynd
\pind that many trivial programming mistakes
\pind are due to confusions between especially 
\pind derived SI units, fractions and prefixes.
\afslut
\afslut

\nbbb{Units are Indivisible}
\begynd
\pind A volt, kg$\times$m$^2\times$s$^{-3}\times$A$^{-1}$, 
\begynd
\pind see Table\,\ref{table:2}, is ``indivisible''.
\pind It is not a composite structure of
\begynd
\pind mass, % kilos,
\pind length, % meters,
\pind time, and % seconds, and
\pind electric current -- % amperes --
\afslut in some intricate relationship.
\afslut
\afslut

\pos{\psno}{\mnewfoil}

\noindent
\treprikker
\begynd
\pind Physical attributes may ascribe mass and volume to endurants.
\begynd
\pind But they do not reveal the substance,
\pind i.e., the material from which the endurant is made.
\pind That is done by chemical attributes.
\afslut
\afslut

\nbbb{Chemical Elements}
\begynd
\pind The chemical elements are, to us, what makes up 
      $\mathbb{MATTER}$.
      \index{defind}{matter@$\mathbb{MATTER}$}%
\pind The \textsl{mole}, \textsf{mol}, substance is about chemical
molecules. 
\begynd
\pind A \textsf{mole} contains exactly $6.022\-14\-076{\times}10^{23}$
(the Avogadro 
      number) constituent particles, usually
      atoms, molecules, or ions -- of the elements,
\pind cf.\,\textsf{'The Periodic Table',} 
      \bcolor{\texttt{en.\-wi\-ki\-pe\-di\-a.\-org\-wi\-ki/\-Pe\-ri\-o\-dic\-\_tab\-le}},
      \nyl cf.\,Fig.\,\ref{period}.
\afslut
\afslut
\mnewfoil
\pos{}{.}
\hDBfigure{pte}{\pos{10.4}{10}cm}{Periodic Table}{period}\LLLL
                                %periodic

\noindent%
\begynd
\pind Any specific molecule is then a compound of two or more elements,
      \nyl for example, \textsf{cal\-ci\-um\-phos\-phat: Ca3(PO4)2}.
\afslut

\mnewfoil%
\begynd%
\pind Moles bring substance to endurants.
\begynd
\pind The physics attributes may ascribe weight and volume
      to endurants,
\pind but they do not explain what it is that gives weight, \nyl
      i.e., fills out the volume.
\afslut
\afslut

\label{si.n}
%%  LocalWords:  behaviours lll mol candela cd llll radian steradian
%%  LocalWords:  sr siemens weber Wb tesla henry lumen lm lux lx deca
%%  LocalWords:  illuminance luminance lllllll hecto giga da tera exa
%%  LocalWords:  peta zetta yotta deci centi milli nano pico femto wi
%%  LocalWords:  atto zepto yocto modelling artefacts endurants ki pe
%%  LocalWords:  endurant di Pe ri dic le ci phos pte adian defind
