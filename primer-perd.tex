
%%%%%%%%%%%%%%%%%%%%%%%%%%%%%%%%%%%%%%%%%%%%%%%%%%%%%%%%%%%%%%%%%%%%%%%%%
%%%%%%%%% Chapter 6 released 27 April 2023 for translation %%%%%%%%%%%%%%
%%%%%%%%% Chapter 6 re-released 23 June 2023 for translation %%%%%%%%%%%%
%%%%%%%%% Chapter 6 re-released 04 Jauary 2024 for translation %%%%%%%%%%
%%%%%%%%%%%%%%%%%%%%%%%%%%%%%%%%%%%%%%%%%%%%%%%%%%%%%%%%%%%%%%%%%%%%%%%%%

\nbbbbbb{Perdurants}\label{chap6.tex.1}\label{chapter:Perdurants}
\pos{\minitoc}{}

\noindent
\begynd
\pind Please consider Fig.\,\vref{onto.fig\pos{2}{xtra}}.
\begynd
\pind The previous \pos{two chapters}{lectures} covered the  left of Fig.\,\ref{\pos{onto.fig2}{onto.figxtra}}.
\pind \pos{This chapter}{These next lectures} cover\pos{s}{} the right of Fig.\,\ref{\pos{onto.fig2}{onto.figxtra}}.
\afslut
\afslut

\pos{}{\hDBfigure{onto}{\pos{10}{10}cm}{{An} Upper Ontology}{onto.figxtra}}

\pos{\treprikker

\noindent
\begynd
\pind This chapter  is a rather ``drastic'' reformulation and
      simplification of \cite[\sfsl{Chapter 7, i.e., pages
        159--196}]{BjornerMonograph2020}. 
\begynd
\pind Besides, Sect.\,\ref{Discrete Dynamic Domains} is new. 
\afslut
\afslut
}{}

\begynd
\pind In this chapter we transcendentally ``morph'' manifest 
\begynd
\pind \sort{parts} into \sort{behaviours,} that is:
\pind \sort{endurants} into \sort{perdurants}.
\afslut
\pind We analyse that notion, \sfsl{perdurants}, and its constituent notions of
\begynd
\pind \sort{actors}, \index{pconind}{actor}
\pind \sort{channels} and \sort{communication},\index{pconind}{channel}
\pind \sort{actions} and \index{pconind}{action}
\pind \sort{events} and \index{pconind}{event}
\pind \sort{behaviours}.  \index{pconind}{behaviour}
\afslut 
\afslut
\mnewfoil
\inthis{perdurants}{state and time-evolving domain phenomena.}{}

\treprikker

\noindent
\mnewfoil
\nysf{%
\begynd
\pind In this  chapter we shall analyse and describe \sfsl{perdurants}, that is, the \sfsl{entities}
\begynd
\pind of domains for which only a
      fragment exists, in \sfsl{space}, if we look at or touch them at
      any given snapshot in
      \sfsl{time}.
\afslut
    
\pind This modelling will focus on the \sort{actions, events} and \sort{behaviours}
      of these perdurants and the means, here referred to as
      \sort{channels}, by means of which the part behaviours interact.

\pind On one hand there are the domain phenomena of perdurants.
\pind On the other hand there are means for analysing and describing these.
\pind The former are not formalized ``before'', or as, we analyse and
      describe them.
\pind The latter, `the means', are assumed formalized.
\afslut

\mnewfoil

\treprikker\LLLL

\noindent
\begynd
\pind The \sort{structure of this chapter} need be explained.
\begynd
\pind The chapter attempts to motivate and explain the ``morphing''
      of endurant parts into perdurant behaviours.
\pind The endurant parts were analyzed and described in terms
      of \texttt{RSL} abstract types, i.e., sorts, and observers -- of both
      external and internal qualities.
\pind The perdurant behaviours will be analyzed and described in terms
      of tail-recursive functions, their signatures and `body'
      definitions -- and their [``output/input''] interaction by means
      of \texttt{CSP} output/input clauses and channels.
\pind To arrive at these analyses and descriptions we
      ``move'' from general motivation and text on behaviours, their
      actions and events and their reliance on channels, to
      increasingly more specific such text.
\pind Hence the seeming ``repetition'' of treatments of behaviours,
      actions, events and channels.
\afslut
\afslut
\mnewfoil

\pmt{Primary Modelling Tool, II}
\noindent
\begynd
\pind The tool with which we describe perdurants will be 
\begynd
\pind the  \sort{tail recursive function} and  
\pind the \sort{channel} 
\afslut concepts of the formal specification language \texttt{RSL} \cite{RSL}.
\pind A special focus will be on the \sfsl{signature} of the 
\begynd
\pind action and behaviour
function
\afslut definitions.
\afslut
\tmp
}

\bookdefn{Description, III}{ %
\begynd
\pind By a \sort{description} \index{pdefind}{description!perdurants} of the perdurants of a domain
\begynd
\pind we shall mean pairs of informal, narrative, and formal text
\pind which characterises the
\begynd
\pind behaviour interaction channels,
\pind actors, actions, events and behaviours,
\pind i.e., the
\begynd
\pind signatures, \index{pconind}{signature}
\pind invocation and \index{pconind}{invocation}
\pind Initialisation \index{pconind}{initialisation}
\afslut
\afslut
\pind of manifest part behaviours \dbsquare
\afslut
\afslut
}

\noindent
\begynd
\pind This chapter explains what is meant by 
\begynd
\pind \sfsl{channel}, \index{pconind}{channel} 
\pind \sfsl{actor}, \index{pconind}{actor}
\pind \sfsl{behaviour},  \index{pconind}{behaviour}
\pind \sfsl{action},  \index{pconind}{action}
\pind \sfsl{event},  \index{pconind}{event}
\pind \sfsl{signature},  \index{pconind}{signature}
\pind \sfsl{invocation} and \index{pconind}{invocation}
\pind \sfsl{initialisation}. \index{pconind}{initialisation}
\afslut
\afslut

\nbbbbb{Parts and their Behaviours}\label{Parts and their Behaviours}

\pos{%
\begynd
\pind By transcendental deduction we ``morph''\ysf{\footnote{Morph: change the form or
      character of ...}} parts into behaviours.
\pind We refer to Sect.\,\ref{Transcendental Deductions}'s
      Example\,\vref{Transcendentality}.
\afslut


\nbbbb{General Notions}}{\bbbb{General Notions}}

\begynd
\pind Parts are manifest entities of domains.
\pind Behaviours are likewise manifest entities of domains.
\pind Behaviours are domain  notions. 
\pind We shall express domain behaviours in terms of the
      \texttt{RSL/CSP} notions of \sfsl{processes}.
\pind And we shall explain domain behaviours in terms of
\begynd
\pind \sfsl{domain actions},
\pind \sfsl{domain events}, and subsidiary, i.e., ``embedded'',
\pind \sfsl{domain behaviours}.
\afslut
\afslut
\mnewfoil

\bookdefn{Behaviour}{
\begynd
\pind We define domain behaviours as
\begynd
\pind sets of sequences of 
\pind \sfsl{domain actions}, 
\pind \sfsl{domain events} and [subsidiary] 
\pind \sfsl{domain behaviours} \dbsquare
\afslut
\afslut}

\bookdefn{Actor}{
\begynd
\pind An actor
\begynd
\pind is anything that can invoke and sustain
\pind a behaviour, an action or an event \dbsquare
\afslut
\afslut}

\bookdefn{Action}{
\begynd
\pind Actions
\begynd
\pind are planned, purposeful state changes \dbsquare 
\afslut
\afslut}

\bookdefn{Event}{
\begynd
\pind Events 
\begynd
\pind are surreptitious state changes \dbsquare
\afslut
\afslut}

\noindent
\begynd
\pind \sfsl{Domain actions} are expressed in terms of the [\texttt{RSL}]
      notion of language clauses which prescribe state changes.
\pind \sfsl{Domain events} are expressed in terms of the \texttt{CSP}
      notion of language clauses which prescribe interaction between
      [\texttt{CSP}] processes.
\pind \sfsl{Domain behaviours}, to repeat, are expressed in terns of
      \texttt{CSP} processes, more specifically in terms of \sfsl{tail
      recursive} function definitions.
\afslut

\mnewfoil

\begynd
\pind Thus there are two notions:
\begynd
\pind the domain notions of endurant parts and perdurant actions,
      events and behaviours, and
\pind the description language, here \texttt{RSL/CSP}, notions of
      part descriptions and expressions and statements, i.e., possibly
      state-changing processes.
\afslut
\afslut

\nbbbb{An Aside: Behaviours versus Processes}\label{An Aside: Behaviours versus Processes}
\index{pconind}{behaviour!versus process}
\index{pconind}{process!versus behaviour}
\begynd
\pind In programming, in general, and in \texttt{CSP} in specific, we use the term \sfsl{process} 
\begynd
\pind to characterize the execution of 
\pind a set of sequences of
\pind {[program]} statements 
\pind and processes\footnote{Yes, we do mean a recursive definition.}.
\afslut
\afslut


\mnewfoil

\bookdefn{Process}{
\begynd
\pind We define [computing] processes as
\begynd
\pind sets of sequences of 
\pind \sfsl{state changes} 
\pind (whether purposefully planned or accidental) \dbsquare
\afslut
\afslut}

\mnewfoil

\noindent
\begynd
\pind In domains we use the term \sfsl{behaviour}, in contrast,
\begynd
\pind to characterize the \sfsl{conduct},
\pind the [organization, as for endurants] 
\pind and the \sfsl{carrying out}
\pind of the \sfsl{intentions} of a part.
\afslut
\afslut

\nbbbb{Multiple, Communicating Behaviours}\bnew

\begynd
\pind On the basis of Kai S{\o}lander's Philosophy
      \cite{kaisorlander1994,kaisorlander1997,kaisorlander2002,kaisorlander2016,kaisorlander2022}
      we can reason that there  
\begynd
\pind is an indefinite number of parts, 
\pind that is, an indefinite number of [part] behaviours,
\pind hence an indefinite number of \texttt{CSP} processes.
\afslut

\pind We shall further reason that these [part] behaviours interact.

\begynd
\pind There is the possibility that parts have dynamic attributes. 
\pind For dynamic attribute values to change there must be an `agent of change'.
\pind There is the possibility that two or more distinct parts
      interact.
\pind \sort{Examples:}
\begynd
\pind (i) \sfsl{automobile}s enter and leave \sfsl{hub}s and \sfsl{link}s;
\pind (ii) \sfsl{retailer}s deposit funds in \sfsl{bank}s;
\pind (iii) \sfsl{container vessel}s load and unload \sfsl{container}s\dbsquare\
\afslut
\pind The part pairs: 
\begynd
\pind \sfsl{(automobile,hub)},
\pind \sfsl{(automobile,link)}, 
\pind \sfsl{(retailer,bank)} and   
\pind\sfsl{(container vessel,container)}, 
\afslut are pairs of `agents of change'.
\afslut

\pind This reasoning, by transcendental deduction, leads us to conclude
\begynd
\pind that these mutual interactions 
\pind can be seen as channel communications
\pind in the sense of, for example, \texttt{CSP}.
\afslut
\afslut


\nbbbb{Domain Behaviours and Domain Actions}

\begynd
\pind What do we mean by behaviours and actions.
\begynd
\pind First of all we must emphasize, as in Sect.\,\vref{An Aside: Behaviours versus Processes},
\begynd
\pind that we are not dealing with with domains, not with computing, 
\pind with domain behaviours, not with computing processes.
\pind Domain parts are not computers.\footnote{Yes, we exclude
      from the domains, for which we put forward the calculi of this
      paper, such parts which acts like general computing devices.}
\afslut 
\pind Then we focus of the conduct of domain actions of domain
behaviours.

\pind Generally speaking a \sort{domain action},  \index{pdefind}{action} of a part, is
\begynd
\pind either updating the state, 
\pind i.e., the mereology or dynamic attributes 
\pind of the part
\pind of which the action is intended.
\afslut
\pind Part actions are seen as ``atomic'',
\begynd
\pind that is \textsl{``taking no time''} to ``occur'' ---
\pind although they may be expressed in terms of \sfsl{sub-actions.}
\pind We shall later elaborate on our notion of sub-actions.
\afslut

\pind \sort{Examples} of domain actions are those of an automobile deciding
\begynd
\pind (i-ii) to remain on a link or at a hub,
\pind (iii) to leave a hub entering a link,
\pind (iv) to leave a link entering a hub,
\pind (v) to leave the road net, i.e., ``disappear'' altogether\dbsquare\
\afslut

\pind Correspondingly, a \sort{domain behaviour}, of a part, is
\begynd
\pind either a sequence of one or more \sfsl{domain actions} \index{pdefind}{behaviour}
\pind or an \sfsl{``alternative'' set} of
\begynd
\pind either internally, non-deterministically chosen (\NONDETCHOICE) sequence of  one
      or more \sfsl{domain actions},
\pind or externally,  non-deterministically chosen (\DETCHOICE) sequence of one or
      more \sfsl{domain actions}, 
\pind or combinations thereof.
\afslut
\pind By \sfsl{``alternative'' set} we mean
\begynd
\pind that each element of the set is one of 
\pind the  internally or externally, non-deterministically 
\pind chosen sequences of domain actions.
\afslut
\afslut
\afslut
\afslut

\pind \sort{Examples} of domain behaviours are those of
\begynd
\pind (a) an automobile which 
\begynd
\pind internally, non-deterministically (\NONDETCHOICE)
\pind alternates between (i--v) above; or
\afslut
\pind (b) a link which 
\begynd
\pind externally, non-deterministically (\DETCHOICE) alternates between
\begynd
\pind (b.i) welcoming incoming automobiles,
\pind (b.ii) accepting that automobiles remain on the link,
\pind (b.iii) ``saying good bye'' to outgoing  automobiles; or
\afslut
\afslut
\pind (c) a hub which 
\begynd
\pind externally, non-deterministically (\DETCHOICE) alternates between
\begynd
\pind (c.i) welcoming incoming automobiles,
\pind (c.ii) accepting that automobiles remain at the hub,
\pind (c.iii) ``saying good bye'' to outgoing  automobiles \dbsquare
\afslut
\afslut
\afslut
\enew

\nbbbbb{Channel Description}\label{Channel Description}\label{Channel Analysis}

\begynd
\pind The \texttt{CSP} concept of \sfsl{channel} \nyl  is to be our way of
      expressing the ``medium'' \nyl  in which behaviours interact.
\begynd
\pind \ysfchg{A c}hannel is thus an abstract concept. \index{pdefind}{channel}
\pind Please do not think of it as a physical, \nyl an \texttt{IT}
      (information technology) device.
\pind As an abstract concept it is defined in terms of, \nyl
      roughly, the laws, the semantics, of \texttt{CSP} \citecsp.
\pind We write `roughly' since the \texttt{CSP} \nyl
      we are speaking of, is ``embedded'' in \texttt{RSL}.
\afslut
\afslut
\mnewfoil

\bookdefn{Channels}{
\begynd
\pind A channel is an abstract notion, not a physical ``gadget''.
\begynd
\pind A channel is anything 
\pind that allow\ysfchg{s } any two behaviours
\pind to interact:
\begynd
\pind to synchronise and
\pind communicate,
\pind i.e., exchange information \dbsquare
\afslut
\afslut
\afslut}

\mnewfoil
\noindent
\begynd
\pind We  simplify the general
      treatment of channel declarations.
\pind Basically all we can say, for any domain,
\begynd
\pind is that any two distinct part behaviours
\pind may need to communicate.
\afslut
\pind Therefore we declare a vector of channels \nyl
indexed by sets of two distinct part identifiers.
\ddprompt{describe\_channels}{describe-channels}{
%\RSLatex
%value
%   describe_channels: Unit -> &\rsltext&
%   describer_channels() is &\bq\,&channel { ch[{ij,ik}] | ij,ik:UI :- {ij,ik}<<= uid&$_{\sigma}$& /\ ij~=ik } M&\,\eq&
%\endRSLatex
\bp
\kw{value}\\
\>\ describe\_channels: \kw{Unit} {\RIGHTARROW} \rsltext\\
\>\ describer\_channels() {\IS} \bq\,\kw{channel} {\LBRACE} ch{\LBRACKET}{\LBRACE}ij,ik{\RBRACE}{\RBRACKET} {\BAR} ij,ik:UI {\RDOT} {\LBRACE}ij,ik{\RBRACE}{\SUBSETEQ} uid$_{\sigma}$ {\WEDGE} ij{\NOTEQ}ik {\RBRACE} M\,\eq
\ep
}
\noindent
\pind Initially we shall leave the type, \textsf{M},  of messages over channels
      further undefined.
\pos{\pind As we, laboriously, work through the definition of behaviours,
      \nyl we shall be able to make \textsf{M} precise.\ysfchg{}}{}
\afslut

\nysf{%
\nbbbbb{Action and Event Description, I}\label{Action and Event Description.i}

\begynd
\pind For each [part] behaviour we identify the zero, one or more
      actions and events which that [part] behaviour initiates,
      respectively is subjected to.
\begynd
\pind Actions, to recall, are planned, purposeful state changes. \index{pdefind}{action}
\pind Events are surreptitious state changes. \index{pdefind}{event}
\pind The actor, \index{pdefind}{actor} i.e., the [part] behaviour plan actions and await events.
\afslut
\afslut
}

\mnewfoil

\nysf{%%%%%%%%%%%%%%%%%%
\monoexample{Road Transport -- Actions}{%%%%%%
  %%\input{actions}
\begin{multicols}{2}
\begin{itemize}
\item \sort{Automobile Actions:}
\begin{enumerate}\setei
\item \label{action010} \texttt{progress\_around\_hub},
\item \label{action020} \texttt{leave\_hub\_enter\_link},
\item \label{action030} \texttt{disappears\_from\_road\_net},
\item \label{action040} \texttt{progress\_along\_link},
\item \label{action050} \texttt{idle\_on\_link}\footnote{We do not
    consider automobiles idling in hubs.} and
\item \label{action060} \texttt{leave\_link\_enter\_hub}.
\savei\end{enumerate}
\item \sort{Hub Actions:} none\,!
\item \sort{Link Actions:} none\,!
\end{itemize}
\end{multicols}
\noindent
\begynd
\pind We omit a number of actions:
      \texttt{accellerate\_auto, brake\_auto,} etc.
\afslut
}%%%%%%%%%%%%%%%%%%%%%%%%%%%%%%%%%%%%%%%%%%%%%%%%%%%%%%%    
}%%%%%%%%%%%%%%%%%%%%%%%

\noindent
We continue our treatment of actions in Sect.\,\vref{Action and Event Description.ii}.

\nbbbbb{Behaviour Signatures}\label{Behaviour Definition Description}\label{Behaviour Signatures}

\begynd
\pind Behaviours have to be described.
\begynd
\pind Behaviour definitions are in the form of \nysf{tail-recursive} function definitions and
      are here expressed in \texttt{RSL} \nyl relying, very much,
      on its \texttt{CSP} component.
\pind \nysf{The tail-recursion expresses that the behaviour goes on,
      potentially ``forever''\,!}
\pind Behaviour definitions describe \nyl the type of the arguments
      \nyl \ysf{that} the function, i.e., the behaviour, \ysf{accepts}.
\afslut

\mnewfoil

\pind Thus there are two elements to a behaviour definition:
\begynd
\pind the behaviour \sfsl{signature}
      \index{pconind}{signature!behaviour}\index{pconind}{behaviour!signature}%  
      and
\pind the behaviour \sfsl{body}
      \index{pconind}{body!function definition}\index{pconind}{function!definition body}%%
\afslut definitions.
\afslut

\begynd
\pind \nysf{Behaviour signatures indicate that behaviours evolve
      around 
\begynd
\pind the internal qualities of the part from which the
      behaviour is transcendentally deduced, and
\pind the interaction with other [part] behaviours. 
\afslut
\pind Thus there are basically two elements to behaviour signatures:
\begynd
\pind The unique identifier, mereology and attributes element, 
\pind and the element of channel interface to potentially interacting
      other behaviours.}
\afslut
\afslut

\mnewfoil

\bookdefn{Signature}{
\begynd
\pind A signature is something that is associated with any behaviour
      and, hence, action.
\begynd
\pind A signature consists of two parts:
\begynd
\pind a name for the behaviour or action; and
\pind a function type expression,
\begynd
\pind the latter is typically of the form \textsf{$A${\RIGHTARROW}$B$},
\pind where $A$ is the type of the behaviour or action input [arguments],
\pind and $B$ is the type of the behaviour or action output, i.e.,
      result \dbsquare
\afslut
\afslut
\afslut
\afslut}

\nbbbb{Domain Behaviour Signatures}\label{Domain Behaviour Signatures}

\bnew
\begynd
\pind The next  \bbcolor{perdurant description functions} are those of
      the description of the [part] behaviour and [part] action signatures.

\pind We shall develop a variety of possible behaviour and action signatures.
\begynd
\pind From a very simple
\pind to a full-fledged ``traditional''signatures.
\afslut

\pind In all cases the [part] behaviour (and [part] action)
      definitions ---
\begynd
\pind such as we have chosen them ---
\pind evolve around the part 
\pind \sfsl{``from which they are transcendentally deduced''},
\pind and the interaction with other [part] behaviours,
\pind what else could there be\,?
\afslut
\afslut

\mnewfoil

\begynd
\pind We shall illustrate two kinds of signatures:
\begynd
\pind simple part argument signatures
\begynd
\pind which imply that the behaviour 
\pind so-to-speak ``carries'' the whole part ``with it'';
\afslut
\pind and internal quality argument signatures
\begynd
\pind which imply that the behaviour,
\pind in the conventional manner of program procedure definitions
\pind has a number of internal quality constant, or variable, or
      reference arguments.
\afslut
\afslut
\afslut

\nbbb{Part Argument Behaviour Signatures}
\begynd
\pind The simplest behaviour signature expresses
\begynd
\pind that the behaviour, as a function definition, 
\pind takes a part,
\pind of the sort
\pind for which the behaviour is defined.
\afslut
\pind For each manifest part sort one behaviour definition.
\afslut

\vspace{2mm}
 
\ddprompt{describe\_behaviour\_signature, \sort{I}}{describe-behaviour-1}{
%\RSLatex
%value b: P -> channels  Unit   
%\endRSLatex 
\bp
\kw{value} b: P {\RIGHTARROW} channels\ \ \kw{Unit}\ \ \ 
\ep
}

\noindent
\begynd
\pind \textsf{b} is an arbitrarily chosen name for behaviours based on
      parts of sort  \textsf{P}.
\pind \textsf{channels} is an expression of channels based on the
      mereology of \textsf{P}.
\pind \sort{Unit} designates the value \textsf{()}, i.e., a
      state-to-state function.
\afslut

\monoexample{Signature}{
\noindent
\begynd
\pind A schematic example:
\afslut
%\RSLatex
%value
%   automobile: a:A -> out {ch[ai,ri]|ai:AI:-ai=&{\uid}&A(a)/\ri:(HI|LI):-ri isin &$his$&union&$lis$&}&\dbsquare&
%\endRSLatex 
\bp
\kw{value}\\
\>\ automobile: a:A {\RIGHTARROW} \kw{out} {\LBRACE}ch{\LBRACKET}ai,ri{\RBRACKET}{\BAR}ai:AI{\RDOT}ai{\EQ}{\uid}A(a){\WEDGE}ri:(HI{\BAR}LI){\RDOT}ri {\ISIN} $his${\UNION}$lis${\RBRACE}\dbsquare
\ep
}


\nbbb{Internal Quality Argument Behaviour Signatures}\label{Internal Quality Argument Behaviour Signatures}


\begynd
\pind The  internal quality argument behaviour signature \nyl ``unfolds'' the internal part
      qualities \nyl into separate arguments:
\begynd
\pind together these arguments 
\pind ``substitute'' for the simple behaviour
      and 
\pind action part argument.
\afslut
\afslut
\afslut

\begynd
\pind The internal quality arguments are:
\begynd
\pind the unique part identifier,
\pind the part mereology,
\pind the static attributes argument,
\pind the monitorable attribute names argument, and
\pind the programmable attributes argument.
\afslut
\pind The signature furthermore 
\begynd
\pind names the behaviour,
\pind the channels, and
\pind the state-to-state function designator \sort{Unit}.
\afslut
\afslut
\enew

\begynd
\pind A schematic form of part ($p$) behaviour signatures is:

\ddprompt{describe\_behaviour\_signature, \sort{II}}{describe-behaviour-2}{
%\RSLatex
%value  b: bi:BI->me:Mer->svl:StaV-list->mvl:MonV-list->prgl:PrgV-list channels  Unit   
%\endRSLatex 
\bp
\kw{value}\ \ b: bi:BI{\RIGHTARROW}me:Mer{\RIGHTARROW}svl:StaV$^{\ast}${\RIGHTARROW}mvl:MonV$^{\ast}${\RIGHTARROW}prgl:PrgV$^{\ast}$ channels\ \ \kw{Unit}\ \ \ 
\ep
}

\noindent
\pind  We shall motivate the general form of part behaviour,
\textsf{B}, signatures, ``step-by-step'':
\afslut

\mnewfoil
 
\noindent 
\begin{tabular}{crl}
$\alpha.$ & \ \  \textsf{\ysfchg{b}} & \ \ the [chosen] name of part $p$ behaviours.\\
$\beta.$ & \ \ \textsf{U{\RIGHTARROW}V{\RIGHTARROW}...{\RIGHTARROW}W{\RIGHTARROW}Z}:
  & \ \ The function signature is expressed in the 
  \texttt{Sch{\"o}nfinkel/Curry}\footnotemark  \\ \ \ & & \ \ style --
  corresponding to the invocation form \textsf{F(u)(v)...(w)} \\
$\gamma.$ & \ \ \textsf{bi:BI}: & \ \ a  general value and the type of part $p$ unique
  identifier\\
$\delta.$ & \ \ \textsf{me:Mer}:  & \ \ a general value and the type of part $p$ mereology\\
$\epsilon.$ & \ \ \textsf{svl:StaV$^{\ast}$}: &  \ \  a general (possibly empty) list of values and types of
  part $p$'s \\ & & \ \ (possibly empty) list of static attributes\\
$\zeta.$ & \ \ \textsf{mvl:MonV$^{\ast}$}: &  \ \   a general list of names of types of
  part $p$'s\\ &  &  \ \ (possibly empty) list of monitorable attributes\\
$\eta.$ & \ \ \textsf{prgl:PrgV$^{\ast }$}:  &  \ \ a general list of values and types of
  part $p$'s \\  & & \ \ (possibly empty)  list of programmable attributes\\ 
 $\theta.$& \ \ \textsf{channels}:  & \ \ are usually of the form:
  \textsf{{\LBRACE}ch{\LBRACKET}{\LBRACE}i,j{\RBRACE}{\RBRACKET}{\BAR}(i,j){\ISIN}$I$(me){\RBRACE}} 
  and express the subset \\ &  &  \ \ of channels over which behaviour  \textsf{B}s interact with other behaviors \\
$\iota$. & \kw{Unit}: &  \ \ designates the single value, (), {\sort{Unit}}
\end{tabular}
\footnotetext{Moses Sch{\"o}nfinkel (1888--1942)  was a %Russian
  logician and mathematician accredited with having
  invented combinatory logic
  [https://en.wikipedia.org/wiki/Moses\_Sch{\"o}nfinkel]. Haskell
  B. Curry (1900--1982) was a %n American
  mathematician and logician known for his
  work in combinatory logic
  [https://en.wikipedia.org/wiki/Haskell\_Curry]}

\vspace{2mm}\normalsize
\mnewfoil

\noindent
In detail:
\begin{itemize}
\item [$\alpha.$] \sort{Behaviour name:} In each domain description there are many sorts, \textsf{B},
  of parts. For each sort there is a generic behaviour, whose name,
  here \textsf{\ysfchg{b}}\ysfchgii{, } is
  chosen to suitably reflect \textsf{B}.
\item [$\beta.$] \sort{Currying} is here used in the pragmatic sense of grouping
  ``same kind of arguments'', i.e., separating these from one-another,
  by means of the {\RIGHTARROW}s.
\item [$\gamma.$] The \sort{unique identifier} of part sort \textsf{B} is here chosen
  to be \textsf{BI}. Its value is a constant.
\item [$\delta.$] The \sort{mereology} is a usually constant. For same part sorts it
  may be a variable.
\end{itemize}\normalsize

\mnewfoil

%\pos{}{\sort{An Aside:}}
\begin{quote}
\monoexample{Variable Mereologies}{%%%
\begynd
\pind For a road transport system  where we focus on the transport the
      mereology is a constant. 
\pind For a road net where we focus on the development of the road
      net: building new roads: inserting and removing hubs and links, the mereology is a variable.
\pind Similar remarks apply to canal systems
      \texttt{{www.imm.\-dtu.\-dk/\~{}dibj/\-2021/\-Graphs/\-Ri\-vers-and-Ca\-nals.pdf}}, 
      pipeline systems \cite{2013pipe}, container terminals \cite{BjornerContainer2018},
      assembly line systems \cite{BjornerTeslaSept2021}, etc. \dbsquare
      \afslut}\pos{}{}
\end{quote}
  
\mnewfoil

\begin{itemize}
\item [$\epsilon$.] \sort{Static attribute values} are constants. The use of static
  attribute values in behaviour body definitions is expressed by an
  identifier of the \textsf{svl} list of identifiers.
\item [$\zeta$.] \sort{Monitorable  attribute values} are generally, 
          ascertainable, i.e., readable, cf.\,Sect.\,\vref{Evaluation
          of Monitorable Attributes}. 
          Some are \sfsl{biddable}, can be changed by a, or the
          behaviour, cf.\,Sect.\,\ref{Update of Biddable Attributes}
          on Page\,\pageref{Update of Biddable Attributes},
          but there is no guarantee, as for programmable 
          attributes, that they remain fixed.
          \pos{\vspace*{1mm}}{}
\begynd
\pind   The use of a[ny] monitorable
  attribute value in behaviour body definitions is expressed by a \textsf{read\_A\_from\_P(mv,bi)}
  where \textsf{mv} is an identifier of the \textsf{mvl} list of
          identifiers and \textsf{bi} is the unique part identifier of
          the behaviour definition in which the \textsf{read} occurs.
\pind  The update of a biddable attribute value  in behaviour body
          definitions is expressed by a
          \textsf{update\_P\_with\_A(bi,mv,a)}.
\afslut
\mnewfoil
\item [$\eta$.] \sort{Programmable attribute values} are just that. They
  vary as specified, i.e., ``programmed'', by the behaviour body
  definition. Tail-recursive invocations of behaviour \textsf{B$_i$}
  ``replace'' relevant programmable attribute argument list elements
  with ``new'' values.
\item [$\theta$.] \sort{channels:} \textsf{$I$(me)} expresses a set of unique
  part identifiers different from \textsf{bi}, hence of behaviours,
  with which behaviour \textsf{b(i)} interacts.
\item [$\iota$.] The \kw{Unit} of the behaviour signature is a short-hand
  for the behaviour either \sort{read}ing the value of a monitorable 
  attribute, hence global state $\sigma$, or performing a
  \sort{write}, i.e., an \sfsl{update}, on $\sigma$. 
\end{itemize}

\nbbbbb{Action Signatures and General Form of Action Definitions}\label{Action and Event Description.ii}

\nysf{%%%%%%%%%%%%%%
\begynd
\pind Actions come in basically one signature form, like for
      behaviours, cf.\ Sect.\,\vref{Internal Quality Argument Behaviour Signatures}:
\begin{enumerate}\setei
\item \label{as010} likewise determine every action of that [part] behaviour.
\savei\end{enumerate}
\noindent
And the action definition, i.e., the ``body'', is of the general form
\begin{enumerate}\setei
\item \label{as020} first a description, \textsf{act}, of the action proper, one in
  which both the part mereology and the programmable attributes may be
  changed, 
\item \label{as030} then the tail-recursive invocation of the possibly
  updated [part] behaviour.
\savei\end{enumerate}
\afslut

\mnewfoil

\ddprompt{describe\_action}{action-act}{\index{dltind}{describe\_act}%%%%%%%%%%%%%%%%%
%\RSLatex
%value
%&\ref{as010}.&   action: bi:BI -> mer:Mer -> svl:StaV-list -> mvl:MonV-list -> prgl:PrgV-list  channels  Unit 
%&\ref{as020}.&   act: bi:BI -> mer:Mer -> svl:StaV-list -> mvl:MonV-list -> prgl:PrgV-list  channels  Unit 
%&\ref{as010}.&   action(bi)(mer)(svl)(mvl)(prgl) is
%&\ref{as020}.&      let (mer&$'$&,prgl&$'$&) = act(bi)(mer)(svl)(mvl)(prgl) in
%&\ref{as030}.&      behaviour(bi)(mer&$'$&)(svl)(mvl)(prgl&$'$&) end
%\endRSLatex
\bp
\kw{value}\\
\ref{as010}.\ \ \ action: bi:BI {\RIGHTARROW} mer:Mer {\RIGHTARROW} svl:StaV$^{\ast}$ {\RIGHTARROW} mvl:MonV$^{\ast}$ {\RIGHTARROW} prgl:PrgV$^{\ast}$\ \ channels\ \ \kw{Unit} \\
\ref{as020}.\ \ \ act: bi:BI {\RIGHTARROW} mer:Mer {\RIGHTARROW} svl:StaV$^{\ast}$ {\RIGHTARROW} mvl:MonV$^{\ast}$ {\RIGHTARROW} prgl:PrgV$^{\ast}$\ \ channels\ \ \kw{Unit} \\
\ref{as010}.\ \ \ action(bi)(mer)(svl)(mvl)(prgl) {\IS}\\
\ref{as020}.\ \ \ \ \ \ \kw{let} (mer$'$,prgl$'$) {\EQ} act(bi)(mer)(svl)(mvl)(prgl) \kw{in}\\
\ref{as030}.\ \ \ \ \ \ behaviour(bi)(mer$'$)(svl)(mvl)(prgl$'$) \kw{end}
\ep
}%%%%%%%%%%%%%%%%%%%%%%%%%%%%%%%%

\mnewfoil

\begin{enumerate}\setei
\item \label{as040} The \textsf{act} is of basically three forms:
\begin{enumerate}
\item \label{as050} either it is an ``active'' form in which it initiates
                    an interaction with another behavior, \textsf{bj},
\item \label{as060} or it is \ysfchg{a } ``passive'' form in which it awaits
                    an interaction with another behavior, \textsf{bj},
\item \label{as070} or it is neither, ie., it is some simple \texttt{RSL} clause.
\end{enumerate}
\savei\end{enumerate}


%\RSLatex
%&\ref{as040}.&  act: ... ch[{bi,bj}] ! val ...
%&\ref{as050}.&  act: ... let id = ch[{bi,bj}] ? in ... end ...
%&\ref{as060}.&  act: ... 
%\endRSLatex 
\bp
\ref{as040}.\ \ act: {\DOTDOTDOT} ch{\LBRACKET}{\LBRACE}bi,bj{\RBRACE}{\RBRACKET} ! val {\DOTDOTDOT}\\
\ref{as050}.\ \ act: {\DOTDOTDOT} \kw{let} id {\EQ} ch{\LBRACKET}{\LBRACE}bi,bj{\RBRACE}{\RBRACKET} ? \kw{in} {\DOTDOTDOT} \kw{end} {\DOTDOTDOT}\\
\ref{as060}.\ \ act: {\DOTDOTDOT} 
\ep

\mnewfoil


\monoexample{Automobile, Hub and Link Signatures}{
\begin{enumerate}\setei              
\item \label{process-040} \textsf{automobile}:
\begin{enumerate}
\item \label{process-042} There is the usual ``triplet'' or arguments:
                          unique identifier, mereology, static (...)
                          and monitorable (...) attributes; 
\item \label{process-044} programmable attributes: automobile location
                          and automobile history; 
\item \label{process-046} and channel references \nyl allowing communication between
                          the automobile and the hub and link behaviours. 
\end{enumerate}
\item \label{process-047} Similar for hubs and link behaviours.
\savei\end{enumerate}

%\cacmboiteepaisseavecuntitre{Formalisation}
%\RSLatex
%value
%&\ref{process-042}& automobile: ai:AI -> (_,uis):AM -> ... -> ... 
%&\ref{process-044}&        -> (A_Loc >< A_Hist)
%&\ref{process-046}&           out {ch[{ai,ui}]|ui:(HI|LI) :- ui isin &$his$&union&$lis$&} Unit
%&\ref{process-047}a&    hub: hi:HI -> (lis,ais,rni):HM -> (H`Omega >< ...) -> ... 
%&\ref{process-047}b&        -> (H`Sigma><H_Hist)
%&\ref{process-047}c&           in {ch[{hi,ui}]|ui:(LI|HI|RNI)-set :- ui isin &lis&union&his&union&rni&} Unit 
%&\ref{process-047}a&    link: li:LI -> (his,ais,rni):LM -> (LEN >< L`Omega >< ...)
%&\ref{process-047}b&        -> (L`Sigma><L_Hist)
%&\ref{process-047}c&           in {ch[{li,ui}]|ui:(LI|HI|RNI)-set :- li isin &lis&union&his&union&rni&} Unit
%\endRSLatex
\bp
\kw{value}\\
\ref{process-042} automobile: ai:AI {\RIGHTARROW} ({\UNDERLINE},uis):AM {\RIGHTARROW} {\DOTDOTDOT} {\RIGHTARROW} {\DOTDOTDOT} \\
\ref{process-044}\ \ \ \ \ \ \ \ {\RIGHTARROW} (A\_Loc {\TIMES} A\_Hist)\\
\ref{process-046}\ \ \ \ \ \ \ \ \ \ \ \kw{out} {\LBRACE}ch{\LBRACKET}{\LBRACE}ai,ui{\RBRACE}{\RBRACKET}{\BAR}ui:(HI{\BAR}LI) {\RDOT} ui {\ISIN} $his${\UNION}$lis${\RBRACE} \kw{Unit}\\
\ref{process-047}a\ \ \ \ hub: hi:HI {\RIGHTARROW} (lis,ais,rni):HM {\RIGHTARROW} (H$\Omega$ {\TIMES} {\DOTDOTDOT}) {\RIGHTARROW} {\DOTDOTDOT} \\
\ref{process-047}b\ \ \ \ \ \ \ \ {\RIGHTARROW} (H$\Sigma${\TIMES}H\_Hist)\\
\ref{process-047}c\ \ \ \ \ \ \ \ \ \ \ \kw{in} {\LBRACE}ch{\LBRACKET}{\LBRACE}hi,ui{\RBRACE}{\RBRACKET}{\BAR}ui:(LI{\BAR}HI{\BAR}RNI)\kw{-set} {\RDOT} ui {\ISIN} lis{\UNION}his{\UNION}rni{\RBRACE} \kw{Unit} \\
\ref{process-047}a\ \ \ \ link: li:LI {\RIGHTARROW} (his,ais,rni):LM {\RIGHTARROW} (LEN {\TIMES} L$\Omega$ {\TIMES} {\DOTDOTDOT})\\
\ref{process-047}b\ \ \ \ \ \ \ \ {\RIGHTARROW} (L$\Sigma${\TIMES}L\_Hist)\\
\ref{process-047}c\ \ \ \ \ \ \ \ \ \ \ \kw{in} {\LBRACE}ch{\LBRACKET}{\LBRACE}li,ui{\RBRACE}{\RBRACKET}{\BAR}ui:(LI{\BAR}HI{\BAR}RNI)\kw{-set} {\RDOT} li {\ISIN} lis{\UNION}his{\UNION}rni{\RBRACE} \kw{Unit}
\ep
}%\endboiteepaisseavecuntitre
}%%%%%%%%%%%%%%%%%%%

\label{before:Behaviour Invocation}

\pos{}{\label{Behaviour Call}}

\nbbbbb{Behaviour Invocation}\label{Behaviour Invocation}

\bookdefn{Invocation}{
\begynd
\pind By action or behaviour, i.e., function, invocation
\begynd
\pind we shall understand 
\pind the act of initiating, invoking, 
\pind that which is prescribed by the action or behaviour definition \dbsquare
\afslut
\afslut}

\begynd
\pind The general form of behaviour invocation is shown below.
\begynd
\pind The invocation follows the ``Currying'' of the behaviour type signature.
\pind {[Normally one would write all this on one line: \textsf{b(i)(m)(s)(m)(p)}.]}
\afslut
\afslut

%\RSLatex
%    behaviour
%       (unique_identifier)
%          (mereology)
%             (static_values)
%                (monitorable_attribute_names)
%                   (programmable_variables)
%\endRSLatex 
\bp
\>\>behaviour\\
\>\>\>\ (unique\_identifier)\\
\>\>\>\>\>(mereology)\\
\>\>\>\>\>\>\ (static\_values)\\
\>\>\>\>\>\>\>\>(monitorable\_attribute\_names)\\
\>\>\>\>\>\>\>\>\>\ (programmable\_variables)
\ep

\mnewfoil

\noindent
\begynd
\pind When first ``invoked''\pos{, that is, transcendentally deduced, \nyl i.e.,
      ``morphed'', from a manifest part, $p$, \nyl  the invocation looks like}{}:
\afslut

\LLLL
\ddprompt{describe\_behaviour\_signature, \sort{II}}{describe-behaviour-signature}{%%%%%%%%%%%%%%%%
\delati{describe\_behaviour\_invocation}
%\RSLatex
%   value
%      describe_behaviour_signature: P -> &\rsltext&
%      describe_behaviour_signature(p) is
%      &\bq\,&behaviour:
%              UId -> Mereo -> StaVL -> MonVL -> ProVL -> channels Unit
%         behaviour
%              (uid_B(p))
%                 (mereo_B(p))
%                    (types_to_values(static_attribute_types(p)))
%                       (mon_attribute_types(p))
%                          (types_to_values(programmable_attribute_types(p)))
%          pre: is_manifest(p) &\eq&
%
%      describe_behaviour_signatures: Unit -> &\rsltext\label{DBSs}&
%      describe_behaviour_signatures() is&\delati{describe\_behaviour\_signatures}&
%          { describe_behaviour_signature(p) | p isin &$\sigma$& /\ is_manifest(p) }    
%\endRSLatex 
\bp
\>\ \kw{value}\\
\>\>\>describe\_behaviour\_signature: P {\RIGHTARROW} \rsltext\\
\>\>\>describe\_behaviour\_signature(p) {\IS}\\
\>\>\>\bq\,behaviour:\\
\>\>\>\>\>\>\>UId {\RIGHTARROW} Mereo {\RIGHTARROW} StaVL {\RIGHTARROW} MonVL {\RIGHTARROW} ProVL {\RIGHTARROW} channels \kw{Unit}\\
\>\>\>\>\ behaviour\\
\>\>\>\>\>\>\>(uid\_B(p))\\
\>\>\>\>\>\>\>\>\ (mereo\_B(p))\\
\>\>\>\>\>\>\>\>\>\>(types\_to\_values(static\_attribute\_types(p)))\\
\>\>\>\>\>\>\>\>\>\>\>\ (mon\_attribute\_types(p))\\
\>\>\>\>\>\>\>\>\>\>\>\>\ \ (types\_to\_values(programmable\_attribute\_types(p)))\\
\>\>\>\>\>\kw{pre}: is\_manifest(p) \eq\\
\\
\>\>\>describe\_behaviour\_signatures: \kw{Unit} {\RIGHTARROW} \rsltext\label{DBSs}\\
\>\>\>describe\_behaviour\_signatures() {\IS}\delati{describe\_behaviour\_signatures}\\
\>\>\>\>\>{\LBRACE} describe\_behaviour\_signature(p) {\BAR} p {\ISIN} $\sigma$ {\WEDGE} is\_manifest(p) {\RBRACE}\ \ \ \ 
\ep
}

\nbbbbb{Behaviour Definition Bodies}\label{Behaviour Definition Bodies}

\bookdefn{Behaviour Definition}{
\begynd
\pind By behaviour definition
\begynd
\pind we shall understand 
\pind the prescription of that characterises the behaviour \dbsquare
\afslut
\afslut}

\noindent
\begynd
\pind In general a \textsf{behaviour} alternates between
\begynd
\pind a number, \textsf{m}, of actions, \textsf{act\_action\_i}, that
      either actively initiates 
      interaction with other behaviours or do not engage in interactions, or
\pind a number, \textsf{n}, of actions, \textsf{pas\_action\_j}, that
      passively seek such interaction.
\afslut
\pind The alternation between the former is \sort{internal
      non-deterministic}, {\NONDETCHOICE}, i.e., it is the
      \textsf{behaviour} that determines in which alternative to engage.
\pind The alternation between the latter is \sort{external
      non-deterministic}, {\DETCHOICE}, i.e., it is the
      \textsf{behaviour} that determines in which alternative to engage.
\afslut
\mnewfoil

\nbbbb{Behaviour Definition Schema I}

\begynd
\pind In Schema I
\begynd
\pind some lines designate non-deterministic actions,
\pind other deterministic actions.
\afslut
\afslut
\afslut


%\cacmboiteepaisseavecuntitre{Behaviour Definition Schema, I}
%\RSLatex   
%value
%   b(bi)(me)(svl)(mvl)(prl) is
%     &\,\,\,\,&non-deterministic_action_1(bi)(me)(svl)(mvl)(prl)
%    |^| non-deterministic_action_2(bi)(me)(svl)(mvl)(prl)
%       ...
%    |^| non-deterministic_action_n(bi)(me)(svl)(mvl)(prl)
%    |=| deterministic_action_1(bi)(me)(svl)(mvl)(prl)
%    |=| deterministic_action_2(bi)(me)(svl)(mvl)(prl)
%       ...
%    |=| deterministic_action_d(bi)(me)(svl)(mvl)(prl)
%\endRSLatex 
\bp
\kw{value}\\
\>\ b(bi)(me)(svl)(mvl)(prl) {\IS}\\
\>\>\ \,\,\,\,non{\MINUS}deterministic\_action\_1(bi)(me)(svl)(mvl)(prl)\\
\>\>{\NONDETCHOICE} non{\MINUS}deterministic\_action\_2(bi)(me)(svl)(mvl)(prl)\\
\>\>\>\ {\DOTDOTDOT}\\
\>\>{\NONDETCHOICE} non{\MINUS}deterministic\_action\_n(bi)(me)(svl)(mvl)(prl)\\
\>\>{\DETCHOICE} deterministic\_action\_1(bi)(me)(svl)(mvl)(prl)\\
\>\>{\DETCHOICE} deterministic\_action\_2(bi)(me)(svl)(mvl)(prl)\\
\>\>\>\ {\DOTDOTDOT}\\
\>\>{\DETCHOICE} deterministic\_action\_d(bi)(me)(svl)(mvl)(prl)
\ep
%\endboiteepaisseavecuntitre

\nbbbb{Behaviour Definition Schema II}

\begynd
\pind In Schema II
\begynd
\pind we have made use of \textsf{Domain Description Prompt} \ref{action-act}'s 
\pind explication of action behaviours.
\afslut
\afslut
\afslut


%\cacmboiteepaisseavecuntitre{Behaviour Definition Schema, II}
%\RSLatex   
%value
%   b(bi)(me)(svl)(mvl)(prl) is
%      let (me&$'$&,prl&$'$&) = non-deterministic_act_1(bi)(me)(svl)(mvl)(prl)
%              &\,&  |^| non-deterministic_act_2(bi)(me)(svl)(mvl)(prl)
%              &\,&  ...
%              &\,&  |^| non-deterministic_act_n(bi)(me)(svl)(mvl)(prl)
%              &\,&  |=| deterministic_act_1(bi)(me)(svl)(mvl)(prl)
%              &\,&  |=| deterministic_act_2(bi)(me)(svl)(mvl)(prl)
%              &\,&  ...
%              &\,&  |=| deterministic_act_d(bi)(me)(svl)(mvl)(prl) in
%      b(bi)(me&$'$&)(svl)(mvl)(prl&$'$&) end   
%\endRSLatex 
\bp
\kw{value}\\
\>\ b(bi)(me)(svl)(mvl)(prl) {\IS}\\
\>\>\>\kw{let} (me$'$,prl$'$) {\EQ} non{\MINUS}deterministic\_act\_1(bi)(me)(svl)(mvl)(prl)\\
\>\>\>\>\>\>\>\,\ \ {\NONDETCHOICE} non{\MINUS}deterministic\_act\_2(bi)(me)(svl)(mvl)(prl)\\
\>\>\>\>\>\>\>\,\ \ {\DOTDOTDOT}\\
\>\>\>\>\>\>\>\,\ \ {\NONDETCHOICE} non{\MINUS}deterministic\_act\_n(bi)(me)(svl)(mvl)(prl)\\
\>\>\>\>\>\>\>\,\ \ {\DETCHOICE} deterministic\_act\_1(bi)(me)(svl)(mvl)(prl)\\
\>\>\>\>\>\>\>\,\ \ {\DETCHOICE} deterministic\_act\_2(bi)(me)(svl)(mvl)(prl)\\
\>\>\>\>\>\>\>\,\ \ {\DOTDOTDOT}\\
\>\>\>\>\>\>\>\,\ \ {\DETCHOICE} deterministic\_act\_d(bi)(me)(svl)(mvl)(prl) \kw{in}\\
\>\>\>b(bi)(me$'$)(svl)(mvl)(prl$'$) \kw{end}\ \ \ 
\ep
%\endboiteepaisseavecuntitre



\nbbbb{Describe Behaviour Definition Bodies}\HHHH\label{Describe Behaviour Definition Bodies}

\begynd
\pind In other words, 
\begynd
\pind for current lack of a more definitive methodology
\pind for ``describing'' the bodies of behaviour definitions
\pind we resort to ``\ldots''\,!
\afslut
\afslut

\ddprompt{describe\_behaviour\_definition[s]}{ describe-behaviour-definitions}{%%%%%%%%%%%%%%%%%
%\RSLatex
%   value
%      describe_behaviour_definition: P -> &\rsltext&
%      describe_behaviour_definition(p) is &[Scheme I or Schema II]&
%      
%      describe_behaviour_definitions: Unit -> &\rsltext\label{DBDs}&
%      describe_behaviour_definitions() is
%         { describe_behaviour_definition(p) | p isin &$\sigma$& /\ is_manifest(p) }
%\endRSLatex
\bp
\>\ \kw{value}\\
\>\>\>describe\_behaviour\_definition: P {\RIGHTARROW} \rsltext\\
\>\>\>describe\_behaviour\_definition(p) {\IS} [Scheme I or Schema II]\\
\>\>\>\\
\>\>\>describe\_behaviour\_definitions: \kw{Unit} {\RIGHTARROW} \rsltext\label{DBDs}\\
\>\>\>describe\_behaviour\_definitions() {\IS}\\
\>\>\>\>\ {\LBRACE} describe\_behaviour\_definition(p) {\BAR} p {\ISIN} $\sigma$ {\WEDGE} is\_manifest(p) {\RBRACE}
\ep
}

\nbbbbb{Behaviour, Action and Event Examples}
      
\monoexample{Automobile Behaviour}{\pos{We remind the \pos{reader}{student}
  of the main, running example  of this \primer, the
   of \sfsl{the road transport system} Example\footnote{%
  That is, examples \vref{Sketch of a Road Transport System UoD},
  \vref{Cartesian Automobiles},
  \vref{A Road Transport System Domain: Cartesians},
  \vref{Road Transport System: Sets of Hubs, Links and Automobiles},
  \vref{The Road Transport System Taxonomy},
  \vref{Manifest Parts and Structures},
  \vref{Unique Identifiers},
  \vref{Uniqueness of Road Net Identifiers},
  \vref{Rail Net Unique Identifiers},
  \vref{Mereology of a Road Net},
  \vref{Invariance of Road Nets},
%%  \vref{Possible Consequences of a Road Net Mereology},
  \vref{Road Net Attributes},
  \vref{Invariance of Road Net Traffic States},
  \vref{Road Transport -- Further Attributes}, and 
  \vref{Intentional Pull -- Road Transport}.}.}{}

\newcommand{\nasachg}[1]{#1}
\newcommand{\vbnfm}{}
\newcommand{\venfm}{}
\newcommand{\vnfm}{}

\mnewfoil

\paragraph*{\bgcolor{\HHHH Definitions: Automobile at a Hub}}

\begin{enumerate}\setei
\item \label{habff1}    We abstract automobile  behaviour \textsf{at} a 
                        \textsf{H}{ub} (\textsf{hi}).
\item \label{habff1ind} Internally non-deterministically, an
                        automobile \ysfchg{ } 
\item \label{habff1bx}  either progresses around the hub
\item \label{habff1bz}  or leaves the hub to enter a link.
\savei\end{enumerate}

%\mnewfoil\pos{\footnotesize\small\normalsize}{\HHHH}\sf
%\RSLatex
%&\ref{habff1}&  automobile(ai)(aai,uis)(...)(apos:atH(fli,hi,tli),ahist) is
%&\ref{habff1bx},\ref{action010}&      automobile_progress_around_hub(ai)(aai,uis)(...)(apos:atH(fli,hi,tli),ahist)
%&\ref{habff1ind}&          |^|
%&\ref{habff1bz},\ref{action020}&      automobile_leave_hub_enter_link(ai)(aai,uis)(...)(apos:atH(fli,hi,tli),ahist)
%\endRSLatex 
\bp
\ref{habff1}\ \ automobile(ai)(aai,uis)({\DOTDOTDOT})(apos:atH(fli,hi,tli),ahist) {\IS}\\
\ref{habff1bx},\ref{action010}\ \ \ \ \ \ automobile\_progress\_around\_hub(ai)(aai,uis)({\DOTDOTDOT})(apos:atH(fli,hi,tli),ahist)\\
\ref{habff1ind}\ \ \ \ \ \ \ \ \ \ {\NONDETCHOICE}\\
\ref{habff1bz},\ref{action020}\ \ \ \ \ \ automobile\_leave\_hub\_enter\_link(ai)(aai,uis)({\DOTDOTDOT})(apos:atH(fli,hi,tli),ahist)
\ep

\mnewfoil
\begin{enumerate}\setei
\item \label{habff-10} {[\ref{action010}]} The automobile progresses around the hub:
\begin{enumerate}
\item \label{habff-20} the automobile  at that hub,  
\item \label{habff-30} informing (``first'') the hub behaviour.
\end{enumerate}
\savei\end{enumerate}

%\RSLatex
%&\ref{habff-10},\ref{action010}&  automobile_progress_around_hub(ai)(aai,uis)(...)(atH(fli,hi,tli),ahist) is
%&\ref{habff-10}&      let `tau = &\recordtime\ & in
%&\ref{habff-30}&     ch[ai,hi] ! `tau ;
%&\ref{habff-20}&     automobile(ai)(aai,uis)(...)(atH(fli,hi,tli),upd_hist(`tau,hi)(ahist))
%&\ref{habff-10}&      end
%
%&\ref{habff-20}&   upd_hist: (&$\mathbb{TIME}$&><UI) -> (AHist->AHist)|(HHist->HHist)|(LHist->LHist)  
%&\ref{habff-20}&   upd_hist(`tau,ui)(hist) is hist !! [ui +> <.`tau.>^hist(ui)]
%\endRSLatex 
\bp
\ref{habff-10},\ref{action010}\ \ automobile\_progress\_around\_hub(ai)(aai,uis)({\DOTDOTDOT})(atH(fli,hi,tli),ahist) {\IS}\\
\ref{habff-10}\ \ \ \ \ \ \kw{let} $\tau$ {\EQ} \recordtime\  \kw{in}\\
\ref{habff-30}\ \ \ \ \ ch{\LBRACKET}ai,hi{\RBRACKET} ! $\tau$ ;\\
\ref{habff-20}\ \ \ \ \ automobile(ai)(aai,uis)({\DOTDOTDOT})(atH(fli,hi,tli),upd\_hist($\tau$,hi)(ahist))\\
\ref{habff-10}\ \ \ \ \ \ \kw{end}\\
\\
\ref{habff-20}\ \ \ upd\_hist: ($\mathbb{TIME}${\TIMES}UI) {\RIGHTARROW} (AHist{\RIGHTARROW}AHist){\BAR}(HHist{\RIGHTARROW}HHist){\BAR}(LHist{\RIGHTARROW}LHist)\ \ \\
\ref{habff-20}\ \ \ upd\_hist($\tau$,ui)(hist) {\IS} hist {\DAGGER} {\LBRACKET}ui {\MAPSTO} {\LANGLE}$\tau${\RANGLE}{\CONCAT}hist(ui){\RBRACKET}
\ep


\mnewfoil

\begin{enumerate}\setei
\item \label{habff49} The automobile leaves the hub entering a link:
\begin{enumerate} 
\item \label{habff5}  \textsf{tli}, whose ``next'' 
                     hub, identified by \textsf{thi}, is obtained from
                     the mereology of the link identified by \textsf{tli};  
\item \label{habff6} informs the hub it is leaving and the link it is
                     entering,
\item \label{habff7} ``whereupon'' the vehicle resumes (i.e., ``while
                     at the same time'' resuming) the vehicle
                     behaviour positioned at the very beginning
                     (\textsf{0}) of that link.
\end{enumerate}
\savei\end{enumerate}
%\RSLatex
%&\ref{habff49}&  automobile_leave_hub_enter_link(ai)(aai,uis)(...)(apos:atH(fli,hi,tli),ahist) is
%&\ref{habff5}&     (let ({fhi,thi},ais) = mereo_L(retr_L(tli)(`sigma)) in &\kw{assert:}& fhi=hi
%&\ref{habff6}&      ( ch[ai,hi] ! `tau || ch[ai,tli] ! `tau ) ;
%&\ref{habff7}&      automobile(ai)(aai,uis)(...)(onL(tli,(hi,thi),0),upd_hist(`tau,tli)(upd_hist(`tau,hi)(ahist))) end)
%\endRSLatex 
\bp
\ref{habff49}\ \ automobile\_leave\_hub\_enter\_link(ai)(aai,uis)({\DOTDOTDOT})(apos:atH(fli,hi,tli),ahist) {\IS}\\
\ref{habff5}\ \ \ \ \ (\kw{let} ({\LBRACE}fhi,thi{\RBRACE},ais) {\EQ} mereo\_L(retr\_L(tli)($\sigma$)) \kw{in} \kw{assert:} fhi{\EQ}hi\\
\ref{habff6}\ \ \ \ \ \ ( ch{\LBRACKET}ai,hi{\RBRACKET} ! $\tau$ {\PARL} ch{\LBRACKET}ai,tli{\RBRACKET} ! $\tau$ ) ;\\
\ref{habff7}\ \ \ \ \ \ automobile(ai)(aai,uis)({\DOTDOTDOT})(onL(tli,(hi,thi),0),upd\_hist($\tau$,tli)(upd\_hist($\tau$,hi)(ahist))) \kw{end})
\ep


\mnewfoil

\begin{enumerate}\setei
\item \label{xxxxx} {[\ref{action030}]}  Or the  automobile ``disappears --- off the radar''~!
\savei\end{enumerate}

%\RSLatex
%&\ref{xxxxx},\ref{action030}&  automobile_stop(ai)(aai,uis),(...)(apos:atH(fli,hi,tli),ahist) is stop
%\endRSLatex 
\bp
\ref{xxxxx},\ref{action030}\ \ automobile\_stop(ai)(aai,uis),({\DOTDOTDOT})(apos:atH(fli,hi,tli),ahist) {\IS} \kw{stop}
\ep
\rm\pos{\normalsize}{\HHHH}

\noindent
\begynd
\pind Similar behaviour definitions can be given for \sfsl{automobiles
  on a link}, for \sfsl{link}s and for \sfsl{hub}s.
\pind Together they must reflect, amongst other things:
\begynd
\pind the time continuity of automobile flow,
\pind that automobiles follow routes,
\pind that automobiles, links and hubs together adhere to the
      intentional pull expressed earlier,
\pind et cetera. 
\afslut
\pind A specification of these aspects must be proved to adhere to
      these properties. 
\afslut
}

\nbbbbb{Domain [Behaviour] Initialisation}\label{Domain Behaviour Initialisation}\HHHH

\bookdefn{Domain Initialisation}{
\begynd
\pind By behaviour initialisation
\begynd
\pind we shall understand 
\pind the [initial] invocation
\pind of all part behaviours \dbsquare
\afslut
\afslut}

\noindent
\begynd
\pind For every manifest part sort
\begynd
\pind there is a single description: signature and definition \nyl (i.e., its syntax).
\afslut
\pind For every manifest part 
\begynd
\pind there is a behaviour \nyl (i.e., its semantics ``realization'').
\afslut
\pind For the total of all manifest domain parts there is their
      initialization:
\begynd
\pind the parallel ``execution'' 
\pind of the behaviour of each manifest part,
\pind properly initialized.
\afslut
\afslut

\ddprompt{describe\_domain\_initialisation}{describe-domain-initialisation}{%%%%
%\RSLatex
%  &\bq& || { b
%      (uid_P(p))
%         (mereo_P(p))
%            &\sort{analyse\_}&static_attribute_type_names_Cartesian(p)
%               &\sort{analyse\_}&monitorable_attribute_type_names_Cartesian(p)
%                  &\sort{analyse\_}&programmable_attribute_type_names_Cartesian(p) 
%                     | p:P :- p isin `sigma } &\eq&
%\endRSLatex 
\bp
\>\bq {\PARL} {\LBRACE} b\\
\>\>\>(uid\_P(p))\\
\>\>\>\>\ (mereo\_P(p))\\
\>\>\>\>\>\>\sort{analyse\_}static\_attribute\_type\_names\_Cartesian(p)\\
\>\>\>\>\>\>\>\ \sort{analyse\_}monitorable\_attribute\_type\_names\_Cartesian(p)\\
\>\>\>\>\>\>\>\>\>\sort{analyse\_}programmable\_attribute\_type\_names\_Cartesian(p) \\
\>\>\>\>\>\>\>\>\>\>\ {\BAR} p:P {\RDOT} p {\ISIN} $\sigma$ {\RBRACE} \eq
\ep
}


\mnewfoil


\monoexample{The Road Transport System Initialisation}{We ``wrap up''
             the main example of this \primer.
\begynd
\pind We omit treatment of monitorable attributes.
\afslut
\begin{enumerate}\setei
\item \label{rtsi000} Let us refer to the system initialisation as a\ysf{
  behaviour}.
\item \label{rtsi010} All links are initialised,
\item \label{rtsi020} all hubs are initialised,
\item \label{rtsi030} all automobiles are initialised,
\item \label{rtsi040} etc.
\savei\end{enumerate}

\mnewfoil\LLLL

%\RSLatex
%value
%&\ref{rtsi000}.&  rts_initialisation: Unit -> Unit
%&\ref{rtsi000}.&  rts_initialisation() is
%&\ref{rtsi010}.&     || { link(uid_L(l))(mereo_L(l))(attr_LEN(l),attr_L`Omega(l))(attr_L_Traffic(l),attr_L`Sigma(l))| l:L :- l isin &$ls$& }   
%&\ref{rtsi020}.&     || || { hub(uid_H(l))(mereo_H(l))(attr_H`Omega(l))(attr_H_Traffic(l),attr_H`Sigma(l))| h:H :- h isin &$hs$& }
%&\ref{rtsi030}.&     || || { automobile(uid_A(a))(mereo_A(a))(attr_RegNo(a))(attr_APos(a)) | a:A :- a isin &$as$& }
%&\ref{rtsi040}.&     || ...
%\endRSLatex 
\bp
\kw{value}\\
\ref{rtsi000}.\ \ rts\_initialisation: \kw{Unit} {\RIGHTARROW} \kw{Unit}\\
\ref{rtsi000}.\ \ rts\_initialisation() {\IS}\\
\ref{rtsi010}.\ \ \ \ \ {\PARL} {\LBRACE} link(uid\_L(l))(mereo\_L(l))(attr\_LEN(l),attr\_L$\Omega$(l))(attr\_L\_Traffic(l),attr\_L$\Sigma$(l)){\BAR} l:L {\RDOT} l {\ISIN} $ls$ {\RBRACE}\ \ \ \\
\ref{rtsi020}.\ \ \ \ \ {\PARL} {\PARL} {\LBRACE} hub(uid\_H(l))(mereo\_H(l))(attr\_H$\Omega$(l))(attr\_H\_Traffic(l),attr\_H$\Sigma$(l)){\BAR} h:H {\RDOT} h {\ISIN} $hs$ {\RBRACE}\\
\ref{rtsi030}.\ \ \ \ \ {\PARL} {\PARL} {\LBRACE} automobile(uid\_A(a))(mereo\_A(a))(attr\_RegNo(a))(attr\_APos(a)) {\BAR} a:A {\RDOT} a {\ISIN} $as$ {\RBRACE}\\
\ref{rtsi040}.\ \ \ \ \ {\PARL} {\DOTDOTDOT}
\ep

\noindent
\begynd
\pind We have here omitted possible monitorable attributes.
\pind We refer to 
\begynd
\pind $ls$: Item\,\vref{srares-030},
\pind $hs$: Item\,\vref{srares-040}, and
\pind $as$: Item\,\vref{srares-080}\dbsquare
\afslut
\afslut}
\label{before Discrete Dynamic Domains}

\pos{}{
\mnewfoil


\label{lects6-7}
\ \vfill
\centerline{\lilacolor{Day \#{7}: Perdurants, II}}
\vfill
}

\nbbbbb{Discrete Dynamic Domains}\label{Discrete Dynamic Domains}

\begynd
\pind Up till now our analysis \& description of a domain, 
\begynd
\pind has, in a sense, been \sfsl{static:}
\pind in analysing a domain we considered its entities
\pind to be of a definite number.
\afslut
\pind In this section we shall consider the case \nyl
      where the number of entities change:
\begynd
\pind where new entities are \sfsl{created}
\pind and existing entities are \sfsl{destroyed},
\pind that is:
\begynd
\pind where new parts, and hence behaviours, arise, and
\pind existing parts, and hence behaviours, cease to exist.
\afslut
\afslut
\afslut

\nbbbb{Create and Destroy Entities}\label{Create and Destroy Entities}

\begynd
\pind In the domain we can expect that its behaviours \nyl
      create and destroy entities.
\afslut

\monoexample{Creation and Destruction of Entities}{%
\begynd
\pind In the \sfsl{road transport} domain 
\begynd
\pind new hubs, links and automobiles \nyl may be inserted into the road net, and
\pind existing links, hubs and automobiles \nyl may be removed from the road net.
\afslut
\pind In a \sfsl{container terminal} domain \cite{db07:container,BjornerContainer2018}
\begynd
\pind new containers are introduced, old are discarded;
\pind new container vessels  are introduced, old are discarded;
\pind new ship-to-shore cranes   are introduced, old are discarded;
\pind et cetera.
\afslut
\mnewfoil
\pind In a \sfsl{retailer} domain \cite{BjornerRetailer2021}
\begynd
\pind new customers are introduced, old are discarded;
\pind new retailers are introduced, old are discarded;
\pind new merchandise is introduced, old is discarded;
\pind et cetera.
\afslut
\pind In a \sfsl{financial system} domain
\begynd
\pind new customers are introduced, old are discarded;
\pind new banks are introduced, old are discarded;
\pind new brokers are introduced, old are discarded;
\pind et cetera\dbsquare
\afslut
\afslut}

\mnewfoil
\noindent
\begynd
\pind \sort{The issue here is:}
\begynd
\pind When hubs and links are inserted or removed
\begynd
\pind the mereologies of ``neighbouring'' road elements change,
\pind and so does the mereology of automobiles.
\afslut
\pind When automobiles are inserted or removed  
\begynd
\pind \ysf{t}he mereology of road elements 
\pind have to be changed 
\pind to take account of the insertions and removals,
\pind and so does the mereology of automobiles.
\afslut
\pind And, some domain laws must be re-expressed:
\begynd
\pind The domain part state, $\sigma$, must be updated\footnote{\LLLL Cf.\ Sect.\,\vref{kap3.States.specific}},
\pind and so must the unique identifier state, ${uid}_{\sigma}$\footnote{\LLLL Cf.\ Sect.\,\vref{all-uniq-ids}}.
\afslut
\afslut
\afslut

\nbbb{Create Entities}\label{Create Entities}

\begynd
\pind It is taken for granted here that there are behaviours, \nyl one or more, which
      take the initiative to and carry out \nyl the creation of specific
      entities. \nyl Let us refer to such a behaviour as the ``creator''.
\pind To create an entity implies the following three major steps
\begynd
\pind {[A.--C.]} the step wise creation of the part \nyl and
                 initialisation of the transduced behaviour, and
\pind {[D.]} the adjustment of all such part behaviours that might
             have their mereologies and attributes updated to accept
             such requests from creators.
\afslut
\afslut
\mnewfoil
\begin{itemize}
\item[A.] To decide on the part sort -- in order to create that part -- that is
\begin{itemize}
\item to obtain a unique identifier -- one hitherto not used;
\item to obtain a mereology, one
\begin{itemize}
\item according to the general mereology for parts of that sort,
\item and how the part specifically is to ``fit'' into its surroundings;
\end{itemize}
\item to obtain an appropriate set of attributes:
\begin{itemize}
\item again according to the attribute types for that part sort
\item and, more specifically, choosing initial attribute
      values.
\end{itemize}
\item This part is then ``joined'' to \pos{the global part state,}{}
      $\sigma$\footnote{\LLLL \pos{}{(the global part state),} Cf.\ Sect.\,\vref{kap3.States.specific}}
      and 
\item its unique identifier ``joined'' to \pos{the global unique identifier
      state,}{} ${uid}_{\sigma}$\footnote{\LLLL \pos{}{(the global unique identifier state),} Cf.\
      Sect.\,\vref{all-uniq-ids}}. 
\end{itemize}
\mnewfoil
\item[B.] Then to transcendentally deduce that part into a behaviour:
\begin{itemize}
\item initialised (according to Sect.\,\ref{Behaviour Signatures}) with
\begin{itemize}
\item the unique identifier,
\item the mereology, and
\item the attribute values
\end{itemize} 
\item This behaviour is then invoked and ``joined'' to the set of
      current behaviours, cf.\,Sect.\,\vref{Domain Behaviour
      Initialisation} -- i.e., just above\,!
\end{itemize} 
\item[C.] Then, finally, to ``adjust'' the mereologies of
          topologically or conceptually related parts,
\begin{itemize}
\item that is, for  each of these parts to update:
\item their mereology and possibly some
\item state and state space
\end{itemize} arguments of their corresponding behaviours.
\end{itemize}
\mnewfoil

\begin{itemize}
\item[D.] The update of the mereologies \nyl of already ``running'' behaviours
      requires the following:
\begin{itemize}
\item that, potentially all, behaviours offers to accept  
\item mereology update requests from the ``creator'' behaviour.
\end{itemize}
\end{itemize}

\noindent
\begynd
\pind The latter means, practically speaking, 
\begynd
\pind that each part/behaviour 
\pind which may be subject to mereology changes
\pind externally non-deterministically 
\pind expresses an offer to accept such a change.
\afslut
\afslut

\monoexample{Road Net Administrator}{%
\begynd
\pind  We introduce the road net behaviour -- based
      on the road net composite part, \textsf{RN}.
\afslut
\mnewfoil\LLLL
\begin{enumerate}\setei
\item \label{ddd:rna:510} The road net has a programmable attribute: a
                          \rngraph.\footnote{\LLLL The presentation of the road
                          net Behaviour, \textsf{rn}, is simplified.}
\begynd
\pind The road net graph consists of a quadruple:
\begynd 
\pind a map that for each hub identifier records ``all'' the
                          information that the road net administrator
                          deems necessary\footnote{\LLLL We presently \label{wpafwthi}
                          abstract from what this information is.} for the maintenance and
                          development of road net hubs;
\pind a map that for each link identifier records ``all'' the
                          information\footnote{\LLLL See
                          footnote\,\vref{wpafwthi}.} that the road net
                          administrator 
                          deems necessary for the maintenance and
                          development of road net links;
\pind and a map from the hub identifiers to the set of identifiers
      of the links it is connected to, and
\pind the set of all automobile identifiers.  
\afslut
\afslut
\mnewfoil 
\item \label{ddd:rna:520} This graph is commensurate with the actual
                          topology of the road net.
\savei\end{enumerate}             
%\RSLatex
%type
%&\ref{ddd:rna:510}.&   G = (HI-m->H_Info) >< (LI-m->L_Info) >< (HI-m->LI-set) >< AI-set
%value
%&\ref{ddd:rna:510}.&   attr_G: RN -> G
%axiom
%&\ref{ddd:rna:510}.&   all (hi_info,li_info,map,ais):G :- 
%&\ref{ddd:rna:510}.&      dom map = dom hi_info = &$his$& /\ union rng map = dom li_info = &$lis$& /\
%&\ref{ddd:rna:520}.&      all hi:HI :- hi isin dom hi_info =>
%&\ref{ddd:rna:520}.&         let h:H :- h isin `sigma /\ uid_H(h)=hi in
%&\ref{ddd:rna:520}.&         let (lis&$'$&,...) = mereo_H(h) in lis&$'$& = map(hi)  
%&\ref{ddd:rna:520}.&         ais <<= &$a{_{ui}}s$\footnotemark& /\ ...
%&\ref{ddd:rna:520}.&         end end
%\endRSLatex
\bp
\kw{type}\\
\ref{ddd:rna:510}.\ \ \ G {\EQ} (HI{\MARROW}H\_Info) {\TIMES} (LI{\MARROW}L\_Info) {\TIMES} (HI{\MARROW}LI\kw{-set}) {\TIMES} AI\kw{-set}\\
\kw{value}\\
\ref{ddd:rna:510}.\ \ \ attr\_G: RN {\RIGHTARROW} G\\
\kw{axiom}\\
\ref{ddd:rna:510}.\ \ \ {\ALL} (hi\_info,li\_info,map,ais):G {\RDOT} \\
\ref{ddd:rna:510}.\ \ \ \ \ \ \kw{dom} map {\EQ} \kw{dom} hi\_info {\EQ} $his$ {\WEDGE} {\UNION} \kw{rng} map {\EQ} \kw{dom} li\_info {\EQ} $lis$ {\WEDGE}\\
\ref{ddd:rna:520}.\ \ \ \ \ \ {\ALL} hi:HI {\RDOT} hi {\ISIN} \kw{dom} hi\_info {\DBLRIGHTARROW}\\
\ref{ddd:rna:520}.\ \ \ \ \ \ \ \ \ \kw{let} h:H {\RDOT} h {\ISIN} $\sigma$ {\WEDGE} uid\_H(h){\EQ}hi \kw{in}\\
\ref{ddd:rna:520}.\ \ \ \ \ \ \ \ \ \kw{let} (lis$'$,{\DOTDOTDOT}) {\EQ} mereo\_H(h) \kw{in} lis$'$ {\EQ} map(hi)\ \ \\
\ref{ddd:rna:520}.\ \ \ \ \ \ \ \ \ ais {\SUBSETEQ} $a{_{ui}}s$\footnotemark {\WEDGE} {\DOTDOTDOT}\\
\ref{ddd:rna:520}.\ \ \ \ \ \ \ \ \ \kw{end} \kw{end}
\ep
\ysfchg{\footnotetext{The $a{_{ui}}s$ was defined in
    Sect.\,\ref{Sect.luis-auis}, Item\,\ref{uic-040} on Page\,\pageref{auis}.}}
\mnewfoil 
\noindent
\begynd
\pind Please note the fundamental difference between
\begynd
\pind the \rngraph\ and
\pind the road net.
\afslut
\pind The latter pretends to be ``the real thing''.
\pind The former is ``just'' an abstraction thereof\,!
\afslut
%\dbeat{%%%%%%%%%%%%%%%%%%%%%%%%%%%%%%%%%%%%%%%%%%%%%%%%%%%%%%%%%%%%%%%%%%%%%%%%%%%%%%%%%%%%%%%
\mnewfoil                       
\begin{enumerate}\setei
\item \label{ddd:rna:100} The road net mereology (``bypasses'') the
                          hub and link aggregates, and comprises a set
                          of hub identifiers and a set of link
                          identifiers -- of the road
                          net\footnote{\LLLL This is a repeat of the
                          hub mereology given in Item\,\vref{mereo-000}.}.
\savei\end{enumerate}            
%\RSLatex 
%type
%&\ref{ddd:rna:100}.&   H_Mer = AI-set >< LI-set
%&\ref{ddd:rna:100}.&   mereo_RN: RN -> RNMer
%axiom
%&\ref{ddd:rna:100}.&   all rts:RTS :- let (_,lis) = mereo_H(obs_RN(rts)) in lis <<= &$l_{ui}s$\footnotemark& end
%\endRSLatex
\bp
\kw{type}\\
\ref{ddd:rna:100}.\ \ \ H\_Mer {\EQ} AI\kw{-set} {\TIMES} LI\kw{-set}\\
\ref{ddd:rna:100}.\ \ \ mereo\_RN: RN {\RIGHTARROW} RNMer\\
\kw{axiom}\\
\ref{ddd:rna:100}.\ \ \ {\ALL} rts:RTS {\RDOT} \kw{let} ({\UNDERLINE},lis) {\EQ} mereo\_H(obs\_RN(rts)) \kw{in} lis {\SUBSETEQ} $l_{ui}s$\footnotemark \kw{end}
\ep
\ysfchg{\footnotetext{The $l{_{ui}}s$ was defined in Sect.\,\ref{Sect.luis-auis}, Item\,\ref{uic-010} on Page\,\pageref{luis}.}}
%}%%%%%%%%%%%%%%%%%%%%%%%%%%%%%%%%%%%%%%%%%%%%%%%%%%%%%%%%%%%%%%%%%%%%%%%%%%%%%%%%%%%%%%%%%%%%%%%%%
\mnewfoil

\begin{enumerate}\setei
\item \label{ddd:rna:110} The road net [administrator] behaviour,
\item \label{ddd:rna:120} amongst other activities (\ldots)
\item \label{ddd:rna:130} internal non-deterministically  decides upon 
\begin{enumerate}
\item \label{ddd:rna:140} either a hub insertion,
\item \label{ddd:rna:150} or a link insertion,
\item \label{ddd:rna:160} or a hub removal,
\item \label{ddd:rna:170} or a link removal;
\end{enumerate}         
\savei\end{enumerate}
\noindent
\begynd
\pind These four sub-behaviours each resume being the road net behaviour.
\afslut
\mnewfoil
\pos{}{\vspace*{-10mm}}
%\RSLatex
%value
%&\ref{ddd:rna:110}.&   rn: RNI -> RNMer -> G -> in,out{ch[{i,j}]|{i,j}<<=&${uid}_{\sigma}$&}
%&\ref{ddd:rna:110}.&   rn(rni)(rnmer)(g) is
%&\ref{ddd:rna:120}.&        ...
%&\ref{ddd:rna:140}.&      |^| insert_hub(g)(rni)(rnmer)
%&\ref{ddd:rna:150}.&      |^| insert_link(g)(rni)(rnmer)        
%&\ref{ddd:rna:160}.&      |^| remove_hub(g)(rni)(rnmer)
%&\ref{ddd:rna:170}.&      |^| remove_link(g)(rni)(rnmer)
%\endRSLatex   
\bp
\kw{value}\\
\ref{ddd:rna:110}.\ \ \ rn: RNI {\RIGHTARROW} RNMer {\RIGHTARROW} G {\RIGHTARROW} \kw{in},\kw{out}{\LBRACE}ch{\LBRACKET}{\LBRACE}i,j{\RBRACE}{\RBRACKET}{\BAR}{\LBRACE}i,j{\RBRACE}{\SUBSETEQ}${uid}_{\sigma}${\RBRACE}\\
\ref{ddd:rna:110}.\ \ \ rn(rni)(rnmer)(g) {\IS}\\
\ref{ddd:rna:120}.\ \ \ \ \ \ \ \ {\DOTDOTDOT}\\
\ref{ddd:rna:140}.\ \ \ \ \ \ {\NONDETCHOICE} insert\_hub(g)(rni)(rnmer)\\
\ref{ddd:rna:150}.\ \ \ \ \ \ {\NONDETCHOICE} insert\_link(g)(rni)(rnmer)\ \ \ \ \ \ \ \ \\
\ref{ddd:rna:160}.\ \ \ \ \ \ {\NONDETCHOICE} remove\_hub(g)(rni)(rnmer)\\
\ref{ddd:rna:170}.\ \ \ \ \ \ {\NONDETCHOICE} remove\_link(g)(rni)(rnmer)
\ep
                                                    
\pos{%%%%%%%%%%%%%%%%%%%%
\noindent 
\begynd
\pind Details on the \textsf{insert} and \textsf{remove} actions are given below.
\afslut
}{}%%%%%%%%%%%%%%%%%%%%%%

\mnewfoil
\begin{enumerate}\setei
\item  \label{ddd:rna:900} These road net sub-behaviours require information about
\begin{enumerate}
\item  \label{ddd:rna:910} a hub to be inserted: its
                           initial state, state space and [empty]
                           traffic history, or 
\item  \label{ddd:rna:920} a link to be inserted:
                           its length, initial state, state space and
                           [empty] traffic history, or
\item  \label{ddd:rna:930} a hub to be removed: its
                           unique identifier, or
\item  \label{ddd:rna:940} a link to be removed: its
                           unique identifier.
\end{enumerate}
\savei\end{enumerate}
%\RSLatex
%type
%&\ref{ddd:rna:900}.&   Info == nHInfo | nLInfo | oHInfo | oLInfo
%&\ref{ddd:rna:900}.&   nHInfo :: H`Sigma >< H`Omega >< H_Traffic
%&\ref{ddd:rna:900}.&   nLInfo :: LEN >< L`Sigma >< L`Omega >< L_Traffic 
%&\ref{ddd:rna:900}.&   oHInfo :: HI
%&\ref{ddd:rna:900}.&   oLInfo :: LI &\dbsquare&
%\endRSLatex
\bp
\kw{type}\\
\ref{ddd:rna:900}.\ \ \ Info {\EQ}{\EQ} nHInfo {\BAR} nLInfo {\BAR} oHInfo {\BAR} oLInfo\\
\ref{ddd:rna:900}.\ \ \ nHInfo :: H$\Sigma$ {\TIMES} H$\Omega$ {\TIMES} H\_Traffic\\
\ref{ddd:rna:900}.\ \ \ nLInfo :: LEN {\TIMES} L$\Sigma$ {\TIMES} L$\Omega$ {\TIMES} L\_Traffic \\
\ref{ddd:rna:900}.\ \ \ oHInfo :: HI\\
\ref{ddd:rna:900}.\ \ \ oLInfo :: LI \dbsquare
\ep
}

\mnewfoil

\monoexample{Road Net Development: Hub Insertion}{%
\begynd
\pind Road net development alternates between design,
\begynd
\pind based on the \rngraph, and
\afslut
\pind actual, ``real life'', construction
\begynd
\pind taking place in the real surroundings of the road net.
\afslut
\afslut
\mnewfoil
\begin{enumerate}\setei    
\item \label{ddd:hi:125} If a hub insertion then the  road net
                          behaviour, \nyl based on the hub and link information and
                          the road net layout \nyl  in the \rngraph\  selects
\begin{enumerate}
\item \label{ddd:hi:125a} an initial mereology for the hub, \textsf{h\_mer},
\item \label{ddd:hi:125b} an initial hub state,  \textsf{h$\sigma$}, and
\item \label{ddd:hi:125c} an initial hub state space, \textsf{h$\omega$}, and
\item \label{ddd:hi:125d} an initial, i.e., empty hub traffic history;
\end{enumerate}   
\item \label{ddd:hi:125e} updates its \rngraph\ with information about the
                         new hub, 
\item \label{ddd:hi:125f} and results in a suitable grouping of these.
\savei\end{enumerate}

\mnewfoil

%\RSLatex
%value
%&\ref{ddd:hi:125}.&   design_new_hub: G -> (nHInfo><G) 
%&\ref{ddd:hi:125}.&   design_new_hub(g) is  
%&\ref{ddd:hi:125a}.&      let h_mer:HMer = &$\mathcal{M}_{ih}$&(g),
%&\ref{ddd:hi:125b}.&          h`sigma:H`Sigma = &$\mathcal{S}_{ih}$&(g),
%&\ref{ddd:hi:125c}.&          h`omega:H`Omega = &$\mathcal{O}_{ih}$&(g),
%&\ref{ddd:hi:125d}.&          h_traffic = [],
%&\ref{ddd:hi:125e}.&            g&$'$& = &$\mathcal{MSO}_{ih}$&(g) in
%&\ref{ddd:hi:125f}.&         ((h_mer,h`sigma,h`omega,h_traffic),g&$'$&) end
%\endRSLatex
\bp
\kw{value}\\
\ref{ddd:hi:125}.\ \ \ design\_new\_hub: G {\RIGHTARROW} (nHInfo{\TIMES}G) \\
\ref{ddd:hi:125}.\ \ \ design\_new\_hub(g) {\IS}\ \ \\
\ref{ddd:hi:125a}.\ \ \ \ \ \ \kw{let} h\_mer:HMer {\EQ} $\mathcal{M}_{ih}$(g),\\
\ref{ddd:hi:125b}.\ \ \ \ \ \ \ \ \ \ h$\sigma$:H$\Sigma$ {\EQ} $\mathcal{S}_{ih}$(g),\\
\ref{ddd:hi:125c}.\ \ \ \ \ \ \ \ \ \ h$\omega$:H$\Omega$ {\EQ} $\mathcal{O}_{ih}$(g),\\
\ref{ddd:hi:125d}.\ \ \ \ \ \ \ \ \ \ h\_traffic {\EQ} {\emptymap},\\
\ref{ddd:hi:125e}.\ \ \ \ \ \ \ \ \ \ \ \ g$'$ {\EQ} $\mathcal{MSO}_{ih}$(g) \kw{in}\\
\ref{ddd:hi:125f}.\ \ \ \ \ \ \ \ \ ((h\_mer,h$\sigma$,h$\omega$,h\_traffic),g$'$) \kw{end}
\ep
\noindent
\begynd
\pind We leave open, in Items\,\ref{ddd:hi:125a}--\ref{ddd:hi:125c},
      as to what \nyl the initial hub mereology, state and state space
      should be initialised, i.e., the $\mathcal{M}_{ih}, \mathcal{S}_{ih},
      \mathcal{O}_{ih}$ and $\mathcal{MSO}_{ih}$  functions. 
\afslut

\mnewfoil

\begin{enumerate}\setei
\item \label{hi:1000}  To insert a new hub the road net administrator
\begin{enumerate}
\item \label{hi:1002}  first designs the new hub,
\item \label{hi:1004}  then selects a hub part 
\item \label{hi:1006}  which satisfies the design,
\item[]                whereupon it updates the global stat\ysfchg{e }
\item \label{hi:1010}  of parts $\sigma$,
\item \label{hi:1020}  of unique identifiers, and
\item \label{hi:1030}  of hub identifiers --
\end{enumerate} in parallel, and in parallel with
\item \label{hi:1040}  initiating a new hub behaviour
\item \label{hi:1050}  and resuming being the road net behaviour.
\savei\end{enumerate}

\mnewfoil


%\RSLatex
%&\ref{hi:1000}.&  insert_hub: G><RNI><RNMer -> Unit
%&\ref{hi:1000}.&  insert_hub(g,rni,rnmer) is
%&\ref{hi:1002}.&      let ((h_mer,h`sigma,h`omega,h_traffic),g&$'$&) = design_new_hub(g) in 
%&\ref{hi:1004}.&      let h:H :- h~isin`sigma :- 
%&\ref{hi:1006}.&                 mereo_H(h)=h_mer /\ h`sigma=attr_H`Sigma(h) /\ 
%&\ref{hi:1006}.&                 h`omega=attr_H`Omega(h) /\ h_traffic=attr_HTraffic(h) in
%&\ref{hi:1010}.&      `sigma := `sigma union {h}
%&\ref{hi:1020}.&   || uid&$_{\sigma}$& := uid&$_{\sigma}$& union {uid_H(h)}
%&\ref{hi:1030}.&    || &$his$& := &$his$& union {uid_H(h)}
%&\ref{hi:1040}.&     || hub(uid_H(h))(attr_H`Sigma(h),attr_H`Omega(h),attr_H`Omega(h))
%&\ref{hi:1050}.&     || rn(rni)(rnmer)(g&$'$&)
%&\ref{hi:1000}.&      end end  &\dbsquare&
%\endRSLatex 
\bp
\ref{hi:1000}.\ \ insert\_hub: G{\TIMES}RNI{\TIMES}RNMer {\RIGHTARROW} \kw{Unit}\\
\ref{hi:1000}.\ \ insert\_hub(g,rni,rnmer) {\IS}\\
\ref{hi:1002}.\ \ \ \ \ \ \kw{let} ((h\_mer,h$\sigma$,h$\omega$,h\_traffic),g$'$) {\EQ} design\_new\_hub(g) \kw{in} \\
\ref{hi:1004}.\ \ \ \ \ \ \kw{let} h:H {\RDOT} h{\NOTISIN}$\sigma$ {\RDOT} \\
\ref{hi:1006}.\ \ \ \ \ \ \ \ \ \ \ \ \ \ \ \ \ mereo\_H(h){\EQ}h\_mer {\WEDGE} h$\sigma${\EQ}attr\_H$\Sigma$(h) {\WEDGE} \\
\ref{hi:1006}.\ \ \ \ \ \ \ \ \ \ \ \ \ \ \ \ \ h$\omega${\EQ}attr\_H$\Omega$(h) {\WEDGE} h\_traffic{\EQ}attr\_HTraffic(h) \kw{in}\\
\ref{hi:1010}.\ \ \ \ \ \ $\sigma$ :{\EQ} $\sigma$ {\UNION} {\LBRACE}h{\RBRACE}\\
\ref{hi:1020}.\ \ \ {\PARL} uid$_{\sigma}$ :{\EQ} uid$_{\sigma}$ {\UNION} {\LBRACE}uid\_H(h){\RBRACE}\\
\ref{hi:1030}.\ \ \ \ {\PARL} $his$ :{\EQ} $his$ {\UNION} {\LBRACE}uid\_H(h){\RBRACE}\\
\ref{hi:1040}.\ \ \ \ \ {\PARL} hub(uid\_H(h))(attr\_H$\Sigma$(h),attr\_H$\Omega$(h),attr\_H$\Omega$(h))\\
\ref{hi:1050}.\ \ \ \ \ {\PARL} rn(rni)(rnmer)(g$'$)\\
\ref{hi:1000}.\ \ \ \ \ \ \kw{end} \kw{end}\ \ \dbsquare
\ep
}

\mnewfoil\LLLL\HHHH

\monoexample{Road Net Development: Link Insertion}{\LLLL\HHHH%
\begin{enumerate}\setei
\item \label{ddd:li:130} If a link insertion then the road net
                         behaviour \nyl based on the hub and link
                         information and 
                         the road net layout \nyl  in the \rngraph\  selects 
\begin{enumerate}
\item \label{ddd:li:140} the  mereology  for the link,
                         \textsf{h\_mer}\footnote{\LLLL that is, the
                         two existing hub identifiers between whose
                         hubs the new link is to be inserted},     
\item \label{ddd:li:145} the (static) length (attribute),
\item \label{ddd:li:150} an initial link state,   \textsf{l$\sigma$},
\item \label{ddd:li:160} an initial link state space
                         \textsf{l$\omega$}, and   
\item \label{ddd:li:170} and initial, i.e., empty, link traffic history;
\end{enumerate}
\item \label{ddd:li:180} updates its \rngraph\ \nyl with information
                         about the new link, 
\item \label{ddd:li:190} and results in a suitable grouping of these.
\savei\end{enumerate}
\mnewfoil

%\RSLatex
%value
%&\ref{ddd:li:130}.&   design_new_link: G -> (nLInfo><G)
%&\ref{ddd:li:130}.&   design_new_link(g) is 
%&\ref{ddd:li:140}.&      let l_mer:LMer = &$\mathcal{M}_{il}$&(g),
%&\ref{ddd:li:145}.&            le:LEN = &$\mathcal{L}_{il}$&(g),
%&\ref{ddd:li:150}.&            l`sigma:L`Sigma = &$\mathcal{S}_{il}$&(g),
%&\ref{ddd:li:160}.&            l`omega:L`Omega = &$\mathcal{O}_{il}$&(g),
%&\ref{ddd:li:170}.&            l_hist:L_Hist = []
%&\ref{ddd:li:180}.&              g&$'$&:G = &$\mathcal{MLSO}_{il}$&(g) in
%&\ref{ddd:li:190}.&        ((l_mer,le,l`sigma,l`omega,l_hist),g&$'$&) end
%\endRSLatex
\bp
\kw{value}\\
\ref{ddd:li:130}.\ \ \ design\_new\_link: G {\RIGHTARROW} (nLInfo{\TIMES}G)\\
\ref{ddd:li:130}.\ \ \ design\_new\_link(g) {\IS} \\
\ref{ddd:li:140}.\ \ \ \ \ \ \kw{let} l\_mer:LMer {\EQ} $\mathcal{M}_{il}$(g),\\
\ref{ddd:li:145}.\ \ \ \ \ \ \ \ \ \ \ \ le:LEN {\EQ} $\mathcal{L}_{il}$(g),\\
\ref{ddd:li:150}.\ \ \ \ \ \ \ \ \ \ \ \ l$\sigma$:L$\Sigma$ {\EQ} $\mathcal{S}_{il}$(g),\\
\ref{ddd:li:160}.\ \ \ \ \ \ \ \ \ \ \ \ l$\omega$:L$\Omega$ {\EQ} $\mathcal{O}_{il}$(g),\\
\ref{ddd:li:170}.\ \ \ \ \ \ \ \ \ \ \ \ l\_hist:L\_Hist {\EQ} {\emptymap}\\
\ref{ddd:li:180}.\ \ \ \ \ \ \ \ \ \ \ \ \ \ g$'$:G {\EQ} $\mathcal{MLSO}_{il}$(g) \kw{in}\\
\ref{ddd:li:190}.\ \ \ \ \ \ \ \ ((l\_mer,le,l$\sigma$,l$\omega$,l\_hist),g$'$) \kw{end}
\ep

\noindent
\begynd
\pind We leave open, in Items\,\ref{ddd:li:140}--\ref{ddd:li:160},
      as to what \nyl the initial link mereology, state and state space
      should be initialised.
\afslut
      
\mnewfoil
\begin{enumerate}\setei
\item \label{ddd:1i:2000} To insert a new link the road net
  administrator 
\begin{enumerate}
\item \label{ddd:1i:2002}  first designs the new link,
\item \label{ddd:1i:2004}  then selects a link part 
\item \label{ddd:1i:2006}  which satisfies the design,
\item[] whereupon it updates the global states
\item \label{ddd:1i:2010} of parts, $\sigma$,
\item \label{ddd:1i:2020} of unique part identifiers, and
\item \label{ddd:1i:2030} of link identifiers --
\end{enumerate} in parallel, and in parallel with
\item \label{ddd:1i:2040} initiating a new link behaviour and
\item \label{ddd:1i:2050} updating the mereologies and possibly the
                          state and the state space attributes of the
                          connected hubs.  
\savei\end{enumerate}
\mnewfoil
%\RSLatex
%value
%&\ref{ddd:1i:2000}.&   insert_link: G -> Unit
%&\ref{ddd:1i:2000}.&   insert_link(rni,l) is 
%&\ref{ddd:1i:2002}.&       let ((l_mer,le,l`sigma,l`omega,l_traffic_hist),g&$'$&) = design_new_link(g) in 
%&\ref{ddd:1i:2006}.&       let l:L :- l~isin`sigma :- mereo_L(l)=l_mer /\ 
%&\ref{ddd:1i:2006}.&                    le=attr_LEN(l) /\ l`sigma=attr_L`Sigma(l) /\ 
%&\ref{ddd:1i:2006}.&                    l`omega=attr_L`Omega(l) /\ l_traffic_hist=attr_HTraffic(l) in
%&\ref{ddd:1i:2010}.&      `sigma := `sigma union {l} 
%&\ref{ddd:1i:2020}.&   || uid&$_{\sigma}$& := uid&$_{\sigma}$& union {uid_L(l)}
%&\ref{ddd:1i:2030}.&    || &$lis$& := &$list$& union {}
%&\ref{ddd:1i:2040}.&     || link(uid_L(l))(l_mer)(le,l`omega)(l`sigma,l_traffic)
%&\ref{ddd:1i:2050}.&     || ch[{rni,hi1}] ! updH(&$\mathcal{M}_{il}$&(g),&$\Sigma_{il}$&(g),&$\Omega_{il}$&(g))
%&\ref{ddd:1i:2050}.&     || ch[{rni,hi2}] ! 
%&\ref{ddd:1i:2000}.&       end end &\dbsquare&
%\endRSLatex 
\bp
\kw{value}\\
\ref{ddd:1i:2000}.\ \ \ insert\_link: G {\RIGHTARROW} \kw{Unit}\\
\ref{ddd:1i:2000}.\ \ \ insert\_link(rni,l) {\IS} \\
\ref{ddd:1i:2002}.\ \ \ \ \ \ \ \kw{let} ((l\_mer,le,l$\sigma$,l$\omega$,l\_traffic\_hist),g$'$) {\EQ} design\_new\_link(g) \kw{in} \\
\ref{ddd:1i:2006}.\ \ \ \ \ \ \ \kw{let} l:L {\RDOT} l{\NOTISIN}$\sigma$ {\RDOT} mereo\_L(l){\EQ}l\_mer {\WEDGE} \\
\ref{ddd:1i:2006}.\ \ \ \ \ \ \ \ \ \ \ \ \ \ \ \ \ \ \ \ le{\EQ}attr\_LEN(l) {\WEDGE} l$\sigma${\EQ}attr\_L$\Sigma$(l) {\WEDGE} \\
\ref{ddd:1i:2006}.\ \ \ \ \ \ \ \ \ \ \ \ \ \ \ \ \ \ \ \ l$\omega${\EQ}attr\_L$\Omega$(l) {\WEDGE} l\_traffic\_hist{\EQ}attr\_HTraffic(l) \kw{in}\\
\ref{ddd:1i:2010}.\ \ \ \ \ \ $\sigma$ :{\EQ} $\sigma$ {\UNION} {\LBRACE}l{\RBRACE} \\
\ref{ddd:1i:2020}.\ \ \ {\PARL} uid$_{\sigma}$ :{\EQ} uid$_{\sigma}$ {\UNION} {\LBRACE}uid\_L(l){\RBRACE}\\
\ref{ddd:1i:2030}.\ \ \ \ {\PARL} $lis$ :{\EQ} $list$ {\UNION} {\LBRACE}{\RBRACE}\\
\ref{ddd:1i:2040}.\ \ \ \ \ {\PARL} link(uid\_L(l))(l\_mer)(le,l$\omega$)(l$\sigma$,l\_traffic)\\
\ref{ddd:1i:2050}.\ \ \ \ \ {\PARL} ch{\LBRACKET}{\LBRACE}rni,hi1{\RBRACE}{\RBRACKET} ! updH($\mathcal{M}_{il}$(g),$\Sigma_{il}$(g),$\Omega_{il}$(g))\\
\ref{ddd:1i:2050}.\ \ \ \ \ {\PARL} ch{\LBRACKET}{\LBRACE}rni,hi2{\RBRACE}{\RBRACKET} ! \\
\ref{ddd:1i:2000}.\ \ \ \ \ \ \ \kw{end} \kw{end} \dbsquare
\ep
\noindent
\begynd
\pind We leave undefined the mereology and the state $\sigma$ and
      state space $\omega$ \textsf{upd}ate functions.
\afslut
}

\nbbb{Destroy Entities}\label{Destroy Entities}

\begynd
\pind The introduction to Sect.\,\vref{Create Entities} \nyl on the \sfsl{creation of entities}
\begynd
\pind outlined a number of creation issues ([A, B, C, D]).
\afslut
\pind For the \sfsl{destruction of entities}
\begynd
\pind description matters are a bit simpler.
\afslut
\pind It is, almost, simply a matter
\begynd
\pind of designating, by its unique identifier,
\pind the entity: part and behaviour to be destroyed.
\afslut
\pind Almost\,!
\begynd
\pind The mereology of the destroyed entity
\pind must be such that the destruction
\pind does not leave ``dangling'' references\,!
\afslut
\afslut

\mnewfoil

\monoexample{Road Net Development: Hub Removal}{%
\begin{enumerate}\setei
\item \label{ddd:hr:100} If a hub removal  then the road net \textsf{design\_remove\_hub}
                         behaviour, based on the \rngraph,  calculates the \sfsl{unique hub identifier}
                         of the ``isolated'' hub to be removed --
                         that is, is not connected to any links,
\item \label{ddd:hr:110} updates the \rngraph, and
\item \label{ddd:hr:120} results in a pair of these.
\savei\end{enumerate}

%\RSLatex
%value
%&\ref{ddd:hr:100}.&   design_remove_hub: G -> (HI><G)
%&\ref{ddd:hr:100}.&   design_remove_hub(g) as (hi,g&$'$&)
%&\ref{ddd:hr:100}.&      let hi:HI :- hi isin &$his$& /\ let (_,lis) = mereo_H(retr_part(hi)) in lis={} end in
%&\ref{ddd:hr:110}.&      let g&$'$& = &$\mathcal{M}_{rh}$&(hi,g) in
%&\ref{ddd:hr:120}.&      (hi,g&$'$&) end end
%\endRSLatex 
\bp
\kw{value}\\
\ref{ddd:hr:100}.\ \ \ design\_remove\_hub: G {\RIGHTARROW} (HI{\TIMES}G)\\
\ref{ddd:hr:100}.\ \ \ design\_remove\_hub(g) \kw{as} (hi,g$'$)\\
\ref{ddd:hr:100}.\ \ \ \ \ \ \kw{let} hi:HI {\RDOT} hi {\ISIN} $his$ {\WEDGE} \kw{let} ({\UNDERLINE},lis) {\EQ} mereo\_H(retr\_part(hi)) \kw{in} lis{\EQ}{\LBRACE}{\RBRACE} \kw{end} \kw{in}\\
\ref{ddd:hr:110}.\ \ \ \ \ \ \kw{let} g$'$ {\EQ} $\mathcal{M}_{rh}$(hi,g) \kw{in}\\
\ref{ddd:hr:120}.\ \ \ \ \ \ (hi,g$'$) \kw{end} \kw{end}
\ep
}

\mnewfoil

\begin{enumerate}\setei
\item \label{ddd:hr:200} To remove a hub  the road net administrator
\begin{enumerate}
\item \label{ddd:hr:205} first designs which old hub is to be removed
\item \label{ddd:hr:210} then removes the designated hub, 
\item[] whereupon it updates the global states
\item \label{ddd:hr:215}  of parts $\sigma$,
\item \label{ddd:hr:220}  of unique identifiers, and
\item \label{ddd:hr:225}  of hub identifiers --
\end{enumerate} in parallel, and in parallel with
\item \label{ddd:hr:230}  stopping the old hub behaviour
\item \label{ddd:hr:235}  and resuming being a road net behaviour.
\savei\end{enumerate}
\mnewfoil

%\RSLatex
%value
%&\ref{ddd:hr:200}.&  remove_hub: G -> RNI -> RNMer -> Unit
%&\ref{ddd:hr:200}.&  remove_hub(g)(rni)(rnmer) is 
%&\ref{ddd:hr:205}.&      let (hi,g&$'$&) = design_remove_hub(g) in
%&\ref{ddd:hr:210}.&      let h:H :- uid_H(h)=hi /\ ... in
%&\ref{ddd:hr:215}.&      `sigma := `sigma \ {h} 
%&\ref{ddd:hr:220}.&   || uid&$_{\sigma}$& := uid&$_{\sigma}$& \ {hi}
%&\ref{ddd:hr:225}.&   || &$his$& := &$his$& \ {hi} 
%&\ref{ddd:hr:230}.&     || ch[{rni,hi}] ! mkStop()
%&\ref{ddd:hr:235}.&     || rn(rni)(rnmer)(g&$'$&)
%&\ref{ddd:hr:200}.&        end end &\dbsquare&
%\endRSLatex 
\bp
\kw{value}\\
\ref{ddd:hr:200}.\ \ remove\_hub: G {\RIGHTARROW} RNI {\RIGHTARROW} RNMer {\RIGHTARROW} \kw{Unit}\\
\ref{ddd:hr:200}.\ \ remove\_hub(g)(rni)(rnmer) {\IS} \\
\ref{ddd:hr:205}.\ \ \ \ \ \ \kw{let} (hi,g$'$) {\EQ} design\_remove\_hub(g) \kw{in}\\
\ref{ddd:hr:210}.\ \ \ \ \ \ \kw{let} h:H {\RDOT} uid\_H(h){\EQ}hi {\WEDGE} {\DOTDOTDOT} \kw{in}\\
\ref{ddd:hr:215}.\ \ \ \ \ \ $\sigma$ :{\EQ} $\sigma$ {\SETMINUS} {\LBRACE}h{\RBRACE} \\
\ref{ddd:hr:220}.\ \ \ {\PARL} uid$_{\sigma}$ :{\EQ} uid$_{\sigma}$ {\SETMINUS} {\LBRACE}hi{\RBRACE}\\
\ref{ddd:hr:225}.\ \ \ {\PARL} $his$ :{\EQ} $his$ {\SETMINUS} {\LBRACE}hi{\RBRACE} \\
\ref{ddd:hr:230}.\ \ \ \ \ {\PARL} ch{\LBRACKET}{\LBRACE}rni,hi{\RBRACE}{\RBRACKET} ! mkStop()\\
\ref{ddd:hr:235}.\ \ \ \ \ {\PARL} rn(rni)(rnmer)(g$'$)\\
\ref{ddd:hr:200}.\ \ \ \ \ \ \ \ \kw{end} \kw{end}  \dbsquare
\ep

\mnewfoil
\dbeat{%%%%%%%%%%%%%%%%%%%%%%%%%%%%%%%%%%%%%%%%%%%%%%%%%%%%%%%%%%%%%%%%%%%%%%%%%%%%%%%%
\monoexample{Road Net Development: Link Removal}{%
\begynd
\pind
\begin{enumerate}\setei
\item \label{ddd:lr:150} If a link removal then the road net
                         behaviour  calculates the \sfsl{unique identifier}
                         of the link to be removed.
\item \label{ddd:lr:200}
\item \label{ddd:lr:210}
\item \label{ddd:lr:220}
\item \label{ddd:lr:230}
\item \label{ddd:lr:240}
\item \label{ddd:lr:250}
\item \label{ddd:lr:260}
\item \label{ddd:lr:270}
\item \label{ddd:lr:280}
\item \label{ddd:lr:290}
  \savei\end{enumerate}

%\RSLatex
%&\ref{ddd:lr:150}.& 
%&\ref{ddd:lr:150}.& 
%&\ref{ddd:lr:200}.&  
%&\ref{ddd:lr:210}.&  
%&\ref{ddd:lr:220}.&  
%&\ref{ddd:lr:230}.&  
%&\ref{ddd:lr:240}.&
%&\ref{ddd:lr:250}.&  
%&\ref{ddd:lr:260}.&  
%&\ref{ddd:lr:270}.&  
%&\ref{ddd:lr:280}.&  
%&\ref{ddd:lr:290}.&     
%\endRSLatex 
\bp
\ref{ddd:lr:150}. \\
\ref{ddd:lr:150}. \\
\ref{ddd:lr:200}.\ \ \\
\ref{ddd:lr:210}.\ \ \\
\ref{ddd:lr:220}.\ \ \\
\ref{ddd:lr:230}.\ \ \\
\ref{ddd:lr:240}.\\
\ref{ddd:lr:250}.\ \ \\
\ref{ddd:lr:260}.\ \ \\
\ref{ddd:lr:270}.\ \ \\
\ref{ddd:lr:280}.\ \ \\
\ref{ddd:lr:290}.\ \ \ \ \ 
\ep

\begynd
\pind 
\pind 
\afslut
\pind 
\afslut
}
}%%%%%%%%%%%%%%%%%%%%%%%%%%%%%%%%%%%%%%%%%%%%%%%%%%%%%%%%%%%%%%%%%%%%%%%%%%%

\nbbbb{Adjustment of Creatable and Destructable Behaviours}\label{Adjustment}

\begynd
\pind When an entity
\begynd
\pind is created or destroyed 
\pind its creation, respectively destruction
\pind affects the neurologically related parts
      and their behaviours.
\begynd
\pind their mereology 
\pind and possibly their programmable state attributes
\pind need be adjusted.
\afslut
\pind And when entities are destroyed \nyl
      their behaviours are \sort{stop}ped\,!
\pind These entities are ``informed'' so by the creator/destructor
      entity \nyl -- as was shown in Examples\,\ref{Road Net
      Development: Hub Insertion}--\ref{Road Net Development: Hub
      Removal}. 
\afslut
\pind The next example will illustrate how such `affected' entities
      \nyl handle such  creator/destructor communication.
\afslut

\mnewfoil

\monoexample{Hub Adjustments}{%
\begynd
\pind We have not yet illustrated hub (nor link) behaviours.
\pind Now we have to\,!
\afslut
\begin{enumerate}\setei
\item \label{hub-adj:0900} The mereology of a hub is a triple: \nyl the
  identification of the set of automobiles that may enter the hub,  \nyl the
  identification of the set of links that connect to the hub, \nyl and
  the identification of the road net.
\item \label{hub-adj:1000} The hub behaviour external
                            non-deterministically ({\DETCHOICE})
                            alternates between
\item \label{hub-adj:1010}  doing ``own work'',
\item \label{hub-adj:1020} or   accepting a stop ``command'' from the road
                            net administrator, or
\item \label{hub-adj:1030} or   accepting mereology \& state update
                            information,
\item \label{hub-adj:1040} or  other.
\savei\end{enumerate}

\mnewfoil

%\RSLatex
%type
%&\ref{hub-adj:0900}.&   HMer = AI-set >< LI-set >< RNI
%value
%&\ref{hub-adj:0900}.&   mereo_H: H -> HMer
%&\ref{hub-adj:1000}.&   hub: hi:HI -> (auis,lis,&$rni$&):HMer -> h`omega:H`Omega -> (h`sigma:H`Sigma><ht:HTraffic) ->
%&\ref{hub-adj:1000}.&            {ch[hi,ui]|ui:(RNI|AI) :- ui=&$rni$&\/ui isin auis}  Unit
%&\ref{hub-adj:1000}.&   hub(hi)(hm:(auis,lis,rni))(h`omega)(h`sigma,ht) is
%&\ref{hub-adj:1010}.&           ...
%&\ref{hub-adj:1020}.&       |=| let mkStop() = ch[hi,&$rni$&] ? in stop end
%&\ref{hub-adj:1030}.&       |=| let mkUpdH(hm&$'$&,h`sigma&$'$&,h`sigma&$'$&) = ch[{&$rni$&,hi}] ? in
%&\ref{hub-adj:1030}.&           hub(hi)(hm&$'$&)(h`omega&$'$&)(h`sigma&$'$&,ht) end     
%&\ref{hub-adj:1040}.&           ... 
%\endRSLatex 
\bp
\kw{type}\\
\ref{hub-adj:0900}.\ \ \ HMer {\EQ} AI\kw{-set} {\TIMES} LI\kw{-set} {\TIMES} RNI\\
\kw{value}\\
\ref{hub-adj:0900}.\ \ \ mereo\_H: H {\RIGHTARROW} HMer\\
\ref{hub-adj:1000}.\ \ \ hub: hi:HI {\RIGHTARROW} (auis,lis,$rni$):HMer {\RIGHTARROW} h$\omega$:H$\Omega$ {\RIGHTARROW} (h$\sigma$:H$\Sigma${\TIMES}ht:HTraffic) {\RIGHTARROW}\\
\ref{hub-adj:1000}.\ \ \ \ \ \ \ \ \ \ \ \ {\LBRACE}ch{\LBRACKET}hi,ui{\RBRACKET}{\BAR}ui:(RNI{\BAR}AI) {\RDOT} ui{\EQ}$rni${\VEE}ui {\ISIN} auis{\RBRACE}\ \ \kw{Unit}\\
\ref{hub-adj:1000}.\ \ \ hub(hi)(hm:(auis,lis,rni))(h$\omega$)(h$\sigma$,ht) {\IS}\\
\ref{hub-adj:1010}.\ \ \ \ \ \ \ \ \ \ \ {\DOTDOTDOT}\\
\ref{hub-adj:1020}.\ \ \ \ \ \ \ {\DETCHOICE} \kw{let} mkStop() {\EQ} ch{\LBRACKET}hi,$rni${\RBRACKET} ? \kw{in} \kw{stop} \kw{end}\\
\ref{hub-adj:1030}.\ \ \ \ \ \ \ {\DETCHOICE} \kw{let} mkUpdH(hm$'$,h$\sigma$$'$,h$\sigma$$'$) {\EQ} ch{\LBRACKET}{\LBRACE}$rni$,hi{\RBRACE}{\RBRACKET} ? \kw{in}\\
\ref{hub-adj:1030}.\ \ \ \ \ \ \ \ \ \ \ hub(hi)(hm$'$)(h$\omega$$'$)(h$\sigma$$'$,ht) \kw{end}\ \ \ \ \ \\
\ref{hub-adj:1040}.\ \ \ \ \ \ \ \ \ \ \ {\DOTDOTDOT} 
\ep
\noindent
\begynd
\pind Observe from formula Item\,\ref{hub-adj:1020} that the hub
      behaviour ends, 
\pind whereas ``from'' Item\,\ref{hub-adj:1030} it tail recurses\,! \dbsquare
\afslut
}


\nbbbb{Summary on Creatable \& Destructable Entities}

\begynd
\pind We have sketched how we may model \nyl
      the dynamics of creating and destroying entities.
\begynd
\pind It is, but a sketch.
\pind We should wish for a more methodological account.
\pind So, that is what we are working on -- amongst other issues -- at the moment.
\afslut
\afslut

\nbbbbb{Domain Engineering: Description and Construction}\label{Domain Engineering: Description and Construction}

\begynd
\pind There are two meanings to the term `Domain Engineering'.
\begin{itemize}
\item the construction of \sfsl{descriptions} of domains, and
\item  the construction of \sfsl{domains}.
\end{itemize}
\noindent
\begynd
\pind Most sections of Chapters\,\ref{primer-extq.1}--\ref{chapter:Perdurants} \nyl are
      ``devoted'' to the former;
\pind the previous section, Sect.\,\ref{Discrete Dynamic Domains} to the latter.
\afslut
\afslut

\ysfchgv{ \dbeat{
\nbbbbb{Domain Laws}\label{perd-Domain Laws}

\begynd
\pind The\footnote{This section is currently under consideration.}
      issue of \sfsl{domain laws} seems to be crucial.
\begynd
\pind Inklings of \sfsl{domain laws} have been hinted at:
\begynd
\pind (i) intentional pulls, Sect.\,\ref{chap4.Intentional Pull} and
\pind (ii) Galois connections, Sect.\,\ref{kap4-Galois Connections}.
\afslut
\afslut
\afslut
}}

\nbbbbb{A Domain Discovery Procedure, III}\label{A Domain Discovery
  Process, IIIa}\label{The Perdurant Analysis and Description
  Procedure} 

%\pos
{The predecessors of this section are Sects.\,\vref{A Domain
    Discovery Process, I} and \vref{A Domain Discovery Process, II}.}{} 

\bbbb{Review of the Endurant Analysis and Description Process}%
      \label{Review of The Endurant Analysis and Description Process}\LLLL

\begynd
\pind The \textsf{describe\_...} functions below were defined in Sects.\,\vref{A Domain
    Discovery Process, I} and \vref{A Domain Discovery Process, II}.
\afslut

\label{endurantanalysisanddescription}
%\RSLatex
%value
%   endurant_analysis_and_description: Unit -> Unit
%   endurant_analysis_and_description() is 
%      discover_sorts();  &\ \ \ \ \ \ \ \ \ \ \ \ \ \ \ \ \ \ \ \ \ \ \ \ \ \ \ \ \ \ \ \ \ \ [Page\,\pageref{discover-sorts}]&
%      discover_internal_endurant_qualities() & \ \ [Page\,\pageref{discover-uids}]&
%\endRSLatex
\bp
\kw{value}\\
\>\ endurant\_analysis\_and\_description: \kw{Unit} {\RIGHTARROW} \kw{Unit}\\
\>\ endurant\_analysis\_and\_description() {\IS} \\
\>\>\>discover\_sorts();\ \ \ \ \ \ \ \ \ \ \ \ \ \ \ \ \ \ \ \ \ \ \ \ \ \ \ \ \ \ \ \ \ \ \ \ [Page\,\pageref{discover-sorts}]\\
\>\>\>discover\_internal\_endurant\_qualities()  \ \ [Page\,\pageref{discover-uids}]
\ep
\noindent
\begynd
\pind We are now to define a
      \textsf{perdurant\_analysis\_and\_description}
      procedure -- 
\pind to follow the above \textsf{endurant\_an\-a\-ly\-sis\_\-and\_\-des\-crip\-tion}
      procedure.
\afslut
 
\nbbbb{A Domain Discovery Process, III}%
      \label{A Domain Discovery Process, III}

\begynd
\pind We define the  \textsf{perdurant\_analysis\_and\_description}
      procedure
\begynd
\pind in the reverse order of that of Sect.\,\vref{A Domain Discovery Process, II},
\pind first the full procedure,
\pind then its sub-procedures.
\afslut
\afslut\boiteepaisseavecuntitre{A Domain
    Endurant Analysis and Description Process}
\HHHH
\label{DiscoverySchema4}
%\RSLatex
%value
%   perdurant_analysis_and_description: Unit -> Unit
%   perdurant_analysis_and_description() is &\label{PAaD}&
%       describe_state(); &\ \ \ \ \ \ \ \ \ \ \ \ \ \ \ \ \ \ \ \ \ \ \ \ \ \ &axiom ... &\ \ \ [ Note (a) ]&
%       describe_channels(); &\ \ \ \ \ \ \ \ \ \ \ \ \ \ \ \ \ \ \ &axiom ... &\ \ \ [ Note (b) ]&
%       describe_behaviour_signatures(); axiom  ... &\ \ \ [ Note (c) ]&
%       describe_behaviour_definitions(); axiom  ... &\ \ \ [ Note (d) ]&
%       describe_initial_system()&\ \ \ \ \ \ \ \ \ \ \ \ \ \ \ & axiom  ... &\ \ \ [ Note (e) ]&
%\endRSLatex
\bp
\kw{value}\\
\>\ perdurant\_analysis\_and\_description: \kw{Unit} {\RIGHTARROW} \kw{Unit}\\
\>\ perdurant\_analysis\_and\_description() {\IS} \label{PAaD}\\
\>\>\>\ describe\_state(); \ \ \ \ \ \ \ \ \ \ \ \ \ \ \ \ \ \ \ \ \ \ \ \ \ \ \kw{axiom} {\DOTDOTDOT} \ \ \ [ Note (a) ]\\
\>\>\>\ describe\_channels(); \ \ \ \ \ \ \ \ \ \ \ \ \ \ \ \ \ \ \ \kw{axiom} {\DOTDOTDOT} \ \ \ [ Note (b) ]\\
\>\>\>\ describe\_behaviour\_signatures(); \kw{axiom}\ \ {\DOTDOTDOT} \ \ \ [ Note (c) ]\\
\>\>\>\ describe\_behaviour\_definitions(); \kw{axiom}\ \ {\DOTDOTDOT} \ \ \ [ Note (d) ]\\
\>\>\>\ describe\_initial\_system()\ \ \ \ \ \ \ \ \ \ \ \ \ \ \  \kw{axiom}\ \ {\DOTDOTDOT} \ \ \ [ Note (e) ]
\ep
\endboiteepaisseavecuntitre\pos{\normalsize}{\HHHH}\rm
\noindent
\pos{\psno}{\mnewfoil}\LLLL
\iinded{%%%%%%%%%%%%%%%%%%%%%%%%%%%%%%%%%%
\begynd
\pind \sort{\sfsl{Notes}:} 
\begin{itemize}
\item (a) \bbcolor{The States: $\sigma$ and $ui_{\sigma}$}
\begynd
\pind We refer to Sect.\,\vref{kap3.States.specific} and Sect.\,\vref{kap4.The Unique Identifier State}.
\pind The state calculation, as shown on
      Page\,\pageref{kap3-gen-state}, must be replicated, i.e.,
      re-described, in any separate 
      domain analysis \& description.
\pind The purpose of the state, i.e., $\sigma$, is to formulate
      appropriate axiomatic constraints and domain laws.
\afslut
\item (b) \bbcolor{The Channels:}
\begynd
\pind We refer to Sect.\,\vref{Channel Analysis}.
\pind Thus we indiscriminately declare a channel for each pair of \nyl
      distinct unique part identifiers \nyl
      whether the corresponding pair of part behaviours, \nyl
      if at all invoked, communicate or not.
\afslut
\pos{\psno}{\mnewfoil}
\item (c) \bbcolor{Behaviour Signatures:}
\begynd
\pind We refer to Sect.\,\vref{Domain Behaviour Signatures}.
\pind We find it more productive to first settle on the signatures of
      all behaviours -- careful thinking has to go into that -- 
\pind before tackling the far more time-consuming work on defining the
      behaviours:
\afslut
\item (d) \bbcolor{Behaviour Definitions:}
\begynd
\pind We refer to Sect.\,\vref{Behaviour Definition Bodies}.
\afslut
\item (e) \bbcolor{The Running System:}
\begynd
\pind We refer to Sect.\,\vref{Domain Behaviour Initialisation}.
\afslut
\afslut
\end{itemize}
}%%%%%%%%%%%%%%%%%%%%%%%%%%%
 
\nbbbbb{Summary}\label{perd-Summary}\label{P:Summary}
\LLLL

\boiteepaisseavecuntitre{\brcolor{Perdurants: Analysis \& Description:
  Method Tools}}
\label{functions-table-3}
\pos{\begin{minipage}[h]{\pos{6}{10}cm}\sf\small\LLLL\HHHH\sf
\begin{quote}
\begin{itemize}
\item \sort{Domain Discovery:} The procedures being described here,
  informally, guides the domain analyser cum describer to do the job\,!

  We have basically finished our listings of the procedural steps of
  the domain engineering methodology of this \primer\,!
\end{itemize}\normalsize\LLLL\HHHH
\end{quote}
\end{minipage}}{}
\pos{\begin{minipage}[h]{5mm} \ \
\end{minipage}}{}
\pos{\begin{minipage}[h]{\pos{6.5}{14}cm}}{}\small\LLLL\HHHH\sf
\qbtabular\\
           %
& \ \ \brcolor{Description Functions} & \\
\iline{describe\_channels}{Channel Description}\\
\iline{describe\_behaviour\_signatures}{DBSs}\\
\iline{describe\_behaviour\_definitions}{DBDs}\\
\iline{describe\_initial\_system}{Domain Behaviour Initialisation}\\
\iline{perdurant\_analysis\_and\_description}{PAaD}\\

\qetabular\normalsize\LLLL\HHHH
\pos{\end{minipage}}{}
\endboiteepaisseavecuntitre
%%  LocalWords:  Perdurants analyser


\treprikker\HHHH

\noindent
\begynd
\pind Please consider Fig.\,\vref{onto.fig2}.
\begynd
\pind This chapter has covered the right of Fig.\,\ref{onto.fig2}.
\afslut
\afslut

\label{chapter:Perdurants.n}

%%  LocalWords:  Perdurants behaviours endurants perdurants analyse
%%  LocalWords:  Behaviour endurantly perdurantly perdure modelling
%%  LocalWords:  artefacts artefactual synchronise deterministically
%%  LocalWords:  args behaviour synchronisation bv cv zm yv zv CSP ij
%%  LocalWords:  Etcetera RSL Monitorable Initialisation Perdurant ik
%%  LocalWords:  UI uniq pconind Mer svl StaV mvl MonV prgl PrgV crl
%%  LocalWords:  nfinkel combinatory mereology monitorable www imm dk
%%  LocalWords:  Mereologies dtu dibj Ri vers nals pdf stvl ny mv uid
%%  LocalWords:  plitind ing Et cetera UId Mereo StaVL MonVL ProVL ai
%%  LocalWords:  mereo mon pre behaviuors Bxn Cartesians Invariance
%%  LocalWords:  attr nasa uis apos APos ahist AHist ui ais li LM lis
%%  LocalWords:  ub aai atH fli tli upd LHist HHist thi fhi retr onL
%%  LocalWords:  amongst et initialised initialisation rts hs RegNo
%%  LocalWords:  srares analysing mereologies neighbouring Endurant
%%  LocalWords:  endurant perdurant ly des crip tion kap transduced
%%  LocalWords:  wpafwthi dom rng RNMer devt nHMsg nLMsg oHMsg oLMsg
%%  LocalWords:  rn RNI rni auis mer HMer attrs AttrVal sta prog LMer
%%  LocalWords:  AttrVAL HSta rnmer nHInfo nLInfo oHInfo oLInfo ih il
%%  LocalWords:  MSO HTraffic le MLSO Creatable Destructable UoD updH
%%  LocalWords:  rh mkStop ped destructor hm mkUpdH recurses Hoare's
%%  LocalWords:  Transcendentality modeller actn xtra figxtra bj isin
%%  LocalWords:  pdefind characterises Kai lander's accellerate ri ie
%%  LocalWords:  designator dltind Loc prl hannel
