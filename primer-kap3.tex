


%%%%%%%%%%%%%%%%%%%%%%%%%%%%%%%%%%%%%%%%%%%%%%%%%%%%%%%%%%%%%%%%%%%%%%%
%%%% Released for translation 1X April 2023  %%%%%%%%%%%%%%%%%%%%%%%%%%
%%%%%%%%%%%%%%%%%%%%%%%%%%%%%%%%%%%%%%%%%%%%%%%%%%%%%%%%%%%%%%%%%%%%%%%

\nbbbbbb{\HHHH Domains}\label{chapter:Domains}\label{chap2.tex.Preview}
\pos{\minitoc}{}
\pos{\label{lect1:label1}}{}

\noindent
\begynd
\pind We now explain \sfsl{the domain description ontology} as
\begynd
\pind \sl a \sfsl{structured set of concepts for modelling domains}, a set 
\pind that shows their properties and the relations between them\rm.
\afslut
\pind \sl In simple terms, ontology seeks the classification and
      explanation of entities\rm.\footnote{\sfsl{Google's English Dictionary} as provided by \sfsl{Oxford Languages.}} 
\afslut

\mnewfoil

{%%%%%%%%%%%%%%%%%%%%
\begynd
\pind Figure\,\vref{onto.fig\pot{}{2}} is a graphical rendition of a
\begynd
\pind \sfsl{structured set of concepts for modelling domains}.
\afslut
\afslut
}%%%%%%%%%%%%%%%%%%%%

%\paperortutorial{}{
\mnewfoil
\hDBfigure{onto}{\pos{10}{11}cm}{The Upper Ontology}{onto.fig}
%}

\nbbbbb{Universe of Discourse}\label{Universe of Discourse}

\begynd
\pind Domain descriptions start
\begynd
\pind with a terse sketch of the main facets of the domain
\pind followed by the naming of the domain.
\afslut
\afslut
 
\addcontentsline{toc}{subparagraph}{{\hspace*{10mm}\dbeat{\brcolor{\arabic{dprompt}:}
    }\small{\brcolor{\texttt{descr\_Universe\_of\_Discourse}}}}}
\bff{Universe of Discourse: \texttt{calc\_UoD}}\newdf{descr\_Universe\_of\_Discourse}
%\RSLatex
%   &\sort{Narration:}& 
%      Text
%   &\sort{Formalisation:}& 
%      type UoD
%\endRSLatex 
\bp
\>\ \sort{Narration:} \\
\>\>\>\kw{Text}\\
\>\ \sort{Formalisation:} \\
\>\>\>\kw{type} UoD
\ep
\eff

\newexampl{Universe of Discourse}{exa-uod}{exa:nasod}
  

\nbbbbb{External and Internal Qualities}\label{External Qualities}

\characterise{External qualities}{external-qualities}{External qualities
      of endurants\footnote{We refer to predicate prompt \#\,2 below
      for a definition of \sfsl{endurant}.} of a domain 
\begynd
\pind are, in a simplifying sense, those properties of endurants that we can
\begynd
\pind see,
\pind touch and which 
\pind have spatial extent.
\afslut
\pind They, so to speak, take form.
\afslut}


\characterise{Internal qualities}{internal-qualities}{Internal qualities of endurants of a domain
\begynd
\pind are, in a less simplifying sense, those which
\begynd
\pind we may not be able to see or ``feel'' when touching an endurant,
\pind but they can, as we now `mandate' them,
\begynd
\pind be reasoned about, \nyl as for \brcolor{unique identifiers} \nyl
and \brcolor{mereologies},\footnote{We refer to Sects.\,\ref{Unique Identification}--\ref{Mereology}.} 
\afslut or
\pind be measured by some \sort{physical/chemical} means,
\pind or be ``spoken of'' by \sort{intentional deduction}, and
\begynd
\pind be reasoned about,
\afslut
\pind as we do when we \brcolor{attribute} and \brcolor{intentional pull} properties\footnote{We
  refer to Sects.\,\ref{Attributes}--\ref{Intentional Pull}.}  to endurants. 
\afslut
\afslut
}


\bbbb{Predicate Analysis of External Qualities of Endurants}

\characterise{Phenomenon}{phenomenon}{ By a
     \sfsl{phenomenon} \sl  we shall understand a fact that is 
     observed to exist or happen\ \dbsquare\ \ \rm Examples of phenomena
     are: emotions of a human, the rivers, lakes, forests, mountains
     and valleys of mother nature; the railway tracks, their units,
     the locomotive of a railway system.}

\bgcolor{Domain Analysis Predicates:}{ %dom-anal-predI}{
     \rm We shall define a number of domain analysis predicates.
     They are all referred to as prompts. Prompts are method tools. The
     domain modeller applies these to ``real'', i.e.,
     actual world phenomena, that is, not to formal values. In the
     next 18 paragraphs we shall ``reveal'' a number of such
     predicates. First with a \sl reasonable definition (in slanted font)\rm, then with
     examples and some comments (in roman font).
\prepro{is\_entity}{is-entity}\newap{is\_entity} \sl% 
\addcontentsline{toc}{subparagraph}{{\hspace*{10mm}\arabic{predctr}. }\small{\bbcolor{\texttt{is\_entity}}}}%
\begynd
\pind By an {entity} we shall understand  something
\begynd
\pind that can be {observe}d, i.e., be
\begynd  
\pind seen or touched by humans,
\pind \sfsl{or} that can be {conceive}d 
\pind as an {abstraction}  of an entity;
\afslut 
\pind alternatively,
\begynd
\pind a phenomenon is an entity, \sfsl{if it exists, it is
      \pdindextermi{``being''}, 
\pind it is that which makes a {``thing''} what it is: \nyl
      essence, essential nature}\footnote{All of the analysis prompts
      are necessary, but few new be explicitly applied\,!}\pos{ \cite[\sfsl{Vol.\,I, pg.\,665}]{OED}}{}
     \dbsquare\ \
\afslut
\afslut \rm Some, but not necessarily
  all aspects of a river can
  be rationally described, hence can be still be considered
  entities. Similarly, many aspects of a
  road net can be rationally described, hence will be considered
     entities. \sl
\prepro{is\_endurant}{is-endurant}\newap{is\_endurant} \sl%
\addcontentsline{toc}{subparagraph}{{\hspace*{10mm}\arabic{predctr}. }\small{\bbcolor{\texttt{is\_endurant}}}}%
     Endurants are those quantities of domains that we can
      observe (see and touch), in \sfsl{space}, as ``complete'' entities at no matter
      which point in \sfsl{time} -- ``material'' entities that
      persists, endures \cite[\sfsl{Vol.\,I, pg.\,656}]{OED} \dbsquare\ \ \rm
      Street segments [links], street intersections [hubs],  
      automobiles standing still in an automobile show room are
      endurants.  Domain endurants, when eventually modelled in
      software, typically 
      become data. Hence the careful analysis of domain endurants is a
      prerequisite for subsequent careful conception and analyses of
      data structures for software, including data bases. \sl
      \sl\rm
     
\prepro{is\_perdurant}{is-perdurant}\newap{is\_perdurant} \sl%  
\addcontentsline{toc}{subparagraph}{{\hspace*{10mm}\arabic{predctr}. }\small{\bbcolor{\texttt{is\_perdurant}}}}%
\begynd
\pind By a \pdindextermi{perdurant} we shall understand an
      entity   
\begynd
\pind for which only a fragment exists\pos{}{\\} if we look at or
      touch them\pos{}{\\} at
      any given snapshot in time.
\pind Were we to freeze time we would only see or
      touch \pos{}{\\} a fragment of the perdurant
      \cite[\sfsl{Vol.\,II, pg.\,1552}]{OED} \eod \ \ 
\afslut
\afslut \rm Automobiles in action, container vessels sailing on the 7
     seas and loading and unloading containers in harbours are
     examples of perdurants. Domain perdurants, when eventually modelled in
     software, typically 
      become processes.} 
    
\begynd
\pind \sfsl{Endurants} are
\begynd
\pind either \sfsl{solid} endurants, or are
\pind \sfsl{fluid} endurants.
\afslut

\prepro{is\_solid}{is-solid}\newap{is\_solid} \sl 
\addcontentsline{toc}{subparagraph}{{\hspace*{10mm}\arabic{predctr}. }\small{\bbcolor{\texttt{is\_solid}}}}
\begynd
\pind By a \pdindextermii{solid}{endurant} we shall understand an endurant
\pind which is 
\begynd
\pind  separate,  
\pind individual or
\pind  distinct in form or concept,
\afslut
\pind or, rephrasing:
\begynd
\pind  a body 
\pind or magnitude
\afslut
  of three-dimensions, \nyl having length, breadth and thickness
  \cite[\sfsl{Vol.\,II, pg.\,2046}]{OED} \eod\ \
\afslut \rm 
\begynd
\pind Wells,
\pind pipes, 
\pind valves, 
\pind pumps, 
\pind forks,
\pind joins,
\pind regulator, and 
\pind sinks
\afslut of a pipeline are solids.
      \sl
\prepro{is\_fluid}{is-fluid}\newap{is\_fluid} \sl 
\addcontentsline{toc}{subparagraph}{{\hspace*{10mm}\arabic{predctr}. }\small{\bbcolor{\texttt{is\_fluid}}}}
\begynd
\pind By a \pdindextermii{fluid}{endurant} we shall understand \nyl   an endurant which is 
\begynd
\pind prolonged, without interruption, \nyl  in an unbroken series or pattern;
\afslut or, rephrasing:
\begynd
\pind a substance (liquid, gas or plasma)  \nyl 
      having the property of flowing,  \nyl  
      consisting of particles \nyl  
      that move  among themselves \cite[\sfsl{Vol.\,I, pg.\,774}]{OED} \eod \  \ \rm
\afslut
\afslut Fluids are otherwise
\begynd
\pind liquid, or
\pind gaseous, or
\pind plasmatic, or
\pind granular\footnote{\label{fn.granular1}\LLLL This is a purely
      pragmatic decision. \nyl ``Of course'' sand, gravel, soil, etc., are not
      fluids, \nyl  but for our modelling purposes it is convenient \nyl  to
      ``compartmentalise'' them as fluids\,!}, or
\pind plant products\footnote{\LLLL i.e., chopped sugar cane, threshed, or
      otherwise. See footnote \vref{fn.granular1}.},
\pind et cetera.
\pind Specific examples of fluids are:
\begynd
\pind water, oil, gas, compressed air, etc.
\afslut
\pind A container, which we consider a solid endurant,
\begynd
\pind may be \sfsl{conjoined} with another, a fluid, 
\pind like a gas pipeline unit may ``contain'' gas.
\afslut

\rm
\pind We  analyse endurants into either of two kinds:
\begynd
\pind \sfsl{parts} and
\pind \sfsl{living species}.
\afslut
\pind The distinction between \sfsl{parts} and \sfsl{living species} \nyl
      is motivated in \kaisorfil\
      \cite{kaisorlander1994,kaisorlander1997,kaisorlander2002,kaisorlander2016,kaisorlander2022}. 
\afslut
      \sl
\prepro{is\_part}{is-part}\newap{is\_part} \sl 
\addcontentsline{toc}{subparagraph}{{\hspace*{10mm}\arabic{predctr}. }\small{\bbcolor{\texttt{is\_part}}}} 
\begynd
\pind By a \sfsl{part} we shall understand
\begynd
\pind a solid endurant existing in time and space and
\pind subject to laws of physics,
\pind including the \sfsl{causality principle} and
\pind \sfsl{gravitational pull}\,\footnotemark
\afslut
\afslut
\footnotetext{\LLLL This characterisation
  is the result of our study of relations between philosophy and
  computing science, notably influenced by \kaisorfil\label{influencex}}\
\ \dbsquare\ \

\pind Natural and man-made parts are either
\begynd
\pind \sfsl{atomic} or
\pind \sfsl{compound}.
\afslut
\afslut 
      \sl
\prepro{is\_atomic}{is-atomic}\newap{is\_atomic}  \sl 
\addcontentsline{toc}{subparagraph}{{\hspace*{10mm}\arabic{predctr}. }\small{\bbcolor{\texttt{is\_atomic}}}}
\begynd
\pind By an \sfsl{atomic part} we shall understand a part
\begynd
\pind which the domain analyser considers to be indivisible
\pind in the sense of not meaningfully,
\pind for the purposes of the domain under consideration,
\pind that is, to not meaningfully consist of sub-parts \dbsquare\ \ \rm
\afslut
\afslut The
\begynd
\pind wells,
\pind pumps, 
\pind valves, 
\pind pipes, 
\pind forks, 
\pind joins and 
\pind sinks
\afslut of a pipeline can be considered atomic.
      \sl
\prepro{is\_compound}{is-compound}\newap{is\_compound} \sl 
\addcontentsline{toc}{subparagraph}{{\hspace*{10mm}\arabic{predctr}. }\small{\bbcolor{\texttt{is\_compound}}}}
\begynd
\pind \pdindextermii{Compound}{part}s are those which are
\begynd
\pind either Cartesian-product-
\pind or are set-
\afslut
\pind oriented parts  \eod \ \rm
\afslut
      \sl
\prepro{is\_Cartesian}{is-Cartesian}\newap{is\_Cartesian} \sl 
\addcontentsline{toc}{subparagraph}{{\hspace*{10mm}\arabic{predctr}. }\small{\bbcolor{\texttt{is\_Cartesian}}}}
\begynd
\pind \pdindextermii{Cartesian}{part}s are those (compound parts)
\begynd
\pind which consists of an ``indefinite number'' 
\pind of two or more parts 
\pind of distinctly named sorts \dbsquare\ \ \rm
\afslut
\afslut
\begynd
\pind Some clarification may be needed.
\begynd
\pind (i) In mathematics, as in \texttt{RSL} \citersl, 
\begynd
\pind a value is a Cartesian (``record'') value 
\pind if it can be expressed, for example as $(a,b,...,c)$,
\pind where $a, b, ..., c$ are mathematical (or, which is the same, \texttt{RSL}) values.
\pind Let the sort names of these be $A, B, ..., C$ --
\pind with these being required to be distinct.
\pind We wrote ``indefinite number'':
\begynd
\pind the meaning being that the number is fixed, finite, but not specific.
\afslut
\afslut
\afslut
\pind (ii) The requirement: `distinctly named' is pragmatic.
\begynd
\pind If the domain modeller thinks 
\pind that two or
      more of the components of a  Cartesian part 
\pind [really] are of the same sort, 
\pind then that person is most likely confused
\pind and must come up with suitably distinct sort names for these
      ``same sort'' parts\,!
\afslut
\pind (iii) Why did we not write ``definite number''\,?
\begynd
\pind Well, at the time of first analysing a Cartesian part,
\pind the  domain modeller
\pind may not have thought of all the consequences, i.e., analysed,
\pind the compound part.
\pind Additional sub-parts, of the Cartesian compound, may be
      ``discovered'', subsequently
\pind and can then, 
\pind with the approach we are taking wrt.\ the modelling of these,
\pind be ``freely'' added subsequently\,!
\afslut
\afslut
\begynd
\pind We refer to the road transport
  system example above.
\pind We there viewed (hubs, links and) automobiles \nyl as atomic parts.
\pind From another point of view \nyl we shall here understand automobiles \nyl
      as Cartesian parts:
\pos{}{\begin{multicols}{2}}
\begynd
\pind the engine train,
\pind the chassis,
\pind the car body,
\pind four doors (left front, left rear, right front, right rear), and
\pind the wheels.
\afslut
\pos{}{\end{multicols}}
\pind These may again be considered Cartesian parts.
\afslut
      \sl
\prepro{is\_part\_set}{is-part-set}\newap{is\_part\_set} \sl 
\addcontentsline{toc}{subparagraph}{{\hspace*{10mm}\arabic{predctr}. }\small{\bbcolor{\texttt{is\_part\_set}}}}
\begynd
\pind \pdindextermii{Part}{set}s are those which,
\begynd
\pind in a given context, 
\pind are deemed to \textsl{meaningfully} consist of
\begynd
\pind an indefinite number of \sfsl{sub-part}s
\pind of the same sort \dbsquare\ \ \rm
\afslut
\pind Examples of set parts are: the set of hubs of a road net hub
aggregate, the set of links of a road net link aggregate, and the set
of automobiles of an automobile aggregate -- all of the road net
transport that we are exemplifying.
\afslut

      \sl
\prepro{is\_living\_species}{is-living-species}\newap{is\_living\_species} \sl 
\addcontentsline{toc}{subparagraph}{{\hspace*{10mm}\arabic{predctr}. }\small{\bbcolor{\texttt{is\_living\_species}}}}
\begynd
\pind By a \sfsl{living species} we shall understand
\begynd
\pind a solid endurant,
\pind subject to laws of physics, and
\pind additionally subject to \sfsl{causality of
      purpose}. 
\pind Living species
\begynd \LLLL
\pind must have some \sfsl{form they can be developed to reach};
\pind a form they must be \sfsl{causally  determined  to maintain}.
\pind This \sfsl{development and maintenance} must further \nyl engage
      in \sfsl{exchanges of matter with an environment} \rm\dbsquare\
\afslut
\afslut
\afslut 
\pind It must be possible that living species occur in two forms:
\begynd
\pind \brcolor{plants}, respectively \brcolor{animals}. 
\pind Although we have not yet come across domains for which \nyl the need
      to model the living species of plants were needed.
\pind Hence:
\afslut

\prepro{is\_plant}{is-plant}\newap{is\_plant} \sl 
\addcontentsline{toc}{subparagraph}{{\hspace*{10mm}\arabic{predctr}. }\small{\bbcolor{\texttt{is\_plant}}}} 
Plants are living species which are characterised by  \nyl
\sfsl{development, form and exchange of matters with the environment} \dbsquare\ \ \sl 

\prepro{is\_animal}{is-animal}\newap{is\_animal} \sl 
\addcontentsline{toc}{subparagraph}{{\hspace*{10mm}\arabic{predctr}. }\small{\bbcolor{\texttt{is\_animal}}}}  
Animals are living species which are {additionally} characterised by
the \sfsl{ability of purposeful movement} \dbsquare\ \ \rm 
 
Within animals we then have humans. \sl

\prepro{is\_human}{is-human}\newap{is\_human} \sl 
\addcontentsline{toc}{subparagraph}{{\hspace*{10mm}\arabic{predctr}. }\small{\bbcolor{\texttt{is\_human}}}}
\begynd
\pind A \sfsl{human} (a \sfsl{person}) is an \sfsl{animal}, \nyl with the additional
      properties of having 
\begynd
\pind \sfsl{language},
\pind being \sfsl{conscious} of \sfsl{having knowledge} (of its own
      situation), and
\pind \sfsl{responsibility} \dbsquare\ \ \rm
\afslut
\afslut


\characterise{Manifest Part}{manifest}{By a manifest part we shall
  understand \label{summary-manifest}
\begynd
\pind a part which `manifests' itself 
\begynd
\pind either in a physical, visible
      manner, ``occupying'' \nyl an $\mathbb{AREA}$ or \nyl a $\mathbb{VOLUME}$
      and \nyl a $\mathbb{POSITION}$  \nyl in  $\mathbb{SPACE}$, 
\pind or in a conceptual manner \nyl  forms an organisation in Your
mind\,!\ \dbsquare \rm\ \ 
\afslut 
\pind As we have already revealed,   
\pind endurant parts can be \nyl transcendentally
      deduced into perdurant behaviours   
\pind -- with manifest parts indeed being so.
\afslut}

\prepro{is\_manifest}{is-manifest}\newap{is\_manifest} \sl \texttt{is\_manifest($e$)}
\addcontentsline{toc}{subparagraph}{{\hspace*{10mm}\arabic{predctr}. }\small{\bbcolor{\texttt{is\_manifest}}}}%
      holds  if $e$ is manifest \dbsquare \sl

\mnewfoil
      
\characterise{Structure}{structure}{By a structure we shall
      understand
\begynd
\pind an endurant concept that allows \nyl  the domain modeller
\begynd
\pind to rationally decompose \nyl  a domain analysis and/or its description 
\pind into manageable, logically relevant sections,  
\pind but where these abstract endurants \nyl  are not further reflected
      upon \nyl  in the domain analysis and description \dbsquare \rm\ \ 
\afslut
\pind Structures are therefore \nyl  not  transcendentally
      deduced into perdurant behaviours.
\afslut}

\prepro{is\_structure}{is-structure}\newap{is\_structure} \sl  \texttt{is\_structure($e$)}
\addcontentsline{toc}{subparagraph}{{\hspace*{10mm}\arabic{predctr}. }\small{\bbcolor{\texttt{is\_structure}}}}
      holds if $e$  is a structure \dbsquare\sl

Examples of structures arise as the result of our analysis of
parts. Thus a road net could be modelled as the composite of two
structures: a set of hubs and a set of links (the stretches between two
adjacent hubs, i.e., road intersections), cf.\ Items\,\ref{final-0030}--\vref{final-0040}.
      
\prepro{is\_stationary}{is-stationary}\newap{is\_stationary}
\addcontentsline{toc}{subparagraph}{{\hspace*{10mm}\arabic{predctr}. }\small{\bbcolor{\texttt{is\_stationary}}}}
        \sl  An endurant part is
      stationary if it never changes position in space \dbsquare\ \ 
      
      \sl
\prepro{is\_mobile}{is-mobile}\newap{is\_mobile} \sl
\addcontentsline{toc}{subparagraph}{{\hspace*{10mm}\arabic{predctr}. }\small{\bbcolor{\texttt{is\_mobile}}}}
      An endurant part is
      mobile if it may possibly change position in space  \dbsquare\ \
      \rm

      We may need, occasionally, the distinction as now outlined:


%\dbeat{%%%%%%%%%%%%%%%%%%%
\begynd
\pind \sfsl{Endurants} are
\begynd
\pind either \sfsl{natural} endurants, or are
\pind \sfsl{artefactual} endurants.
\afslut

      \sl
\prepro{is\_natural}{is-natural} \sl  
By a \sfsl{natural endurant} we shall understand one which has been
created by nature.
\rm

      \sl
\prepro{is\_artefactual}{is-artefactual} \sl  
By an \sfsl{artefactual} endurant we shall understand one which has been
created by humans.\rm

\begynd
\pind \bgcolor{Discrete Dynamic and Artefactual Domains:}
\begynd
\pind In our initial characterisation of domains, Page\,\pageref{What is a Domain},
\pind an emphasis was put on their \sfsl{discrete dynamics and human assistedness}.
\pind The analysis and description calculi and, hence, our domain modelling,
\pind are therefore ``geared'' in that direction.
\afslut

\pind We are not offering to model time continuous domains.\label{noContinuity}
\pind See Sect.\,\vref{Continuity}.
\afslut
%}%%%%%%%%%%%%%%%%%%%%%%%%

      
%\noindent
We summarise\footnotemark:

\bff{Analysis Predicates}\sf
%%\characterise{Analysis Predicate Signatures}{anprsi}{%
\begin{multicols}{3}\small
%\RSLatex
%value
%&\cref{is-entity}&  is_entity: &$\Phi$& -> Bool 
%&\cref{is-endurant}&  is_endurants: E -> Bool 
%&\cref{is-perdurant}&  is_perdurant: E -> Bool 
%&\cref{is-solid}&  is_solid: E -> Bool 
%&\cref{is-fluid}&  is_fluid: E -> Bool 
%&\cref{is-part}&  is_part: E -> Bool 
%&\cref{is-living-species}&  is_living_species: E -> Bool 
%&\cref{is-atomic}&  is_atomic: E -> Bool 
%&\cref{is-compound}&  is_compound: E -> Bool 
%&\cref{is-animal}&  is_animal: E -> Bool 
%&\cref{is-animal}&  is_plant: E -> Bool 
%&\cref{is-Cartesian}&  is_Cartesian: E -> Bool 
%&\cref{is-part-set}&  is_part_set: E -> Bool 
%&\cref{is-human}&  is_human: E -> Bool 
%&\cref{is-manifest}&  is_manifest: E -> Bool 
%&\cref{is-structure}&  is_structure: E -> Bool 
%&\cref{is-stationary}&  is_structure: E -> Bool 
%&\cref{is-mobile}&  is_structure: E -> Bool 
%&\cref{is-natural}&  is_natural: E -> Bool 
%&\cref{is-artefactual}&  is_artefactual: E -> Bool 
%\endRSLatex 
\bp
\kw{value}\\
\cref{is-entity}\ \ is\_entity: $\Phi$ {\RIGHTARROW} \kw{Bool} \\
\cref{is-endurant}\ \ is\_endurants: E {\RIGHTARROW} \kw{Bool} \\
\cref{is-perdurant}\ \ is\_perdurant: E {\RIGHTARROW} \kw{Bool} \\
\cref{is-solid}\ \ is\_solid: E {\RIGHTARROW} \kw{Bool} \\
\cref{is-fluid}\ \ is\_fluid: E {\RIGHTARROW} \kw{Bool} \\
\cref{is-part}\ \ is\_part: E {\RIGHTARROW} \kw{Bool} \\
\cref{is-living-species}\ \ is\_living\_species: E {\RIGHTARROW} \kw{Bool} \\
\cref{is-atomic}\ \ is\_atomic: E {\RIGHTARROW} \kw{Bool} \\
\cref{is-compound}\ \ is\_compound: E {\RIGHTARROW} \kw{Bool} \\
\cref{is-animal}\ \ is\_animal: E {\RIGHTARROW} \kw{Bool} \\
\cref{is-animal}\ \ is\_plant: E {\RIGHTARROW} \kw{Bool} \\
\cref{is-Cartesian}\ \ is\_Cartesian: E {\RIGHTARROW} \kw{Bool} \\
\cref{is-part-set}\ \ is\_part\_set: E {\RIGHTARROW} \kw{Bool} \\
\cref{is-human}\ \ is\_human: E {\RIGHTARROW} \kw{Bool} \\
\cref{is-manifest}\ \ is\_manifest: E {\RIGHTARROW} \kw{Bool} \\
\cref{is-structure}\ \ is\_structure: E {\RIGHTARROW} \kw{Bool} \\
\cref{is-stationary}\ \ is\_stationary: E {\RIGHTARROW} \kw{Bool} \\
\cref{is-mobile}\ \ is\_mobile: E {\RIGHTARROW} \kw{Bool} \\
\cref{is-natural}\ \ is\_natural: E {\RIGHTARROW} \kw{Bool} \\
\cref{is-artefactual}\ \ is\_artefactual: E {\RIGHTARROW} \kw{Bool} 
\ep
\end{multicols}\normalsize 
\eff\rm\footnotetext{Framed texts highlight \sfsl{domain analysis \&
    description} prompts.}

\begin{exam}
\stepcounter{examctr}\bbcolor{Example.}  \brcolor{\textsf{Analysis Predicates:}} In the example of
Appendix\,\vref{final-example}--\pageref{final-example.n} we do not [explicitly]  show the
``application'' of analysis predicates. They are tacitly assumed.
\end{exam}

\dbeat{
\nbbbb{On Interpreting the Analysis \& Description Ontology}

                                %\begin{multicols}{2}
\begin{minipage}{12cm}
\begynd
\pind We interpret the kind of analysis \& description methodology
      ontology diagrams of which Fig.\,\vref{onto.fig\pot{}{2}} is an example.
\pind The figure to the right illustrates a fragment of such diagrams.
\begynd
\pind At node \sort{A}, i.e., observing an endurant, \sort{a}, of sort \sort{A},
\pind the downward diverging two lines express that \sort{a}
\begynd
\pind is either of sort \sort{B}
\pind or of sort \sort{C};
\afslut
\pind that is:
\xtut{%
%\RSLatex
%    pre: is_B(a), is_C(a): is_A(a)
%    (is_B(a)=>~is_C(a)) /\ (is_C(a)=>~is_B(a))
%    (is_B(a)is~is_C(a)) /\ (~is_B(a)is&i&s_C(a))
%\endRSLatex
\bp
\>\>\kw{pre}: is\_B(a), is\_C(a): is\_A(a)\\
\>\>(is\_B(a){\DBLRIGHTARROW}{\SIM}is\_C(a)) {\WEDGE} (is\_C(a){\DBLRIGHTARROW}{\SIM}is\_B(a))\\
\>\>(is\_B(a){\IS}{\SIM}is\_C(a)) {\WEDGE} ({\SIM}is\_B(a){\IS}is\_C(a))
\ep
}
\afslut
\afslut
\end{minipage}
\begin{minipage}{4cm}
\epsfig{file=ABC.eps,height=3cm}%
\end{minipage}

%\end{multicols}
}

\nbbbb{Functional Analysis of External Qualities of Endurants}

\begynd
\pind Given a compound endurant, that is,
\begynd
\pind either a Cartesian
\pind or a part set,
\afslut
\pind we analyse that compound, 
\pind at the two $\bullet$'s of Fig.\,\vref{onto.fig\pot{}{2}},
\pind into its constituent endurants, respectively parts, 
\pind and the name of the sort:
\afslut

\addcontentsline{toc}{subparagraph}{{\hspace*{10mm} \small{\lilacolor{determine\_Cartesian\_parts}}}}
\bff{determine\_Cartesian\_parts, determine\_part\_set}\newaf{determine\_Cartesian\_parts}\newaf{determine\_part\_set}
%\RSLatex
%value
%   determine_Cartesian_parts: E -> (E1><`eta&\,&`Phi)><(E2><`eta&\,&`Phi)><...><(Ec><`eta&\,&`Phi)
%   determine_Cartesian_parts(e) as (e1:`eta&E1&,e2:`eta&E2&,...,ec:`eta&Ec&)
%
%   determine_part_set: E -> P-set><`eta&\,&`Phi
%   determine_part_set(e) as ({p1,p2,...,ps}:`eta&P&,)
%\endRSLatex 
\bp
\kw{value}\\
\>\ determine\_Cartesian\_parts: E {\RIGHTARROW} (E1{\TIMES}$\eta$\,$\Phi$){\TIMES}(E2{\TIMES}$\eta$\,$\Phi$){\TIMES}{\DOTDOTDOT}{\TIMES}(Ec{\TIMES}$\eta$\,$\Phi$)\\
\>\ determine\_Cartesian\_parts(e) \kw{as} (e1:$\eta$E1,e2:$\eta$E2,{\DOTDOTDOT},ec:$\eta$Ec)\\
\\
\>\ determine\_part\_set: E {\RIGHTARROW} P\kw{-set}{\TIMES}$\eta$\,$\Phi$\\
\>\ determine\_part\_set(e) \kw{as} ({\LBRACE}p1,p2,{\DOTDOTDOT},ps{\RBRACE}:$\eta$P,)
\ep
\eff
\addcontentsline{toc}{subparagraph}{{\hspace*{10mm} \small{\lilacolor{determine\_part\_set}}}}\newaf{determine\_part\_set}

\noindent
\begynd
\pind The above \textsf{calc}ulation function signatures and
      characterisations illustrate two extensions to \texttt{RSL}
      \cite{RSL}:
\begynd
\pind \textsf{$\eta$P} expresses the name of a sort \textsf{P}, and
\pind \textsf{$\eta$\,$\Phi$} expresses the  type of sort names.
\afslut

\pind Again we emphasize that these \textsf{calc}ulations 
\pind are performed by the domain modeller.
\pind They are used in subsequent schemas for describing external
      qualities of endurants.
\afslut

\nbbbb{Descriptions of External Qualities of Endurants}

\begynd
\pind Similarly,
\begynd
\pind again at the two \dbsquare's of Fig.\,\vref{onto.fig\pot{}{2}},
\pind we are now ready to describe respectively
\begynd
\pind Cartesian parts and
\pind part set parts.
\afslut
\afslut
\afslut

\bbb{Describing Cartesian Parts}\label{Describing Cartesian Parts}

\addcontentsline{toc}{subparagraph}{{\hspace*{10mm} \small{\brcolor{descr\_Cartesian}}}}\newdf{descr\_Cartesian}
\bff{descr\_Cartesian}
%\RSLatex
%value 
%   descr_Cartesian: P -> &\rsltext&
%   descr_Cartesian(p) is
%      &\bq& &\sort{Narrative:}&
%          [s] text on sorts
%          [o] text on observers
%          [a] text on axioms and/or proof obligations
%       &\,\,\sort{Formalisation:}&
%          [s] type 
%                 E1, E2, ..., Ec
%          [o] value 
%                 obs_E1: E->E1, obs_E2: E->E2, ..., obs_Ec: E->Ec
%          [a] &\sort{axiom} and/or \sort{proof obligation}&
%                 &$\mathcal{A}$/$\mathcal{P}$&(...) &\eq&
%\endRSLatex 
\bp
\kw{value} \\
\>\ descr\_Cartesian: P {\RIGHTARROW} \rsltext\\
\>\ descr\_Cartesian(p) {\IS}\\
\>\>\>\bq \sort{Narrative:}\\
\>\>\>\>\>{\LBRACKET}s{\RBRACKET} text on sorts\\
\>\>\>\>\>{\LBRACKET}o{\RBRACKET} text on observers\\
\>\>\>\>\>{\LBRACKET}a{\RBRACKET} text on axioms and/or proof obligations\\
\>\>\>\ \,\,\sort{Formalisation:}\\
\>\>\>\>\>{\LBRACKET}s{\RBRACKET} \kw{type} \\
\>\>\>\>\>\>\>\>\ E1, E2, {\DOTDOTDOT}, Ec\\
\>\>\>\>\>{\LBRACKET}o{\RBRACKET} \kw{value} \\
\>\>\>\>\>\>\>\>\ obs\_E1: E{\RIGHTARROW}E1, obs\_E2: E{\RIGHTARROW}E2, {\DOTDOTDOT}, obs\_Ec: E{\RIGHTARROW}Ec\\
\>\>\>\>\>{\LBRACKET}a{\RBRACKET} \sort{axiom} and/or \sort{proof obligation}\\
\>\>\>\>\>\>\>\>\ $\mathcal{A}$/$\mathcal{P}$({\DOTDOTDOT}) \eq
\ep
\eff

\newexampl{Cartesians}{exa-cart}{exa:Cartesian Examples}
  
\nbbb{Describing Part Sets}\label{Describing Part Sets}

\addcontentsline{toc}{subparagraph}{{\hspace*{10mm} \small{\brcolor{descr\_part\_set}}}}
\bff{descr\_part\_set}\newdf{descr\_part\_set}
%\RSLatex
%value 
%   descr_part-set: P -> &\rsltext&
%   descr_part_set(p) is
%      &\bq& &\sort{Narrative:}&
%          [s] text on sorts
%          [o] text on observers
%          [a] text on axioms and/or proof obligations
%       &\,\,\sort{Formalisation:}&
%          [s] type 
%                 P, Ps = P-set
%          [o] value 
%                 obs_Ps: E->Ps
%          [a] &\sort{axiom} and/or \sort{proof obligation}&
%                 &$\mathcal{A}$/$\mathcal{P}$&(...) &\eq&
%\endRSLatex 
\bp
\kw{value} \\
\>\ descr\_part\kw{-set}: P {\RIGHTARROW} \rsltext\\
\>\ descr\_part\_set(p) {\IS}\\
\>\>\>\bq \sort{Narrative:}\\
\>\>\>\>\>{\LBRACKET}s{\RBRACKET} text on sorts\\
\>\>\>\>\>{\LBRACKET}o{\RBRACKET} text on observers\\
\>\>\>\>\>{\LBRACKET}a{\RBRACKET} text on axioms and/or proof obligations\\
\>\>\>\ \,\,\sort{Formalisation:}\\
\>\>\>\>\>{\LBRACKET}s{\RBRACKET} \kw{type} \\
\>\>\>\>\>\>\>\>\ P, Ps {\EQ} P\kw{-set}\\
\>\>\>\>\>{\LBRACKET}o{\RBRACKET} \kw{value} \\
\>\>\>\>\>\>\>\>\ obs\_Ps: E{\RIGHTARROW}Ps\\
\>\>\>\>\>{\LBRACKET}a{\RBRACKET} \sort{axiom} and/or \sort{proof obligation}\\
\>\>\>\>\>\>\>\>\ $\mathcal{A}$/$\mathcal{P}$({\DOTDOTDOT}) \eq
\ep
\eff

\newexampl{Part Sets}{exa-sets}{exa:Part Set Examples}
  
\nbbbb{Endurant States}\label{Endurant States}

\characterise{Endurant State}{endurant-state}{%
\begynd
\pind By an \sfsl{endurant state} we shall understand
\begynd
\pind any collection of endurant parts \dbsquare
\afslut
\afslut}

\bff{obs\_$\Sigma$}
%\RSLatex
%value
%   `Sigma = P-set
%value
%   obs_`Sigma: E -> `Sigma
%   obs_`Sigma(e) is
%      if is_manifest(e) 
%         then
%            is_atom(e) -> {e},
%            is_Cartesian(e) ->
%               let (p1:`eta&E1&,p2:`eta&E2&,...,pc:`eta&Ec&) =  calc_cartesian_parts_and_sorts(e) in
%               {p1,p2,...,pc} union obs_`Sigma(p1) union obs_`Sigma(p2) union ... union obs_`Sigma(pc) end
%            is_part-set(e) -> 
%               let ({p1,p2,...,ps}:`eta&P&) =  calc_part_sets_parts_and_sort(e) in
%               {p1,p2,...,ps} union obs_`Sigma(p1) union obs_`Sigma(p2) union ... union obs_`Sigma(ps) end
%         else {}
%      end     
%\endRSLatex 
\bp
\kw{value}\\
\>\ $\Sigma$ {\EQ} P\kw{-set}\\
\kw{value}\\
\>\ obs\_$\Sigma$: E {\RIGHTARROW} $\Sigma$\\
\>\ obs\_$\Sigma$(e) {\IS}\\
\>\>\>\kw{if} is\_manifest(e) \\
\>\>\>\>\ \kw{then}\\
\>\>\>\>\>\>is\_atom(e) {\RIGHTARROW} {\LBRACE}e{\RBRACE},\\
\>\>\>\>\>\>is\_Cartesian(e) {\RIGHTARROW}\\
\>\>\>\>\>\>\>\ \kw{let} (p1:$\eta$E1,p2:$\eta$E2,{\DOTDOTDOT},pc:$\eta$Ec) {\EQ}\ \ calc\_cartesian\_parts\_and\_sorts(e) \kw{in}\\
\>\>\>\>\>\>\>\ {\LBRACE}p1,p2,{\DOTDOTDOT},pc{\RBRACE} {\UNION} obs\_$\Sigma$(p1) {\UNION} obs\_$\Sigma$(p2) {\UNION} {\DOTDOTDOT} {\UNION} obs\_$\Sigma$(pc) \kw{end}\\
\>\>\>\>\>\>is\_part\kw{-set}(e) {\RIGHTARROW} \\
\>\>\>\>\>\>\>\ \kw{let} ({\LBRACE}p1,p2,{\DOTDOTDOT},ps{\RBRACE}:$\eta$P) {\EQ}\ \ calc\_part\_sets\_parts\_and\_sort(e) \kw{in}\\
\>\>\>\>\>\>\>\ {\LBRACE}p1,p2,{\DOTDOTDOT},ps{\RBRACE} {\UNION} obs\_$\Sigma$(p1) {\UNION} obs\_$\Sigma$(p2) {\UNION} {\DOTDOTDOT} {\UNION} obs\_$\Sigma$(ps) \kw{end}\\
\>\>\>\>\ \kw{else} {\LBRACE}{\RBRACE}\\
\>\>\>\kw{end}\ \ \ \ \ 
\ep
\eff

\newexampl{Endurant State Examples}{exa-sta-exa}{exa:Endurant States}

\nbbbb{A Proof-theoretic Explication, I}\label{AProofTheoreticExplicationI}

\begynd
\pind The concept of \sfsl{analysis predicates} and \sfsl{part
  observer functions} is due to McCarthy \cite[\sfsl{Sect.\.12-13}]{Mc62a}.

\pind In \cite{Mc62a} McCarthy introduces a notion of \sfsl{abstract
      syntax}, Sect.\,12, and \sfsl{semantics}, Sect.\,13.
\pind So far we have dealt, in our domain analysis, with syntax.
\pind There are three elements, according to McCarthy, to consider:
\begynd
\pind the \texttt{is\_...} predicates,
\pind the \texttt{obs\_...} [``destructor''] functions, and, not shown, so far, in this
      paper, 
\pind the \texttt{mk\_...} constructor functions.
\afslut
\pind For compound abstract syntactic entities they are related as follows:
%\RSLatex
%      is_Cartesian(p) is
%          let (p1:`eta&P1&,p2:`eta&P2&,...,pc:`eta&Pc&) = calc_Cartesian_parts_and_sorts(p) in
%          p = mk_Cartesian(obs_P1(p),obs_P2(p),...,obs_Pc(p)) end
%
%      is_part_set(p) is
%          let ({p1,p2,...,ps},`eta&P1&) = calc_part_sets_parts_and_sort(p) in
%          p = mk_part_set({p1,p2,...,ps}) end
%\endRSLatex 
\bp
\>\>\>is\_Cartesian(p) {\IS}\\
\>\>\>\>\>\kw{let} (p1:$\eta$P1,p2:$\eta$P2,{\DOTDOTDOT},pc:$\eta$Pc) {\EQ} calc\_Cartesian\_parts\_and\_sorts(p) \kw{in}\\
\>\>\>\>\>p {\EQ} mk\_Cartesian(obs\_P1(p),obs\_P2(p),{\DOTDOTDOT},obs\_Pc(p)) \kw{end}\\
\\
\>\>\>is\_part\_set(p) {\IS}\\
\>\>\>\>\>\kw{let} ({\LBRACE}p1,p2,{\DOTDOTDOT},ps{\RBRACE},$\eta$P1) {\EQ} calc\_part\_sets\_parts\_and\_sort(p) \kw{in}\\
\>\>\>\>\>p {\EQ} mk\_part\_set({\LBRACE}p1,p2,{\DOTDOTDOT},ps{\RBRACE}) \kw{end}
\ep
\noindent
\pind The \texttt{mk\_...} constructors were not introduced above.
\pind The reason is simple; a pragmatic decision:
\begynd
\pind As the domain modeller proceeds in their work
\pind they may, when encountering Cartesian compounds,
\pind be free to leave some components (of the Cartesian) out,
\pind components that they may later introduce.
\afslut
\pind So really, the first of the identities above ought be expressed as
\afslut
%\RSLatex 
%      is_Cartesian(p) is
%          let (p1`eta:&P1&,p2:`eta&P2&,...,pc:`eta&Pc&,...) = calc_Cartesian_parts_and_sorts(p) in
%          p = mk_Cartesian(obs_P1(p),obs_P2(p),...,obs_Pc(p),...) end 
%\endRSLatex 
\bp
\>\>\>is\_Cartesian(p) {\IS}\\
\>\>\>\>\>\kw{let} (p1$\eta$:P1,p2:$\eta$P2,{\DOTDOTDOT},pc:$\eta$Pc,{\DOTDOTDOT}) {\EQ} calc\_Cartesian\_parts\_and\_sorts(p) \kw{in}\\
\>\>\>\>\>p {\EQ} mk\_Cartesian(obs\_P1(p),obs\_P2(p),{\DOTDOTDOT},obs\_Pc(p),{\DOTDOTDOT}) \kw{end} 
\ep

\noindent
\begynd
\pind We continue this explication in Sect.\,\vref{AProofTheoreticExplicationII}.
\afslut

%%  \input{/home/db/2023/final/space-time}
\nbbbbb{Space and Time}\label{final:Space and Time}

\begynd
\pind The concepts of space and time can be \sfsl{transcendentally
      deduced},
\begynd
\pind by rational reasoning, 
\pind as has been shown in \cite[\sfsl{Kai
      S{\o}rlander}]{kaisorlander1994,kaisorlander1997,kaisorlander2002,kaisorlander2016,kaisorlander2022}, 
\pind from the facts of \sfsl{symmetry, asymmetry, transitivity} and \sfsl{intransitivity} relations.
\afslut

\pind \sfsl{They are therefore facts of every possible universe.}
\afslut

\nbbbb{Space}\label{final:Space}

\begynd
\pind There is one given space.
\pind As a type we name it $\mathbb{SPACE}$.
\pind We do not bother, here, about textual representation of spatial
      locations, 
\pind but here is an example that would work in or near this globe we
      call our earth: 
\pind \texttt{Latitude\,55.805600, Longitude\,12.448160,
  Altitude\,35\,m}\footnote{The author's house location\,!}.
\afslut

\mnewfoil

\begynd
\pind Also, in this paper, we do not present models of $\mathbb{SPACE}$.
\pind But we do introduce such notions as
\begynd
\pind (i)   $\mathbb{POINT}$: as $\mathbb{SPACE}$ being some dense and
            infinite collection of points;
\pind (ii)  $\mathbb{LOCATION}$: as the location in space of some point; 

%\RSLatex
%value  &\textsf{\sort{record\_$\mathbb{LOCATION}$}}:& E -> &$\mathbb{LOCATION}$\index{symbind}{zzzzzz@\protect{\sort{record\_$\mathbb{LOCATION}$}}}&
%\endRSLatex 
\bp
\kw{value}\ \ \textsf{\sort{record\_$\mathbb{LOCATION}$}}: E {\RIGHTARROW} $\mathbb{LOCATION}$\index{symbind}{zzzzzz@\protect{\sort{record\_$\mathbb{LOCATION}$}}}
\ep
\newdf{\sort{record\_$\mathbb{LOCATION}$}}
\pind (iii) $\mathbb{CURVE}$: as an infinite collection of points
            forming a mathematical curve -- having a (finite or
            infinite) \sfsl{length};
\pind (iv)  $\mathbb{SURFACE}$: as an infinite collection of points
            forming a mathematical surface -- having a (finite or
            infinite) \sfsl{area}; and
\pind (v)   $\mathbb{VOLUME}$: as an infinite collection of points
            forming a mathematical volume -- having a (finite or
            infinite) \sfsl{volume}.
\afslut
\pind We suggest it, as a domain science \& engineering research topic,
\begynd
\pind that somebody studies \sfsl{a calculus or calculi of spatial modelling}.
\afslut
\afslut

\nbbbb{Time}\label{final:Time}

\begynd
\pind There is one given time.
\pind As a type we name it $\mathbb{TIME}$.
\begynd
\pind We do not bother, here, about textual representation of time, 
\pind but here is an example: \texttt{\todaytime}\footnote{The time
      this text was last compiled\,!}.
\afslut
\pind But we do introduce such crucial notions as
\begynd
\pind \sfsl{time interval} $\mathbb{TI}$ and
\pind operations on $\mathbb{TIME}$ and $\mathbb{TI}$:

%\RSLatex
%value
%    -: &$\mathbb{TIME}$&><&$\mathbb{TIME}$&->&$\mathbb{TI}$& 
%    +: &$\mathbb{TIME}$&><&$\mathbb{TI}$&->&$\mathbb{TIME}$& 
%    *: Real><&$\mathbb{TI}$&->&$\mathbb{TI}$& 
%\endRSLatex 
\bp
\kw{value}\\
\>\>{\MINUS}: $\mathbb{TIME}${\TIMES}$\mathbb{TIME}${\RIGHTARROW}$\mathbb{TI}$ \\
\>\>{\PLUS}: $\mathbb{TIME}${\TIMES}$\mathbb{TI}${\RIGHTARROW}$\mathbb{TIME}$ \\
\>\>{\AST}: \kw{Real}{\TIMES}$\mathbb{TI}${\RIGHTARROW}$\mathbb{TI}$ 
\ep
\afslut
\noindent
\pind A crucial time-related operation is that of \textsf{\sort{record\_$\mathbb{TIME}$}}.
\pind It applies to ``nothing'':
      \textsf{\sort{record\_$\mathbb{TIME}$}()} and yields $\mathbb{TIME}$.
\afslut

%\RSLatex
%value  &\textsf{\sort{record\_$\mathbb{TIME}$}}:& Unit -> &$\mathbb{TIME}$\index{symbind}{zzzzzz@\protect{\sort{record\_$\mathbb{TIME}$}}}&
%\endRSLatex 
\bp
\kw{value}\ \ \textsf{\sort{record\_$\mathbb{TIME}$}}: \kw{Unit} {\RIGHTARROW} $\mathbb{TIME}$\index{symbind}{zzzzzz@\protect{\sort{record\_$\mathbb{TIME}$}}}
\ep
\newdf{\sort{record\_$\mathbb{TIME}$}}
%\nbbbbb{Space and Time}\label{final:Space and Time}



\nbbbbb{Internal Qualities}\label{Internal Qualities}

\begynd
\pind We refer to the \sfsl{Internal Qualities} characterisation on
      Page\,\pageref{char:internal-qualities}.
\pind We can justify the grouping of internal endurant qualities into
three kinds:
\begynd
\pind \sfsl{unique identifiers}, cf.\,Sect.\,\ref{Unique Identification},
\pind \sfsl{mereologies}, cf.\,Sect.\,\ref{Mereology}, and
\pind \sfsl{attributes}, cf.\,Sect.\,\ref{Attributes}.
\pind To this we add the concept of \sfsl{intentional pull},
      cf.\,Sect.\,\ref{Intentional Pull}. 
\afslut
\afslut

\nbbbb{Unique Identification}\label{Unique Identification}

\begynd
\pind On the basis of \sfsl{philosophical reasoning}, within \sfsl{metaphysics}, 
\begynd
\pind we [can] argue that parts are uniquely identifiable
     \cite[\sfsl{Kai S{\o}rlander}]{kaisorlander1994,kaisorlander1997,kaisorlander2002,kaisorlander2016,kaisorlander2022} 
\afslut
\afslut

\bbb{Calculate Unique Identifiers}\label{Calculate Unique Identifiers}

\addcontentsline{toc}{subparagraph}{{\hspace*{10mm} \small{\brcolor{descr\_unique\_identifier}}}}\newdf{descr\_unique\_identifier}
\bff{descr\_unique\_identifier}\newdf{descr\_unique\_identifier}
%\RSLatex
%value 
%   descr_unique_identifier: P -> &\rsltext&
%   descr_unique_identifier(p) is
%      &\bq& &\sort{Narrative:}&
%          [s] text on unique identifier sort
%          [o] text on unique identifier observer
%          [a] text on axioms and/or proof obligations
%       &\,\,\sort{Formalisation:}&
%          [s] type 
%                 PI
%          [o] value 
%                 uid_P: P -> PI
%          [a] &\sort{axiom} and/or \sort{proof obligation}&
%                 &$\mathcal{A}$/$\mathcal{P}$&(...) &\eq&
%\endRSLatex 
\bp
\kw{value} \\
\>\ descr\_unique\_identifier: P {\RIGHTARROW} \rsltext\\
\>\ descr\_unique\_identifier(p) {\IS}\\
\>\>\>\bq \sort{Narrative:}\\
\>\>\>\>\>{\LBRACKET}s{\RBRACKET} text on unique identifier sort\\
\>\>\>\>\>{\LBRACKET}o{\RBRACKET} text on unique identifier observer\\
\>\>\>\>\>{\LBRACKET}a{\RBRACKET} text on axioms and/or proof obligations\\
\>\>\>\ \,\,\sort{Formalisation:}\\
\>\>\>\>\>{\LBRACKET}s{\RBRACKET} \kw{type} \\
\>\>\>\>\>\>\>\>\ PI\\
\>\>\>\>\>{\LBRACKET}o{\RBRACKET} \kw{value} \\
\>\>\>\>\>\>\>\>\ uid\_P: P {\RIGHTARROW} PI\\
\>\>\>\>\>{\LBRACKET}a{\RBRACKET} \sort{axiom} and/or \sort{proof obligation}\\
\>\>\>\>\>\>\>\>\ $\mathcal{A}$/$\mathcal{P}$({\DOTDOTDOT}) \eq
\ep
\eff

\newexampl{Unique Identifiers}{ui}{exa:Unique Identifiation}

\nbbb{Endurant Identifier States}

\begynd
\pind Given the endurant state values, 
\begynd
\pind for the whole domain or 
\pind for respective, manifest part sorts,
\afslut
\pind one can define corresponding unique identifier values.
\afslut

\newexampl{Unique Identifier State}{uis}{exa:Unique Identifier State}

\nbbb{Axioms}

\begynd
\pind The number of manifest parts
\pind is the sames as 
\pind the number of manifest part unique identifiers.
\afslut

\newexampl{Unique Identifier Axiom}{uia}{exa:Unique Identifier Axiom}

\nbbb{Endurant Retrieval}\label{intq-retr}

\begynd
\pind Given a unique identifier, $\pi$, 
\pind of a manifest part, $p$, 
\pind of an endurant state, $\sigma$, of a domain 
\pind one can retrieve that part:
\afslut

%\RSLatex
%    value
%        `sigma:`Sigma = gen_`Sigma(uod)
%        retr_P: `Pi -> `Sigma -> P
%        retr_P(`pi)(`sigma) is let p:P :- p isin `sigma /\ uid_P(p)=`pi in p end
%\endRSLatex 
\bp
\>\>\kw{value}\\
\>\>\>\>$\sigma$:$\Sigma$ {\EQ} gen\_$\Sigma$(uod)\\
\>\>\>\>retr\_P: $\Pi$ {\RIGHTARROW} $\Sigma$ {\RIGHTARROW} P\\
\>\>\>\>retr\_P($\pi$)($\sigma$) {\IS} \kw{let} p:P {\RDOT} p {\ISIN} $\sigma$ {\WEDGE} uid\_P(p){\EQ}$\pi$ \kw{in} p \kw{end}
\ep


\nbbbb{Mereology}\label{Mereology}

\begynd
\pind Mereology is the study and knowledge of parts and part relations.
\pind It was first put forward, around 1916, by the Polish logician
\sfsl{Stanis{\l}aw Le{\'s}niewski} \cite{Lesniewski1920s,CasatiVarzi1999}. 
\afslut

\begynd
\pind Which are the relations that can be relevant for ``endurant-hood''\,? 
\pind There are basically two relations: (i) physical ones, and (ii) conceptual ones.
\begynd
\pind (i) Physically two or more endurants 
\begynd
\pind may be topologically either adjacent to one another,
\begynd
\pind like rails of a line, or within an endurant, 
\pind like links and hubs of a road net, 
\pind or an atomic part is conjoined to one or more fluids, 
\pind or a fluid is conjoined to one or more parts.
\afslut 
\pind The latter two could also be considered conceptual ``adjacencies''.
\afslut
\pind (ii) Conceptually some parts, 
\begynd
\pind like automobiles, ``belong'' to an embedding endurant, 
\pind like to an automobile club, 
\pind or are registered in the local department of vehicles, 
\pind or are ‘intended’
to drive on roads.
\afslut
\afslut

\nbbb{Calculate Mereologies}\label{Calculate Mereologies}

\bff{descr\_mereology}\newdf{descr\_mereology}
\addcontentsline{toc}{subparagraph}{{\hspace*{10mm} \small{\lilacolor{descr\_mereology}}}}
%\RSLatex
%value 
%   descr_mereology: P -> &\rsltext&
%   descr_mereology(p) is
%      &\bq& &\sort{Narrative:}&
%          [s] text on mereology &type&
%          [o] text on  mereology observer
%          [a] text on axioms and/or proof obligations
%       &\,\,\sort{Formalisation:}&
%          [s] type 
%                 MT = &$\mathcal{M}$&(p)
%          [o] value 
%                 mereo_P: P -> MT
%          [a] &\sort{axiom} and/or \sort{proof obligation}&
%                 &$\mathcal{A}$/$\mathcal{P}$&(...) &\eq&
%\endRSLatex 
\bp
\kw{value} \\
\>\ descr\_mereology: P {\RIGHTARROW} \rsltext\\
\>\ descr\_mereology(p) {\IS}\\
\>\>\>\bq \sort{Narrative:}\\
\>\>\>\>\>{\LBRACKET}s{\RBRACKET} text on mereology type\\
\>\>\>\>\>{\LBRACKET}o{\RBRACKET} text on\ \ mereology observer\\
\>\>\>\>\>{\LBRACKET}a{\RBRACKET} text on axioms and/or proof obligations\\
\>\>\>\ \,\,\sort{Formalisation:}\\
\>\>\>\>\>{\LBRACKET}s{\RBRACKET} \kw{type} \\
\>\>\>\>\>\>\>\>\ MT {\EQ} $\mathcal{M}$(p)\\
\>\>\>\>\>{\LBRACKET}o{\RBRACKET} \kw{value} \\
\>\>\>\>\>\>\>\>\ mereo\_P: P {\RIGHTARROW} MT\\
\>\>\>\>\>{\LBRACKET}a{\RBRACKET} \sort{axiom} and/or \sort{proof obligation}\\
\>\>\>\>\>\>\>\>\ $\mathcal{A}$/$\mathcal{P}$({\DOTDOTDOT}) \eq
\ep
\eff
\noindent
\begynd
\pind \textsf{$\mathcal{M}$(p)} is usually a type expression over
      unique identifiers of mereology-related parts.
\afslut

\newexampl{Mereology}{mer}{exa:Mereology}

\noindent
\begynd
\pind Given the definition of
\begynd
\pind external qualities of a domain,
\pind and its unique identifier and mereology internal qualities
\pind one can analyse and describe many properties of that domain.
\afslut
\pind The \sfsl{routes} subsection (Page\,\pageref{exa:Routes}) of the mereology example,
      Example\,\vref{exam:mer}, illustrates one such property.
\afslut

\nbbbb{Attributes}\label{Attributes}

\begynd
\pind Parts and fluids are typically recognised
\begynd
\pind because of their spatial form and are otherwise characterised by their intangible,
\pind but measurable attributes.
\afslut
\pind That is, whereas endurants, 
\begynd
\pind whether solid (as are parts) or fluids,
\pind are physical, tangible,
\pind in the sense of being spatial 
\pind [or being abstractions, i.e., concepts, of spatial endurants], 
\pind attributes are intangible: 
\begynd
\pind cannot normally be touched, or seen,  
\pind but can be objectively measured. 
\afslut
\pind Thus, in our quest for describing domains where humans play an
      active r{\^{o}}le, 
\begynd
\pind we rule out subjective ``attributes'':
\pind feelings, sentiments, moods. 
\pind Thus we shall abstain, in our domain science 
\pind also from matters of psychology and aesthetics.
\afslut
\afslut

\bbb{Functional Analysis of Attributes}\label{Functional Analysis of Attributes}

\begynd
\pind Given a manifest part, \textsf{p}, that is,
\begynd
\pind either an atom,
\pind or a Cartesian,
\pind or a part set,
\afslut
\pind we calculate from that part, 
\pind its constituent attributes values and types:
\afslut

\bff{determine\_attributes}
%\RSLatex
%    value
%        determine_attributes: P -> (a1><`eta&A1&) ><  (a2><`eta&A2&) ><  ... >< (aa><`eta&Aa&) 
%\endRSLatex 
\bp
\>\>\kw{value}\\
\>\>\>\>determine\_attributes: P {\RIGHTARROW} (a1{\TIMES}$\eta$A1) {\TIMES}\ \ (a2{\TIMES}$\eta$A2) {\TIMES}\ \ {\DOTDOTDOT} {\TIMES} (aa{\TIMES}$\eta$Aa) 
\ep
\eff
\addcontentsline{toc}{subparagraph}{{\hspace*{10mm} \small{\brcolor{determine\_attributes}}}}\newaf{determine\_attributes}


\bbb{Describe Attributes}\label{Calculate Attributes}

\bff{descr\_attributes}\newdf{descr\_attributes}
%\RSLatex
%value 
%   descr_attributes: P -> &\rsltext&
%      let ((_,`eta&A1&),(_,`eta&A2&),...,(_,`eta&Aa&)) = determine_attributes(p:P) in
%   descr_attributes(p) is
%      &\bq& &\sort{Narrative:}&
%          [s] text on attribute &types&
%          [o] text on attribute observers
%          [a] text on axioms and/or proof obligations
%       &\,\,\sort{Formalisation:}&
%          [s] type 
%                 A1 [ = ... ], A2 [ = ... ], ..., Aa [ = ... ], 
%          [o] value 
%                 attr_A1: P -> A1, attr_A2: P -> A2, ..., attr_Aa: P -> Aa,
%          [a] &\sort{axiom} and/or \sort{proof obligation}&
%                 &$\mathcal{A}$/$\mathcal{P}$&(...) &\eq&
%      end
%\endRSLatex 
\bp
\kw{value} \\
\>\ descr\_attributes: P {\RIGHTARROW} \rsltext\\
\>\>\>\kw{let} (({\UNDERLINE},$\eta$A1),({\UNDERLINE},$\eta$A2),{\DOTDOTDOT},({\UNDERLINE},$\eta$Aa)) {\EQ} determine\_attributes(p:P) \kw{in}\\
\>\ descr\_attributes(p) {\IS}\\
\>\>\>\bq \sort{Narrative:}\\
\>\>\>\>\>{\LBRACKET}s{\RBRACKET} text on attribute types\\
\>\>\>\>\>{\LBRACKET}o{\RBRACKET} text on attribute observers\\
\>\>\>\>\>{\LBRACKET}a{\RBRACKET} text on axioms and/or proof obligations\\
\>\>\>\ \,\,\sort{Formalisation:}\\
\>\>\>\>\>{\LBRACKET}s{\RBRACKET} \kw{type} \\
\>\>\>\>\>\>\>\>\ A1 {\LBRACKET} {\EQ} {\DOTDOTDOT} {\RBRACKET}, A2 {\LBRACKET} {\EQ} {\DOTDOTDOT} {\RBRACKET}, {\DOTDOTDOT}, Aa {\LBRACKET} {\EQ} {\DOTDOTDOT} {\RBRACKET}, \\
\>\>\>\>\>{\LBRACKET}o{\RBRACKET} \kw{value} \\
\>\>\>\>\>\>\>\>\ attr\_A1: P {\RIGHTARROW} A1, attr\_A2: P {\RIGHTARROW} A2, {\DOTDOTDOT}, attr\_Aa: P {\RIGHTARROW} Aa,\\
\>\>\>\>\>{\LBRACKET}a{\RBRACKET} \sort{axiom} and/or \sort{proof obligation}\\
\>\>\>\>\>\>\>\>\ $\mathcal{A}$/$\mathcal{P}$({\DOTDOTDOT}) \eq\\
\>\>\>\kw{end}
\ep
\eff
\noindent
\begynd
\pind The domain modeller has thus determined/decided 
\begynd
\pind that \textsf{A1, A2, ..., Aa} are the ``interesting'' attributes of
\pind of parts of sort \textsf{P}.
\afslut
\pind Attributes are often given a ``concrete'' form, hence the \textsf{[ = ... ]} 
\pind where the \textsf{...} is some type expression.
\afslut

\newexampl{Attributes}{attrs}{exa:Attributes}

\nbbb{Attribute Categories}

\begynd
\pind Michael A.\ Jackson has proposed a structure of attributes \cite{lexicon}.
\afslut

\attrcat{Static}{static}
\begynd
\pind By a static attribute we shall understand an attribute whose values are constants, i.e., cannot change.
\afslut

\attrcat{Dynamic}{dynamic}
\begynd
\pind By a dynamic attribute we shall understand an attribute whose values are variable, i.e., can change. 
\pind Dynamic attributes are either inert, reactive or active attributes.
\afslut

\attrcat{Inert}{inert}
\begynd
\pind By an inert attribute we shall understand a dynamic attribute
      whose values only change as the result of external stimuli where 
      these stimuli prescribe new values.   
\afslut

\attrcat{Reactive}{reactive}
\begynd
\pind By a reactive attribute we shall
      understand a dynamic attribute whose values, if they vary, change in response to external
      stimuli, where these stimuli either come from outside the domain of interest or from other
      endurants.
\afslut

\attrcat{Active}{active}
\begynd
\pind By an active attribute we shall understand a dynamic attribute
whose values change (also) of its own volition.  
\pind Active attributes
      are either autonomous, or  biddable or programmable attributes.
\afslut

\attrcat{Autonomous}{autonomous}
\begynd
\pind By an autonomous attribute we
      shall understand a dynamic active attribute whose values change
      only ``on their own volition''. 
\pind The values of an autonomous attributes are a ``law onto
      themselves and their surroundings''.
\afslut

\attrcat{Biddable}{biddable}
\begynd
\pind By a  biddable attribute we shall understand a dynamic active
      attribute whose values are prescribed but may fail to be observed as such.
\afslut

\attrcat{Programmable}{programmable}
\begynd
\pind  By a  programmable attribute we shall understand a dynamic active attribute whose values can be prescribed.
\afslut

\begynd
\pind We modify Jackson's categorisation.
\pind This is done in preparation for our expos{\'e} of behaviour
      signatures, cf.\,Sect.\,\vref{Behaviour Signatures}.
\pind Figure\,\vref{attributes.fig} shows groupings of some  of M.\,A.\,Jackson's  basic categories.
\afslut

\noindent
\hDBfigure{attributes}{4cm}{An Attribute Ontology}{attributes.fig}

\begynd
\pind Our motivation for modifying Jackson's attribute categories is
      as follows:
\begynd
\pind when transcendentally deducing behaviours from parts
\pind we find that there are basically a need for distinguishing 
\pind between only three major attribute categories
\begynd
\pind the static,
\pind the monitorable, and
\pind the programmable attributes.
\afslut
\pind Static attributes have their values passed ``by value'',  as constants,
\pind programmable attributes  have their values passed by ``by
      reference'', as variables who value can be changed, and 
\pind monitorable attributes  have their values passed by ``by
      name'' --  as we shall see\,!
\afslut
\afslut

\nbbbb{Intentional Pull}\label{Intentional Pull}
\bbb{Characterisations}

\begynd
\pind \sort{Intentionality} as a philosophical concept 
is defined by the Stanford Encyclopedia of
Philosophy\footnote{Jacob, P. (Aug 31, 2010). Intentionality. Stanford 
  Encyclopedia of Philosophy
  (\texttt{\-seop.\-illc.\-uva.\-nl/\-en\-tries/\-in\-ten\-tion\-a\-li\-ty/} October 15, 2014,
  retrieved April 3, 2018.} as \sfsl{``the power of minds to be about,
  to represent, or to stand for, things, properties and states of 
  affairs.''}

\pind \sort{Intent} is then a usually clearly formulated or planned intention.
\pind An example of \sfsl{intent} is that of roads made for automobiles
      \sort{and} automobiles meant for roads.
\afslut

\mnewfoil

\begynd
\pind \sort{Intentional Pull\footnote{The term \sfsl{intentional pull}
  is chosen so as to connote with the term \sfsl{gravitational
    pull}.}:} Two or more artefactual parts of different sorts, but
with overlapping sets of intents may 
excert an intentional ``pull'' on one another. This intentional ``pull'' may take many forms. Let $p_x$$:X$ and
$p_y$:$Y$ be two parts of different sorts ($X$,$Y$), and with common intent, $\iota$. Manifestations of these, their common
intent, must somehow be subject to constraints, and these must be expressed predicatively.
When a composite artefact
has an intentionality then its constituents have individual
intentionalities that relate to
these. The composite road transport system has intentionality of the roads serving the automobiles, and
the automobiles have the intent of being served by the roads.
\afslut

\newexampl{Intentional Pull: Road Transport}{intent-pull}{exa:Intentional Pull}

\inlinenewexampl{Intentional Pull: Double-entry Bookkeeping}{debook}{
\begynd
\pind Double-entry bookkeeping, also known as double-entry accounting,
is a method of bookkeeping that relies on a two-sided accounting entry
to maintain financial information. Every entry to an account requires
a corresponding and opposite entry to a different account. The
double-entry system has two equal and corresponding sides known as
debit and credit. A transaction in double-entry bookkeeping always
affects at least two accounts, always includes at least one debit and
one credit, and always has total debits and total credits that are
equal.\footnote{https://en.wikipedia.org/wiki/Double-entry\_bookkeeping}
\afslut}

\nbbbb{A Proof-theoretic Explication, II}\label{AProofTheoreticExplicationII}

\begynd
\pind We remind You of Sect.\,\vref{AProofTheoreticExplicationI}.
\afslut

\begynd
\pind With the introduction of \sfsl{analysis functions} and \sfsl{observers} 
\begynd
\pind for \sfsl{unique identifiers, mereology} and \sfsl{attributes} 
\pind we can now augment the \texttt{is\_..., uid\_..., mereo\_..., attr\_A...} observers 
\pind introduced since Page\,\pageref{AProofTheoreticExplicationI}.
\afslut
\afslut

%\RSLatex
%    is_manifest(p:P) is
%        let ((_,`eta&A1&),(_,`eta&A2&),(_,`eta&Aa&)) = calc_attributes(p) in
%        p = mk_P(uid_P(p),mereo_P(p),(attr_A1(p),attr_A2(p),...,attr_Aa(p))) end
%\endRSLatex 
\bp
\>\>is\_manifest(p:P) {\IS}\\
\>\>\>\>\kw{let} (({\UNDERLINE},$\eta$A1),({\UNDERLINE},$\eta$A2),({\UNDERLINE},$\eta$Aa)) {\EQ} calc\_attributes(p) \kw{in}\\
\>\>\>\>p {\EQ} mk\_P(uid\_P(p),mereo\_P(p),(attr\_A1(p),attr\_A2(p),{\DOTDOTDOT},attr\_Aa(p))) \kw{end}
\ep


\nbbbbb{Perdurants}\label{Perdurants}

\begynd
\pind A key point of our domain science \& engineering approach is this:
\begynd
\pind to \bbcolor{every manifest part}
\pind we \b{transcendentally deduce}
\pind a \bbcolor{unique behaviour.}
\afslut

\begynd
\pind By \bbcolor{transcendental}
\begynd
\pind \sl  we shall understand the philosophical notion: 
\pind the a priori or intuitive basis of knowledge, \nyl independent
      of experience.\rm
\afslut
\afslut
  
\begynd
\pind  By a \bbcolor{transcendental deduction}  
\begynd
\pind \sl we shall understand the philosophical notion: 
\pind a transcendental `conversion' of one kind of knowledge 
\pind into a seemingly different kind of know\-ledge.\rm
\afslut
\afslut

\nbbbb{Channels}\label{Channels}

\begynd
\pind Part behaviours may interact.
\begynd
\pind To express part behaviours and their interaction
\pind we use \textsf{Hoare}'s \texttt{CSP} \cite{Hoa78a,Hoare85,Hoare85+2004}.
\begynd
\pind One may question this choice. 
\pind In \cite[\sfsl{2009--2017}]{dines:hoare:2009,BjornerMereology2013,BjornerMereologyCSP2017}
      we show \sfsl{``that to every mereology there is a
      \texttt{CSP} expression''}.
\pind On that background we maintain that \texttt{CSP} is a reasonable
      choice ---
\pind but invite the reader to suggest more appropriate mechanisms for
      handling behaviours and their communication.\footnote{%%%%%%%%%%%%%%%
\pind Please bear in mind that the use, here, of  \texttt{CSP}, is in the following context:
\begynd
\pind the \texttt{CSP} clauses are not to be ``interpreted'' on a computer
\pind where this ``computerisation'' has to be ``shared''
\pind with other computations;
\pind hence \texttt{CSP} \sfsl{synchcronisation \& communication} is ``ideal''
\pind and reflects reality.
\afslut}
\afslut

\pind So, in general, we declare a \texttt{RSL/CSP} \sfsl{channel}:
\afslut

\bff{\kw{channel} declaration}
%\RSLatex
%    channel { ch[{ui,uj}] | ui,uj:UI :- {ui,uj}<<=&$uis$& } : M
%\endRSLatex 
\bp
\>\>\kw{channel} {\LBRACE} ch{\LBRACKET}{\LBRACE}ui,uj{\RBRACE}{\RBRACKET} {\BAR} ui,uj:UI {\RDOT} {\LBRACE}ui,uj{\RBRACE}{\SUBSETEQ}$uis$ {\RBRACE} : M
\ep
\eff

\noindent
\begynd
\pind Here \textsf{ch} is the name of the indexed array of channels
      and the indexes are, in general, any two element sets of unique
      part identifiers.
\pind That is: For every pair of part behaviours -- 
\pind identified but their unique part identifiers (\textsf{ui,uj}) --
\pind there is a channel, say \textsf{ch[$\{$ui,uj$\}$]}.

\pind \textsf{M} is the type of the messages communicate between
behaviours of index \textsf{ui,uj}.

\pind We shall develop, in Sect.\,\vref{final-perd:Events}, the specifics of the type, \textsf{M}, of channel messages.
\afslut

      
\nbbbb{Actors}\label{final-perd:Actors}

\begynd
\pind By an \sfsl{actor} we shall understand
\begynd
\pind either an \sfsl{action},
\pind \xtut{or an \sfsl{event},}
\pind or a \sfsl{behaviour}.
\afslut
\afslut

\nbbb{Actions}\label{final-perd:Actions}

\begynd 
\pind By an \sfsl{action of a behaviour} we shall understand 
\begynd
\pind something which is local to a behaviour,
\pind and, which, when applied,
\pind potentially changes the states.
\afslut
\pind Generally action clauses are expressed in \texttt{RSL} \citersl.
\afslut

\newexampl{Road Transport Actions}{road-trans-act}{exa:Domain Actions}

\nbbb{Events}\label{final-perd:Events}

\xtut{%%%%%%%%%%%%%%%%%%
\begynd
\pind By an \sfsl{event of a behaviour} we shall understand 
\begynd
\pind something that involves two behaviours,
\pind and, which, when applied,  
\pind potentially changes the states of both behaviours.
\afslut
\pind Event clauses are expressed using the  \texttt{CSP} elements of \texttt{RSL}.
\pind That is, the \texttt{CSP} output ``\sfsl{!}'' and input events ``\sfsl{?}'': 
\afslut

%\RSLatex
%   ch[{ui,uj}] ! expr
%   let val = ch[{ui,uj}] ? ... end
%\endRSLatex 
\bp
\>\ ch{\LBRACKET}{\LBRACE}ui,uj{\RBRACE}{\RBRACKET} ! expr\\
\>\ \kw{let} val {\EQ} ch{\LBRACKET}{\LBRACE}ui,uj{\RBRACE}{\RBRACKET} ? {\DOTDOTDOT} \kw{end}
\ep
}%%%%%%%%%%%%%%%%%%%%%%%

\newexampl{Road Transport Events}{road-trans-eve}{exa:Domain Events}

\xtut{%%%%%%%%%%%%%%%%%%%%%%%%%%%%%%%%%
\nbbbb{State Access and Updates}\label{State Access and Updates}

\begynd
\pind We need define two functionals:
\begynd
\pind one for changing the mereology of a part and
\pind another for changing the attribute value of a part.
\afslut
\pind We therefore informally define the following functionals:
\afslut

\nbbb{Update Mereologies}\label{Update Mereologies}

\begin{itemize}
\item \sort{part\_update\_mereology} is a functional: it takes the following
      arguments: a part \textsf{p} of type \textsf{P} and a mereology value and yields a part of type \textsf{P}.
\item The yielded result, \textsf{p$'$}, has the same unique identifier,
      as the argument part \textsf{p},
\item a new, the argument, mereology,
      as the argument part \textsf{p}, 
\item and the same attribute values for all attributes,
      as the argument part \textsf{p}.
\end{itemize}

\addcontentsline{toc}{subparagraph}{{\hspace*{10mm} \small{\lilacolor{part\_update\_mereology}}}}
%\RSLatex
%value
%    &\sort{part\_update\_mereology}&: P -> M -> P
%    &\sort{part\_update}&(p)(m) is
%        let ((_,`eta&A1&),(_,`eta&A2&),...,(_,`eta&Aa&)) = determine_attributes(p) in
%        let p&$'$&:P :- uid_P(p&$'$&)=uid_P(p)/\mereo_P(p&$'$&)=m/\
%            all `eta&A&:`eta&$\Phi$&:-`eta&A&isin{`eta&A1&,`eta&A2&,...,`eta&Aa&}=>attr_A(p&$'$&)=attr_A(p) in
%        p&$'$& end end
%\endRSLatex 
\bp
\kw{value}\\
\>\>\sort{part\_update\_mereology}: P {\RIGHTARROW} M {\RIGHTARROW} P\\
\>\>\sort{part\_update}(p)(m) {\IS}\\
\>\>\>\>\kw{let} (({\UNDERLINE},$\eta$A1),({\UNDERLINE},$\eta$A2),{\DOTDOTDOT},({\UNDERLINE},$\eta$Aa)) {\EQ} determine\_attributes(p) \kw{in}\\
\>\>\>\>\kw{let} p$'$:P {\RDOT} uid\_P(p$'$){\EQ}uid\_P(p){\WEDGE}mereo\_P(p$'$){\EQ}m{\WEDGE}\\
\>\>\>\>\>\>{\ALL} $\eta$A:$\eta$$\Phi${\RDOT}$\eta$A{\ISIN}{\LBRACE}$\eta$A1,$\eta$A2,{\DOTDOTDOT},$\eta$Aa{\RBRACE}{\DBLRIGHTARROW}attr\_A(p$'$){\EQ}attr\_A(p) \kw{in}\\
\>\>\>\>p$'$ \kw{end} \kw{end}
\ep


\nbbb{Update Attributes}\label{Update Attributes}

\begin{itemize}
\item \sort{part\_update\_attribute} is a functional: it takes the following
      arguments: a part \textsf{p} of type \textsf{P} and a pair of an attribute name
      and value, and yields a part \textsf{p$'$} of type \textsf{P}.
\item The argument attribute name must be that of an attribute of the part.
\item The yielded result \textsf{p$'$} has the same unique identifier
  and mereology as the argument part \textsf{p},
\item and the same attribute values for all attributes,
      as the argument part \textsf{p},
\pind except for argument attribute (name) for which it now yields the
      argument attribute value.
\end{itemize}

\addcontentsline{toc}{subparagraph}{{\hspace*{10mm} \small{\lilacolor{part\_update\_attribute}}}}
%\RSLatex
%value
%    &\sort{part\_update\_attribute}&: P -> `Phi&A& >< A -> P
%    &\sort{part\_update\_attribute}&(p)(`eta&A&,a) is
%        let ((_,`eta&A1&),(_,`eta&A2&),...,(_,`eta&Aa&)) = determine_attributes(p) in
%            &\sort{assert:}& `eta&A&isin{`eta&A1&,,`eta&A2&,...,,`eta&Aa&}
%        let p&$'$&:P :- uid_P(p)=uid_P(p&$'$&)/\mereo_P(p)=mereo_P(p&$'$&)/\
%            all `eta&A&:`eta&$\Phi$&:-`eta&A&isin{`eta&A1&,,`eta&A2&,...,,`eta&Aa&}\`eta&A&=>attr_A(p&$'$&)=attr_A(p) in
%        p&$'$& end end
%\endRSLatex 
\bp
\kw{value}\\
\>\>\sort{part\_update\_attribute}: P {\RIGHTARROW} $\Phi$A {\TIMES} A {\RIGHTARROW} P\\
\>\>\sort{part\_update\_attribute}(p)($\eta$A,a) {\IS}\\
\>\>\>\>\kw{let} (({\UNDERLINE},$\eta$A1),({\UNDERLINE},$\eta$A2),{\DOTDOTDOT},({\UNDERLINE},$\eta$Aa)) {\EQ} determine\_attributes(p) \kw{in}\\
\>\>\>\>\>\>\sort{assert:} $\eta$A{\ISIN}{\LBRACE}$\eta$A1,,$\eta$A2,{\DOTDOTDOT},,$\eta$Aa{\RBRACE}\\
\>\>\>\>\kw{let} p$'$:P {\RDOT} uid\_P(p){\EQ}uid\_P(p$'$){\WEDGE}mereo\_P(p){\EQ}mereo\_P(p$'$){\WEDGE}\\
\>\>\>\>\>\>{\ALL} $\eta$A:$\eta$$\Phi${\RDOT}$\eta$A{\ISIN}{\LBRACE}$\eta$A1,,$\eta$A2,{\DOTDOTDOT},,$\eta$Aa{\RBRACE}{\SETMINUS}$\eta$A{\DBLRIGHTARROW}attr\_A(p$'$){\EQ}attr\_A(p) \kw{in}\\
\>\>\>\>p$'$ \kw{end} \kw{end}
\ep

\noindent
\begynd
\pind Monitorable attributes usually change their values surreptitiously. 
\begynd
\pind That is, ``behind the back'', so-to-speak, of the part behaviour.
\afslut
\afslut
}%%%%%%%%%%%%%%%%%%%%%%%

\input{/home/db/2023/final/new-perd}

\nbbbb{Domain Initialisation}\label{Domain Initialisation}

\begynd
\pind By \sfsl{domain initialisation} we mean the ``start-up'' of a
      behaviour for all manifest parts.
\afslut

\newexampl{Road Transport Domain Initialisation}{hrtd-init}{exa:Domain Initialisation}

\paperortutorial{%% Paper:
\nbbbb{End of Domain Modelling Presentation}
\begynd
\pind This concludes the four sections, Sects.\,2, 3, 4 and 6, on
domain modelling.
\afslut
}{%% Tutorial:
\nbbbb{Summary of One Approach to Modelling Perdurants}
\begynd
\pind Section\,\ref{Perdurants} models one approach to modelling
\begynd
\pind the transcendental deduction of behaviours from manifest parts.
\pind Chapter\,\ref{chapter:Perdurants} will illustrate another, less simple, approach.
\afslut
\afslut
}



%%  LocalWords:  Perdurants behaviour priori Endurant rtd ps Hoare ui
%%  LocalWords:  behaviours CSP RSL uj UI uis expr mer Mer sta Attrs
%%  LocalWords:  mon prg Prgr monitorable endurant vals args Attr lis
%%  LocalWords:  ais als progr li ai hlis analysing uid mereo progrs
%%  LocalWords:  attr pre upd le fhi thi deterministically tHub hst
%%  LocalWords:  mk mereology ttl Initialisation Perdurant uod ereo
%%  LocalWords:  computerisation synchcronisation hla exa elipsed Aa
%%  LocalWords:  initialisation hrtd init functionals Mereologies
%%  LocalWords:  Modelling modelling

%%  LocalWords:  characterisation endurant mereologies Kai rlander hs
%%  LocalWords:  calc Formalisation uid lis ais rtd RTD Mereology Mer
%%  LocalWords:  Stanis niewski endurants adjacencies mereology mereo
%%  LocalWords:  ris recognised characterised le Aa attr analyser li
%%  LocalWords:  lj Pos APos apos fhi thi retr Formalised seop illc
%%  LocalWords:  Intentionality uva nl tion ty artefactual excert dom
%%  LocalWords:  artefact intentionality intentionalities rng inds ui
%%  LocalWords:  Characterisations descr exa Identifiation uis uia aa
%%  LocalWords:  uod mer analyse attrs categorisation behaviour pre
%%  LocalWords:  monitorable AProofTheoreticExplicationI mk modeller
%%  LocalWords:  behaviours debook

%%  LocalWords:  endurants endurant mereologies dom predI analyser fn
%%  LocalWords:  modelled perdurant harbours perdurants plasmatic et
%%  LocalWords:  modelling compartmentalise cetera analyse RSL wrt ec
%%  LocalWords:  characterisation artefactual characterised analysing
%%  LocalWords:  analysed organisation behaviours anprsi Bool calc Ec
%%  LocalWords:  ps ulations schemas ulation characterisations exa pc
%%  LocalWords:  Formalisation Cartesians uod cartesian sta Kai UoD
%%  LocalWords:  rlander intransitivity assistedness summarise nasod
%%  LocalWords:  descr modeller pre destructor mk
%%  LocalWords:  AProofTheoreticExplicationII
