
\nbbbbbb{A \texttt{RAISE} Specification Language Primer}\label{chap2.tex.RSL}
\label{begin-of-rsl}\label{RSL-intro}
\minitoc

\renewcommand{\nrslframebox}[2]{\vspace*{2mm} 
                               \boiteepaisseavecuntitre{#1}
                               #2
                               \endboiteepaisseavecuntitre
}

%\input{primer-chap2-rsl}

We present an \texttt{RSL Primer}. Indented text, in slanted font, such as
  this, presents informal material and examples. Non-indented text, in
  {{\rm roman font}}, presents narrative and formal explanation of
  \texttt{RSL} constructs.

This  \texttt{RSL Primer} omits treatment of a number of language
constructs, notably the \texttt{RSL} module concepts of
\sfsl{schemes, classes} and \sfsl{objects}. Although we do cover the 
imperative language construct of [declaration of] variables and,
hence, assignment, we shall omit treatment of structured imperative 
constructs like  \sort{for} ...,
       \sort{do} $s$ \sort{while} $b$,
       \sort{while} $b$ \sort{do} $s$ loops.

Section\,\vref{chap2.tex.rsltext} introduces additional language constructs, thereby
motivation the $^+$ in the \texttt{RSL$^+$} name.

\lastchg{%%%%%%%%%%%%%%%%%%%%%%%%%%%%%%%%%%%%%%%%%%%%%%%%
\bbbbb{Specification Units}

An \texttt{RSL} specification consists of a set (textually ordered in
any linear sequence) of \sfsl{specification units}.
We shall only treat five kinds of such units:

\begin{itemize}
\item \sfsl{type:} Prefixed by the keyword (literal) \sort{type}, type
  specification units introduce distinct type names. 
  The general form of a type specification unit is:

%\RSLatex
%    type T = Type_Expression
%\endRSLatex 
\bp
\>\>\kw{type} T {\EQ} Type\_Expression
\ep

\noindent
where \textsf{T} is a distinct (type) identifier.
Type specification units may introduce several (new, distinct) types:

%\RSLatex
%    type T1 = Type_Expression_1, T2 = Type_Exression_2, ..., Tn = Type_Expression_n
%\endRSLatex 
\bp
\>\>\kw{type} T1 {\EQ} Type\_Expression\_1, T2 {\EQ} Type\_Exression\_2, {\DOTDOTDOT}, Tn {\EQ} Type\_Expression\_n
\ep
\noindent
  For more on types, see Sect.\,\ref{tseb.rsl.Types}.

\item \sfsl{value:} Prefixed by the keyword (literal) \sort{value} value
  specification units  introduce  distinct value names.
  The general form of a value specification unit is:

%\RSLatex
%    value a:A = &$\mathcal{E}$&(...)
%\endRSLatex 
\bp
\>\>\kw{value} a:A {\EQ} $\mathcal{E}$({\DOTDOTDOT})
\ep

\noindent
where \textsf{$\mathcal{E}$(...)} is a value expression.
Value specification units may introduce several (new, distinctly named) values:

%\RSLatex
%    value a_1:A_1 = &$\mathcal{E}_1$&(...), a_2:A_2 = &$\mathcal{E}_2$&(...), ..., a_n:A_n = &$\mathcal{E}_n$&(...)
%\endRSLatex 
\bp
\>\>\kw{value} a\_1:A\_1 {\EQ} $\mathcal{E}_1$({\DOTDOTDOT}), a\_2:A\_2 {\EQ} $\mathcal{E}_2$({\DOTDOTDOT}), {\DOTDOTDOT}, a\_n:A\_n {\EQ} $\mathcal{E}_n$({\DOTDOTDOT})
\ep
Quite often the value specification is of the form:

%\RSLatex
%    value f: A -> B, f(arg) is &$\mathcal{E}$&(...[f]...)
%\endRSLatex 
\bp
\>\>\kw{value} f: A {\RIGHTARROW} B, f(arg) {\IS} $\mathcal{E}$({\DOTDOTDOT}{\LBRACKET}f{\RBRACKET}{\DOTDOTDOT})
\ep

\noindent
that is: the value is a function, $f$, whose \sfsl{signature} gives
the name, \textsf{f}, and the type of the functionality \textsf{A {\RIGHTARROW}
  B} (or \textsf{A {\PARRIGHTARROW} B}).
 Most of this
  chapter will cover aspects of value expressions.
  
\item \sfsl{axiom:} Prefixed by the keyword (literal) \sort{axiom}, axiom
  specification units  serve to limit values.
  The general form of an axiom specification unit is:

%\RSLatex
%    axiom &$\mathcal{A}$&(...)
%\endRSLatex 
\bp
\>\>\kw{axiom} $\mathcal{A}$({\DOTDOTDOT})
\ep

\noindent
where \textsf{$\mathcal{A}$(...)} is some predicate expression over
(specification unit) defined quantities.
  For more on axiomatic expressions, see Sect.\,\ref{tseb.rsl.The RSL
  Predicate Calculus}.
  
\item \sfsl{variable:} Prefixed by the keyword (literal) \sort{variable}, variable
  specification units  introduce  distinct variable names.
  The general form of a variable specification unit is:

%\RSLatex
%    variable v:T := expression
%\endRSLatex 
\bp
\>\>\kw{variable} v:T :{\EQ} expression
\ep

\noindent
where \textsf{v} is ..., \textsf{T} is ..., and \textsf{expression} is
  a value expression.
  For variables, see Sect.\,\ref{tseb.rsl.Variables and Assignment}.
  
\item \sfsl{channel:} Prefixed by the keyword (literal) \sort{channel}, channel
  specification units introduce  distinct channel names. 

  The general form of a channel specification unit is:

%\RSLatex
%    channel { ch[{i,j}] | i,j:UI :- ... } M
%\endRSLatex 
\bp
\>\>\kw{channel} {\LBRACE} ch{\LBRACKET}{\LBRACE}i,j{\RBRACE}{\RBRACKET} {\BAR} i,j:UI {\RDOT} {\DOTDOTDOT} {\RBRACE} M
\ep

\noindent
where \textsf{ch} is a distinct, here channel, name,we are declaring
an array of channels.
\textsf{ch[$\{i,j\}$]} expresses that ,
$\{i,j\}$ ranges of so-called unique identifier indices of type
\textsf{UI}, and \textsf{M} is a type expression.
For more on
  channels, see Sect.\,\ref{tseb.rsl.Process Channels}. 
  
\end{itemize}
\noindent
This chapter will otherwise be conventionally structured.
}%%%%%%%%%%%%%%%%%%%%%%%%%%%%%%%%%%%%%%%%%%%%%%%%%%%%%%%%

\bbbbb{Types and Values}\label{tseb.rsl.Types}%chap2.tex.rsltext

\irsltxt{Types\pconindex{type} are, in general, set-like structures\footnote{We shall
  not, in this primer, go into details as to the mathematics of
  types.} of things, i.e., values\pconindex{value}, having common
characteristics.

A bunch of zero, one or more apples (type \sfsl{apples}) may thus form a [sub]set of type \sfsl{Belle de Boskoop} apples.
A bunch of zero, one or more  pears (type \sfsl{pears}) may thus form a [sub]set of type \sfsl{Concorde} pears.
A union of zero, one or more of these apples and pears then form a [sub]set of entities of type \sfsl{fruits}.}

\bbbb{Sort and Type Expressions}\label{tseb.rsl.Type Expressions}
\index{symbind}{typeea@\protect{\underline{\textbf{Type Expressions}}}}

\begynd
\pind Sort\pdefindex{sort!expression}\pconindex{expression!sort} and
      type expressions\pdefindex{type!expression}\pconindex{expression!type}
      are expressions whose values are types, that is, 
\pind possibly infinite set-like structures of values (of ``that'' type).
\afslut

\bbb{Atomic Types: Identifier Expressions and Type Values}\pdefindex{atomic!type}\pconindex{type!atomic}

\begynd
\pind Atomic types have (atomic) values.
\pind That is, values which we consider to have no proper constituent
      (sub-)values, 
\pind i.e., cannot, to us, be meaningfully ``taken apart''. 
\afslut

\pos{\texttt{RSL} has a number of [so-called] \emph{built-in} atomic
  types. They are expressed in terms of literal identifiers.
These are the \sort{Boolean}s\plitindex{\sort{Boolean}}, \sort{int}{eger}s\plitindex{\sort{Int}}, \sort{Nat}{ural
number}s\plitindex{\sort{Nat}}, \sort{Real}s\plitindex{\sort{Real}}, \sort{Char}{acters\plitindex{\sort{Char}}, and \sort{Text}{s.}}\plitindex{\sort{Text}} 
\sort{Text}s are free-form texts and are more general than just texts of \texttt{RSL-like}
formulas. \rsltext's will be introduced in Sect.\,\vref{chap2.tex.rsltext}.}{}

We shall not need the base types \sort{Char}acters, nor the general
type \sort{Text}s
for domain modelling in this primer.
They will be listed below, but not mentioned further.

The base types are:

\nrslframebox{Basic Types}{\pdefindex{basic!type}\pconindex{type!basic}{\pdefindex{basic!type!expression}}\pconindex{type!basic!expression}
%\RSLatex
%type
%  [1] Bool&\index{symbind}{typeeb@\kw{Bool}}&
%  [2] Int&\index{symbind}{typeeb@\kw{Int}}& 
%  [3] Nat&\index{symbind}{typeeb@\kw{Nat}}& 
%  [4] Real&\index{symbind}{typeeb@\kw{Real}}& 
%  [5] Char&\index{symbind}{typeeb@\kw{Char}}& 
%  [6] Text&\index{symbind}{typeeb@\kw{Text}}&
%\endRSLatex
\bp
\kw{type}\\
\>{\LBRACKET}1{\RBRACKET} \kw{Bool}\index{symbind}{typeeb@\kw{Bool}}\\
\>{\LBRACKET}2{\RBRACKET} \kw{Int}\index{symbind}{typeeb@\kw{Int}} \\
\>{\LBRACKET}3{\RBRACKET} \kw{Nat}\index{symbind}{typeeb@\kw{Nat}} \\
\>{\LBRACKET}4{\RBRACKET} \kw{Real}\index{symbind}{typeeb@\kw{Real}} \\
\>{\LBRACKET}5{\RBRACKET} \kw{Char}\index{symbind}{typeeb@\kw{Char}} \\
\>{\LBRACKET}6{\RBRACKET} \kw{Text}\index{symbind}{typeeb@\kw{Text}}
\ep
}

\begin{enumerate}
\item\label{tseb.rsl.1}
      The Boolean type of truth values \kw{false} and \kw{true}.\index{symbind}{aaab@\kw{false}}\index{symbind}{aaab@\kw{true}}\index{symbind}{logic@\kw{false}}\index{symbind}{logic@\kw{true}}
\item\label{tseb.rsl.2}
      The integer type on integers ..., --2, --1, 0, 1, 2, ... .
\item\label{tseb.rsl.3}
      The natural number type of positive integer values 0, 1, 2, ...
\item\label{tseb.rsl.4}
      The real number type of real values, i.e., values whose numerals
  can be written as an integer, followed by a period (``.''), followed
  by a natural number (the fraction).
\item\label{tseb.rsl.5}
      The character type of character values
  \textsf{{\PRIM}{\PRIM}a{\PRIM}{\PRIM},
    {\PRIM}{\PRIM}bbb{\PRIM}{\PRIM}, ...} 
\item\label{tseb.rsl.6}
      The text type of character string values
    \textsf{{\PRIM}{\PRIM}aa{\PRIM}{\PRIM},
    {\PRIM}{\PRIM}aaa{\PRIM}{\PRIM}, ..., 
    {\PRIM}{\PRIM}abc{\PRIM}{\PRIM}, ...}
\savei\end{enumerate}

\bbb{Composite Types: Expressions and Type Values}\label{Composite Types}
\hhhh\pdefindex{composite!type}\pconindex{type!composite}\hhhh\pdefindex{composite!type!expression}\pconindex{type!composite!expression} 

\begynd
\pind Composite types have composite values.
\pind That is, values which we consider to have proper constituent
      (sub-)values, 
\pind i.e., can, to us, be meaningfully ``taken apart''. 
\afslut

\pos{
From these one can form type expressions: finite sets, infinite sets, 
Cartesian products, lists, maps, etc. 

Let \textsf{A}, \textsf{B} and \textsf{C} be any type names or type
expressions, then these are the composite types, hence, type expressions:}{}

\nrslframebox{Composite Type Expressions}{
%\RSLatex
%  [7] A-set&\index{symbind}{typeeb@T\kw{-set}}&
%  [8] A-infset&\index{symbind}{typeeb@T\kw{-infset}}&
%  [9] A >< B >< ... >< C&\index{symbind}{typeeb@\protect{T$_1$ \TIMES\ T$_2$\ \TIMES\ ... \TIMES\ T$_n$}}\label{rsl[9]}\index{symbind}{typeeb@\protect{(T$_1$\TIMES T$_2$\TIMES ... \TIMES T$_n$)}}&
%  [10] A-list&\index{symbind}{typeeb@\protect{T$^{\ast}$}}\label{rsl[10]}& 
%  [11] A-inflist&\index{symbind}{typeeb@\protect{T$^{\omega}$}}\label{rsl[11]}&
%  [12] A -m-> B&\index{symbind}{typeeb@\protect{T$_i$}\protect{\MARROW}\protect{T$_j$}}&
%  [13] A -> B&\index{symbind}{typeeb@\protect{T$_i$\protect{\RIGHTARROW}{T$_j$}}}& 
%  [14] A -~-> B&\index{symbind}{typeeb@\protect{T$_i$\protect{\PARRIGHTARROW}{T$_j$}}}&
%  [15] A | B | ... | C&\index{symbind}{typeeb@\protect{T$_1\mid\ $T$_2\mid\ ...\mid\ $T$_1\mid\ $T$_n$}}&
%  [16] mk_id(sel_a:A,...,sel_b:B)&\index{symbind}{typeeb@\protect{mk\_id(s$_1$:T$_1$,s$_2$:T$_2$,...,s$_n$:T$_n$)}}&
%  [17] sel_a:A ... sel_b:B&\index{symbind}{typeeb@\protect{s$_1$:T$_1$\ s$_2$:T$_2$\ ...\ s$_n$:T$_n$}}&
%\endRSLatex
\bp
\>{\LBRACKET}7{\RBRACKET} A\kw{-set}\index{symbind}{typeeb@T\kw{-set}}\\
\>{\LBRACKET}8{\RBRACKET} A\kw{-infset}\index{symbind}{typeeb@T\kw{-infset}}\\
\>{\LBRACKET}9{\RBRACKET} A {\TIMES} B {\TIMES} {\DOTDOTDOT} {\TIMES} C\index{symbind}{typeeb@\protect{T$_1$ \TIMES\ T$_2$\ \TIMES\ ... \TIMES\ T$_n$}}\label{rsl[9]}\index{symbind}{typeeb@\protect{(T$_1$\TIMES T$_2$\TIMES ... \TIMES T$_n$)}}\\
\>{\LBRACKET}10{\RBRACKET} A$^{\ast}$\index{symbind}{typeeb@\protect{T$^{\ast}$}}\label{rsl[10]} \\
\>{\LBRACKET}11{\RBRACKET} A$^{\omega}$\index{symbind}{typeeb@\protect{T$^{\omega}$}}\label{rsl[11]}\\
\>{\LBRACKET}12{\RBRACKET} A {\MARROW} B\index{symbind}{typeeb@\protect{T$_i$}\protect{\MARROW}\protect{T$_j$}}\\
\>{\LBRACKET}13{\RBRACKET} A {\RIGHTARROW} B\index{symbind}{typeeb@\protect{T$_i$\protect{\RIGHTARROW}{T$_j$}}} \\
\>{\LBRACKET}14{\RBRACKET} A {\PARRIGHTARROW} B\index{symbind}{typeeb@\protect{T$_i$\protect{\PARRIGHTARROW}{T$_j$}}}\\
\>{\LBRACKET}15{\RBRACKET} A {\BAR} B {\BAR} {\DOTDOTDOT} {\BAR} C\index{symbind}{typeeb@\protect{T$_1\mid\ $T$_2\mid\ ...\mid\ $T$_1\mid\ $T$_n$}}\\
\>{\LBRACKET}16{\RBRACKET} mk\_id(sel\_a:A,{\DOTDOTDOT},sel\_b:B)\index{symbind}{typeeb@\protect{mk\_id(s$_1$:T$_1$,s$_2$:T$_2$,...,s$_n$:T$_n$)}}\\
\>{\LBRACKET}17{\RBRACKET} sel\_a:A {\DOTDOTDOT} sel\_b:B\index{symbind}{typeeb@\protect{s$_1$:T$_1$\ s$_2$:T$_2$\ ...\ s$_n$:T$_n$}}
\ep
}

\pt{
\noindent
The following are generic type expressions:

\begin{enumerate}\setei
\item\label{tseb.rsl.7}
      The set type of finite cardinality set values.
\item\label{tseb.rsl.8}
      The set type of infinite and finite cardinality set values.
\item\label{tseb.rsl.9}
      The Cartesian type of Cartesian values.
\item\label{tseb.rsl.10}
      The list type of finite length list values.
\item\label{tseb.rsl.11}
      The list type of infinite and finite length list values.
\item\label{tseb.rsl.12}
      The map type of finite definition set map values.
\item\label{tseb.rsl.13}
      The function type of total function values.
\item\label{tseb.rsl.14}
      The function type of partial function values.
\dbeat{%%%%%%%%%%%%%%%%%%%%%%\item\label{tseb.rsl.15}
      In \textsf{(A)} \textsf{A} is constrained to be:
\begin{itemize}
\item either a Cartesian \textsf{B {\TIMES} C {\TIMES} {\DOTDOTDOT}
      {\TIMES} D}, in which case it is identical to type expression
    kind 9,
\item or not to be the name of a built-in type (cf., 1--6)
      or of a type, in which case the parentheses serve as simple
      delimiters, e.g.,  \textsf{(A {\MARROW} B),} or
      \textsf{(A$^{\ast}$)\kw{-set},} or \textsf{(A\kw{-set})$^{\ast}$}, or
      \textsf{(A{\BAR}B) {\MARROW} (C{\BAR}D{\BAR}(E{\MARROW}F))}, etc.
\end{itemize}
}%%%%%%%%%%%%%%%%%%%%%%%%%%%%%%%%%%%%%%%%%%%%%%%%    
\item\label{tseb.rsl.16}
      The postulated disjoint union of types \textsf{A,} \textsf{B,}
  \ldots, and \textsf{C.} 
\item\label{tseb.rsl.17}
      The  record type of \textsf{mk\_id}-named record values
  \textsf{mk\_id(av,{\DOTDOTDOT},bv),} where \textsf{av,} \ldots,
  \textsf{bv,} are values of respective types. The distinct
  identifiers \textsf{sel\_a,} etc., designate selector functions.
\item\label{tseb.rsl.18}
      The record type of unnamed record values
  \textsf{(av,{\DOTDOTDOT},bv),} where \textsf{av,} \ldots,
  \textsf{bv,} are values of respective types. The distinct
  identifiers \textsf{sel\_a,} etc., designate selector functions.
\savei\end{enumerate}
}{}
\index{symbind}{typeea@\protect{\underline{\textbf{Type Expressions}}}}

\noindent
Section\,\vref{chap2.tex.rsltext} introduces the extended \texttt{RSL}
concepts of type name values and the type, $\mathbb{T}$, of type names.
  
\bbbb{Type Definitions}\label{tseb.rsl.Type Definitions}
\bbb{Sorts --- Abstract Types}\label{tseb.rsl.Sorts}

\begynd
\pind Types can be (abstract) sorts 
\pind in which case their structure is not specified:
\afslut

\nrslframebox{Sorts}{ 
%\RSLatex
%type
%  A, B, ..., C
%\endRSLatex
\bp
\kw{type}\\
\>A, B, {\DOTDOTDOT}, C
\ep
}
 
\bbb{Concrete Types}\hhhh
\index{symbind}{utypeec@\protect{\underline{\textbf{Type Definitions}}}|(}
 
\begynd
\pind Types can be concrete
\pind in which case the structure of the type is specified by type
      expressions:
\afslut

\nrslframebox{Type Definition}{\label{rsl.Type Definition}
%\RSLatex
%type
%  A = Type_expr&\index{symbind}{utypeed@T = Type\_Expr}&
%\endRSLatex
\bp
\kw{type}\\
\>A {\EQ} Type\_expr\index{symbind}{utypeed@T = Type\_Expr}
\ep
}
\mnewfoil

\noindent
\xrsltxt{\sort{RSL Example:} \brcolor{Sets.}
  \textsf{\textbf{Narrative:}}\index{pdefind}{sets} $H$ stand for the
  domain type of 
  street intersections -- we shall call then \textsf{hub}s, and let $L$ stand for the domain type of
  segments of streets between immediately neighboring hubs -- we shall
  call then \textsf{link}s.
  Then $Hs$ and $Ls$ are to designate the types of finite sets of
  zero, one or more hubs, respectively links. \textsf{\textbf{Formalisation:}}
%\RSLatex
%   type H, L, Hs=H-set, Ls=L-set &$\bullet$&
%\endRSLatex   
\bp
\>\ \kw{type} H, L, Hs{\EQ}H\kw{-set}, Ls{\EQ}L\kw{-set} $\bullet$
\ep
}

\noindent
\xrsltxt{ \sort{RSL Example:} \brcolor{Cartesians.}
  \textsf{\textbf{Narrative:}} \index{pdefind}{Cartesian} 
  Let $RN$ stand for the
  domain type of road nets consisting of hub aggregates, $HA$, and
  link aggregates, $LA$. Hub and link aggregates can be observed from
  road nets, and hub sets and link sets can be observed from hub,
  respectively link aggregates. 
\textsf{\textbf{Formalisation:}}
%\RSLatex
%   type RN = HA><LA, Hs, Ls
%   value obs_HA: RN->HA, obs_LA: RN- LA, obs_Hs: HA->Hs, obs_Ls: LA->Ls 
%\endRSLatex 
\bp
\>\ \kw{type} RN {\EQ} HA{\TIMES}LA, Hs, Ls\\
\>\ \kw{value} obs\_HA: RN{\RIGHTARROW}HA, obs\_LA: RN{\MINUS} LA, obs\_Hs: HA{\RIGHTARROW}Hs, obs\_Ls: LA{\RIGHTARROW}Ls 
\ep
\noindent
Observer functions, \textsf{obs\_...} are not further defined -- beyond their signatures.
They will (subsequently) be defined through axioms over their results\,$\bullet$}

\mnewfoil

\noindent 
\begynd
\pind \pos{Some s}{S}chematic type definitions\pos{ are}{}:
\afslut

\nrslframebox{Variety of Type Definitions}{
%\RSLatex
%[18]  Type_name = Type_expr /* without |&\,&s or subtypes */
%[19]  Type_name = Type_expr_1 | Type_expr_2 | ... | Type_expr_n
%[20]  Type_name == &\index{symbind}{utypeed@\protect{T==TE$_1$\,\protect{$\mid$}TE$_2$\,\protect{$\mid$}\,...\,\protect{$\mid$}\,TE$_n$}}&
%        mk_id_1(s_a1:Type_name_a1,...,s_ai:Type_name_ai) |  
%        ... | 
%        mk_id_n(s_z1:Type_name_z1,...,s_zk:Type_name_zk) 
%[21]  Type_name :: sel_a:Type_name_a  ...  sel_z:Type_name_z
%[22]  Type_name = {| v:Type_name' :- &$\mathcal{P}$&(v) |}&\index{symbind}{utypeed@\protect{T=\protect{$\{\mid$}v:T\protect{$'\bullet$}P(v)\protect{$\mid\}$}}}&
%\endRSLatex
\bp
{\LBRACKET}18{\RBRACKET}\ \ Type\_name {\EQ} Type\_expr {\LCOMMENT} without {\BAR}\,s or subtypes {\RCOMMENT}\\
{\LBRACKET}19{\RBRACKET}\ \ Type\_name {\EQ} Type\_expr\_1 {\BAR} Type\_expr\_2 {\BAR} {\DOTDOTDOT} {\BAR} Type\_expr\_n\\
{\LBRACKET}20{\RBRACKET}\ \ Type\_name {\EQ}{\EQ} \index{symbind}{utypeed@\protect{T==TE$_1$\,\protect{$\mid$}TE$_2$\,\protect{$\mid$}\,...\,\protect{$\mid$}\,TE$_n$}}\\
\>\>\>\>mk\_id\_1(s\_a1:Type\_name\_a1,{\DOTDOTDOT},s\_ai:Type\_name\_ai) {\BAR}\ \ \\
\>\>\>\>{\DOTDOTDOT} {\BAR} \\
\>\>\>\>mk\_id\_n(s\_z1:Type\_name\_z1,{\DOTDOTDOT},s\_zk:Type\_name\_zk) \\
{\LBRACKET}21{\RBRACKET}\ \ Type\_name :: sel\_a:Type\_name\_a\ \ {\DOTDOTDOT}\ \ sel\_z:Type\_name\_z\\
{\LBRACKET}22{\RBRACKET}\ \ Type\_name {\EQ} {\LBRACE}{\BAR} v:Type\_name{\PRIM} {\RDOT} $\mathcal{P}$(v) {\BAR}{\RBRACE}\index{symbind}{utypeed@\protect{T=\protect{$\{\mid$}v:T\protect{$'\bullet$}P(v)\protect{$\mid\}$}}}
\ep
}

\mnewfoil
\noindent 
\begynd
\pind where a form of [19--20] is provided by combining the types:
\afslut

\nrslframebox{Record Types}{\label{tseb.rsl.Record Types}
%\RSLatex
%[23]  Type_name = A | B | ... | Z
%[24]  A == mk_id_1(s_a1:A_1,...,s_ai:A_i) 
%[25]  B == mk_id_2(s_b1:B_1,...,s_bj:B_j) 
%[26]  ...  
%[27]  Z == mk_id_n(s_z1:Z_1,...,s_zk:Z_k)
%\endRSLatex
\bp
{\LBRACKET}23{\RBRACKET}\ \ Type\_name {\EQ} A {\BAR} B {\BAR} {\DOTDOTDOT} {\BAR} Z\\
{\LBRACKET}24{\RBRACKET}\ \ A {\EQ}{\EQ} mk\_id\_1(s\_a1:A\_1,{\DOTDOTDOT},s\_ai:A\_i) \\
{\LBRACKET}25{\RBRACKET}\ \ B {\EQ}{\EQ} mk\_id\_2(s\_b1:B\_1,{\DOTDOTDOT},s\_bj:B\_j) \\
{\LBRACKET}26{\RBRACKET}\ \ {\DOTDOTDOT}\ \ \\
{\LBRACKET}27{\RBRACKET}\ \ Z {\EQ}{\EQ} mk\_id\_n(s\_z1:Z\_1,{\DOTDOTDOT},s\_zk:Z\_k)
\ep
\noindent
Of these we shall almost exclusively make use of [23--27].
\mnewfoil

\xrsltxt{\sort{Disjoint Types.}
  \textsf{\textbf{Narrative:}}\index{pdefind}{disjoint!types}
A pipeline consists of a finite set of zero, one or more [interconnected]\footnote{\LLLL
  Although interconnected we shall not model pipelines as lists.} pipe
units. Pipe units are either wells, or are pumps, or are valves, or
are joins, or are forks, or are sinks.
\textsf{\textbf{Formalisation:}} 
%\RSLatex
%type PL = P-set, P == WU|PU|VA|JO|FO|SI, Wu,Pu,Vu,Ju,Fu,Su
%   &\,\,\,&WU::mkWU(swu:Wu), PU::mkPU(spu:Pu), VA::mkVU(svu:Vu), 
%   &\,\,\,&JO::mkJu(sju:Ju), FO::mkFu(sfu:Fu), SI::mkSi(ssu:Su)
%\endRSLatex 
\bp
\kw{type} PL {\EQ} P\kw{-set}, P {\EQ}{\EQ} WU{\BAR}PU{\BAR}VA{\BAR}JO{\BAR}FO{\BAR}SI, Wu,Pu,Vu,Ju,Fu,Su\\
\>\ \,\,\,WU::mkWU(swu:Wu), PU::mkPU(spu:Pu), VA::mkVU(svu:Vu), \\
\>\ \,\,\,JO::mkJu(sju:Ju), FO::mkFu(sfu:Fu), SI::mkSi(ssu:Su)
\ep
\noindent
where we leave types \textsf{Wu, Pu, Vu, Ju, Fu} and \textsf{Su} further undefined\,$\bullet$%
}%%%%%%%%%%%%%%%%%%%%%%%%%%%%%

\mnewfoil

\pos{
\noindent
Types \textsf{A, B, ..., Z} are disjoint, i.e., shares no values,
provided all \textsf{mk\_id\_k} are distinct and due to the use of the
disjoint record type constructor \textsf{{\EQ}{\EQ}}.}{}

%\RSLatex
%axiom
%  all a1:A_1, a2:A_2, ..., ai:Ai :-
%    s_a1(mk_id_1(a1,a2,...,ai))=a1 /\ s_a2(mk_id_1(a1,a2,...,ai))=a2 /\
%    ... /\ s_ai(mk_id_1(a1,a2,...,ai))=ai /\ 
%  all a:A :- let mk_id_1(a1',a2',...,ai') = a in 
%    a1' = s_a1(a) /\ a2' = s_a2(a) /\ ... /\ ai' = s_ai(a) end 
%\endRSLatex
\bp
\kw{axiom}\\
\>{\ALL} a1:A\_1, a2:A\_2, {\DOTDOTDOT}, ai:Ai {\RDOT}\\
\>\>s\_a1(mk\_id\_1(a1,a2,{\DOTDOTDOT},ai)){\EQ}a1 {\WEDGE} s\_a2(mk\_id\_1(a1,a2,{\DOTDOTDOT},ai)){\EQ}a2 {\WEDGE}\\
\>\>{\DOTDOTDOT} {\WEDGE} s\_ai(mk\_id\_1(a1,a2,{\DOTDOTDOT},ai)){\EQ}ai {\WEDGE} \\
\>{\ALL} a:A {\RDOT} \kw{let} mk\_id\_1(a1{\PRIM},a2{\PRIM},{\DOTDOTDOT},ai{\PRIM}) {\EQ} a \kw{in} \\
\>\>a1{\PRIM} {\EQ} s\_a1(a) {\WEDGE} a2{\PRIM} {\EQ} s\_a2(a) {\WEDGE} {\DOTDOTDOT} {\WEDGE} ai{\PRIM} {\EQ} s\_ai(a) \kw{end} 
\ep
}

\noindent
\sort{Note:}\label{note-on-mk-type}
\begynd
\pind Values of type \textsf{A},
\pind where that type is defined by \textsf{A::B{\TIMES}C{\TIMES}D},
\pind can be expressed \textsf{A(b,c,d)} for \textsf{b:B, c:D, d:D}.
\afslut

\bbb{Subtypes}\label{tseb.rsl.Subtypes}%\hhhh

\begynd
\pind In {\dbrsl}, each type represents a set of values. Such a set can be
      delimited by means of predicates.
\pind The set of values \textsf{b} which have type \textsf{B} and which
      satisfy the predicate $\mathcal{P}$, constitute the subtype A: 
\afslut

\nrslframebox{Subtypes}{
%\RSLatex
%type
%  A = {| b:B :- &$\mathcal{P}$&(b) |}&\index{symbind}{utypeed@\protect{T=\protect{$\{\mid$}v:T\protect{$'\bullet$}P(v)\protect{$\mid\}$}}}&
%\endRSLatex
\bp
\kw{type}\\
\>A {\EQ} {\LBRACE}{\BAR} b:B {\RDOT} $\mathcal{P}$(b) {\BAR}{\RBRACE}\index{symbind}{utypeed@\protect{T=\protect{$\{\mid$}v:T\protect{$'\bullet$}P(v)\protect{$\mid\}$}}}
\ep
}

\noindent
\xrsltxt{\sort{Subtype.} \textsf{\textbf{Narrative:}}\index{pdefind}{subtype} The subtype of even
                                  natural numbers.

\textsf{\textbf{Formalisation:}} \sf \kw{type} ENat {\EQ}
{\LBRACE}{\BAR} en {\BAR} en:\kw{Nat} {\RDOT}
is\_even\_natural\_number(en) {\BAR}{\RBRACE}\,\rm$\bullet$}%%%%%%%%%%%%%%%%%%%%%%%%%%%%%%%%%%%%

\label{tseb.rsl.Types.n} 

\index{symbind}{utypeec@\protect{\underline{\textbf{Type Definitions}}}|)}
\index{symbind}{loga@\protect{\underline{\textbf{Logic Constructs}}}|(}

\nbbbbb{The Propositional and  Predicate Calculi}\label{tseb.rsl.The RSL
  Predicate Calculus} 
\bbbb{Propositions}\label{tseb.rsl.Propositional Expressions}

\irsltxt{In logic, a proposition\pdefindex{proposition} is the meaning of a declarative
  sentence. [A declarative sentence\pdefindex{declarative sentence} is a type of sentence that makes a statement]}

\bbb{Propositional Expressions}

\irsltxt{Propositional expressions\pdefindex{propositional!expression}, informally speaking, are
  quantifier-free expressions having truth (or \sort{chaos})
  values. {\ALL}, {\EXISTS} and {\EXISTS\,!} are quantifiers\pconindex{quantifier}, see below.
  
Below, we will first treat propositional expressions all of whose
identifiers denote truth values. As we progress, in sections on
\textsf{arithmetic, sets, list, maps}, etc., we shall extend the range of
propositional expressions}

\begynd
\pind Let identifiers (or propositional expressions) \textsf{a, b, ..., c}
      designate Boolean values (\kw{true} or \kw{false} [or \kw{chaos}]). 
\pind Then: 
\afslut

\nrslframebox{Propositional Expressions}{
%\RSLatex
%  false, true&\index{symbind}{logic@\kw{false}}\index{symbind}{aaab@\kw{false}}\index{symbind}{aaab@\kw{true}}\index{symbind}{logic@\kw{true}}&
%  a, b, ..., c ~a, a/\b, a\/b, a=>b, a=b, a~=b&\index{symbind}{logic@\protect{\SIM\ b}}\index{symbind}{logic@\protect{b$_i$ \WEDGE\ b$_j$}}\index{symbind}{logic@\protect{$b_i$ \VEE\ b$_j$}}\index{symbind}{logic@\protect{b$_i$ \DBLRIGHTARROW\ b$_j$}}&
%\endRSLatex
\bp
\>\kw{false}, \kw{true}\index{symbind}{logic@\kw{false}}\index{symbind}{aaab@\kw{false}}\index{symbind}{aaab@\kw{true}}\index{symbind}{logic@\kw{true}}\\
\>a, b, {\DOTDOTDOT}, c {\SIM}a, a{\WEDGE}b, a{\VEE}b, a{\DBLRIGHTARROW}b, a{\EQ}b, a{\NOTEQ}b\index{symbind}{logic@\protect{\SIM\ b}}\index{symbind}{logic@\protect{b$_i$ \WEDGE\ b$_j$}}\index{symbind}{logic@\protect{$b_i$ \VEE\ b$_j$}}\index{symbind}{logic@\protect{b$_i$ \DBLRIGHTARROW\ b$_j$}}
\ep
}

\noindent 
\begynd
\pind are propositional expressions having Boolean values.
\pind {\SIM}, \index{symbind}{booleanop@\protect{{\SIM}}}
      {\WEDGE}, \index{symbind}{booleanop@\protect{{\WEDGE}}}
      {\VEE}, \index{symbind}{booleanop@\protect{{\VEE}}}
      {\DBLRIGHTARROW}, \index{symbind}{booleanop@\protect{{\DBLRIGHTARROW}}}
      {\EQ}, \index{symbind}{booleanop@\protect{{\EQ}}}
      {\NOTEQ} and \index{symbind}{booleanop@\protect{{\NOTEQ}}}
      {\ALWAYS} \index{symbind}{booleanop@\protect{{\ALWAYS}}}
      are Boolean connectives (i.e., operators). 
\pind They can be read as: \sort{not}, \sort{and},
      \sort{or, if then} (or \sort{implies}), \sort{equal,
      not equal} and \sort{always}. 
\afslut

\nbbb{Propositional Calculus}

\irsltxt{Propositional calculus\pdefindex{propositional!calculus} is a branch of logic. It is also
  called propositional logic, statement logic, sentential calculus,
  sentential logic, or sometimes zeroth-order logic. It deals with
  propositions (which can be true or false) and relations between
  propositions, including the construction of arguments based on
  them. Compound propositions are formed by connecting propositions by
  logical connectives. Propositions that contain no logical
  connectives are called atomic propositions \wiki} 

\noindent
A simple two-value Boolean logic can be defined as follows:

%\RSLatex
%  type
%    Bool
%  value
%    true, false
%    ~: Bool -> Bool
%    /\, \/, =>, =, ~=, is: Bool >< Bool -> Bool
%  axiom &\label{ch2.logic.algebra.axioms}&
%    all b,b':Bool :-
%      ~b is if b then false else true end
%      b /\ b' is if b then b' else false end
%      b \/ b' is if b then true else b' end
%      b => b' is if b then b' else true end
%      b = b' is if (b/\b')\/(~b/\~b') then true else false end
%      (b ~= b') is ~(b = b')
%      (b is b') is (b = b')
%\endRSLatex
\bp
\>\kw{type}\\
\>\>\kw{Bool}\\
\>\kw{value}\\
\>\>\kw{true}, \kw{false}\\
\>\>{\SIM}: \kw{Bool} {\RIGHTARROW} \kw{Bool}\\
\>\>{\WEDGE}, {\VEE}, {\DBLRIGHTARROW}, {\EQ}, {\NOTEQ}, {\IS}: \kw{Bool} {\TIMES} \kw{Bool} {\RIGHTARROW} \kw{Bool}\\
\>\kw{axiom} \label{ch2.logic.algebra.axioms}\\
\>\>{\ALL} b,b{\PRIM}:\kw{Bool} {\RDOT}\\
\>\>\>{\SIM}b {\IS} \kw{if} b \kw{then} \kw{false} \kw{else} \kw{true} \kw{end}\\
\>\>\>b {\WEDGE} b{\PRIM} {\IS} \kw{if} b \kw{then} b{\PRIM} \kw{else} \kw{false} \kw{end}\\
\>\>\>b {\VEE} b{\PRIM} {\IS} \kw{if} b \kw{then} \kw{true} \kw{else} b{\PRIM} \kw{end}\\
\>\>\>b {\DBLRIGHTARROW} b{\PRIM} {\IS} \kw{if} b \kw{then} b{\PRIM} \kw{else} \kw{true} \kw{end}\\
\>\>\>b {\EQ} b{\PRIM} {\IS} \kw{if} (b{\WEDGE}b{\PRIM}){\VEE}({\SIM}b{\WEDGE}{\SIM}b{\PRIM}) \kw{then} \kw{true} \kw{else} \kw{false} \kw{end}\\
\>\>\>(b {\NOTEQ} b{\PRIM}) {\IS} {\SIM}(b {\EQ} b{\PRIM})\\
\>\>\>(b {\IS} b{\PRIM}) {\IS} (b {\EQ} b{\PRIM})
\ep

\noindent
\begynd
\pind We shall, however, make use of a three-value Boolean logic.
\pind The model-theory explanation of the meaning of propositional expressions 
\begynd
\pind is now given in terms of the \sfsl{truth tables} for the logic connectives:
\afslut
\afslut
\begin{center}
  \textbf{$\vee, \wedge,$ and \DBLRIGHTARROW\ Syntactic Truth Tables}
  
\vspace{2mm}
\begin{tabular}{|l||l|l|l|}\hline \VEE
           & \kw{true}& \kw{false}& \kw{chaos}\\ \hline\hline
\kw{true}  & \kw{true} & \kw{true} & \kw{true} \\ \hline  
\kw{false} & \kw{true} & \kw{false} & \kw{chaos} \\ \hline
\kw{chaos} & \kw{chaos} & \kw{chaos} & \kw{chaos} \\ \hline 
\end{tabular}\ \ \
\begin{tabular}{|l||l|l|l|}\hline \WEDGE
           & \kw{true}& \kw{false}& \kw{chaos}\\ \hline\hline  
\kw{true}  & \kw{true} & \kw{false} & \kw{chaos} \\ \hline  
\kw{false} & \kw{false} & \kw{false} & \kw{false} \\ \hline
\kw{chaos} & \kw{chaos} & \kw{chaos} & \kw{chaos} \\ \hline
\end{tabular}\ \ \ \label{rsl-truth tables}
\begin{tabular}{|l||l|l|l|}\hline \DBLRIGHTARROW
           & \kw{true}& \kw{false}& \kw{chaos}\\ \hline\hline
\kw{true}  & \kw{true} & \kw{false} & \kw{chaos} \\ \hline  
\kw{false} & \kw{true} & \kw{true} & \kw{true} \\ \hline
\kw{chaos} & \kw{chaos} & \kw{chaos} & \kw{chaos} \\ \hline
\end{tabular}
\end{center}
\noindent
The two-value logic defined earlier `transpires' from the
\kw{true,false} columns and rows of the above truth tables.
 
\mnewfoil
\bbbb{Predicates}\label{tseb.rsl.Simple Predicate
  Expressions}\hhhh

\irsltxt{Predicates are mathematical assertions that contains
  variables, sometimes referred to as predicate variables, and may be
  true or false depending on those variables' value or
  values\footnote{\LLLL
  https://calcworkshop.com/logic/predicate-logic/, and: predicate
  logic, first-order logic or quantified logic is a formal language in
  which propositions are expressed in terms of predicates, variables
  and quantifiers. It is different from propositional logic which
  lacks quantifiers https://brilliant.org/wiki/predicate-logic/.}}  

\bbb{Predicate Expressions}

\noindent
\begynd
\pind Let \textsf{x, y, ..., z} (or term expressions) designate non-Boolean
      values,
\pind and let $\mathcal{P}(x)$, $\mathcal{Q}(y)$ and $\mathcal{R}(z)$
      be propositional or predicate expressions, then:
\afslut

\nrslframebox{Simple Predicate Expressions}{
%\RSLatex
%[28]  all&x&:X :- &$\mathcal{P}(x)$\index{symbind}{logic@\protect{\ALL\ a:A $\bullet$ P(a)}}& 
%[29]  exists&y&:Y :- &$\mathcal{Q}(y)$\index{symbind}{logic@\protect{\EXISTS\ a:A $\bullet$ P(a)}}& 
%[30]  exists!&z&:Z :- &$\mathcal{R}(z)$\index{symbind}{logic@\protect{\EXISTS "!\ a:A $\bullet$ P(a)}}&
%\endRSLatex
\bp
{\LBRACKET}28{\RBRACKET}\ \ {\ALL}x:X {\RDOT} $\mathcal{P}(x)$\index{symbind}{logic@\protect{\ALL\ a:A $\bullet$ P(a)}} \\
{\LBRACKET}29{\RBRACKET}\ \ {\EXISTS}y:Y {\RDOT} $\mathcal{Q}(y)$\index{symbind}{logic@\protect{\EXISTS\ a:A $\bullet$ P(a)}} \\
{\LBRACKET}30{\RBRACKET}\ \ {\EXISTS}!z:Z {\RDOT} $\mathcal{R}(z)$\index{symbind}{logic@\protect{\EXISTS "!\ a:A $\bullet$ P(a)}}
\ep
}

\noindent 
\begynd
\pind are quantified, i.e., predicate expressions. 
\pind {\ALL}, {\EXISTS} and {\EXISTS\,!} are the quantifiers.
\afslut

\nbbb{Predicate Calculus}

\pt{They are ``read'' as:
  
  [28] For all $x$ (values in type $X$) the predicate $\mathcal{P}(x)$ holds -- 
       if that is not the case the expression yields truth value \sort{false}. 

  [29] There exists (at least) one $y$ (value in type $Y$) such that
       the predicate $\mathcal{Q}(y)$ holds -- 
       if that is not the case the expression yields truth value \sort{false}.

  [30] There exists a unique $z$ (value in type $Z$) such that the
       predicate $\mathcal{R}(z)$ holds -- 
       if that is not the case the expression yields truth value \sort{false}.}{}
 
  [28--30]   The predicates $\mathcal{P}(x)$, $\mathcal{Q}(y)$ or
     $\mathcal{R}(z)$ may yield \sort{chaos} in which case the whole
     expression yields  \sort{chaos}.
\index{symbind}{loga@\protect{\underline{\textbf{Logic Constructs}}}|)}

%\nbbbbb{Concrete \dbrsl\ Types: Values and Operations}
%\label{tseb.rsl.Concrete RSL Types}
\bbbbb{Arithmetics}
\irsltxt{\texttt{RSL} offers the usual set of arithmetic
  operators. From these the usual kind of arithmetic expressions can
  be formed.}
\label{tseb.rsl.Artithmetic}
\index{symbind}{baritha@\protect{\underline{\textbf{Arithmetic Constructs}}}|(}

\nrslframebox{Arithmetic}{
%\RSLatex
%type
%  Nat, Int, Real
%value
%  +,-,*: Nat><Nat->Nat | Int><Int->Int | Real><Real->Real
%  /: Nat><Nat-~->Nat | Int><Int-~->Int | Real><Real-~->Real
%  <,<=,=,~=,>=,> (Nat|Int|Real) -> (Nat|Int|Real)
%\endRSLatex
\bp
\kw{type}\\
\>\kw{Nat}, \kw{Int}, \kw{Real}\\
\kw{value}\\
\>{\PLUS},{\MINUS},{\AST}: \kw{Nat}{\TIMES}\kw{Nat}{\RIGHTARROW}\kw{Nat} {\BAR} \kw{Int}{\TIMES}\kw{Int}{\RIGHTARROW}\kw{Int} {\BAR} \kw{Real}{\TIMES}\kw{Real}{\RIGHTARROW}\kw{Real}\\
\>/: \kw{Nat}{\TIMES}\kw{Nat}{\PARRIGHTARROW}\kw{Nat} {\BAR} \kw{Int}{\TIMES}\kw{Int}{\PARRIGHTARROW}\kw{Int} {\BAR} \kw{Real}{\TIMES}\kw{Real}{\PARRIGHTARROW}\kw{Real}\\
\>{\LT},{\LEQ},{\EQ},{\NOTEQ},{\GEQ},{\GT} (\kw{Nat}{\BAR}\kw{Int}{\BAR}\kw{Real}) {\RIGHTARROW} (\kw{Nat}{\BAR}\kw{Int}{\BAR}\kw{Real})
\ep
}
\index{symbind}{barithb@\protect{a$_i$+a$_j$}}%
\index{symbind}{barithb@\protect{a$_i-$a$_j$}}%
\index{symbind}{barithb@\protect{a$_i$*a$_j$}}%
\index{symbind}{barithb@\protect{a$_i$/a$_j$}}%
\index{symbind}{barithb@\protect{a$_i$\protect{\LT}a$_j$}}%
\index{symbind}{barithb@\protect{a$_i$\protect{\LEQ}a$_j$}}%
\index{symbind}{barithb@\protect{a$_i$\protect{\EQ}a$_j$}}%
\index{symbind}{barithb@\protect{a$_i$\protect{\NOTEQ}a$_j$}}%
\index{symbind}{barithb@\protect{a$_i$\protect{\GEQ}a$_j$}}%
\index{symbind}{barithb@\protect{a$_i$\protect{\GT}a$_j$}}%
\index{symbind}{baritha@\protect{\underline{\textbf{Arithmetic Constructs}}}|)}

\mnewfoil
\bbbbb{Comprehensive Expressions}\label{mono-rsl.Set Expressions}
\label{tseb.rsl.Set Enumerations}
\index{symbind}{sea@\protect{\underline{\textbf{Set Constructs}}}|(}
\irsltxt{Comprehensive expressions are common in mathematics
  texts. They capture properties conveniently abstractly}

\nbbbb{Set Enumeration and Comprehension}\hhhh

\nbbb{Set Enumeration}\hhhh


%\pt
{Let the below $a$'s denote values of type $A$:}{} 

\nrslframebox{Set Enumerations}{
%\RSLatex
%  {{}, {a}, {e&$_1$&,e&$_2$&,...,e&$_n$&}, ...} isin A-set&\index{symbind}{seta@\protect{$\{\}$}}\index{symbind}{setb@\protect{$\{e_1,e_2,...,e_n\}$}}&
%  {{}, {a}, {e&$_1$&,e&$_2$&,...,e&$_n$&}, ..., {e&$_1$&,e&$_2$&,...}} isin A-infset
%\endRSLatex
\bp
\>{\LBRACE}{\LBRACE}{\RBRACE}, {\LBRACE}a{\RBRACE}, {\LBRACE}e$_1$,e$_2$,{\DOTDOTDOT},e$_n${\RBRACE}, {\DOTDOTDOT}{\RBRACE} {\ISIN} A\kw{-set}\index{symbind}{seta@\protect{$\{\}$}}\index{symbind}{setb@\protect{$\{e_1,e_2,...,e_n\}$}}\\
\>{\LBRACE}{\LBRACE}{\RBRACE}, {\LBRACE}a{\RBRACE}, {\LBRACE}e$_1$,e$_2$,{\DOTDOTDOT},e$_n${\RBRACE}, {\DOTDOTDOT}, {\LBRACE}e$_1$,e$_2$,{\DOTDOTDOT}{\RBRACE}{\RBRACE} {\ISIN} A\kw{-infset}
\ep
}

\nbbb{Set Comprehension}\hhhh
\noindent 
\begynd
\pind The expression, last line below, to the right of the \IS, expresses
      set comprehension. 
\pind The expression ``builds'' the set of values
      satisfying the given predicate. 
\pind It is abstract in the sense
      that it does not do so by following a concrete algorithm.
\afslut

\nrslframebox{Set Comprehension}{\label{tseb.rsl.Set Comprehension}
%\RSLatex
%type
%  A, B
%  P = A -> Bool
%  Q = A -~-> B
%value
%  comprehend: A-infset >< P >< Q -> B-infset
%  comprehend(s,P,Q) is { Q(a) | a:A :- a isin s /\ P(a)}&\index{symbind}{setc@\protect{$\{${Q}(a){$\mid$}a:A{$\bullet$}a{$\in$}s{$\wedge$}{P}(a)$\}$}}&
%\endRSLatex
\bp
\kw{type}\\
\>A, B\\
\>P {\EQ} A {\RIGHTARROW} \kw{Bool}\\
\>Q {\EQ} A {\PARRIGHTARROW} B\\
\kw{value}\\
\>comprehend: A\kw{-infset} {\TIMES} P {\TIMES} Q {\RIGHTARROW} B\kw{-infset}\\
\>comprehend(s,P,Q) {\IS} {\LBRACE} Q(a) {\BAR} a:A {\RDOT} a {\ISIN} s {\WEDGE} P(a){\RBRACE}\index{symbind}{setc@\protect{$\{${Q}(a){$\mid$}a:A{$\bullet$}a{$\in$}s{$\wedge$}{P}(a)$\}$}}
\ep
}
\index{symbind}{sea@\protect{\underline{\textbf{Set Constructs}}}|)}

%\nbbbb{Cartesian Expressions}
\bbb{Cartesian Enumeration}\label{rsl.Cartesian Enumerations}
\label{tseb.rsl.Cartesian Enumerations}
\index{symbind}{cartesiaa@\protect{\underline{\textbf{Cartesian Constructs}}}|(}
\begynd
\pind Let $e$ range over values of Cartesian types involving $A$, $B$,
      $\ldots$, $C$,
\pind then the below expressions are simple Cartesian enumerations: 
\afslut

\nrslframebox{Cartesian Enumerations}{
%\RSLatex
%type
%  A, B, ..., C
%  A >< B >< ... >< C
%value
%  (e1,e2,...,en)&\index{symbind}{cartesian@\protect{(e$_1$,e$_2$,...,e$_n$)}}&
%\endRSLatex
\bp
\kw{type}\\
\>A, B, {\DOTDOTDOT}, C\\
\>A {\TIMES} B {\TIMES} {\DOTDOTDOT} {\TIMES} C\\
\kw{value}\\
\>(e1,e2,{\DOTDOTDOT},en)\index{symbind}{cartesian@\protect{(e$_1$,e$_2$,...,e$_n$)}}
\ep
}
\index{symbind}{cartesiaa@\protect{\underline{\textbf{Cartesian Constructs}}}|)}

%\bbbb{List Expressions}\label{rsl.List Enumerations}\hhhh
\bbbb{List Enumeration and Comprehension}\hhhh\label{rsl.List Enumerations}
\label{tseb.rsl.List Enumerations}

\bbb{List Enumeration}
\index{symbind}{klista@\protect{\underline{\textbf{List Constructs}}}|(}

\begynd
\pind Let $a$ range over values of type $A$,
\pind then the below expressions are simple list enumerations:
\afslut

\nrslframebox{List Enumerations}{
%\RSLatex
%  {<..>, <.e.>, ..., <.e1,e2,...,en.>, ...} isin A-list&\index{symbind}{klistb@\protect{e$_{1}<$e$_2$,e$_2$,...,e$_{n}>$}}\index{symbind}{klistb@\protect{$<>$}}&
%  {<..>, <.e.>, ..., <.e1,e2,...,en.>, ..., <.e1,e2,...,en,... .>, ...} isin A-inflist
%
%  <. a$_i$ .. a$_j$ .>&\index{symbin}{klistb@\protect{{\LANGLE}ei{\DOTDOT}ej{\RANGLE}}}&
%\endRSLatex
\bp
\>{\LBRACE}{\LANGLE}{\RANGLE}, {\LANGLE}e{\RANGLE}, {\DOTDOTDOT}, {\LANGLE}e1,e2,{\DOTDOTDOT},en{\RANGLE}, {\DOTDOTDOT}{\RBRACE} {\ISIN} A$^{\ast}$\index{symbind}{klistb@\protect{e$_{1}<$e$_2$,e$_2$,...,e$_{n}>$}}\index{symbind}{klistb@\protect{$<>$}}\\
\>{\LBRACE}{\LANGLE}{\RANGLE}, {\LANGLE}e{\RANGLE}, {\DOTDOTDOT}, {\LANGLE}e1,e2,{\DOTDOTDOT},en{\RANGLE}, {\DOTDOTDOT}, {\LANGLE}e1,e2,{\DOTDOTDOT},en,{\DOTDOTDOT} {\RANGLE}, {\DOTDOTDOT}{\RBRACE} {\ISIN} A$^{\omega}$\\
\\
\>{\LANGLE} a$\_i$ {\DOTDOT} a$\_j$ {\RANGLE}\index{symbin}{klistb@\protect{{\LANGLE}ei{\DOTDOT}ej{\RANGLE}}}
\ep
}

\noindent 
\begynd
\pind The last line above assumes $a_i$ and $a_j$ to be integer-valued
      expressions. 
\pind It then expresses the set of integers from the value of
      $e_i$ to and including the value of $e_j$. 
\pind If the latter is smaller than the former, then the list is empty.
\afslut

\nbbb{List Comprehension}\hhhh

\begynd
\pind The last line below expresses list comprehension.
\afslut

\nrslframebox{List Comprehension}{\label{tseb.rsl.List Comprehension}
%\RSLatex
%type
%  A, B, P = A -> Bool, Q = A -~-> B
%value
%  comprehend: A-inflist >< P >< Q -~-> B-inflist
%  comprehend(l,P,Q) is <. Q(l(i)) | i in <.1..len l.> :- P(l(i)).>&\index{symbind}{klistb@\protect{$<$Q(l(i))$\mid$i \kw{in}$<$1..\kw{len}l$>\bullet$P(a)$>$}}&
%\endRSLatex
\bp
\kw{type}\\
\>A, B, P {\EQ} A {\RIGHTARROW} \kw{Bool}, Q {\EQ} A {\PARRIGHTARROW} B\\
\kw{value}\\
\>comprehend: A$^{\omega}$ {\TIMES} P {\TIMES} Q {\PARRIGHTARROW} B$^{\omega}$\\
\>comprehend(l,P,Q) {\IS} {\LANGLE} Q(l(i)) {\BAR} i \kw{in} {\LANGLE}1{\DOTDOT}\kw{len} l{\RANGLE} {\RDOT} P(l(i)){\RANGLE}\index{symbind}{klistb@\protect{$<$Q(l(i))$\mid$i \kw{in}$<$1..\kw{len}l$>\bullet$P(a)$>$}}
\ep
}
\index{symbind}{klista@\protect{\underline{\textbf{List Constructs}}}|)}

\nbbbb{Map Enumeration and Comprehension}
\label{tseb.rsl.Map Enumerations}
\index{symbind}{maa@\protect{\underline{\textbf{Map Constructs}}}|(}
\nbbb{Map Enumeration}

\begynd
\pind Let (possibly indexed) $u$ and $v$ range over values of type $T1$
      and $T2$, respectively, 
\pind then the below expressions are simple map enumerations: 
\afslut


\nrslframebox{Map Enumerations}{
%\RSLatex
%type 
%  T1, T2
%  M = T1 -m-> T2
%value
%  u,u1,u2,...,un:T1, v,v1,v2,...,vn:T2
%  [],  [u+>v],  ...,  [u1+>v1,u2+>v2,...,un+>vn] all isin M&\index{symbind}{mapa@\protect{[\,]}}\index{symbind}{mapb@\protect{[u$_1${\protect\MAPSTO}v$_1$,u$_2${\protect\MAPSTO}v$_2$,...,u$_n${\protect\MAPSTO}v$_n$]}}&
%\endRSLatex
\bp
\kw{type} \\
\>T1, T2\\
\>M {\EQ} T1 {\MARROW} T2\\
\kw{value}\\
\>u,u1,u2,{\DOTDOTDOT},un:T1, v,v1,v2,{\DOTDOTDOT},vn:T2\\
\>{\emptymap},\ \ {\LBRACKET}u{\MAPSTO}v{\RBRACKET},\ \ {\DOTDOTDOT},\ \ {\LBRACKET}u1{\MAPSTO}v1,u2{\MAPSTO}v2,{\DOTDOTDOT},un{\MAPSTO}vn{\RBRACKET} {\ALL} {\ISIN} M\index{symbind}{mapa@\protect{[\,]}}\index{symbind}{mapb@\protect{[u$_1${\protect\MAPSTO}v$_1$,u$_2${\protect\MAPSTO}v$_2$,...,u$_n${\protect\MAPSTO}v$_n$]}}
\ep
}

\nbbb{Map Comprehension}

\noindent
\begynd
\pind The last line below expresses map comprehension:
\afslut
 

\nrslframebox{Map Comprehension}{\label{tseb.rsl.Map Comprehension}
%\RSLatex
%type
%  U, V, X, Y
%  M = U -m-> V
%  F = U -~-> X
%  G = V -~-> Y
%  P = U -> Bool
%value
%  comprehend: M><F><G><P -> (X -m-> Y)
%  comprehend(m,F,G,P) is [ F(u) +> G(m(u)) | u:U :- u isin dom m /\ P(u)]&\index{symbind}{mapc@\protect{[F(e)\protect{\MAPSTO}G(m(e))$\mid$e:E$\bullet$e{\ISIN}\kw{dom}\,m{\WEDGE}P(e)]}}& 
%\endRSLatex
\bp
\kw{type}\\
\>U, V, X, Y\\
\>M {\EQ} U {\MARROW} V\\
\>F {\EQ} U {\PARRIGHTARROW} X\\
\>G {\EQ} V {\PARRIGHTARROW} Y\\
\>P {\EQ} U {\RIGHTARROW} \kw{Bool}\\
\kw{value}\\
\>comprehend: M{\TIMES}F{\TIMES}G{\TIMES}P {\RIGHTARROW} (X {\MARROW} Y)\\
\>comprehend(m,F,G,P) {\IS} {\LBRACKET} F(u) {\MAPSTO} G(m(u)) {\BAR} u:U {\RDOT} u {\ISIN} \kw{dom} m {\WEDGE} P(u){\RBRACKET}\index{symbind}{mapc@\protect{[F(e)\protect{\MAPSTO}G(m(e))$\mid$e:E$\bullet$e{\ISIN}\kw{dom}\,m{\WEDGE}P(e)]}} 
\ep
}
\index{symbind}{maa@\protect{\underline{\textbf{Map Constructs}}}|)}

\nbbbbb{Operations}

\nbbbb{Set Operations}\label{mono-rsl.Set Operations}\LLL
\bbb{Set Operator Signatures}\LLL
\index{symbind}{sea@\protect{\underline{\textbf{Set Constructs}}}|(}
\label{tseb.rsl.Set Operations}
\wrtrt{\volchapsec{1}{13}{2}}

\nrslframebox{Set Operator Signatures}{ %\begin{multicols}{2}
%\RSLatex
%value
%  &\ref{tseb.rsl.so1}&  isin: A >< A-infset -> Bool&\index{symbind}{set@\protect{e\ISIN{s}}}&
%  &\ref{tseb.rsl.so2}&  ~isin: A >< A-infset -> Bool&\index{symbind}{set@\protect{e\NOTISIN{s}}}&
%  &\ref{tseb.rsl.so3}&  union: A-infset >< A-infset -> A-infset&\index{symbind}{set@\protect{s$_i$\UNION{s$_j$}}}&
%  &\ref{tseb.rsl.so4}&  union: (A-infset)-infset -> A-infset&\index{symbind}{set@\protect{\UNION{$\{$s$_1$,s$_2$,...,s$_{n}\}$}}}&
%  &\ref{tseb.rsl.so5}&  inter: A-infset >< A-infset -> A-infset&\index{symbind}{set@\protect{s$_i$\INTER{s$_j$}}}&
%  &\ref{tseb.rsl.so6}&  inter: (A-infset)-infset -> A-infset&\index{symbind}{set@\protect{\INTER{$\{$s$_1$,s$_2$,...,s$_{n}\}$}}}&
%  &\ref{tseb.rsl.so7}&  \: A-infset >< A-infset -> A-infset&\index{symbind}{set@\protect{s$_{i}${\protect$\setminus$}{s$_j$}}}&
%  &\ref{tseb.rsl.so8}&  <<: A-infset >< A-infset -> Bool&\index{symbind}{set@\protect{s$_i$\SUBSET{s$_j$}}}&
%  &\ref{tseb.rsl.so9}&  <<=: A-infset >< A-infset -> Bool&\index{symbind}{set@\protect{s$_i$\SUBSETEQ{s$_j$}}}&
%  &\ref{tseb.rsl.so10}&  =: A-infset >< A-infset -> Bool&\index{symbind}{set@\protect{s$_i$\EQ{s$_j$}}}&
%  &\ref{tseb.rsl.so11}&  ~=: A-infset >< A-infset -> Bool&\index{symbind}{set@\protect{s$_i$\NOTEQ{s$_j$}}}&
%  &\ref{tseb.rsl.so12}&  card: A-infset -~-> Nat&\index{symbind}{set@\protect{\kw{card}\,s}}&
%\endRSLatex
\bp
\kw{value}\\
\>\ref{tseb.rsl.so1}\ \ {\ISIN}: A {\TIMES} A\kw{-infset} {\RIGHTARROW} \kw{Bool}\index{symbind}{set@\protect{e\ISIN{s}}}\\
\>\ref{tseb.rsl.so2}\ \ {\NOTISIN}: A {\TIMES} A\kw{-infset} {\RIGHTARROW} \kw{Bool}\index{symbind}{set@\protect{e\NOTISIN{s}}}\\
\>\ref{tseb.rsl.so3}\ \ {\UNION}: A\kw{-infset} {\TIMES} A\kw{-infset} {\RIGHTARROW} A\kw{-infset}\index{symbind}{set@\protect{s$_i$\UNION{s$_j$}}}\\
\>\ref{tseb.rsl.so4}\ \ {\UNION}: (A\kw{-infset})\kw{-infset} {\RIGHTARROW} A\kw{-infset}\index{symbind}{set@\protect{\UNION{$\{$s$_1$,s$_2$,...,s$_{n}\}$}}}\\
\>\ref{tseb.rsl.so5}\ \ {\INTER}: A\kw{-infset} {\TIMES} A\kw{-infset} {\RIGHTARROW} A\kw{-infset}\index{symbind}{set@\protect{s$_i$\INTER{s$_j$}}}\\
\>\ref{tseb.rsl.so6}\ \ {\INTER}: (A\kw{-infset})\kw{-infset} {\RIGHTARROW} A\kw{-infset}\index{symbind}{set@\protect{\INTER{$\{$s$_1$,s$_2$,...,s$_{n}\}$}}}\\
\>\ref{tseb.rsl.so7}\ \ {\SETMINUS}: A\kw{-infset} {\TIMES} A\kw{-infset} {\RIGHTARROW} A\kw{-infset}\index{symbind}{set@\protect{s$_{i}${\protect$\setminus$}{s$_j$}}}\\
\>\ref{tseb.rsl.so8}\ \ {\SUBSET}: A\kw{-infset} {\TIMES} A\kw{-infset} {\RIGHTARROW} \kw{Bool}\index{symbind}{set@\protect{s$_i$\SUBSET{s$_j$}}}\\
\>\ref{tseb.rsl.so9}\ \ {\SUBSETEQ}: A\kw{-infset} {\TIMES} A\kw{-infset} {\RIGHTARROW} \kw{Bool}\index{symbind}{set@\protect{s$_i$\SUBSETEQ{s$_j$}}}\\
\>\ref{tseb.rsl.so10}\ \ {\EQ}: A\kw{-infset} {\TIMES} A\kw{-infset} {\RIGHTARROW} \kw{Bool}\index{symbind}{set@\protect{s$_i$\EQ{s$_j$}}}\\
\>\ref{tseb.rsl.so11}\ \ {\NOTEQ}: A\kw{-infset} {\TIMES} A\kw{-infset} {\RIGHTARROW} \kw{Bool}\index{symbind}{set@\protect{s$_i$\NOTEQ{s$_j$}}}\\
\>\ref{tseb.rsl.so12}\ \ \kw{card}: A\kw{-infset} {\PARRIGHTARROW} \kw{Nat}\index{symbind}{set@\protect{\kw{card}\,s}}
\ep
}

\nbbb{Set Operation Examples}\hhhh

\nrslframebox{Set Operation Examples}{
%\RSLatex
%&\kw{examples}&
%  a isin {a,b,c}
%  a ~isin {}, a ~isin {b,c}
%  {a,b,c} union {a,b,d,e} = {a,b,c,d,e}       
%  union{{a},{a,b},{a,d}} = {a,b,d}
%  {a,b,c} inter {c,d,e} = {c}
%  inter{{a},{a,b},{a,d}} = {a}
%  {a,b,c} \ {c,d} = {a,b}
%  {a,b} << {a,b,c}
%  {a,b,c} <<= {a,b,c}
%  {a,b,c} = {a,b,c}
%  {a,b,c} ~= {a,b}
%  card {} = 0, card {a,b,c} = 3
%\endRSLatex
\bp
\kw{examples}\\
\>a {\ISIN} {\LBRACE}a,b,c{\RBRACE}\\
\>a {\NOTISIN} {\LBRACE}{\RBRACE}, a {\NOTISIN} {\LBRACE}b,c{\RBRACE}\\
\>{\LBRACE}a,b,c{\RBRACE} {\UNION} {\LBRACE}a,b,d,e{\RBRACE} {\EQ} {\LBRACE}a,b,c,d,e{\RBRACE}\ \ \ \ \ \ \ \\
\>{\UNION}{\LBRACE}{\LBRACE}a{\RBRACE},{\LBRACE}a,b{\RBRACE},{\LBRACE}a,d{\RBRACE}{\RBRACE} {\EQ} {\LBRACE}a,b,d{\RBRACE}\\
\>{\LBRACE}a,b,c{\RBRACE} {\INTER} {\LBRACE}c,d,e{\RBRACE} {\EQ} {\LBRACE}c{\RBRACE}\\
\>{\INTER}{\LBRACE}{\LBRACE}a{\RBRACE},{\LBRACE}a,b{\RBRACE},{\LBRACE}a,d{\RBRACE}{\RBRACE} {\EQ} {\LBRACE}a{\RBRACE}\\
\>{\LBRACE}a,b,c{\RBRACE} {\SETMINUS} {\LBRACE}c,d{\RBRACE} {\EQ} {\LBRACE}a,b{\RBRACE}\\
\>{\LBRACE}a,b{\RBRACE} {\SUBSET} {\LBRACE}a,b,c{\RBRACE}\\
\>{\LBRACE}a,b,c{\RBRACE} {\SUBSETEQ} {\LBRACE}a,b,c{\RBRACE}\\
\>{\LBRACE}a,b,c{\RBRACE} {\EQ} {\LBRACE}a,b,c{\RBRACE}\\
\>{\LBRACE}a,b,c{\RBRACE} {\NOTEQ} {\LBRACE}a,b{\RBRACE}\\
\>\kw{card} {\LBRACE}{\RBRACE} {\EQ} 0, \kw{card} {\LBRACE}a,b,c{\RBRACE} {\EQ} 3
\ep
%\end{multicols}
}

\nbbb{Informal Set Operator Explication}\hhhh

The following is \sort{not} a definition of \texttt{RSL} semantics.
In  \texttt{RSL} formulas we present an explication of  \texttt{RSL}
operators. Read, what appears as definitions, {\IS}, as [a kind of]
identities. 

\begin{enumerate}\setei
\item{\ISIN:}\label{tseb.rsl.so1} The membership operator expresses that an
  element is 
  a member of a set. 
\item \NOTISIN:\label{tseb.rsl.so2} The nonmembership operator expresses that an
  element is not a member of a set. 
\item{\UNION:}\label{tseb.rsl.so3} The infix union operator. When applied to
        two sets, the operator gives 
        the set whose members are in either or both of the two operand 
        sets.
\item{\UNION:}\label{tseb.rsl.so4} The distributed prefix union operator. When
        applied to a set of sets, the 
        operator gives 
        the set whose members are in some of the operand sets.
\item{\INTER:}\label{tseb.rsl.so5} The infix intersection operator. When applied to two
  sets, the operator gives the set whose members are in both of the
  two operand sets. 
\item{\INTER:}\label{tseb.rsl.so6} The prefix distributed intersection
  operator. When applied to a set of  
  sets, the operator gives the set whose members are in some of the
  operand sets. 
                             
\mnewfoil
  
\item{\SETMINUS:}\label{tseb.rsl.so7} The set complement (or set subtraction)
  operator. When applied to two sets, the operator gives the set whose
  members are those of the left operand set which are not in the right
  operand set.  
\item{\SUBSETEQ:}\label{tseb.rsl.so8} The proper subset operator expresses that
  all members 
  of the left operand set are also in the right operand set. 
\item{\SUBSET:}\label{tseb.rsl.so9} The proper subset operator expresses that
  all members 
  of the left operand set are also in the right operand set, and that
  the two sets are not identical. 
\item{\EQ:}\label{tseb.rsl.so10} The equal operator expresses that the two
  operand sets are identical.  
\item{\NOTEQ:}\label{tseb.rsl.so11} The nonequal operator expresses that the two operand
  sets are \emph{not} identical.  
\item{\kw{card}:}\label{tseb.rsl.so12} The cardinality operator gives the number
  of elements 
  in a finite set. 
\savei\end{enumerate}

\nbbb{Set Operator Explications}\hhhh

The set operations can be ``equated'' as follows:

\nrslframebox{Set Operator Explications}{
%\RSLatex
%value
%  s' union s'' is { a | a:A :- a isin s' \/ a isin s'' }
%  s' inter s'' is { a | a:A :- a isin s' /\ a isin s'' }
%  s' \ s'' is { a | a:A :- a isin s' /\ a ~isin s'' }
%  s' <<= s'' is all a:A :- a isin s' => a isin s'' 
%  s' << s'' is s' <<= s'' /\ exists a:A :- a isin s'' /\ a ~isin s' 
%  s' = s'' is all a:A :- a isin s' is a isin s'' is s<<=s' /\ s'<<=s 
%  s' ~= s'' is s' inter s'' ~= {}
%  card s is
%    if s = {} then 0 else
%    let a:A :- a isin s in 1 + card (s \ {a}) end end
%    pre s /* is a finite set */&\index{symbind}{aaaa@\protect{\underline{\textbf{Literals}}}|(}& 
%  card s is chaos /* tests for infinity of s */&\index{symbind}{aaab@\kw{chaos}}&
%\endRSLatex
\bp
\kw{value}\\
\>s{\PRIM} {\UNION} s{\PRIM}{\PRIM} {\IS} {\LBRACE} a {\BAR} a:A {\RDOT} a {\ISIN} s{\PRIM} {\VEE} a {\ISIN} s{\PRIM}{\PRIM} {\RBRACE}\\
\>s{\PRIM} {\INTER} s{\PRIM}{\PRIM} {\IS} {\LBRACE} a {\BAR} a:A {\RDOT} a {\ISIN} s{\PRIM} {\WEDGE} a {\ISIN} s{\PRIM}{\PRIM} {\RBRACE}\\
\>s{\PRIM} {\SETMINUS} s{\PRIM}{\PRIM} {\IS} {\LBRACE} a {\BAR} a:A {\RDOT} a {\ISIN} s{\PRIM} {\WEDGE} a {\NOTISIN} s{\PRIM}{\PRIM} {\RBRACE}\\
\>s{\PRIM} {\SUBSETEQ} s{\PRIM}{\PRIM} {\IS} {\ALL} a:A {\RDOT} a {\ISIN} s{\PRIM} {\DBLRIGHTARROW} a {\ISIN} s{\PRIM}{\PRIM} \\
\>s{\PRIM} {\SUBSET} s{\PRIM}{\PRIM} {\IS} s{\PRIM} {\SUBSETEQ} s{\PRIM}{\PRIM} {\WEDGE} {\EXISTS} a:A {\RDOT} a {\ISIN} s{\PRIM}{\PRIM} {\WEDGE} a {\NOTISIN} s{\PRIM} \\
\>s{\PRIM} {\EQ} s{\PRIM}{\PRIM} {\IS} {\ALL} a:A {\RDOT} a {\ISIN} s{\PRIM} {\IS} a {\ISIN} s{\PRIM}{\PRIM} {\IS} s{\SUBSETEQ}s{\PRIM} {\WEDGE} s{\PRIM}{\SUBSETEQ}s \\
\>s{\PRIM} {\NOTEQ} s{\PRIM}{\PRIM} {\IS} s{\PRIM} {\INTER} s{\PRIM}{\PRIM} {\NOTEQ} {\LBRACE}{\RBRACE}\\
\>\kw{card} s {\IS}\\
\>\>\kw{if} s {\EQ} {\LBRACE}{\RBRACE} \kw{then} 0 \kw{else}\\
\>\>\kw{let} a:A {\RDOT} a {\ISIN} s \kw{in} 1 {\PLUS} \kw{card} (s {\SETMINUS} {\LBRACE}a{\RBRACE}) \kw{end} \kw{end}\\
\>\>\kw{pre} s {\LCOMMENT} is a finite set {\RCOMMENT}\index{symbind}{aaaa@\protect{\underline{\textbf{Literals}}}|(} \\
\>\kw{card} s {\IS} \kw{chaos} {\LCOMMENT} tests for infinity of s {\RCOMMENT}\index{symbind}{aaab@\kw{chaos}}
\ep
}
\index{symbind}{sea@\protect{\underline{\textbf{Set Constructs}}}|)}

\nbbbb{Cartesian Operations}\label{rsl.Cartesian Operations}
\label{tseb.rsl.Cartesian Operations}
\index{symbind}{cartesiaa@\protect{\underline{\textbf{Cartesian Constructs}}}|(}

\nrslframebox{Cartesian Operations}{
\begin{multicols}{2}
%\RSLatex
%type
%  A, B, C
%  g0: G0 = A >< B >< C
%  g1: G1 = ( A >< B >< C )
%  g2: G2 = ( A >< B ) >< C
%  g3: G3 = A >< ( B >< C )
%
%value
%  va:A, vb:B, vc:C, vd:D
%  (va,vb,vc):G0, 
%  (va,vb,vc):G1
%  ((va,vb),vc):G2 
%  (va3,(vb3,vc3)):G3
%
%&\kw{decomposition expressions}&
%  let (a1,b1,c1) = g0, 
%      (a1',b1',c1') = g1 in .. end
%  let ((a2,b2),c2) = g2 in .. end
%  let (a3,(b3,c3)) = g3 in .. end 
%\endRSLatex
\bp
\kw{type}\\
\>A, B, C\\
\>g0: G0 {\EQ} A {\TIMES} B {\TIMES} C\\
\>g1: G1 {\EQ} ( A {\TIMES} B {\TIMES} C )\\
\>g2: G2 {\EQ} ( A {\TIMES} B ) {\TIMES} C\\
\>g3: G3 {\EQ} A {\TIMES} ( B {\TIMES} C )\\
\\
\kw{value}\\
\>va:A, vb:B, vc:C, vd:D\\
\>(va,vb,vc):G0, \\
\>(va,vb,vc):G1\\
\>((va,vb),vc):G2 \\
\>(va3,(vb3,vc3)):G3\\
\\
\kw{decomposition expressions}\\
\>\kw{let} (a1,b1,c1) {\EQ} g0, \\
\>\>\>(a1{\PRIM},b1{\PRIM},c1{\PRIM}) {\EQ} g1 \kw{in} {\DOTDOT} \kw{end}\\
\>\kw{let} ((a2,b2),c2) {\EQ} g2 \kw{in} {\DOTDOT} \kw{end}\\
\>\kw{let} (a3,(b3,c3)) {\EQ} g3 \kw{in} {\DOTDOT} \kw{end} 
\ep
\end{multicols}
}\LLL
\index{symbind}{cartesiaa@\protect{\underline{\textbf{Cartesian Constructs}}}|)}

\nbbbb{List Operations}\label{rsl.List Operations}
\bbb{List Operator Signatures}
\label{tseb.rsl.List Operations}
\index{symbind}{klista@\protect{\underline{\textbf{List Constructs}}}|(}

\nrslframebox{List Operator Signatures}{
%\RSLatex    
%value
%  hd: A-inflist -~-> A&\index{symbind}{klistb@\protect{\kw{hd}\,$\ell$}}&
%  tl: A-inflist -~-> A-inflist&\index{symbind}{klistb@\protect{\kw{tl}\,$\ell$}}&
%  len: A-inflist -~-> Nat&\index{symbind}{klistb@\protect{\kw{len}\,$\ell$}}&
%  inds: A-inflist -> Nat-infset&\index{symbind}{klistb@\protect{\kw{inds}\,$\ell$}}&
%  elems: A-inflist -> A-infset&\index{symbind}{klistb@\protect{\kw{elems}\,$\ell$}}&
%  .(.): A-inflist >< Nat -~-> A&\index{symbind}{klistb@\protect{$\ell$(i)}}&
%  ^: A-list >< A-inflist -> A-inflist&\index{symbind}{klistb@\protect{${\ell}'\,\protect{\widehat{\;}}\,{\ell}''$}}&
%  =: A-inflist >< A-inflist -> Bool&\index{symbind}{klistb@\protect{$\ell'$\,\EQ\,$\ell''$}}&
%  ~=: A-inflist >< A-inflist -> Bool&\index{symbind}{klistb@\protect{$\ell'$\,\NOTEQ\,$\ell''$}}&
%\endRSLatex
\bp
\kw{value}\\
\>\kw{hd}: A$^{\omega}$ {\PARRIGHTARROW} A\index{symbind}{klistb@\protect{\kw{hd}\,$\ell$}}\\
\>\kw{tl}: A$^{\omega}$ {\PARRIGHTARROW} A$^{\omega}$\index{symbind}{klistb@\protect{\kw{tl}\,$\ell$}}\\
\>\kw{len}: A$^{\omega}$ {\PARRIGHTARROW} \kw{Nat}\index{symbind}{klistb@\protect{\kw{len}\,$\ell$}}\\
\>\kw{inds}: A$^{\omega}$ {\RIGHTARROW} \kw{Nat}\kw{-infset}\index{symbind}{klistb@\protect{\kw{inds}\,$\ell$}}\\
\>\kw{elems}: A$^{\omega}$ {\RIGHTARROW} A\kw{-infset}\index{symbind}{klistb@\protect{\kw{elems}\,$\ell$}}\\
\>.(.): A$^{\omega}$ {\TIMES} \kw{Nat} {\PARRIGHTARROW} A\index{symbind}{klistb@\protect{$\ell$(i)}}\\
\>{\CONCAT}: A$^{\ast}$ {\TIMES} A$^{\omega}$ {\RIGHTARROW} A$^{\omega}$\index{symbind}{klistb@\protect{${\ell}'\,\protect{\widehat{\;}}\,{\ell}''$}}\\
\>{\EQ}: A$^{\omega}$ {\TIMES} A$^{\omega}$ {\RIGHTARROW} \kw{Bool}\index{symbind}{klistb@\protect{$\ell'$\,\EQ\,$\ell''$}}\\
\>{\NOTEQ}: A$^{\omega}$ {\TIMES} A$^{\omega}$ {\RIGHTARROW} \kw{Bool}\index{symbind}{klistb@\protect{$\ell'$\,\NOTEQ\,$\ell''$}}
\ep
}

\nbbb{List Operation Examples}
\nrslframebox{List Operation Examples}{
%\RSLatex
%&\kw{examples}&
%  hd<.a1,a2,...,am.>=a1
%  tl<.a1,a2,...,am.>=<.a2,...,am.> 
%  len<.a1,a2,...,am.>=m
%  inds<.a1,a2,...,am.>={1,2,...,m}
%  elems<.a1,a2,...,am.>={a1,a2,...,am}
%  <.a1,a2,...,am.>(i)=ai
%  <.a,b,c.>^<.a,b,d.> = <.a,b,c,a,b,d.>
%  <.a,b,c.>=<.a,b,c.>
%  <.a,b,c.> ~= <.a,b,d.>
%\endRSLatex
\bp
\kw{examples}\\
\>\kw{hd}{\LANGLE}a1,a2,{\DOTDOTDOT},am{\RANGLE}{\EQ}a1\\
\>\kw{tl}{\LANGLE}a1,a2,{\DOTDOTDOT},am{\RANGLE}{\EQ}{\LANGLE}a2,{\DOTDOTDOT},am{\RANGLE} \\
\>\kw{len}{\LANGLE}a1,a2,{\DOTDOTDOT},am{\RANGLE}{\EQ}m\\
\>\kw{inds}{\LANGLE}a1,a2,{\DOTDOTDOT},am{\RANGLE}{\EQ}{\LBRACE}1,2,{\DOTDOTDOT},m{\RBRACE}\\
\>\kw{elems}{\LANGLE}a1,a2,{\DOTDOTDOT},am{\RANGLE}{\EQ}{\LBRACE}a1,a2,{\DOTDOTDOT},am{\RBRACE}\\
\>{\LANGLE}a1,a2,{\DOTDOTDOT},am{\RANGLE}(i){\EQ}ai\\
\>{\LANGLE}a,b,c{\RANGLE}{\CONCAT}{\LANGLE}a,b,d{\RANGLE} {\EQ} {\LANGLE}a,b,c,a,b,d{\RANGLE}\\
\>{\LANGLE}a,b,c{\RANGLE}{\EQ}{\LANGLE}a,b,c{\RANGLE}\\
\>{\LANGLE}a,b,c{\RANGLE} {\NOTEQ} {\LANGLE}a,b,d{\RANGLE}
\ep
}

\nbbb{Informal List Operator Explication}

The following is \sort{not} a definition of \texttt{RSL} semantics.
In  \texttt{RSL} formulas we present an explication of  \texttt{RSL}
operators. Read, what appears as definitions, {\IS}, as [a kind of]
identities. 

\begin{itemize}
\item{\kw{hd}:} Head gives the first element in a nonempty list. 
\item{\kw{tl}:} Tail gives the remaining list of a nonempty list
  when Head is removed. 
\item{\kw{len}:} Length gives the number of elements in a finite list.
\item{\kw{inds}:} Indices give the set of indices from \textsf{1} to
  the length of a nonempty list. For empty lists, this set is the
  empty set as well. 
\item{\kw{elems}:} Elements gives the possibly infinite set of all
  distinct elements in a list. 
\item{$\ell(i)$:} Indexing with a natural number, \textsl{i} larger
  than 0, into a list $\ell$ having a number of elements larger than
  or equal to \textsl{i}, gives the \textsl{i}th element of the
  list. 
\ninlineiii{Concrete \dbrsl\ Types: Values and Operations}{List
  Operations}{Informal Explication} 
\item{\CONCAT:} Concatenates two operand lists into one. The elements
  of the left operand list are followed by the elements of the
  right. The order with respect to each list is maintained.  
\item{\EQ:} The equal operator  expresses that the two operand lists
  are identical.  
\item{\NOTEQ:} The nonequal operator expresses that the two operand
  lists are \emph{not} identical.  
\end{itemize}
\noindent
The operations can also be defined as follows:

\nbbb{List Operator Explications}\hhhh

The following is \sort{not} a definition of \texttt{RSL} semantics.
In  \texttt{RSL} formulas we present an explication of  \texttt{RSL}
operators. Read, what appears as definitions, {\IS}, as [a kind of]
identities. 

\nrslframebox{List Operator Explications}{
%\RSLatex
%value
%  is_finite_list: A-inflist -> Bool
%
%  len q is 
%    case is_finite_list(q) of
%      true -> if q = <..> then 0 else 1 + len tl q end,
%      false -> chaos end&\index{symbind}{aaab@\kw{chaos}}&
%
%  inds q is 
%    case is_finite_list(q) of
%      true -> { i | i:Nat :- 1 <= i <= len q },
%      false -> { i | i:Nat :- i~=0 } end 
%
%  elems q is { q(i) | i:Nat :- i isin inds q }
%
%  q(i) is 
%    if i=1 
%      then 
%        if q~=<..> 
%          then let a:A,q':Q :- q=<.a.>^q' in a end 
%          else chaos end&\index{symbind}{aaab@\kw{chaos}}&
%      else q(i-1) end
%
%  fq ^ iq is 
%      <. if 1 <= i <= len fq then fq(i) else iq(i - len fq) end
%       | i:Nat :- if len iq~=chaos then i <= len fq+len end .>
%    pre is_finite_list(fq)
%
%  iq' = iq'' is 
%    inds iq' = inds iq'' /\ all i:Nat :- i isin inds iq' => iq'(i) = iq''(i)
%
%  iq' ~= iq'' is ~(iq' = iq'')
%\endRSLatex
\bp
\kw{value}\\
\>is\_finite\_list: A$^{\omega}$ {\RIGHTARROW} \kw{Bool}\\
\\
\>\kw{len} q {\IS} \\
\>\>\kw{case} is\_finite\_list(q) \kw{of}\\
\>\>\>\kw{true} {\RIGHTARROW} \kw{if} q {\EQ} {\LANGLE}{\RANGLE} \kw{then} 0 \kw{else} 1 {\PLUS} \kw{len} \kw{tl} q \kw{end},\\
\>\>\>\kw{false} {\RIGHTARROW} \kw{chaos} \kw{end}\index{symbind}{aaab@\kw{chaos}}\\
\\
\>\kw{inds} q {\IS} \\
\>\>\kw{case} is\_finite\_list(q) \kw{of}\\
\>\>\>\kw{true} {\RIGHTARROW} {\LBRACE} i {\BAR} i:\kw{Nat} {\RDOT} 1 {\LEQ} i {\LEQ} \kw{len} q {\RBRACE},\\
\>\>\>\kw{false} {\RIGHTARROW} {\LBRACE} i {\BAR} i:\kw{Nat} {\RDOT} i{\NOTEQ}0 {\RBRACE} \kw{end} \\
\\
\>\kw{elems} q {\IS} {\LBRACE} q(i) {\BAR} i:\kw{Nat} {\RDOT} i {\ISIN} \kw{inds} q {\RBRACE}\\
\\
\>q(i) {\IS} \\
\>\>\kw{if} i{\EQ}1 \\
\>\>\>\kw{then} \\
\>\>\>\>\kw{if} q{\NOTEQ}{\LANGLE}{\RANGLE} \\
\>\>\>\>\>\kw{then} \kw{let} a:A,q{\PRIM}:Q {\RDOT} q{\EQ}{\LANGLE}a{\RANGLE}{\CONCAT}q{\PRIM} \kw{in} a \kw{end} \\
\>\>\>\>\>\kw{else} \kw{chaos} \kw{end}\index{symbind}{aaab@\kw{chaos}}\\
\>\>\>\kw{else} q(i{\MINUS}1) \kw{end}\\
\\
\>fq {\CONCAT} iq {\IS} \\
\>\>\>{\LANGLE} \kw{if} 1 {\LEQ} i {\LEQ} \kw{len} fq \kw{then} fq(i) \kw{else} iq(i {\MINUS} \kw{len} fq) \kw{end}\\
\>\>\>\ {\BAR} i:\kw{Nat} {\RDOT} \kw{if} \kw{len} iq{\NOTEQ}\kw{chaos} \kw{then} i {\LEQ} \kw{len} fq{\PLUS}\kw{len} \kw{end} {\RANGLE}\\
\>\>\kw{pre} is\_finite\_list(fq)\\
\\
\>iq{\PRIM} {\EQ} iq{\PRIM}{\PRIM} {\IS} \\
\>\>\kw{inds} iq{\PRIM} {\EQ} \kw{inds} iq{\PRIM}{\PRIM} {\WEDGE} {\ALL} i:\kw{Nat} {\RDOT} i {\ISIN} \kw{inds} iq{\PRIM} {\DBLRIGHTARROW} iq{\PRIM}(i) {\EQ} iq{\PRIM}{\PRIM}(i)\\
\\
\>iq{\PRIM} {\NOTEQ} iq{\PRIM}{\PRIM} {\IS} {\SIM}(iq{\PRIM} {\EQ} iq{\PRIM}{\PRIM})
\ep
}


\index{symbind}{klista@\protect{\underline{\textbf{List Constructs}}}|)}

\ninlinei{Concrete \dbrsl\ Types: Values and Operations}
\bbbb{Map Operations}
\bbb{Map Operator Signatures}
\label{tseb.rsl.Map Operations}
\index{symbind}{maa@\protect{\underline{\textbf{Map Constructs}}}|(}

\nrslframebox{Map Operator Signatures}{%%%%%%%%%%%
%\RSLatex
%value
%[30]  &$\cdot$&(&$\cdot$&): M -> A -~-> B&\index{symbind}{map@\protect{m(e)}}&
%[31]  dom: M -> A-infset [domain &of& map]&\index{symbind}{map@\protect{\kw{dom}\,m}}&
%[32]  rng: M -> B-infset [range &of& map]&\index{symbind}{map@\protect{\kw{rng}\,m}}&
%[33]  !!: M >< M -> M [override extension]&\index{symbind}{map@\protect{m$_i$}\,\protect{$\dagger$}\,\protect{m$_j$}}&
%[34]  union: M >< M -> M [merge union]&\index{symbind}{map@\protect{m$_i$\,$\cup$\,m$_j$}}&
%[35]  \: M >< A-infset -> M [restriction by]&\index{symbind}{map@\protect{\,m$_i$}\,\protect{\SETMINUS}\,\protect{m$_j$}}&
%[36]  /: M >< A-infset -> M [restriction to]&\index{symbind}{map@\protect{\,m$_i$\,/\,m$_j$}}&
%[37]  =,~=: M >< M -> Bool&\index{symbind}{map@\protect{m$_i$\,$=$\,m$j$}}\index{symbind}{map@\protect{m$_i$\,$\neq$\,m$j$}}&
%[38]  #: (A -m-> B) >< (B -m-> C) -> (A -m-> C) [composition]&\index{symbind}{map@\protect{\,m$_i$\,\protect{$\circ$}\,m$_j$}}&
%\endRSLatex
\bp
\kw{value}\\
{\LBRACKET}30{\RBRACKET}\ \ $\cdot$($\cdot$): M {\RIGHTARROW} A {\PARRIGHTARROW} B\index{symbind}{map@\protect{m(e)}}\\
{\LBRACKET}31{\RBRACKET}\ \ \kw{dom}: M {\RIGHTARROW} A\kw{-infset} {\LBRACKET}domain of map{\RBRACKET}\index{symbind}{map@\protect{\kw{dom}\,m}}\\
{\LBRACKET}32{\RBRACKET}\ \ \kw{rng}: M {\RIGHTARROW} B\kw{-infset} {\LBRACKET}range of map{\RBRACKET}\index{symbind}{map@\protect{\kw{rng}\,m}}\\
{\LBRACKET}33{\RBRACKET}\ \ {\DAGGER}: M {\TIMES} M {\RIGHTARROW} M {\LBRACKET}override extension{\RBRACKET}\index{symbind}{map@\protect{m$_i$}\,\protect{$\dagger$}\,\protect{m$_j$}}\\
{\LBRACKET}34{\RBRACKET}\ \ {\UNION}: M {\TIMES} M {\RIGHTARROW} M {\LBRACKET}merge {\UNION}{\RBRACKET}\index{symbind}{map@\protect{m$_i$\,$\cup$\,m$_j$}}\\
{\LBRACKET}35{\RBRACKET}\ \ {\SETMINUS}: M {\TIMES} A\kw{-infset} {\RIGHTARROW} M {\LBRACKET}restriction by{\RBRACKET}\index{symbind}{map@\protect{\,m$_i$}\,\protect{\SETMINUS}\,\protect{m$_j$}}\\
{\LBRACKET}36{\RBRACKET}\ \ /: M {\TIMES} A\kw{-infset} {\RIGHTARROW} M {\LBRACKET}restriction to{\RBRACKET}\index{symbind}{map@\protect{\,m$_i$\,/\,m$_j$}}\\
{\LBRACKET}37{\RBRACKET}\ \ {\EQ},{\NOTEQ}: M {\TIMES} M {\RIGHTARROW} \kw{Bool}\index{symbind}{map@\protect{m$_i$\,$=$\,m$j$}}\index{symbind}{map@\protect{m$_i$\,$\neq$\,m$j$}}\\
{\LBRACKET}38{\RBRACKET}\ \ {\HASH}: (A {\MARROW} B) {\TIMES} (B {\MARROW} C) {\RIGHTARROW} (A {\MARROW} C) {\LBRACKET}composition{\RBRACKET}\index{symbind}{map@\protect{\,m$_i$\,\protect{$\circ$}\,m$_j$}}
\ep
}

\bbb{Map Operation Examples}

\nrslframebox{Map Operation Examples}{%%%%%%%%%%%
%\RSLatex
%value
%[30]  m(a) = b&\index{symbind}{map@\protect{m(e)}}&
%[31]  dom [a1+>b1,a2+>b2,...,an+>bn] = {a1,a2,...,an}&\index{symbind}{map@\protect{\kw{dom}\,m}}&
%[32]  rng [a1+>b1,a2+>b2,...,an+>bn] = {b1,b2,...,bn}&\index{symbind}{map@\protect{\kw{rng}\,m}}&   
%[33]  [a+>b,a'+>b',a''+>b''] !! [a'+>b'',a''+>b'] = [a+>b,a'+>b'',a''+>b']&\index{symbind}{map@\protect{m$_i$}\,\protect{$\dagger$}\,\protect{m$_j$}}&
%[34]  [a+>b,a'+>b',a''+>b''] union [a'''+>b'''] = [a+>b,a'+>b',a''+>b'',a'''+>b''']&\index{symbind}{map@\protect{m$_i$\,$\cup$\,m$_j$}}&
%[35]  [a+>b,a'+>b',a''+>b'']\{a} = [a'+>b',a''+>b'']&\index{symbind}{map@\protect{\,m$_i$}\,\protect{\SETMINUS}\,\protect{m$_j$}}&
%[37]  [a+>b,a'+>b',a''+>b'']/{a',a''} = [a'+>b',a''+>b'']&\index{symbind}{map@\protect{\,m$_i$\,/\,m$_j$}}&
%[38]  [a+>b,a'+>b'] # [b+>c,b'+>c',b''+>c''] = [a+>c,a'+>c']&\index{symbind}{map@\protect{\,m$_i$\,\protect{$\circ$}\,m$_j$}}&
%\endRSLatex
\bp
\kw{value}\\
{\LBRACKET}30{\RBRACKET}\ \ m(a) {\EQ} b\index{symbind}{map@\protect{m(e)}}\\
{\LBRACKET}31{\RBRACKET}\ \ \kw{dom} {\LBRACKET}a1{\MAPSTO}b1,a2{\MAPSTO}b2,{\DOTDOTDOT},an{\MAPSTO}bn{\RBRACKET} {\EQ} {\LBRACE}a1,a2,{\DOTDOTDOT},an{\RBRACE}\index{symbind}{map@\protect{\kw{dom}\,m}}\\
{\LBRACKET}32{\RBRACKET}\ \ \kw{rng} {\LBRACKET}a1{\MAPSTO}b1,a2{\MAPSTO}b2,{\DOTDOTDOT},an{\MAPSTO}bn{\RBRACKET} {\EQ} {\LBRACE}b1,b2,{\DOTDOTDOT},bn{\RBRACE}\index{symbind}{map@\protect{\kw{rng}\,m}}\ \ \ \\
{\LBRACKET}33{\RBRACKET}\ \ {\LBRACKET}a{\MAPSTO}b,a{\PRIM}{\MAPSTO}b{\PRIM},a{\PRIM}{\PRIM}{\MAPSTO}b{\PRIM}{\PRIM}{\RBRACKET} {\DAGGER} {\LBRACKET}a{\PRIM}{\MAPSTO}b{\PRIM}{\PRIM},a{\PRIM}{\PRIM}{\MAPSTO}b{\PRIM}{\RBRACKET} {\EQ} {\LBRACKET}a{\MAPSTO}b,a{\PRIM}{\MAPSTO}b{\PRIM}{\PRIM},a{\PRIM}{\PRIM}{\MAPSTO}b{\PRIM}{\RBRACKET}\index{symbind}{map@\protect{m$_i$}\,\protect{$\dagger$}\,\protect{m$_j$}}\\
{\LBRACKET}34{\RBRACKET}\ \ {\LBRACKET}a{\MAPSTO}b,a{\PRIM}{\MAPSTO}b{\PRIM},a{\PRIM}{\PRIM}{\MAPSTO}b{\PRIM}{\PRIM}{\RBRACKET} {\UNION} {\LBRACKET}a{\PRIM}{\PRIM}{\PRIM}{\MAPSTO}b{\PRIM}{\PRIM}{\PRIM}{\RBRACKET} {\EQ} {\LBRACKET}a{\MAPSTO}b,a{\PRIM}{\MAPSTO}b{\PRIM},a{\PRIM}{\PRIM}{\MAPSTO}b{\PRIM}{\PRIM},a{\PRIM}{\PRIM}{\PRIM}{\MAPSTO}b{\PRIM}{\PRIM}{\PRIM}{\RBRACKET}\index{symbind}{map@\protect{m$_i$\,$\cup$\,m$_j$}}\\
{\LBRACKET}35{\RBRACKET}\ \ {\LBRACKET}a{\MAPSTO}b,a{\PRIM}{\MAPSTO}b{\PRIM},a{\PRIM}{\PRIM}{\MAPSTO}b{\PRIM}{\PRIM}{\RBRACKET}{\SETMINUS}{\LBRACE}a{\RBRACE} {\EQ} {\LBRACKET}a{\PRIM}{\MAPSTO}b{\PRIM},a{\PRIM}{\PRIM}{\MAPSTO}b{\PRIM}{\PRIM}{\RBRACKET}\index{symbind}{map@\protect{\,m$_i$}\,\protect{\SETMINUS}\,\protect{m$_j$}}\\
{\LBRACKET}37{\RBRACKET}\ \ {\LBRACKET}a{\MAPSTO}b,a{\PRIM}{\MAPSTO}b{\PRIM},a{\PRIM}{\PRIM}{\MAPSTO}b{\PRIM}{\PRIM}{\RBRACKET}/{\LBRACE}a{\PRIM},a{\PRIM}{\PRIM}{\RBRACE} {\EQ} {\LBRACKET}a{\PRIM}{\MAPSTO}b{\PRIM},a{\PRIM}{\PRIM}{\MAPSTO}b{\PRIM}{\PRIM}{\RBRACKET}\index{symbind}{map@\protect{\,m$_i$\,/\,m$_j$}}\\
{\LBRACKET}38{\RBRACKET}\ \ {\LBRACKET}a{\MAPSTO}b,a{\PRIM}{\MAPSTO}b{\PRIM}{\RBRACKET} {\HASH} {\LBRACKET}b{\MAPSTO}c,b{\PRIM}{\MAPSTO}c{\PRIM},b{\PRIM}{\PRIM}{\MAPSTO}c{\PRIM}{\PRIM}{\RBRACKET} {\EQ} {\LBRACKET}a{\MAPSTO}c,a{\PRIM}{\MAPSTO}c{\PRIM}{\RBRACKET}\index{symbind}{map@\protect{\,m$_i$\,\protect{$\circ$}\,m$_j$}}
\ep
}

\nbbb{Informal Map Operation Explication}%\hhhh

\begin{itemize}
\item{$m$($a$):} Application gives the element that $a$ maps to in
  the map $m$.
\item{\kw{dom}:} Domain/Definition Set gives the set of values which
  \emph{maps to} in a map. 
\item{\kw{rng}:} Range/Image Set gives the set of values which
  \emph{are mapped to} in a map.   
\item{\DAGGER:} Override/Extend. When applied to two operand maps, it
      gives the map which is like an 
      override of the left operand map by all or some ``pairings'' of
      the right operand map.
\item{\UNION:} Merge. When applied to two operand maps, it
  gives a merge of these maps. 
\pos{\psno}{\ninlineiii{Concrete \dbrsl\ Types: Values and Operations}{Map
  Operations}{Map Operation Explication}}\hhhh
\item{$\backslash$:} Restriction. When applied to two operand maps, it
      gives the map which is a restriction of the left operand map to
      the  elements that are not in the right operand set. 
\item{/:} Restriction. When applied to two operand maps, it gives the
  map which is a restriction of the left operand map to the elements
  of the right operand set. 
\item{\EQ:} The equal operator expresses that the two operand maps are
  identical. 
\item{\NOTEQ:} The nonequal operator expresses that the two
  operand maps are \emph{not} identical. 
\item{\HASH:} Composition. When applied to two operand maps, it gives
  the map from definition set elements of the left operand map, $m_1$,
  to the range elements of the right operand map, $m_2$, such that if
  $a$ is in the definition set of $m_1$ and maps into $b$, and if $b$ is
  in the definition set of $m_2$ and maps into $c$, then $a$, in the
  composition, maps into $c$.  
\end{itemize} 

\nbbb{Map Operator Explication}\LLll

The following is \sort{not} a definition of \texttt{RSL} semantics.
In  \texttt{RSL} formulas we present an explication of  \texttt{RSL}
operators. Read, what appears as definitions, {\IS}, as [a kind of]
identities. 

\pos{
\noindent
The map operations can also be defined as follows:}{}

\nrslframebox{Map Operator Explications}{
%\RSLatex
%value
%  rng m is { m(a) | a:A :- a isin dom m }
%
%  m1 !! m2 is 
%    [ a+>b | a:A,b:B :-  
%      a isin dom m1 \ dom m2 /\ b=m1(a) \/ a isin dom m2 /\ b=m2(a) ]
%
%  m1 union m2 is [ a+>b | a:A,b:B :-
%           a isin dom m1 /\ b=m1(a) \/ a isin dom m2 /\ b=m2(a) ]
%
%  m \ s is [ a+>m(a) | a:A :- a isin dom m \ s ]
%  m / s is [ a+>m(a) | a:A :- a isin dom m inter s ]
%
%  m1 = m2 is 
%    dom m1 = dom m2 /\ all a:A :- a isin dom m1 => m1(a) = m2(a)
%  m1 ~= m2 is ~(m1 = m2)
%
%  m#n is 
%    [ a+>c | a:A,c:C :- a isin dom m /\ c = n(m(a)) ]
%    pre rng m <<= dom n 
%\endRSLatex
\bp
\kw{value}\\
\>\kw{rng} m {\IS} {\LBRACE} m(a) {\BAR} a:A {\RDOT} a {\ISIN} \kw{dom} m {\RBRACE}\\
\\
\>m1 {\DAGGER} m2 {\IS} \\
\>\>{\LBRACKET} a{\MAPSTO}b {\BAR} a:A,b:B {\RDOT}\ \ \\
\>\>\>a {\ISIN} \kw{dom} m1 {\SETMINUS} \kw{dom} m2 {\WEDGE} b{\EQ}m1(a) {\VEE} a {\ISIN} \kw{dom} m2 {\WEDGE} b{\EQ}m2(a) {\RBRACKET}\\
\\
\>m1 {\UNION} m2 {\IS} {\LBRACKET} a{\MAPSTO}b {\BAR} a:A,b:B {\RDOT}\\
\>\>\>\>\>\ a {\ISIN} \kw{dom} m1 {\WEDGE} b{\EQ}m1(a) {\VEE} a {\ISIN} \kw{dom} m2 {\WEDGE} b{\EQ}m2(a) {\RBRACKET}\\
\\
\>m {\SETMINUS} s {\IS} {\LBRACKET} a{\MAPSTO}m(a) {\BAR} a:A {\RDOT} a {\ISIN} \kw{dom} m {\SETMINUS} s {\RBRACKET}\\
\>m / s {\IS} {\LBRACKET} a{\MAPSTO}m(a) {\BAR} a:A {\RDOT} a {\ISIN} \kw{dom} m {\INTER} s {\RBRACKET}\\
\\
\>m1 {\EQ} m2 {\IS} \\
\>\>\kw{dom} m1 {\EQ} \kw{dom} m2 {\WEDGE} {\ALL} a:A {\RDOT} a {\ISIN} \kw{dom} m1 {\DBLRIGHTARROW} m1(a) {\EQ} m2(a)\\
\>m1 {\NOTEQ} m2 {\IS} {\SIM}(m1 {\EQ} m2)\\
\\
\>m{\HASH}n {\IS} \\
\>\>{\LBRACKET} a{\MAPSTO}c {\BAR} a:A,c:C {\RDOT} a {\ISIN} \kw{dom} m {\WEDGE} c {\EQ} n(m(a)) {\RBRACKET}\\
\>\>\kw{pre} \kw{rng} m {\SUBSETEQ} \kw{dom} n 
\ep
}{}
\index{symbind}{maa@\protect{\underline{\textbf{Map Constructs}}}|)}

\nbbbbb{$\lambda$\--Calculus + Functions}
\label{tseb.rsl.Lambda Calculus + Functions}

\irsltxt{The $\lambda$\--Calculus is a foundation for the abstract
  specification language that \texttt{RSL} is}

\bbbb{The $\lambda$-Calculus Syntax}\LLLL
\label{tseb.rsl.The Lambda Calculus Syntax}

\nrslframebox{$\lambda$-Calculus Syntax}{
%\begin{multicols}{2}
%\RSLatex
%type /* A BNF Syntax: */
%   <.L.> ::= <.V.> | <.F.> | <.A.> | ( <.A.> )            
%   <.V.> ::= /* variables, i.e. identifiers */
%   <.F.> ::= &\LAMBDA& <.V.> :- <.L.>
%   <.A.> ::= ( <.L.><.L.> )
%value /* Examples */
%   <.L.>: e, f, a, ...
%   <.V.>: x, ... 
%   <.F.>: -\ x :- e, ...
%   <.A.>: f a, (f a), f(a), (f)(a), ...&\index{symbind}{function@f(a)}& 
%\endRSLatex
\bp
\kw{type} {\LCOMMENT} A BNF Syntax: {\RCOMMENT}\\
\>\ {\LANGLE}L{\RANGLE} ::{\EQ} {\LANGLE}V{\RANGLE} {\BAR} {\LANGLE}F{\RANGLE} {\BAR} {\LANGLE}A{\RANGLE} {\BAR} ( {\LANGLE}A{\RANGLE} )\ \ \ \ \ \ \ \ \ \ \ \ \\
\>\ {\LANGLE}V{\RANGLE} ::{\EQ} {\LCOMMENT} variables, i.e. identifiers {\RCOMMENT}\\
\>\ {\LANGLE}F{\RANGLE} ::{\EQ} \LAMBDA {\LANGLE}V{\RANGLE} {\RDOT} {\LANGLE}L{\RANGLE}\\
\>\ {\LANGLE}A{\RANGLE} ::{\EQ} ( {\LANGLE}L{\RANGLE}{\LANGLE}L{\RANGLE} )\\
\kw{value} {\LCOMMENT} Examples {\RCOMMENT}\\
\>\ {\LANGLE}L{\RANGLE}: e, f, a, {\DOTDOTDOT}\\
\>\ {\LANGLE}V{\RANGLE}: x, {\DOTDOTDOT} \\
\>\ {\LANGLE}F{\RANGLE}: {\LAMBDA} x {\RDOT} e, {\DOTDOTDOT}\\
\>\ {\LANGLE}A{\RANGLE}: f a, (f a), f(a), (f)(a), {\DOTDOTDOT}\index{symbind}{function@f(a)} 
\ep
%\end{multicols}
}

\nbbbb{Free and Bound Variables}
\label{tseb.rsl.Free and Bound Variables}

\nrslframebox{Free and Bound Variables}{
Let $x,y$ be variable names and $e,f$ be \LAMBDA-expressions.
\begin{itemize}
\item \LANGLE V\RANGLE: Variable $x$ is free in $x$.
\item \LANGLE F\RANGLE: $x$ is free in \LAMBDA $y$ \RDOT $e$
if $x\neq y$ and $x$ is free in $e$.
\item \LANGLE A\RANGLE: $x$ is free in $f(e)$ if it is free in
either $f$ or $e$ (i.e., also in both).
\end{itemize}
}

\nbbbb{Substitution}
\label{tseb.rsl.Substitution}
\wrtrt{\volchapsec{1}{7}{4}}
\pt{
In {\dbrsl}, the following rules for substitution apply:}{}\LLL

\nrslframebox{Substitution}{
\begin{itemize}
\item  \textsf{\aesubb{N}{x}{x}} \IS\ \textsf{N};
\item  \textsf{\aesubb{N}{x}{a}} \IS\ \textsf{a},
\begin{itemize}
\item[] for all variables \textsf{a$\neq$ x};
\end{itemize}
\item \textsf{\aesubb{N}{x}{{(P Q)}}} \IS\
   (\textsf{\aesubb{N}{x}{P} \aesubb{N}{x}{Q}});
\item \textsf{\aesubb{N}{x}{{($\lambda x\RDOT P$)}}} \IS\  
                     \textsf{$\lambda$ y\RDOT P};
\item \textsf{\aesubb{N}{x}{{($\lambda$ y\RDOT P)}}} \IS\ 
         \textsf{$\lambda y\RDOT$ \aesubb{N}{x}{P}}, 
\begin{itemize}
\item[] if \textsf{x$\neq$y} and \textsf{y}
    is not free in \textsf{N}
     or \textsf{x} is not free in \textsf{P};
\end{itemize}
\item \textsf{\aesubb{N}{x}{{($\lambda${y}\RDOT{P})}}} \IS\ 
  \textsf{$\lambda$z\RDOT\aesubb{N}{z}{{\mbox{\aesubb{z}{y}{P}}}}},
\begin{itemize}
\item []if \textsf{y$\neq$x} and \textsf{y} is free in \textsf{N}
     and \textsf{x} is free in \textsf{P} 
\item[] (where \textsf{z} is not free in \textsf{(N P)}).
\end{itemize}
\end{itemize} 
}

\nbbbb{$\alpha$-Renaming and $\beta$-Reduction}
\label{tseb.rsl.alpha-Renaming and beta-Reduction}

\wrtrt{\volchapsec{1}{7}{4}}

\nrslframebox{$\alpha$ and $\beta$ Conversions}{
\begin{itemize}
\item $\alpha$-renaming:
\textsf{\LAMBDA x\RDOT M}

   If \textsf{x}, \textsf{y} are distinct variables then replacing
   \textsf{x} by \textsf{y} in
   \textsf{\LAMBDA x\RDOT M} results in
   \textsf{\LAMBDA y\RDOT \aesubb{y}{x}{M}}. We can rename
   the formal parameter of
   a \LAMBDA-function expression provided that no
   free variables of its body \textsf{M} thereby become bound.

\item $\beta$-reduction:
\textsf{(\LAMBDA x\RDOT M)(N)}

      All free occurrences of \textsf{x}
      in \textsf{M} are replaced by  the expression \textsf{N}
      provided that no free variables of \textsf{N} thereby
      become bound in the result.
    \textsf{(\LAMBDA x\RDOT M)(N) \IS\ \aesubb{N}{x}{M}}

\end{itemize}
}

\nbbbb{Function Signatures}
\label{tseb.rsl.Function Signatures}

For sorts we may want to postulate some functions:

\nrslframebox{Sorts and Function Signatures}{
%\RSLatex
%type
%  A, B, C
%value
%  obs_B: A -> B,
%  obs_C: A -> C,
%  gen_A: B><C -> A
%\endRSLatex
\bp
\kw{type}\\
\>A, B, C\\
\kw{value}\\
\>obs\_B: A {\RIGHTARROW} B,\\
\>obs\_C: A {\RIGHTARROW} C,\\
\>gen\_A: B{\TIMES}C {\RIGHTARROW} A
\ep
}

\nbbbb{Function Definitions}\label{tseb.rsl.Function Definitions}

Functions can be defined explicitly:

\index{symbind}{functioa@\protect{\underline{\textbf{Function Constructs}}}|(}
\renewcommand{\dbeat}[1]{}

\nrslframebox{Explicit Function Definitions}{{
%\RSLatex
%value
%  f: Arguments -> Result
%  f(args) is DValueExpr
%
%  g: Arguments -~-> Result
%  g(args) is ValueAndStateChangeClause
%  pre P(args)
%\endRSLatex
\bp
\kw{value}\\
\>f: Arguments {\RIGHTARROW} Result\\
\>f(args) {\IS} DValueExpr\\
\\
\>g: Arguments {\PARRIGHTARROW} Result\\
\>g(args) {\IS} ValueAndStateChangeClause\\
\>\kw{pre} P(args)
\ep
}}
\index{symbind}{function@f(args)\ \protect\IS\ expr}
\index{symbind}{function@{\kw{pre}\ P(args)}}

\mnewfoil

\noindent
Or functions can be defined implicitly:
 
\index{symbind}{function@{f(args)\ \kw{as}\ result}}
\index{symbind}{function@{\kw{post}\ P(args,result)}}
\nrslframebox{Implicit Function Definitions}{
%\RSLatex
%value
%  f: Arguments -> Result
%  f(args) as result
%  post P1(args,result)
%  
%  g: Arguments -~-> Result
%  g(args) as result
%  pre P2(args)
%  post P3(args,result)
%\endRSLatex
\bp
\kw{value}\\
\>f: Arguments {\RIGHTARROW} Result\\
\>f(args) \kw{as} result\\
\>\kw{post} P1(args,result)\\
\>\\
\>g: Arguments {\PARRIGHTARROW} Result\\
\>g(args) \kw{as} result\\
\>\kw{pre} P2(args)\\
\>\kw{post} P3(args,result)
\ep
}
\index{symbind}{function@{\kw{pre}\ P(args)}}
\index{symbind}{function@{f(args)\ \kw{as}\ result}}
\index{symbind}{function@{\kw{post}\ P(args,result)}}

\pt{
\noindent
The symbol {\PARRIGHTARROW} indicates that the function is partial and
thus not defined for all arguments. Partial functions should be
assisted by preconditions stating the criteria for arguments to be
meaningful to the function.}{}
\index{symbind}{functioa@\protect{\underline{\textbf{Function Constructs}}}|)}

\nbbbbb{Other Applicative Expressions}
\label{tseb.rsl.Other Applicative Expressions}

\irsltxt{\texttt{RSL} offers the usual collection of applicative
  constructs that
functional programming languages (\texttt{Standard ML}
\cite{MilnerTofte,MilnerTofte} or \texttt{F\#} \cite{Hansen+Rischel}) offer}

\bbbb{Simple \kw{let} Expressions}
\label{tseb.rsl.Let Expressions}

Simple (i.e., nonrecursive) \textbf{let} expressions:

\index{symbind}{dcoma@\protect{\underline{\textbf{Combinators}}}|(}
\nrslframebox{Let Expressions}{
%\RSLatex
%  let a = &$\mathcal{E}_d$& in &$\mathcal{E}_b$&(a) end&\index{symbind}{dcomb@\protect{\kw{let} pa = e \kw{in} c \kw{end}}}&
%\endRSLatex
\bp
\>\kw{let} a {\EQ} $\mathcal{E}_d$ \kw{in} $\mathcal{E}_b$(a) \kw{end}\index{symbind}{dcomb@\protect{\kw{let} pa = e \kw{in} c \kw{end}}}
\ep

\noindent is an ``expanded'' form of:

%\RSLatex
%  (-\a.&$\mathcal{E}_b$&(a))(&$\mathcal{E}_d$&)
%\endRSLatex
\bp
\>({\LAMBDA}a.$\mathcal{E}_b$(a))($\mathcal{E}_d$)
\ep
}

\nbbbb{Recursive \kw{let} Expressions}

\noindent Recursive \textbf{let} expressions are written as:

\nrslframebox{Recursive \textbf{let} Expressions}{
%\RSLatex
%  let f = -\a:A :- E(f) in B(f,a) end
%
%&is ``the same'' as&:
% 
%  let f = &\kw{Y}&F in B(f,a) end
%
%where:
%
%  F is -\g:--\a:-(E(g)) and YF = F(YF)
%\endRSLatex
\bp
\>\kw{let} f {\EQ} {\LAMBDA}a:A {\RDOT} E(f) \kw{in} B(f,a) \kw{end}\\
\\
is ``the same'' as:\\
 \\
\>\kw{let} f {\EQ} \kw{Y}F \kw{in} B(f,a) \kw{end}\\
\\
where:\\
\\
\>F {\IS} {\LAMBDA}g{\RDOT}{\LAMBDA}a{\RDOT}(E(g)) and YF {\EQ} F(YF)
\ep
}

\nbbbb{Predicative \kw{let} Expressions}

\noindent Predicative \kw{let} expressions:

\nrslframebox{Predicative \kw{let}
  Expressions}{\label{tseb.rsl.Predicative Let Expressions} 
%\RSLatex
%  let a:A :- &$\mathcal{P}$&(a) in $\mathcal{B}$(a) end&\index{symbind}{dcomb@\protect{\kw{let} a:A $\bullet$ P(a) \kw{in} c \kw{end}}}&
%\endRSLatex
\bp
\>\kw{let} a:A {\RDOT} $\mathcal{P}$(a) \kw{in} $\mathcal{B}$(a) \kw{end}\index{symbind}{dcomb@\protect{\kw{let} a:A $\bullet$ P(a) \kw{in} c \kw{end}}}
\ep
}
\noindent express the selection of a value \textsf{a} of type \textsf{A}
which satisfies a predicate \textsf{$\mathcal{P}$(a)} for evaluation in the body \textsf{$\mathcal{B}$(a).}

\nbbbb{Pattern and ``Wild Card'' \kw{let} Expressions}

\noindent \emph{Patterns} and \emph{wild cards} can be used:

\nrslframebox{Patterns}{
%\RSLatex
%  let {a} union s = set in ... end
%  let {a,_} union s = set in ... end
%
%  let (a,b,...,c) = cart in ... end
%  let (a,_,...,c) = cart in ... end
%
%  let <.a.>^&$\ell$& = list in ... end
%  let <.a,_,b.>^&$\ell$& = list in ... end
%
%  let [a+>b] union m = map in ... end
%  let [a+>b,_] union m = map in ... end
%\endRSLatex
\bp
\>\kw{let} {\LBRACE}a{\RBRACE} {\UNION} s {\EQ} set \kw{in} {\DOTDOTDOT} \kw{end}\\
\>\kw{let} {\LBRACE}a,{\UNDERLINE}{\RBRACE} {\UNION} s {\EQ} set \kw{in} {\DOTDOTDOT} \kw{end}\\
\\
\>\kw{let} (a,b,{\DOTDOTDOT},c) {\EQ} cart \kw{in} {\DOTDOTDOT} \kw{end}\\
\>\kw{let} (a,{\UNDERLINE},{\DOTDOTDOT},c) {\EQ} cart \kw{in} {\DOTDOTDOT} \kw{end}\\
\\
\>\kw{let} {\LANGLE}a{\RANGLE}{\CONCAT}$\ell$ {\EQ} list \kw{in} {\DOTDOTDOT} \kw{end}\\
\>\kw{let} {\LANGLE}a,{\UNDERLINE},b{\RANGLE}{\CONCAT}$\ell$ {\EQ} list \kw{in} {\DOTDOTDOT} \kw{end}\\
\\
\>\kw{let} {\LBRACKET}a{\MAPSTO}b{\RBRACKET} {\UNION} m {\EQ} map \kw{in} {\DOTDOTDOT} \kw{end}\\
\>\kw{let} {\LBRACKET}a{\MAPSTO}b,{\UNDERLINE}{\RBRACKET} {\UNION} m {\EQ} map \kw{in} {\DOTDOTDOT} \kw{end}
\ep
}

\nbbb{Conditionals}\LLLL
\label{tseb.rsl.Conditionals}

\pt{
Various kinds of conditional expressions are offered by {\dbrsl}:}{}

\nrslframebox{Conditionals}{
%\RSLatex
%    if b_expr then c_expr else a_expr
%    end&\index{symbind}{dcomb@\protect{\kw{if} b$_e$ \kw{then} c$_c$ \kw{else} c$_a$ \kw{end}}}&
%
%    if b_expr then c_expr end is /* same &as:& */
%      if b_expr then c_expr else skip end
%
%    if b_expr_1 then c_expr_1
%    elsif b_expr_2 then c_expr_2&\index{symbind}{dcomb@\protect{... \kw{elsif} ...}}&
%    elsif b_expr_3 then c_expr_3
%    ...
%    elsif b_expr_n then c_expr_n end
%
%    case expr of&\index{symbind}{dcomb@\protect{\kw{case} b$_e$ \kw{of} pa$_1$ {\RIGHTARROW} c$_1$, ... pa$_n$ {\RIGHTARROW} c$_n$ \kw{end}}}&
%      choice_pattern_1 -> expr_1,
%      choice_pattern_2 -> expr_2,
%      ...
%      choice_pattern_n_or_wild_card -> expr_n
%    end
%\endRSLatex
\bp
\>\>\kw{if} b\_expr \kw{then} c\_expr \kw{else} a\_expr\\
\>\>\kw{end}\index{symbind}{dcomb@\protect{\kw{if} b$_e$ \kw{then} c$_c$ \kw{else} c$_a$ \kw{end}}}\\
\\
\>\>\kw{if} b\_expr \kw{then} c\_expr \kw{end} {\IS} {\LCOMMENT} same as: {\RCOMMENT}\\
\>\>\>\kw{if} b\_expr \kw{then} c\_expr \kw{else} \kw{skip} \kw{end}\\
\\
\>\>\kw{if} b\_expr\_1 \kw{then} c\_expr\_1\\
\>\>\kw{elsif} b\_expr\_2 \kw{then} c\_expr\_2\index{symbind}{dcomb@\protect{... \kw{elsif} ...}}\\
\>\>\kw{elsif} b\_expr\_3 \kw{then} c\_expr\_3\\
\>\>{\DOTDOTDOT}\\
\>\>\kw{elsif} b\_expr\_n \kw{then} c\_expr\_n \kw{end}\\
\\
\>\>\kw{case} expr \kw{of}\index{symbind}{dcomb@\protect{\kw{case} b$_e$ \kw{of} pa$_1$ {\RIGHTARROW} c$_1$, ... pa$_n$ {\RIGHTARROW} c$_n$ \kw{end}}}\\
\>\>\>choice\_pattern\_1 {\RIGHTARROW} expr\_1,\\
\>\>\>choice\_pattern\_2 {\RIGHTARROW} expr\_2,\\
\>\>\>{\DOTDOTDOT}\\
\>\>\>choice\_pattern\_n\_or\_wild\_card {\RIGHTARROW} expr\_n\\
\>\>\kw{end}
\ep
}\LLL

\nbbbb{Operator/Operand Expressions}\LLL
\label{tseb.rsl.Operator/Operand Expressions}

\nrslframebox{Operator/Operand Expressions}{
%\RSLatex
%  <.Expr.> ::= 
%         <.Prefix_Op.> <.Expr.>
%        | <.Expr.> <.Infix_Op.> <.Expr.>
%        | <.Expr.> <.Suffix_Op.> 
%        | ...
%  <.Prefix_Op.> ::= 
%        - | ~ | union | inter | card | len | inds | elems | hd | tl | dom | rng
%  <.Infix_Op.> ::= 
%         = | ~= | is | + | - | * | ** | / | < | <= | >= | > | /\ | \/ | => 
%        | isin | ~isin | union | inter | \ | << | <<= | >>= | >> | ^ | !! | #                    
%  <.Suffix_Op.> ::= ! 
%\endRSLatex
\bp
\>{\LANGLE}Expr{\RANGLE} ::{\EQ} \\
\>\>\>\>\ {\LANGLE}Prefix\_Op{\RANGLE} {\LANGLE}Expr{\RANGLE}\\
\>\>\>\>{\BAR} {\LANGLE}Expr{\RANGLE} {\LANGLE}Infix\_Op{\RANGLE} {\LANGLE}Expr{\RANGLE}\\
\>\>\>\>{\BAR} {\LANGLE}Expr{\RANGLE} {\LANGLE}Suffix\_Op{\RANGLE} \\
\>\>\>\>{\BAR} {\DOTDOTDOT}\\
\>{\LANGLE}Prefix\_Op{\RANGLE} ::{\EQ} \\
\>\>\>\>{\MINUS} {\BAR} {\SIM} {\BAR} {\UNION} {\BAR} {\INTER} {\BAR} \kw{card} {\BAR} \kw{len} {\BAR} \kw{inds} {\BAR} \kw{elems} {\BAR} \kw{hd} {\BAR} \kw{tl} {\BAR} \kw{dom} {\BAR} \kw{rng}\\
\>{\LANGLE}Infix\_Op{\RANGLE} ::{\EQ} \\
\>\>\>\>\ {\EQ} {\BAR} {\NOTEQ} {\BAR} {\IS} {\BAR} {\PLUS} {\BAR} {\MINUS} {\BAR} {\AST} {\BAR} {\UPARROW} {\BAR} / {\BAR} {\LT} {\BAR} {\LEQ} {\BAR} {\GEQ} {\BAR} {\GT} {\BAR} {\WEDGE} {\BAR} {\VEE} {\BAR} {\DBLRIGHTARROW} \\
\>\>\>\>{\BAR} {\ISIN} {\BAR} {\NOTISIN} {\BAR} {\UNION} {\BAR} {\INTER} {\BAR} {\SETMINUS} {\BAR} {\SUBSET} {\BAR} {\SUBSETEQ} {\BAR} {\SUPSETEQ} {\BAR} {\SUPSET} {\BAR} {\CONCAT} {\BAR} {\DAGGER} {\BAR} {\HASH}\ \ \ \ \ \ \ \ \ \ \ \ \ \ \ \ \ \ \ \ \\
\>{\LANGLE}Suffix\_Op{\RANGLE} ::{\EQ} ! 
\ep
}

\nbbbbb{Imperative Constructs}


\irsltxt{\texttt{RSL} offers the usual collection of imperative
  constructs that imperative programming languages (\texttt{Java}
  \cite{JavaLangSpec96,sestoft-Java} or \texttt{Oberon} (!) \cite{Wirth88b}) offer}


\label{tseb.rsl.Imperative Constructs}
\bbbb{Statements and State Changes}

\pt{
Often, following the RAISE method, software development starts with
highly abstract-applicative constructs 
which, through stages of refinements, are
turned into concrete and imperative constructs. 
Imperative constructs are thus inevitable in {\dbrsl}.}{}

\nrslframebox{Statements and State Change}{
%\RSLatex
%  Unit&\index{symbind}{typeeb@\kw{Unit}}&
%value
%  stmt: Unit -> Unit
%  stmt()&\index{symbind}{functionb@f()}& 
%\endRSLatex
\bp
\>\kw{Unit}\index{symbind}{typeeb@\kw{Unit}}\\
\kw{value}\\
\>stmt: \kw{Unit} {\RIGHTARROW} \kw{Unit}\\
\>stmt()\index{symbind}{functionb@f()} 
\ep
\begin{itemize}
\item Statements accept no arguments.
\item Statement execution changes the state (of declared variables).
\item \kw{Unit} {\RIGHTARROW} \kw{Unit} designates a function from
  states to states.
\item Statements, \textsf{stmt}, denote state-to-state changing functions.
\item Writing () as ``only'' arguments to a function ``means'' that ()
  is an argument of type \kw{Unit}.
\end{itemize}
}

\nbbbb{Variables and Assignment}
\label{tseb.rsl.Variables and Assignment}

\nrslframebox{Variables and Assignment}{
%\RSLatex
%   0. variable v:Type := expression&\index{symbind}{dcomb@\protect{\kw{variable} v:Type := expression}}&
%   1. v := expr&\index{symbind}{dcomb@\protect{v := expression}}&
%\endRSLatex
\bp
\>\ 0. \kw{variable} v:Type :{\EQ} expression\index{symbind}{dcomb@\protect{\kw{variable} v:Type := expression}}\\
\>\ 1. v :{\EQ} expr\index{symbind}{dcomb@\protect{v := expression}}
\ep
}


\nbbbb{Statement Sequences and \kw{skip}}
\label{tseb.rsl.Statement Sequences and skip}

\pt{
Sequencing is expressed using the `;' operator. \kw{skip} is the empty
statement having no value or side-effect.}{}

\nrslframebox{Statement Sequences and \kw{skip}}{
%\RSLatex
%   2. skip&\index{symb}{dcomb@\protect{\kw{skip}}}&
%   3. stm_1;stm_2;...;stm_n&\index{symb}{dcomb@\protect{stm$_1$;stm$_2$;...;stm$_n$;}}&
%\endRSLatex
\bp
\>\ 2. \kw{skip}\index{symb}{dcomb@\protect{\kw{skip}}}\\
\>\ 3. stm\_1;stm\_2;{\DOTDOTDOT};stm\_n\index{symb}{dcomb@\protect{stm$_1$;stm$_2$;...;stm$_n$;}}
\ep
}

\nbbbb{Imperative Conditionals}
\label{tseb.rsl.Imperative Conditionals}

\nrslframebox{Imperative Conditionals}{
%\RSLatex
%   4. if expr then stm_c else stm_a end&\index{symbind}{dcomb@\protect{\kw{if} b$_e$ \kw{then} c$_c$ \kw{else} c$_a$ \kw{end}}}&
%   5. case e of: p_1->S_1(p_1),...,p_n->S_n(p_n) end&\index{symbind}{dcomb@\protect{\kw{case} b$_e$ \kw{of} pa$_1$ {\RIGHTARROW} c$_1$, ... pa$_n$ {\RIGHTARROW} c$_n$ \kw{end}}}&
%\endRSLatex
\bp
\>\ 4. \kw{if} expr \kw{then} stm\_c \kw{else} stm\_a \kw{end}\index{symbind}{dcomb@\protect{\kw{if} b$_e$ \kw{then} c$_c$ \kw{else} c$_a$ \kw{end}}}\\
\>\ 5. \kw{case} e \kw{of}: p\_1{\RIGHTARROW}S\_1(p\_1),{\DOTDOTDOT},p\_n{\RIGHTARROW}S\_n(p\_n) \kw{end}\index{symbind}{dcomb@\protect{\kw{case} b$_e$ \kw{of} pa$_1$ {\RIGHTARROW} c$_1$, ... pa$_n$ {\RIGHTARROW} c$_n$ \kw{end}}}
\ep
}

\nbbbb{Iterative Conditionals}
\label{tseb.rsl.Iterative Conditionals}

\nrslframebox{Iterative Conditionals}{
%\RSLatex
%   6. while expr do stm end&\index{symbind}{dcomb@\protect{\kw{while} b$e$ \kw{do} stm \kw{end}}}&
%   7. do stmt until expr end&\index{symbind}{dcomb@\protect{\kw{do} stmt \kw{until} b$e$ \kw{end}}}&
%\endRSLatex
\bp
\>\ 6. \kw{while} expr \kw{do} stm \kw{end}\index{symbind}{dcomb@\protect{\kw{while} b$e$ \kw{do} stm \kw{end}}}\\
\>\ 7. \kw{do} stmt \kw{until} expr \kw{end}\index{symbind}{dcomb@\protect{\kw{do} stmt \kw{until} b$e$ \kw{end}}}
\ep
}

\bbbb{Iterative Sequencing}
\label{tseb.rsl.Iterative Sequencing}

\nrslframebox{Iterative Sequencing}{
%\RSLatex
%   8. for e in list_expr :- P(b) do S(b) end&\index{symbind}{dcomb@\protect{\kw{for} e \kw{in} list$_{expr}$ {$\bullet$} P(b) \kw{do} stm(e) \kw{end}}}&
%\endRSLatex
\bp
\>\ 8. \kw{for} e \kw{in} list\_expr {\RDOT} P(b) \kw{do} S(b) \kw{end}\index{symbind}{dcomb@\protect{\kw{for} e \kw{in} list$_{expr}$ {$\bullet$} P(b) \kw{do} stm(e) \kw{end}}}
\ep
}

\index{symbind}{dcoma@\protect{\underline{\textbf{Combinators}}}|)}

\nbbbbb{Process Constructs}
\label{tseb.rsl.Process Constructs}

\irsltxt{\texttt{RSL} offers several of the constructs that
  \texttt{CS} \citecsp\ offers}

\bbbb{Process Channels}
\label{tseb.rsl.Process Channels}
\index{symbind}{processa@\protect{\underline{\textbf{Process Constructs}}}|(}


As for channels we deviate from common \texttt{RSL} \cite{RSL} in that
we directly \sfsl{declare} channels -- and not via common \texttt{RSL}
\sfsl{objects} etc.

Let \textsf{A} and \textsf{B}  stand for two types of (channel)
messages and \textsf{i:KIdx} for channel array indexes, then:
\nrslframebox{Process Channels}{
%\RSLatex
%  channel c:A&\index{symbind}{processb@\protect{\kw{channel} c:T}}&
%  channel { k[i]:B :- i:Idx }&\index{symbind}{processb@\protect{\kw{channel}\,$\{$k[i]:T$\bullet$i:Idx$\}$}}&
%  channel { k[i,j,...,k]:B :- i:Idx,j:Jdx,...,k:Kdx }
%\endRSLatex
\bp
\>\kw{channel} c:A\index{symbind}{processb@\protect{\kw{channel} c:T}}\\
\>\kw{channel} {\LBRACE} k{\LBRACKET}i{\RBRACKET}:B {\RDOT} i:Idx {\RBRACE}\index{symbind}{processb@\protect{\kw{channel}\,$\{$k[i]:T$\bullet$i:Idx$\}$}}\\
\>\kw{channel} {\LBRACE} k{\LBRACKET}i,j,{\DOTDOTDOT},k{\RBRACKET}:B {\RDOT} i:Idx,j:Jdx,{\DOTDOTDOT},k:Kdx {\RBRACE}
\ep
}

\pt{
\noindent declare a channel, c, and a set (an array)
of channels, k[i], capable of 
communicating values of the designated types (\textsf{A} and
\textsf{B}).}{} 
 
\nbbbb{Process Composition}
\label{tseb.rsl.Process Composition}

\begynd
\pind Let \textsf{P} and \textsf{Q} stand for names of process functions,
\pind i.e., of functions which express willingness to engage in input and/or
      output events, 
\pind thereby communicating over declared channels.
\pind Let \textsf{P()} and \textsf{Q} stand for process expressions, then:
\afslut

\nrslframebox{Process Composition}{
%\RSLatex
%  P || Q    Parallel composition&\index{symbind}{processb@\protect{p$_i$\protect{\PARL}$p_j$}}&     
%  P |=| Q    Nondeterministic external choice (either/or)&\index{symbind}{processb@\protect{p$_i$\protect{\DETCHOICE}$p_j$}}&    
%  P |^| Q    Nondeterministic internal choice (either/or)&\index{symbind}{processb@\protect{p$_i$\protect{\NONDETCHOICE}$p_j$}}&  
%  P ++ Q     Interlock parallel composition&\index{symbind}{processb@\protect{p$_i$\protect{\TIE}$p_j$}}&    
%\endRSLatex
\bp
\>P {\PARL} Q\ \ \ \ Parallel composition\index{symbind}{processb@\protect{p$_i$\protect{\PARL}$p_j$}}\ \ \ \ \ \\
\>P {\DETCHOICE} Q\ \ \ \ Nondeterministic external choice (either/or)\index{symbind}{processb@\protect{p$_i$\protect{\DETCHOICE}$p_j$}}\ \ \ \ \\
\>P {\NONDETCHOICE} Q\ \ \ \ Nondeterministic internal choice (either/or)\index{symbind}{processb@\protect{p$_i$\protect{\NONDETCHOICE}$p_j$}}\ \ \\
\>P {\TIE} Q\ \ \ \ \ Interlock parallel composition\index{symbind}{processb@\protect{p$_i$\protect{\TIE}$p_j$}}\ \ \ \ 
\ep
}

\pt{
\noindent express the parallel (\PARL) of two processes, or the
nondeterministic choice between two processes: either external
(\DETCHOICE) or internal (\NONDETCHOICE). The interlock (\TIE)
composition expresses that the two processes  
are forced to communicate only with one another,
until one of them terminates.
}{} 

\nbbbb{Input/Output Events}
\label{tseb.rsl.Input/Output Events}

Let \textsf{c, k[i]} and \textsf{e} designate channels of type
\textsf{A} and \textsf{B}, then:  

\nrslframebox{Input/Output Events}{
%\RSLatex
%  c ?, k[i] ?      Input&\index{symbind}{processb@\protect{c\,?}}\index{symbind}{processb@\protect{k[i]\,?}}&
%  c ! e, k[i] ! e  Output&\index{symbind}{processb@\protect{c\,"!\,e}}\index{symbind}{processb@\protect{k[i]\,"!\,e}}&
%\endRSLatex
\bp
\>c ?, k{\LBRACKET}i{\RBRACKET} ?\ \ \ \ \ \ Input\index{symbind}{processb@\protect{c\,?}}\index{symbind}{processb@\protect{k[i]\,?}}\\
\>c ! e, k{\LBRACKET}i{\RBRACKET} ! e\ \ Output\index{symbind}{processb@\protect{c\,"!\,e}}\index{symbind}{processb@\protect{k[i]\,"!\,e}}
\ep
}

\noindent 
\begynd
\pind expresses the willingness of a process to engage in an event that 
\begynd
\pind ``reads'' an input, respectively
\pind  ``writes'' an output.
\afslut
\afslut

\nbbbb{Process Definitions}
\label{tseb.rsl.Process Definitions}

The below signatures are just examples. They emphasise that process
functions must somehow express, in their signature, via which channels
they wish to engage in input and output events. 


\nrslframebox{Process Definitions}{
%\RSLatex
%value
%  P: Unit -> in c out k[i]
%  Unit&\index{symbind}{aaab@\kw{Unit}}\index{symbind}{processb@\protect{P:\,\kw{Unit}\,{\RIGHTARROW}\,\kw{in}\,c\,\kw{out}\,k[i]\,\kw{Unit}}}\index{symbind}{aaaa@\protect{\underline{\textbf{Literals}}}|)}&
%  Q: i:KIdx ->  out c in k[i] Unit&\index{symbind}{processb@\protect{Q:\,i:KIdx}\,{\RIGHTARROW}\,\kw{out}\,c\,\kw{in}\,k[i]\,\kw{Unit}}&
%
%  P() is ... c ? ... k[i] ! e ...
%  Q(i) is ... k[i] ? ... c ! e ...
%\endRSLatex
\bp
\kw{value}\\
\>P: \kw{Unit} {\RIGHTARROW} \kw{in} c \kw{out} k{\LBRACKET}i{\RBRACKET}\\
\>\kw{Unit}\index{symbind}{aaab@\kw{Unit}}\index{symbind}{processb@\protect{P:\,\kw{Unit}\,{\RIGHTARROW}\,\kw{in}\,c\,\kw{out}\,k[i]\,\kw{Unit}}}\index{symbind}{aaaa@\protect{\underline{\textbf{Literals}}}|)}\\
\>Q: i:KIdx {\RIGHTARROW}\ \ \kw{out} c \kw{in} k{\LBRACKET}i{\RBRACKET} \kw{Unit}\index{symbind}{processb@\protect{Q:\,i:KIdx}\,{\RIGHTARROW}\,\kw{out}\,c\,\kw{in}\,k[i]\,\kw{Unit}}\\
\\
\>P() {\IS} {\DOTDOTDOT} c ? {\DOTDOTDOT} k{\LBRACKET}i{\RBRACKET} ! e {\DOTDOTDOT}\\
\>Q(i) {\IS} {\DOTDOTDOT} k{\LBRACKET}i{\RBRACKET} ? {\DOTDOTDOT} c ! e {\DOTDOTDOT}
\ep
}
\index{symbind}{processa@\protect{\underline{\textbf{Process Constructs}}}|)}

\noindent The process function definitions (i.e., their bodies)
express possible events. 

\nbbbbb{\dbrsl\ Module Specifications}

\begynd
\pind We shall not include coverage nor use of the
      \texttt{RSL} module concepts of 
\begynd
\pind \sfsl{schemes, 
\pind classes} and
\pind \sfsl{objects}.
\afslut
\afslut

\nbbbbb{Simple {\dbrsl} Specifications}
\label{tseb.rsl.Simple RSL Specifications}

\pt{
Often, we do not want to encapsulate small specifications in schemas,
classes, and objects, as is often done in {\dbrsl}. 
An {\dbrsl} specification is simply a sequence of one or more types,
values (including functions), variables, channels and axioms:}{}

\nrslframebox{Simple {\dbrsl} Specifications}{
%\RSLatex
%  type
%    ...
%  variable
%    ...
%  channel
%    ...
%  value
%    ...
%  axiom
%    ...
%\endRSLatex
\bp
\>\kw{type}\\
\>\>{\DOTDOTDOT}\\
\>\kw{variable}\\
\>\>{\DOTDOTDOT}\\
\>\kw{channel}\\
\>\>{\DOTDOTDOT}\\
\>\kw{value}\\
\>\>{\DOTDOTDOT}\\
\>\kw{axiom}\\
\>\>{\DOTDOTDOT}
\ep

}

\nbbbbb{\rslplus: Extended \texttt{RSL}}\label{chap2.tex.rsltext}

Section\,\vref{tseb.rsl.Types} covered standard \texttt{RSL} types. To them we now
add two new types: Type names and \rsltxt.

We refer  to  Sect.\,\ref{An RSL Extension}
(the \sfsl{An RSL Extension} box) Page\,\pageref{An RSL Extension}  
for a first introduction to extended \texttt{RSL}.

\dbeat{%%%%%%%%%%%%%%%%
For uses of type name type and type name values and for the
``generation'' of \rsltxt\ to
Sect.\,\ref{sec:Analyse Composite Parts} (the \texttt{de\-term\-ine\_\-Car\-te\-si\-an\_\-parts} function),
Sect.\,\ref{calc-Cartesian-parts} (the \texttt{calc\_\-Car\-te\-si\-an\_\-parts} description prompt),  
%Sect.\,\ref{sec:Analyse Same Sorts Part Sets} (the \texttt{de\-term\-ine\_\-same\_\-sort\_\-parts\_\-set} prompt),
%Sect.\,\ref{sec:Analyse Alternative Sorts Part Sets} (the \texttt{de\-term\-ine\_\-al\-tern\-a\-tive\_\-sorts\_\-part\_\-set} function),
Sect.\,\ref{descr.Calc-Sgl-Sort.Sets} (the \texttt{calc\_\-single\_\-sort\_\-parts\_\-sort} description prompt), and
Sect.\,\ref{descr.Calc-Alt-Sort.Sets} (the \texttt{calc\_\-al\-tern\-a\-tive\_\-sort\_\-parts\_\-sort} prompt).
}%%%%%%%%%%%%%%%%%%%%%


\bbbb{Type Names and Type Name Values}

\bbb{Type Names}

\begin{itemize}
\item Let \textsf{T} be a type name.
\item Then \textsf{$\eta$T} is a type name
  value.\index{symbind}{aaab@$\eta$}\index{symbind}{utypeen@$\eta$T}
\item And \textsf{$\eta\mathbb{T}$} is the type of type names.
\end{itemize}

\bbb{Type Name Operations}

\begin{itemize}
\item $\eta$ can be considered an operator.
\begin{itemize} 
\item It (prefix) applies, then, to type (\textsf{T})
      identifiers and yields the name of that type.
\item Two type names, \textsf{nT$_i$, nT$_j$}, can be compared for
      equality: \textsf{nT$_i$ = nT$_j$} iff $i=j$.
\end{itemize}
\item It, vice-versa, suffix  applies to type name (\textsf{nT})\index{symbind}{aaab@$\eta$}
      identifiers and yields the name, \textsf{T}, of that type: \textsf{nT$\eta$ = T}.
\end{itemize}

\nbbbb{\rsltxt}
\bbb{The \rsltxt\ Type and Values}

\begin{itemize}
\item \rsltxt\ is the type name for ordinary, non-extended \texttt{RSL} texts.\index{symbind}{aaab@\rsltxt}
\end{itemize}

\noindent
\begynd
\pind We shall not here give a syntax for ordinary, non-extended
      \texttt{RSL} texts -- but refer to \cite{RSL}.
\afslut

\bbb{\rsltxt\ Operations}

\begin{itemize}
\item {\rsltxt}s can be compared and concatenated:
\begin{itemize}
\item \textsf{rsl-text$_a$}{\EQ}\textsf{rsl-text$_b$}\index{symbind}{aaab@{\EQ}}
\item \textsf{rsl-text$_a$}{\CONCAT}\textsf{rsl-text$_b$}\index{symbind}{aaab@{\CONCAT}}
\end{itemize}
\end{itemize}

\noindent
The {\CONCAT} operator thus also applies, besides, lists (tuples), to
\texttt{RSL} texts -- treating \texttt{RSL} texts as (if they were)
lists of characters.


\nbbbbb{Distributive Clauses}

We clarify:

\bbbb{Over Simple Values}

%\RSLatex
%   &$\oplus$& { a | a:A :- a isin {a_1,a_2,...,a_n} } =
%      if n>0 then a_1&$\oplus$&a_1&$\oplus$&...&$\oplus$&a_n else
%         case &$\oplus$& of
%            + -> 0, - -> 0, * -> 1, / -> chaos, union -> {}, inter -> {}, ...
%         end end 
%\endRSLatex 
\bp
\>\ $\oplus$ {\LBRACE} a {\BAR} a:A {\RDOT} a {\ISIN} {\LBRACE}a\_1,a\_2,{\DOTDOTDOT},a\_n{\RBRACE} {\RBRACE} {\EQ}\\
\>\>\>\kw{if} n{\GT}0 \kw{then} a\_1$\oplus$a\_1$\oplus${\DOTDOTDOT}$\oplus$a\_n \kw{else}\\
\>\>\>\>\ \kw{case} $\oplus$ \kw{of}\\
\>\>\>\>\>\>{\PLUS} {\RIGHTARROW} 0, {\MINUS} {\RIGHTARROW} 0, {\AST} {\RIGHTARROW} 1, / {\RIGHTARROW} \kw{chaos}, {\UNION} {\RIGHTARROW} {\LBRACE}{\RBRACE}, {\INTER} {\RIGHTARROW} {\LBRACE}{\RBRACE}, {\DOTDOTDOT}\\
\>\>\>\>\ \kw{end} \kw{end} 
\ep

%\RSLatex
%   (f_1,f_2,...,f_n)(a) is if n>0 then (f_1(a),f_2(a),...,f_n(a)) else chaos end
%\endRSLatex 
\bp
\>\ (f\_1,f\_2,{\DOTDOTDOT},f\_n)(a) {\IS} \kw{if} n{\GT}0 \kw{then} (f\_1(a),f\_2(a),{\DOTDOTDOT},f\_n(a)) \kw{else} \kw{chaos} \kw{end}
\ep

\bbbb{Over Processes}

%\RSLatex
%   || { p(i) | i:I :- i isin {i_1,i_2,...,i_n} } is if n>0 then p(i_1)||p(i_2)||...||p(i_n) else () end
%  |^| { p(i) | i:I :- i isin {i_1,i_2,...,i_n} } is if n>0 then p(i_1)|^|p(i_2)|^|...|^|p(i_n) else () end
%  |=| { p(i) | i:I :- i isin {i_1,i_2,...,i_n} } is if n>0 then p(i_1)|=|p(i_2)|=|...|=|p(i_n) else () end
%\endRSLatex 
\bp
\>\ {\PARL} {\LBRACE} p(i) {\BAR} i:I {\RDOT} i {\ISIN} {\LBRACE}i\_1,i\_2,{\DOTDOTDOT},i\_n{\RBRACE} {\RBRACE} {\IS} \kw{if} n{\GT}0 \kw{then} p(i\_1){\PARL}p(i\_2){\PARL}{\DOTDOTDOT}{\PARL}p(i\_n) \kw{else} () \kw{end}\\
\>{\NONDETCHOICE} {\LBRACE} p(i) {\BAR} i:I {\RDOT} i {\ISIN} {\LBRACE}i\_1,i\_2,{\DOTDOTDOT},i\_n{\RBRACE} {\RBRACE} {\IS} \kw{if} n{\GT}0 \kw{then} p(i\_1){\NONDETCHOICE}p(i\_2){\NONDETCHOICE}{\DOTDOTDOT}{\NONDETCHOICE}p(i\_n) \kw{else} () \kw{end}\\
\>{\DETCHOICE} {\LBRACE} p(i) {\BAR} i:I {\RDOT} i {\ISIN} {\LBRACE}i\_1,i\_2,{\DOTDOTDOT},i\_n{\RBRACE} {\RBRACE} {\IS} \kw{if} n{\GT}0 \kw{then} p(i\_1){\DETCHOICE}p(i\_2){\DETCHOICE}{\DOTDOTDOT}{\DETCHOICE}p(i\_n) \kw{else} () \kw{end}
\ep

% LocalWords:  rsl ch pecification anguage Asger Eir's Bool Int infset inflist
% LocalWords:  mk sel aa aaa abc cf eg av bv expr ai zk bj isin ei ej lst len
% LocalWords:  dom pre va vb vc vd decompostion hd tl inds elems fq iq bn rng
% LocalWords:  BNF obs gen bs dm al YF elsif exprt Op stm stmt KIdx emphasise
% LocalWords:  nonmembership nonequal nonrecursive Booleans reals symbind aaab

% LocalWords:  typeea typeeb utypeec chematic utypeed loga barithb baritha seta
% LocalWords:  setb setc cartesiaa cartesian klista klistb symbin maa un vn th
% LocalWords:  mapa mapb mapc aaaa Indices indices functioa args DValueExpr
% LocalWords:  ValueAndStateChangeClause dcoma Combinators dcomb symb processa
% LocalWords:  processb Nondeterministic nondeterministic bb subtypes
%%  LocalWords:  cardinality subtype Redefinitions Applicative Idx de
%%  LocalWords:  applicative Jdx Kdx booleanop schemas nnnmodulea Hs
%%  LocalWords:  nnnmoduleb Etcetera Boskoop eger ural acters FO Vu

%%  \nbbbbb{Types and Values}\label{chap2.tex.Types and Values}

%%  \nbbbbb{The Functional RSL$^+$}\label{chap2.tex.The Functional RSL-plus}

%%  \nbbbbb{The Imperative RSL$^+$}\label{chap2.tex.The Imperative RSL}

%%  \nbbbbb{The RSL/CSP}\label{chap2.tex.The RSL/CSP}
%%  LocalWords:  modelling Formalisation Cartesians Ju Fu Su swu spu
%%  LocalWords:  svu sju sfu ssu ENat bbb pdefind sentential nT iff
%%  LocalWords:  Arithmetics ine te si calc tive arg UI
