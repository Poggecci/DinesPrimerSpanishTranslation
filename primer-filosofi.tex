 

%%%%%%%%%%%%%%%%%%%%%%%%%%%%%%%%%%%%%%%%%%%%%%%%%%%%%%%%%%%%%%%%%%%%
%%%%%%%% Released for translation 11.4.2023, 13:00 %%%%%%%%%%%%%%%%%
%%%%%%%% Re-released 22.6.2023, 14:45 %%%%%%%%%%%%%%%%%%%%%%%%%%%%%%
%%%%%%%%%%%%%%%%%%%%%%%%%%%%%%%%%%%%%%%%%%%%%%%%%%%%%%%%%%%%%%%%%%%%

\pos{\nbbbbbb{Kai S{\o}rlander's Philosophy}}{}\label{chap2.tex.Philosophy}\label{chapter:Philosophy}
\minitoc

\index{pconind}{philosophy!S{\o}rlander's|(}

\bookdefn{Philosophy}{\ysfchgii{Philosophy }\pdefindex{philosophy}\footnote{\HHHH From Greek:
      $\phi\iota\lambda{o}\sigma{o}\phi\iota\alpha$, philosophia, 'love of
      wisdom'} % φιλοσοφία
\begynd
\pind is the study of general and fundamental questions, such as those about
\pos{}{\begin{multicols}{3}}
\begynd
\pind existence,
\pind reason,
\pind knowledge,
\pind values,
\pind mind, and
\pind language\pos{\footnote{\LLLL Many of the `definitions' in this \primer\ are
      in the style used in philosophy. They are not in the
      `precise' style commonly used in mathematics and computer
      science. You may wish to call them
      \textbf{characterisation}s. In mathematics and computer science
      the definer usually has a formal base on which to build. In
      domain science \& engineering we do not have a formal base, we
      have the ``material'' world of natural and man-made
      phenomena.}}{}\dbsquare
\afslut
\pos{}{\end{multicols}}
\afslut
}

\pof{\nbbbbb{Introduction}}{}

\pof{}{\noindent}
\begynd
\pind In philosophising questions are asked.
\pind One does not necessarily  get answers to these questions.
\begynd
\pind Questions are examined. 
\pind Light is thrown on the questions and their derivative questions.
\afslut
      
\pind Philosophy is man's endeavour, our quest, \nyl for uncovering the
      necessary characteristics of our world and \nyl
      our situation as humans \ysfchg{in } that world.

\pind We shall focus on the issues of metaphysics. \index{pconind}{metaphysics}%
\afslut

\mnewfoil
     
\begynd
\pind The treatment in this \pof{chapter}{paper}
\begynd
\pind is based very much on the works of \nyl 
      the Danish philosopher \bbcolor{\textsf{Kai S{\o}rlander}} (1944) \nyl
      \cite[1994--2022]{kaisorlander1994,kaisorlander1997,kaisorlander2002,kaisorlander2016,kaisorlander2022}
\pind both in contrast to and inspired by \nyl  the German philosopher
      \bbcolor{\textsf{Immanuel Kant}} (1724--1804) \cite{Kant1992}.
\afslut
\pind In 2023, in collaboration with \ysfchg{Kai } S{\o}rlander,
      \cite{kaisorlander2022} \ysfchg{was translated } into English \cite{kaisorlander2023}.
\afslut

\mnewfoil
      
\begynd
\pind The reason why \ysfchg{we}, as computer scientist\ysfchg{s, should be } interested in philosophy,
\begynd
\pind is that philosophers over more than 2500
      years{\footnote{\LLLL -- \label{citations-1}
      starting, one could claim, with:\scriptsize\footnotesize
\begin{multicols}{2}
\begin{itemize}
\item \sfsl{Thales of Milet} 624--545 [everything originates from
      {\texttt{water}}] \cite{Thales:Dines};
\item \sfsl{Anaximander} 610--546 [{\texttt{`apeiron'}} (the
      `un\--dif\-fer\-ren\-ti\-a\-ted', `the unlimited') is the
      origin] \cite{Anaximander:Dines};
\item \sfsl{Anaximenes} 586--526 [{\texttt{air}} is the basis for
      everything] \cite{Anaximenes:Dines};
\item  \sfsl{Heraklit of Efesos} 540--480 [\texttt{fire} is the
      basis and everything in nature {\texttt{is in never-ending
          ``battle''}}]  \cite{Heraklit:Dines};
\item \sfsl{Empedokles} 490--430 [there are four base elements:
      {\texttt{fire, water, air}} and {\texttt{soil}}] \cite{Empedokles:Dines};
\item \sfsl{Parminedes} 515--470 [everything that exists is \texttt{eternal and
          immutable}] \cite{Parminedes:Dines};
\item \sfsl{Demokrit} 460--370 [all is built from {\texttt{atoms}}] \cite{Demokrit:Dines};
\item the Sophists: Protagoras, Gorgias (fifth and fourth centuries BC),
\item \sfsl{Socrates} (470--399) \cite{Socrates:Dines},
\item \sfsl{Plato} (424--347) \cite{Plato:Dines},
\item \sfsl{Aristotle} (384--322) \cite{Aristotle:Dines-x},
\item etcetera.
\end{itemize}
\noindent
After  more than 1800 years came
\begin{itemize}
\item \sfsl{Ren{\'e} Descartes} (1596--1650) \cite{Descartes:Dines},
\item \sfsl{Baruch Spinoza} (1632--1677) \cite{Spinoza:Dines},
\item \sfsl{John Locke} (1632--1704) \cite{Locke:Dines},
\item \sfsl{George Berkeley} (1685--1753) \cite{Berkeley:Dines},
\item \sfsl{David Hume} (1711--1776) \cite{Hume:Dines},
\item \sfsl{Immanuel Kant} (1724--1804) \cite{Kant:Dines},
\item \sfsl{Johan Gottlieb Fichte} (1762--1814) \cite{Fichte:Dines},
\item \sfsl{Georg Wilhelm Friedrich Hegel} (1770--1831) \cite{Hegel:Dines},
\item \sfsl{Friedrich Wilhelm Schelling} (1775--1864) \cite{Schelling:Dines},
\item \sfsl{Edmund Husserl} (1859--1938) \cite{Husserl:Dines},
\item \sfsl{Bertrand Russell} (1872--1970) \cite{Rus12,Russell1910-1913,Russell1905,Russell1919},
\item \sfsl{Ludwig Wittgenstein} (1889--1951) \cite{Wittgenstein21,Wittgenstein58},
\item \sfsl{Martin Heidegger} (1889--1976) \cite{Heidegger27},
\item \sfsl{Rudolf Karnap} (1891--1970) \cite{RudolfCarnap1928,Car37,Car42},
\item \sfsl{Karl Popper} (1902--1994) \cite{popper-tlosd59,popper-car-tgask63},
\item etcetera.
\end{itemize}
\end{multicols}
\noindent
(This list is ``pilfered''
      from \cite[Pages 33--127]{kaisorlander2016}.) \label{citations-2}
      \cite{kaisorlander2016} presents an analysis of the metaphysics
      of these philosophers. Except for those of Russell, Wittgenstein,
      Karnap and Popper, these references are just that.\normalsize}}{} \nyl  have
      thought about existence: \nyl why is the world as it is -- \nyl and
      computer scientists, \nyl like other scientists (notably
      physicists and economists), \nyl repeatedly model fragments of
      the world;
\mnewfoil
\pind and the reason why \ysfchg{we } focus on Kai S{\o}rlander, \nyl is
      that his philosophy addresses issues \nyl that are crucial to our
      understanding \nyl \ysfchg{of } how we must proceed when modelling domains --
      \nyl and,
      \ysfchg{we } think, in a way that helps us model domains \nyl with a high
      assurance that our models are reasonable, \nyl can withstand close scrutiny.
\pind Kai S{\o}rlander thinks and writes logically, rationally.
\pind The area of his philosophy that \ysfchg{we are } focusing on here is metaphysics.
\afslut
\afslut
%%  LocalWords:  Milet Anaximander apeiron un dif ren Anaximenes Kai
%%  LocalWords:  Heraklit Efesos Empedokles Parminedes Demokrit Johan
%%  LocalWords:  Gorgias etcetera Immanuel Gottlieb Georg Friedrich
%%  LocalWords:  Karnap rlander modelling phenomen ShaoFa mov ese

\pos{
\nbbbb{Metaphysics} \index{pconind}{metaphysics|(}%

\begynd
\pind The branch of philosophy that we are focusing on \nyl  is referred to
      as metaphysics.
\pos{\pind To explain that concept \ysfchg{we } quote from \wiki:}{}
\afslut

\sl ``Metaphysics is the branch of philosophy that studies the fundamental
nature of reality, the first principles of being, identity and change,
space and time, causality, necessity, and
possibility.%
\footnote{www.\-en\-cy\-clo\-pe\-di\-a.\-com/\-phi\-lo\-so\-phy-\-and-\-re\-li\-gion/\-phi\-lo\-so\-phy/\-phi\-lo\-so\-phy-\-terms-\-and-\-con\-cepts/\-me\-ta\-phy\-sics}
It includes 
questions about the nature of consciousness and the relationship
between mind and matter, between substance and attribute, and between
potentiality and actuality.\footnote{Metaphysics. American Heritage
  Dictionary of the English Language (5th ed.). 2011.} The word
``metaphysics'' comes from two 
Greek words that, together, literally mean ``after or behind or among
[the study of] the natural''. It has been suggested that the term might
have been coined by a first century editor who assembled various
small selections of Aristotle's works into the treatise we now know by
the name Metaphysics ($\mu\epsilon\tau\alpha$ $\tau\alpha$
$\phi\upsilon\sigma\iota\kappa\alpha$\dbeat{μετὰ τὰ φυσικά}, meta ta
  physika, \ysfchg{literally } 'after the 
Physics ', another of Aristotle's works) \cite{CohenSMarc2018}.

Metaphysics studies questions related to what it is for something to
exist and what types of existence there are. Metaphysics seeks to
answer, in an abstract and fully general manner, the
questions:\footnote{What is it (that is, whatever it is that there is)
  like? Hall, Ned (2012). "David Lewis's Metaphysics". In Edward
  N. Zalta (ed.). The Stanford Encyclopedia of Philosophy (Fall 2012
  ed.). Center for the Study of Language and Information, Stanford
  University.} 

\begin{multicols}{2}
\begin{itemize}
\item What is there\,?
\item What is it like\,?
\end{itemize}
\end{multicols}
    
\noindent
Topics of metaphysical investigation include existence, objects and
their properties, space and time, cause and effect, and
possibility. Metaphysics is considered one of the four main branches
of philosophy, along with epistemology, logic, and ethics''\rm\
\texttt{en.m.wikipedia.org/\-wi\-ki/\-Me\-ta\-phy\-sics}. }{}
\index{pconind}{metaphysics|)}%

\nbbbb{Transcendental Deductions}\label{Transcendental Deductions}\label{sec:Transcendence}
\index{pconind}{transcendental!deduction|(}%

\begynd
\pind A crucial element in Kant's and S{\o}rlander's philosophies
\pind is that of \sfsl{transcendental deduction}.
\afslut

\begynd
\pind It should be clear to the reader that in \nyl \daadm\
\begynd
\pind we are reflecting on a number of philosophical issues;
\pind first and foremost on those of \sfsl{ontology}. 
\pind For this \pof{chapter}{paper} we reflect on a sub-field of epistemology, \nyl
      we reflect on issues of \sfsl{transcendental}
      nature. \pos{%%%
\pind Should you wish to follow-up on the concept of transcendentality, \nyl we refer to
      \cite[Immanuel Kant]{Kant1992}, \nyl
      \cite[Oxford Companion to Philosophy,
      \ysfchgii{pages}\,878--880]{oxford.dict.phil95}, \nyl
      \cite[The Cambridge Dictionary of Philosophy,
      \ysfchgii{pages}\,807--810]{cambridge.dict.phil95}, \nyl
      \cite[The Blackwell Dictionary of Philosophy, \ysfchgii{pages}\,54--55
      (1998)]{blackwell96}, \nyl and
      \cite[S{\o}rlander]{kaisorlander2016}.}{}
\afslut
\afslut

\nbbb{Some Definitions}\label{5.Some Definitions} % \pos{\psno}{\mnewfoil}
  
\bookdefn{Transcendental}{By \brcolor{transcendental}  \nyl we shall
    understand the philosophical
    notion:  \nyl \bbcolor{the a priori or intuitive basis of knowledge,  \nyl
    independent of experience \eod}}
\pos{\vspace*{-2mm}}{}
\noindent
\begynd
\pind A priori knowledge or intuition is central:
\begynd
\pind By \sfsl{a priori} we mean that it not only precedes,
\pind but also determines rational thought.
\afslut
\afslut


\pos{\psno}{\mnewfoil}

\bookdefn{Transcendental Deduction}{ \nyl
By a \brcolor{transcendental deduction} \nyl
    we shall 
    understand the philosophical
    notion: \nyl \bbcolor{a 
    \textsf{transcendental} ``conversion'' \nyl of one kind of 
    knowledge \nyl into a seemingly different kind of knowledge \eod}}

\nbbb{Some Informal Examples}\label{5.Some Informal Examples}
\index{eind}{Transcendental Deductions: Informal Examples}

\tosemb

\monoexample{Transcendental Deductions -- Informal Examples}{\smallish\HHHH\label{exs-of-td-some}%
\begynd
\pind We give some intuitive examples of transcendental deductions.
\pind They are from the ``domain'' of programming languages.
\begynd
\pind There is the syntax of a programming language, \nyl
      and there are the programs that supposedly adhere to this syntax.
\pind Given that, the following are now transcendental deductions.

%\pos{\psno}{\mnewfoil}
      
\begynd
\pind The software tool, \bbcolor{\textsl{a syntax checker}}\pos{, that takes a program and
      checks whether it satisfies the syntax, including the statically
      decidable context conditions, i.e., the \textsl{statics
        semantics} -- such a tool is one of several forms of
      transcendental deductions}{}.
    
\pind The software tools, \bbcolor{\textsl{an automatic theorem
      prover}} and  \nyl \bbcolor{\textsl{a  model checker}}\pos{, for example
      \texttt{SPIN} \cite{Holzmann03},  that takes a program 
      and some \texttt{theorem}, 
      respectively a \texttt{Promela} statement, and proves,
      respectively checks, the program correct with respect the
      theorem, or the statement}{}.
      
\pind A \bbcolor{\textsl{compiler}} and an \bbcolor{\textsl{interpreter}}\pos{ for any programming language}{}.
\afslut

\pos{\psno}{\mnewfoil}
     
\pind Yes, indeed, any \bbcolor{\textsl{abstract interpretation}}
      \cite{Cousot77a,PatrickCousot2021} \nyl reflects a transcendental
      deduction:
\begynd
\pind firstly, these examples show that \nyl there are many
      transcendental deductions; 
\pind secondly, they show that \nyl there is no single-most preferred
      transcendental deduction. 
\afslut
\afslut
\afslut
}%
\toseme

\pos{\psno}{\mnewfoil}

\noindent
\begynd
\pind A transcendental deduction, crudely speaking,
\begynd
\pind is just any abstraction
\pind that can be ``linked'' to another,
\pind not by logical necessity,
\pind but by logical (and philosophical) possibility\,!
\afslut
\afslut

\bookdefn{Transcendentality}{By \bbcolor{transcendentality} we shall
  here mean the philosophical notion:  \nyl \textsf{\sf ``the state or
    condition of being transcendental''}\dbsquare}

\pos{\psno}{\mnewfoil}

\monoexample{Transcendentality}{\vspace*{2mm}
\noindent%
\begynd
\pind We\footnotemark\ can speak of a\ysf{n automobile} in at least three \sfsl{senses:}
\begin{itemize}
\item[(i)] \ \  The \ysf{automobile} as it is being \texttt{"maintained, serviced, refueled"};
\item[(ii)] \ \ the  \ysf{automobile} as it \texttt{"speeds"} down its route; and
\item[(iii)] \ \ the  \ysf{automobile} as it \texttt{"appears"} \ysf{}
  in a\ysfchg{n } \ysf{advertisement}.
\end{itemize}
\pind The three \sfsl{senses} are:
\begin{itemize}
\item[(i)] \ \   as an \bmcolor{endurant} (here a \sfsl{part}),
\item[(ii)] \ \  as a \bmcolor{perdurant} (as we shall see, a \sfsl{behaviour}), and
\item[(iii)] \ \ as an \bmcolor{attribute}\eox
\end{itemize}
\afslut\normalsize\rm
}
%\addtocounter{footnote}{-1}
\footnotetext{\ysfchgii{We } first came across this example when it was presented
  to \ysfchgii{us } by Paul Lindgreen, an early Danish computer scientist (1936--2021) -- and then
  as a problem of data modelling \cite[1983]{PaulLindgreen1983}.}
%\addtocounter{footnote}{1}
%\footnotetext{\HHHH --
%             in this case rather: as a
%             fragment of a bus time table \sfsl{attribute}.}
\noindent
\mnewfoil 
\begynd
\pind The above example, we claim, reflects
      \sfsl{transcendentality} as follows:
\afslut
\begin{itemize}
\item[(i)] \ \ We have knowledge of an endurant (i.e., a part) being an endurant.
\item[(ii)] \ \  We are then to assume that the perdurant referred to in
             (ii) is an aspect of the endurant mentioned in (i) -- where
             perdurants are to be assumed to represent a different
             kind of knowledge.
\item[(iii)] \ \  And, finally, we are to further assume that the attribute
             mentioned in (iii) is somehow related to both (i) and (ii) -- where
             at least this attribute is to be assumed to represent yet
             a different kind of knowledge.
\end{itemize}
\noindent%
\mnewfoil%
\begynd%
\pind In other words:
\begynd
\pind two (i--ii) kinds of different knowledge;
\pind that they relate \sfsl{must indeed}  be based on \sfsl{a
      priori knowledge}.  
\pind Someone claims that they relate\,!    
\afslut
\pind The two statements (i--ii) are claimed to relate
      transcendentally.\footnote{\LLLL -- the attribute statement was
      ``thrown'' in ``for good measure'', \nyl i.e., to highlight the
      issue\,!}  
\afslut

\nbbb{Bibliographical Note}\label{5.Bibliographical Note}

\begynd
\pind The philosophical concept of \sfsl{transcendental deduction} \nyl
      \ysf{is} a subtle one.
\pind Arguments of transcendental nature, \nyl across the literature of
      philosophy, \nyl \ysf{do} not follow set principles and techniques.
\pind We refer to
\begynd
\pind  \cite[\sfsl{The Cambridge Dictionary of Philosophy},
      pages 807--810]{cambridge.dict.phil95}  and
\pind  \cite[\sfsl{The
      Blackwell Companion to Philosophy}, Chapter 22: Kant (David
      Bell), pages 589--606, Bunnin and Tsui-James,
      eds.]{blackwell96} 
\afslut  for more on `transcendence'. 
\afslut 
\index{pconind}{transcendental!deduction|)}%
%%%%%%%%%%%%%%%%%%%%%%%%%%%%%%

\nbbbbb{The Philosophical Question}\label{The Philosophical Question}

\pos{}{\vspace*{-10mm}}

\begynd
\pind S{\o}rlander focuses on the philosophical question of \sort{``what is
      thus necessary that it could not, under any circumstances, be
      otherwise\,?''}.
    
\pind To study and try answer that question S{\o}rlander thinks
      rationally\pconindex{rational thinking}, that is,
      \sfsl{reasons}\pconindex{reasoning}, rather than
      express\ysfchg{es }
      emotions. 
\pind The German philosopher \textsf{Immanuel Kant (1724--1804)}
      suggests that our philosophising as to the  philosophical
      question above must build on \textsl{``something which no person can
      consistently \ysfchg{\dbeat{can} } deny, and thus, something that every person can
      rationally justify, as a consequence of be\ysfchg{ing } able to think at all''.}
\pind Kant then goes on to build his philosophy \cite{Kant:Dines} on
      \textsl{the possibility of self-awareness}\pconindex{possibility!of self-awareness}
      -- something of which we all are aware.
\pind S{\o}rlander then, in for example \cite{kaisorlander2016}, shows
      that this leads to solipsism\footnote{\LLLL Solipsism: the
      view or theory that the self is all that can be known to
      exist.}, i.e., to nothing.
\afslut
   
\nbbbbb{Three Principles}
 
\bbbb{The Possibility of Truth}\label{The Possibility of Truth}

\begynd
\pind Instead S{\o}rlander suggests that  \sort{the possibility of
      truth} be the basis for the thinking of an answer to the
      highlighted question above.
\pind \sfsl{The possibility of truth} is shared by all of us.
\afslut

\nbbbb{The Principle of Contradiction}\label{The Principle of Contradiction}

\begynd
\pind Once we accept  that \textsl{the possibility of
      truth}\pconindex{possibility!of truth} cannot be
      denied, we have also accepted \sort{the principle of
      contradiction}\pconindex{principle!of contradiction}, that is,
    that an assertion and its negation 
      cannot both be true.
\afslut

\nbbbb{The Implicit Meaning Theory}\label{The Implicit Meaning Theory}

\begynd
\pind We must thus also accept \textsl{the implicit meaning
      theory}\pconindex{theory!the implicit
      meaning}. 
    
\pind \newestpdefn{Implicit Meaning Theory}{The implicit meaning
      theory implies that there is
      a \sfsl{mutual relationship} between the  ($\alpha$)
      \sfsl{meaning of designations} and 
      ($\beta$) \sfsl{consistency relations between assertions}\dbsquare}

\noindent
\begynd
\pind As an example of 
\begynd
\pind what ``goes into'' the \sfsl{implicit meaning theory},
\pind  we bring, albeit from 
      the world of computer science, 
\pind that of the description of the \sort{stack} data type \nyl
      (its endurant data types and perdurant operations).
\afslut
\afslut
\afslut
\pos{\psno}{\mnewfoil}


\vspace{2mm}\label{algebra.1}
\newestxample{An Implicit Meaning Theory}{
\vspace{2mm}
\noindent\sort{Narrative:}\pos{}{\HHHH}%

\noindent
\bbcolor{$\alpha.$ The Designations:}
\begin{enumerate}\setei
\item \label{dum0} Stacks, \textsf{s:S}, have elements, \textsf{e:E};
\item \label{dum1} the \textsf{empty\_S} operation takes no arguments 
  and yields a result stack;
\item \label{dum2} the \textsf{is\_empty\_S} operation takes an
   argument stack 
   and yields a Boolean value result.
\item \label{dum3} the \textsf{stack} operation takes two arguments: an element and a
  stack  and yields a result stack.
\item \label{dum4} the \textsf{unstack} operation takes a\ysfchg{ } non-empty argument
  stack  and yields a stack result. 
\item \label{dum5} the \textsf{top} operation  takes a\ysfchg{ } non-empty argument
  stack  and yields an element result. 
\savei\end{enumerate}
\noindent
\mnewfoil
\bbcolor{$\beta.$ The Consistency Relations:}
\begin{enumerate}\setei
\item \label{dum6} an \textsf{empty\_S} stack \textsf{is\_empty}, and 
                   a stack with at least one element is not;
\item \label{dum7} \textsf{unstack}ing an argument stack,
                   \textsf{stack(e,s)}, results in the stack
                   \textsf{s}; and 
\item \label{dum8} inquiring the \textsf{top} of a non-empty
                   argument stack, \textsf{stack(e,s)}, yields
                   \textsf{e}. 
\savei\end{enumerate}
\pos{\psno}{\mnewfoil}

\noindent
\sort{Formalisation:}      
\pos{}{\HHHH}% 
\noindent
\pos{\begin{multicols}{2}}{}  
\noindent
The designations:
%\RSLatex
%type
%&{1}.&   E, S
%value
%&{2}.&   empty_S: Unit -> S
%&{3}.&   is_empty_S: S -> Bool 
%&{4}.&   stack: E >< S -> S
%&{5}.&   unstack: S -~-> S
%&{6}.&   top: S -~-> E
%\endRSLatex 
\bp
\kw{type}\\
{1}.\ \ \ E, S\\
\kw{value}\\
{2}.\ \ \ empty\_S: \kw{Unit} {\RIGHTARROW} S\\
{3}.\ \ \ is\_empty\_S: S {\RIGHTARROW} \kw{Bool} \\
{4}.\ \ \ stack: E {\TIMES} S {\RIGHTARROW} S\\
{5}.\ \ \ unstack: S {\PARRIGHTARROW} S\\
{6}.\ \ \ top: S {\PARRIGHTARROW} E
\ep
%\pos{\end{multicols}}{}
\noindent
The consistency relations:
%\pos{\begin{multicols}{2}}{} 
%\RSLatex
%axiom
%&{7}.&   is_empty(empty_S()) = true
%&{7}.&   is_empty(stack(e,s)) = false
%&{8}.&   unstack(stack(e,s)) = s
%&{9}.&   top(stack(e,s)) = e &\dbsquare&
%\endRSLatex 
\bp
\kw{axiom}\\
{7}.\ \ \ is\_empty(empty\_S()) {\EQ} \kw{true}\\
{7}.\ \ \ is\_empty(stack(e,s)) {\EQ} \kw{false}\\
{8}.\ \ \ unstack(stack(e,s)) {\EQ} s\\
{9}.\ \ \ top(stack(e,s)) {\EQ} e \dbsquare
\ep
\pos{\end{multicols}}{}
%%%%%%%%%%%%%%
%\afslut
\label{algebra.n}
}

\nbbbb{A Domain Analysis \& Description Core}

\begynd
\pind The three concepts:
\begynd
\pind (i) \sfsl{the possibility of truth},
\pind (ii) \sfsl{the principle of contradiction} and
\pind (iii) \sfsl{the implicit meaning theory}
\afslut
\pind thus form the core -- and imply that 
\begynd
\pind (a) \sfsl{the indispensably necessary characteristics of
                any possible world, i.e., domain},
\pind are \sfsl{equivalent}  with
\pind (b) \sfsl{the similarly  indispensably necessary conditions
                for any possible domain description}. 
\afslut
\afslut

\nbbbbb{The Deductions}

\bbbb{Assertions}

\begynd
\pind \newestpdefn{Assertion}{\pdefindex{assertion} An assertion is a
              declaration, an utterance, that something is the case\dbsquare} 
\pind Assertions may typically be either propositions\index{pconind}{proposition}
      or predicates\index{pconind}{predicate}.
\afslut

\nbbbb{The Logical Connectives}

\begynd
\pind Any domain description must necessarily contain assertions.
\pind Assertions\pconindex{assertion} are expressed in terms of
      negation\pconindex{negation}, {\SIM}\psymindex{\SIM, negation ("not")}, 
      conjunction\pconindex{conjunction}, {\WEDGE}\psymindex{\WEDGE,
        conjunction ("and")},
      disjunction\pconindex{disjunction}, {\VEE}\psymindex{\VEE,
        disjunction ("or")}, and
      implication\pconindex{implication}, 
      {\DBLRIGHTARROW}\psymindex{\DBLRIGHTARROW, implication ("if then")}.
\afslut

\nbbb{{\SIM}: Negation}

\begynd
\pind Negation is defined by the principle of contradiction.
\pind If an assertion, $a$, holds, then its negation, {\SIM}$a$, does not hold.
\afslut

\nbbb{Simple Assertions}

\begynd
\pind Simple assertions, i.e., propositions, 
\pind are formed from
      assertions, \ysfchg{for example } $a,b$, by means of the logical connectives.
\afslut

\nbbb{{\WEDGE}: Conjunction}

\begynd
\pind The simple assertion $a$\WEDGE$b$ holds if both $a$ and $b$
hold\ysfchg{. }
\afslut

\nbbb{{\VEE}: Disjunction}

\begynd
\pind The simple assertion $a$\VEE$b$ holds if either \ysfchg{of } or
both  \ysfchg{of } $a$ and
$b$ hold\ysfchg{. }
\afslut

\nbbb{{\DBLRIGHTARROW}: Implication}

\begynd
\pind The simple assertion $a$\DBLRIGHTARROW$b$ holds if $a$ is
      \sfsl{inconsistent} with the negation of $b$.
\afslut

\dbeat{\pos{\nbbb{Model Theory Explication of The Logical Connectives}

\begynd
\pind A model theory explication of the binary logical connectives is
      given on Page\,\pageref{rsl-truth tables}.
\afslut}{}
}

\nbbbb{Modalities}\pconindex{modality}
\bbb{Necessity}

\noindent
\newestpdefn{Necessity}{ \pdefindex{necessity!modality}\pdefindex{modality!necessity}
      An assertion is \sfsl{necessarily true}  
\begynd
\pind if its truth ("true")
\pind follows from the definition of 
\pind the designations by means of which it is expressed. 
\pind Such an assertion holds under all circumstances\dbsquare
\afslut }

\monoexample{Necessity}{%
\begynd
\pind \sfsl{``It may rain someday''} is necessarily true.
\afslut
}%

\nbbb{Possibility}
\begynd
\pind \newestpdefn{Possibility}{ \pdefindex{possibility!modality}\pdefindex{modality!possibility}
      An assertion is \sfsl{possibly true}
\begynd
\pind if its negation is not \sfsl{necessarily true}\dbsquare
\afslut}
\afslut

\monoexample{Possibility}{%
\begynd
\pind \sfsl{``\ysfchgv{I}t will rain tomorrow''} is possibly true.
\afslut
}%

\nbbbb{Empirical Assertions}

\bookdefn{Empirical Knowledge}{ %%%%%%%%%%%%%%%
  \pdefindex{empirical!knowledge}\pdefindex{knowledge!empirical}
\begynd
\pind In {{\rm philosophy}}, knowledge gained from experience -- rather than from
      innate ideas or deductive reasoning -- is empirical knowledge.  
\pind In {{\rm the sciences}}, knowledge gained from experiment and
      observation -- rather than from theory -- is empirical knowledge\dbsquare
\afslut}

\monoexample{Expressing Empirical Knowledge}{%
\begynd
\pind There are innumerable ways of expressing empirical knowledge.
\afslut
\begin{itemize}
\item[a.] There are two automobiles in that garage.\footnote{\sfsl{The automobiles
   are solid {{\rm endurants}}, and so is the garage, that is, they
   are both {{\rm parts.}}}}
\item[b.] The two automobiles in that garage are
  distinct.\footnote{\sfsl{Their distinctness gives rise to their
  respective, distinct, i.e., {{\rm unique identifiers}}.}}
\item[c.] The two automobiles in that garage are parked next to one
  another.\footnote{\sfsl{The topological ordering of the two
  automobiles is an example of their {{\rm mereology.}}}}
\mnewfoil
\item[d.] That automobile, the one to the left, in that garage is
  [painted] red.\footnote{\sfsl{The red colour of the automobile is an
  {{\rm attribute}} of that automobile.}}
\item[e.] The automobile to the right in that garage has just returned
  from a drive.\footnote{\sfsl{The fact that that automobile, to the
  right in the garage, has just returned from a drive, is a {{\rm
  possibly time-stamped attribute}} of that automobile.}}
\item[f.]  The automobile, with Danish registration number
  \texttt{AB\,12345}, is currently driving on the \texttt{Copenhagen
  area} city \texttt{Holte} road \texttt{Fredsvej} at position
  \texttt{`top of the hill'}.\footnote{\sfsl{The automobile in question is
  now a {{\rm perdurant}} having a so-called {{\rm time-stamped progammable
  event attribute}} of the Copenhagen area city of Holte, ``top of the
hill''.}}
\item[g.] The automobile on the roof of that garage is pink.
\end{itemize}
\mnewfoil

\noindent
\begynd
\pind {The pronoun `that' shall be taken to mean that someone gestures at, points out, the
   garage in question.}
\pind If there is no such garage then the assertion denotes the
      \kw{chaos} value\,!
\pind Statements (a.--g.) are assertions. 
%\pind Except for assertion (g.) they all have truth value \kw{true} or \kw{false}. 
\begynd
\pind The assertions contain \sfsl{references} to
      quantities ``outside the assertions'' ---  
\pind `outside' in the sense that
      they are not defined in the assertions.
\afslut
\pind Assertion (g.) does not make sense, i.e., yields \kw{chaos}.
\begynd
\pind The term `roof' has not been defined\,\dbsquare
\afslut
\afslut
} 

\pos{
\noindent
\irsltxt{\sort{The Object Language.}\footnote{The prefix $\mathcal{I}$
  indented paragraph designates an $\mathcal{I}$nformal explication.} The language used in the above
  assertions is quite `free-wheeling'. The language to be used\dbeat{, and
  first properly introduced in
  Chapters\,\ref{chap3.tex.1}--\ref{chap6.tex.1},} in ``our'' domain
  descriptions is, i.e., will be, more rigid}
}{}

\mnewfoil

\noindent
\newestpdefn{Empirical Assertion}{ \pdefindex{empirical!assertion}\pdefindex{assertion!empirical}
\begynd
\pind The domain description language of assertions, contain 
\pind \sort{references}, i.e., \sfsl{designators}, and \sort{operators}.
\pind All of these shall be properly defined  in terms of names of 
\begynd
\pind {{\sl endurants}} and
\begynd
\pind their {{\sl unique identifiers}},
\pind {{\sl mereologies}} and
\pind {{\sl attributes}};
\afslut
\pind and in terms of their {{\sl perdurant ``counterparts''}}\,\dbsquare
\afslut
\afslut}
\mnewfoil

\mnewfoil

\pos{\treprikker}{}

\noindent
\ysfchg{\textsf{\textbf{From Possible Predicates to Conceptual Logic
      Description Framework.}} } 
\begynd
\pind The ability to deduce which type of predicates
      that a phenomen\ysfchg{on } of any domain can be ascribed 
\pind {{\rm is thus equivalent}}
\pind to deducing the conceptual logical conditions \nyl for every
      \ysfchg{\dbeat{possibly} }
      possible domain description.
\afslut

\mnewfoil

\pos{\treprikker}{}

\noindent
\begynd
\pind By a so-called \sfsl{transcendental
      deduction}\pconindex{transcendental!deduction}\pconindex{deduction!transcendental} 
\begynd
\pind we have shown that simple empirical assertions  consist of
\begynd
\pind a \sort{subject} which \sort{refers} to an independently existing
      entity and
\pind a \sort{predicate} which ascribes a \sort{property} to the
      referred entity \ksref{146}{1--5}\footnote{The reference \ksref{150}{1--5}
        refers to the \cite{kaisorlander2016} book by Kai
        S{\o}rlander, $\pi$age 150, $\ell$ines 1--5.}.
\afslut
\afslut
\afslut

\begynd
\pind \ysfchg{If t}he world, or as we shall put it, the domains, \nyl that we shall
      be concerned with,
\begynd
\pind are \sfsl{what can be described in simple assertions},
\pind then any possible such world, i.e., domain
\pind must \sfsl{primarily consist of such entities} \ksref{146}{5--7}.
\afslut

\mnewfoil

\pind We shall therefore, in the following, explicate \nyl
      a system of \bmcolor{concepts}
\pind by means of which the entities,
\pind that may be referred to in simple assertions,
\pind can be described \ksref{146}{8--11}.
\afslut

\irsltxt{These \bmcolor{concepts} are those of
\pos{}{\begin{multicols}{3}}
\begynd
\pind {{\rm entities, 
\pind endurants,
\pind perdurants,
\pind unique identity,
\pind mereo\-logy}} and
\pind {{\rm attributes.}}
\afslut
\pos{}{\end{multicols}}}

\nbbbb{Identity and Difference}\label{Identity and Relations}

\begynd
\pind We can now assume that \nyl the world consists of an indefinite
      number of entities:
\begynd
\pind Different empirical assertions may refer to distinct entities.
\afslut
%%
\pind Most immediately we can define two interconnected concepts:
\begynd 
\pind \sort{identity} and
\pind \sort{diversity}.
\afslut
\afslut

\nbbb{Identity}\label{IdentityKS}

\newestpdefn{Identical}{ ``An entity referred to by the name $A$
      \nyl is \sfsl{identical} to an entity referred to by the name $B$
\begynd
\pind if $A$ cannot be ascribed a property
\pind which is incommensurable
\pind with a property 
\pind ascribed to $B$'' \ksref{146}{14-23}\,\dbsquare
\afslut}

\nbbb{Difference}

\newestpdefn{Different}{ 
\begynd
\pind ``$A$ and $B$ are \sfsl{distinct}, differ\ysfchg{ } from one another,
\pind if they can be ascribed incommensurable properties.''
      \ksref{146}{23-26}\,\dbsquare
\afslut}
\mnewfoil

\treprikker

\noindent
\begynd
\pind ``These  definitions, by transcendental deduction,\pconindexii{transcendental}{deduction}
      introduce\ysfchg{ } the concepts of
\begynd
\pind \sort{identity} and \pconindex{identity}
\pind \sort{difference}\pconindex{difference}.
\afslut
\pind They can thus be assumed in any transcendental deduction
\begynd
\pind of a domain description
\pind which, in principle, must be expressed \nyl in any possible language''. \ksref{147}{1-5}
\afslut
\afslut
\mnewfoil

\begynd
\pind \newestpdefn{Unique Identification}{ By a \sfsl{transcendental
     deduction}\pconindexii{transcendental}{deduction}\pdefindexii{unique}{identification}
\begynd
\pind we introduce the concept of 
\pind manifest, physical entities 
\pind each being uniquely identified\pdefindexii{unique}{identifier}\,\dbsquare
\afslut}

\noindent
\pind We make no assumptions about \nyl any representation of unique
      identifiers.\pconindexiii{unique}{identifier}{representation} 
\afslut

\nbbbb{Relations}\label{primer.Relations}

\bbb{Identity and Difference}

\newestpdefn{Relation}{ ``Implicitly'', from the two concepts of
      \sfsl{identity} and \sfsl{difference},  follows
\begynd
\pind the concept of \sort{relation}s.\pconindex{relation}
\pind ``$A$ identical to $B$ is a relational assertion.
\pind So is $A$ different from $B$'' \ksref{147}{6-10}\,\dbsquare
\afslut}
\mnewfoil

\bbb{Symmetry}

\newestpdefn{Symmetry}{ \pdefindex{symmetric!relation}\pdefindex{relation!symmetric} 
\begynd
\pind If $A$ is identical to $B$ then $B$ must be identical to $A$.
\pind This expresses that the \sfsl{identical to} relation is \sfsl{symmetric}.
\pind And,
\begynd
\pind \ysfchg{i}f $A$ is different from $B$ then
\pind $B$ must be different from $A$.
\afslut
\pind This expresses  that the \sfsl{different from} relation is also \sfsl{symmetric}\,\dbsquare
\afslut}

\nbbb{Asymmetry}

\newestpdefn{Asymmetry}{ \pdefindex{asymmetric!relation}\pdefindex{relation!asymmetric} A relation
\begynd
\pind which holds between $A$ and $B$
\pind but does not hold between $B$ and $A$
\pind is \sfsl{asymmetric} \ksref{147}{25--27}\,\dbsquare
\afslut}

\nbbb{Transitivity}

\newestpdefn{Transitivity}{ \pdefindex{transitive!relation}\pdefindex{relation!transitive}
\begynd
\pind ``If $A$ is identical to $B$ 
\pind and if $B$ is identical to $C$
\pind then $A$ must be identical to $C$.
\pind So the relation \sfsl{identical to} is \sfsl{transitive}''  \ksref{147-148}{28-30,1-4}\,\dbsquare
\afslut}

\noindent
The relation \sfsl{different from} is not transitive.

\nbbb{Intransitivity}

\newestpdefn{Intransitivity}{\ysfchg{} \pdefindex{intransitive!relation}\pdefindex{relation!intransitive}
If, on the other hand,
\begynd
\pind we can logically deduce that
\begynd
\pind a relation, $\mathcal{R}$ holds\ysfchg{ } from $A$ to $B$
\pind and the same relation, $\mathcal{R}$, holds from $B$ to $C$
\pind but $\mathcal{R}$ does not hold from $A$ to $C$
\afslut 
\pind then relation  $\mathcal{R}$ is \sfsl{intransitive}
      \ksref{148}{9--12}\,\dbsquare
\afslut}

\nbbbb{Sets, Quantifiers and Numbers}

\bbb{Sets}

\begynd
\pind The possibility now exists that two or more entities may be
prescribed the same property.
\afslut

\newestpdefn{Sets}{ The ``same properties'' could, for example, be
\begynd
\pind that two or more uniquely distinguished entities, $x, y, ..., z$,
\pind have [at least] one attribute kind (type) and value, $(t,v)$,  in common.
\pind This means that $(t,v)$ distinguishes a set\pconindex{set} $s_{(s,v)}$ -- by a
      \sfsl{transcendental deduction}.\pconindexii{transcendental}{deduction}
\pind A fact, just $t$ likewise distinguishes a possibly other, \nyl most
      likely ``larger'', set $s_t$\,\dbsquare
\afslut}

\noindent
\mnewfoil
\begynd
\pind From the transcendentally deduced notion of set follows the relations:
\begynd
\pind equality, \sort{$=$}, \pconindex{equality,
      \sort{$=$}}\psymindex{\sort{$=$} equality}
\pind inequality, \sort{$\neq$}, \pconindex{inequality, \sort{$\neq$}}\psymindex{\sort{$\neq$} inequality}
\pind proper subset, \sort{$\subset$}, \pconindex{proper subset,
      \sort{$\subset$}}\psymindex{\sort{$\subset$} proper subset}
\pind subset, \sort{$\subseteq$}, \pconindex{subset,
      \sort{$\subseteq$}}\psymindex{\sort{$\subseteq$} subset}
\pind set membership, \sort{$\in$}, \pconindex{set!membership,
  \sort{$\in$}}\psymindex{\sort{$\in$} set membership} 
\pind set intersection, \sort{$\cap$}, \pconindex{set!intersection,
      \sort{$\cap$}}\psymindex{\sort{$\cap$} set intersection}
\pind set union, \sort{$\cup$}, \pconindex{set!union,
      \sort{$\cup$}}\psymindex{\sort{$\cup$} set union}
\pind set subtraction, \sort{$\setminus$}, \pconindex{set!subtraction,
      \sort{$\setminus$}}\psymindex{\sort{$\setminus$} set subtraction}
\pind set cardinality, \sort{card}, \pconindex{set!cardinality,
      \sort{card}}\psymindex{\sort{card} set cardinality}
\pind etc.\,! 
\afslut
\afslut

\bbb{Quantifiers}

\begynd
\pind By a further \sfsl{transcendental deduction}\pconindexii{transcendental}{deduction} \nyl we can place
      the \sfsl{quantifiers}\pconindex{quantifier} among the concepts
      \nyl that are necessary in order to describe domains.
\afslut
      
\newestpdefn{The Universal Quantifier}{ \pdefindex{universal!quantifier}\pdefindex{quantifier!universal}
\begynd
\pind The universal quantifier 
\pind expresses that all members, $x$, of a
      set, $s$, 
\pind possess a certain $\mathcal{P}$roperty:
      {\ALL}$x:S\bullet\mathcal{P}(x)$\,\dbsquare
\afslut}

\newestpdefn{The Existential
  Quantifier}{ \pdefindex{existential!quantifier}\pdefindex{quantifier!existential}  
\begynd
\pind The existential quantifier 
\pind expresses that at least one member, $x$, of a
      set, $s$, 
\pind possess\ysfchg{es } a certain $\mathcal{P}$roperty:
      {\EXISTS}$x:S\bullet\mathcal{P}(x)$\,\dbsquare 
\afslut}

\bbb{Numbers}\pdefindexi{number}

\begynd
\pind Numbers can, again by \sfsl{transcendental
      deduction}\pconindexii{transcendental}{deduction}, be
      introduced, 
\begynd
\pind not as observable phenomena,
\pind but as a rational, logic consequence of sets. 
\afslut
\afslut

\mnewfoil

\newestpdefn{Numbers}{ Numbers can be motivated, for example, as follows:%
\begin{itemize}
\item Start with an empty set, say $\{\,\}$. It can be said
      to represent the number zero.\footnote{Which, in the decimal notation
      is written as 0.}
\item Then add the empty set  $\{\}$ to  $\{\}$ and You get  $\{\{\}\}$ said to represent 1.
\item Continue with adding   $\{\{\}\}$  to $\{\{\}\}$  and \ysf{You} get
      $\{\{\},\{\{\}\}\}$, said to represent 2.
\item And so forth -- ad infinitum\dbsquare
\end{itemize}}

\noindent
\begynd
\pind In this way
      one\footnote{https://en.wikipedia.org/wiki/Set-theoretic\_definition\_of\_natural\_numbers}
      can define the natural numbers. 
\pind We could also do it by just postulating distinct entities which
      are then added, one by one to an initially empty set
      \ksref{150}{8-13}.
\afslut

\mnewfoil

\begynd
\pind We can then, still in the realm of philosophy, \nyl  proceed with
      the introduction of
\begynd
\pind the arithmetic operations designated by
\pos{}{\begin{multicols}{3}}
\begynd
\pind addition, \sort{+}\pconindexii{addition}{arithmetic
  operator}\psymindex{\sort{+} addition},
\pind subtraction, \sort{--}\pconindexii{subtraction}{arithmetic
  operator}\psymindex{\sort{$-$} subtraction},
\pind multiplication,
\sort{$\ast$}\pconindexii{multiplication}{arithmetic
  operator}\psymindex{\sort{$\ast$} multiplication},
\pind division, \sort{$\div$}\pconindexii{division}{arithmetic
  operator}\psymindex{\sort{$\div$} division},
\pind equality, \sort{=}\pconindexii{equality}{relational operator}\psymindex{\sort{=} equality},
\pind inequality, \sort{$\neq$}\pconindexii{inequality}{relational
  operator}\psymindex{\sort{$\neq$} inequality},
\pind larger than, \sort{$>$}\pconindexii{larger than,}{relational operator}\psymindex{\sort{$>$}larger than},
\pind larger than or equal, \sort{$\geq$}\pconindexii{larger than or
  equal}{relational operator}\psymindex{\sort{$\geq$} larger than or equal},
\pind smaller than, \sort{$<$}\pconindexii{smaller than,}{relational operator}\psymindex{\sort{$>$}smaller than},
\pind smaller than or equal, \sort{$\leq$}\pconindexii{smaller than or
  equal}{relational operator}\psymindex{\sort{$\leq$} smaller than or equal},
\pind etcetera\,!
\afslut
\pos{}{\end{multicols}}
\afslut

\pind From explaining numbers on a purely philosophical basis
\begynd
\pind one can now proceed mathematically
\pind into the realm of \sfsl{number
      theory}\pconindexii{number}{theory} \cite{HardyWright2008}.
\afslut
\afslut


\nbbbb{Primary Entities}

\begynd
\pind We now examine the concept of \sfsl{primary
      objects}\index{pconind}{primary!object}\index{pconind}{object!primary}.
\afslut 

\begynd
\pind The next two definitions, in a sense, ``fall outside'' the line of
      the present philosophical inquiry.
\pind They will be ``corrected'' to then ``fall inside'' our inquiry.
\afslut

\bookdefn{Object}{ By an \sfsl{object}\index{pdefind}{object}\index{pconind}{object}
  we, in our context, mean something material that
  may be perceived by the senses\footnote{www.merriam-webster.com/dictionary/object}\dbsquare} 

\bookdefn{Primary Object}{ By a \sfsl{primary object}
  \index{pconind}{object!primary}\index{pconind}{object!primary}%
  \index{pdefind}{object!primary}\index{pdefind}{primary!object}%
  we\footnote{help.hcltechsw.com/commerce/8.0.0/tutorials/tutorial/ttf\_cmcdefineprimaryobject.html}
  mean an object 
  that exists as its own
  \sfsl{entity}\index{pconind}{entity}
  independent\footnote{Yes, we know: we have not defined what is meant
  by `as its own' and `independent'\,!} of other objects\dbsquare} 

\noindent
\begynd
\pind In the last definition we have used the term \sfsl{entity}. 
\pind That term, `entity', will be used henceforth instead of the term `object'\index{pconind}{entity!= object}.
\afslut

\mnewfoil

\begynd
\pind We have deduced the relations
\pos{}{\begin{multicols}{3}}
\begynd
\pind \sfsl{identity,
\pind difference,
\pind symmetry,
\pind asymmetry,
\pind transitivity} and
\pind \sfsl{intransitivity}
\afslut
\pos{}{\end{multicols}} in Sects.\,\ref{Identity and Relations}--\ref{primer.Relations}. 
\begynd
\pind You may ask: \sfsl{for what purpose\,?}
\pind And our answer is: \sfsl{to justify the next set of deductions.} 
\pind First we reason that 
\begynd
\pind there is the possibility of there being
      many entities.
\afslut 
\pind We argue that
\begynd
\pind  that is possible due to there being the relation
      of asymmetry.
\afslut 
\pind If it holds between two entities 
\begynd
\pind then they must necessarily be ascribed different predicates,
\afslut 
\pind hence be distinct.
\afslut 
\afslut 
\mnewfoil

\begynd
\pind Similarly we can argue that two entities, $B$ and $C$ which both
      are asymmetric \ysf{with respect} to entity $A$
\begynd
\pind may stand in a symmetric relation to one another.
\afslut 
\pind This opens for the \sfsl{possibility} that every pair of
      distinct entities may stand in a pair of mutual relations.
\begynd
\pind First the asymmetry relation that expresses their distinctness.
\pind Secondly, the possibility of a symmetry relation which expresses
      the two entities individually with respect to one-another.
\afslut
\pind \sfsl{The above forms a transcendental basis for how two or more
      [primary] entities must necessarily be characterised by predicates.}
\afslut

\nbbbb{Space and Time}\label{f:Space and Time}

\begynd
\pind The asymmetry and symmetry relations between entities
\begynd
\pind cannot be \sfsl{necessary} characteristics of every possibl\ysf{e} reality
\pind if they cannot also posses\ysfchg{s } an \sfsl{unavoidable r{\^{o}}le} in
      our own concrete reality.
\afslut
\pind Next we examine two such \sfsl{unavoidable r{\^{o}}les}.
\afslut

\nbbb{Space}

\begynd
\pind One pair of such r{\^{o}}les are \sfsl{distance} and \sfsl{direction}.
\begynd
\pind \sfsl{Distance} is a relation \nyl that holds between any pair of
      distinct entities.\index{pconind}{distance}
\pind It is a symmetric relation.
\pind \sfsl{Direction} is an asymmetric relation \nyl that also holds
between \ysfchg{any } pair of
      distinct entities.\index{pconind}{direction}
\afslut
\mnewfoil
\pind Hence we conclude that \nyl \sort{space} is an unavoidable
      characteristics of every possibl\ysf{e}
      reality.\index{pconind}{space@\sort{space}}
\begynd
\pind Hence we conclude that entities exist in space.
\pind They must ``fill'' some space, have \sfsl{extension},\index{pconind}{extension}
\pind they must \sfsl{fill} some space,
\pind have \sfsl{surface} and \sfsl{form}.
      \index{pconind}{surface!spatial}\index{pconind}{form!spatial}%
      \index{pconind}{spatial!surface}\index{pconind}{spatial!form}%
\pind From this we can define the notions of 
\begynd
\pind spatial point,
      \index{pconind}{point!spatial}\index{pconind}{spatial!point}%
\pind  spatial straight line,
      \index{pconind}{line!spatial}\index{pconind}{spatial!line}%
\pind  spatial surface,
      \index{pconind}{surface!spatial}\index{pconind}{spatial!surface}%
\pind etcetera.
\afslut
\pind Thus we can philosophically motivate geometry.
\afslut
\afslut

\nbbb{Time}\label{primer-filosofi-time}

\begynd
\pind Primary empirical entities \nyl may \ysfchgii{ } accrue predicates  \nyl that it is
      not logically necessary that they accrue.
\begynd
\pind That is, it is logically possible that primary entities  
\pind accrue predicates that they  \ysfchgii{do not } actually accrue.
\pind How is it possible that one and the same primary entity \nyl may
      accrue incommensurable predicates\,?
\afslut
\afslut

\mnewfoil

\begynd
\pind That is only possible if one and the same primary entity can 
\pind exist in \sort{different states}.\index{pconind}{state}
\begynd
\pind It may exist in one state in which it accrue\ysfchg{s } a certain predicate.
\pind And it may exist in another state \nyl in which it accrue\ysfchg{s } a therefrom
      incommensurable predicate.
\afslut
\afslut

\mnewfoil

\begynd
\pind What can we say about these states\,?
\begynd
\pind First that these states accrue different, \nyl incommensurable predicates.
\pind How can we assure that\,!
\pind Only if the states stand in an asymmetric relation to one another.
\pind From this we can conclude \nyl that primary entities necessarily \nyl
      may exist in a number of states 
\pind each of which stand \nyl  in an asymmetric relation to
      \ysfchg{its }
      predecessor state.
\pind So these states also stand in a \sfsl{transitive} relation.
\afslut
\afslut

\mnewfoil

\begynd
\pind This is a necessary characteristics of any possible world.
\pind So it is also a characteristics of our world.
\pind That relation is \sort{time}.\index{pconind}{time@\sort{time}}
\begynd
\pind It possesses the \sfsl{before}, \sfsl{after}, \sfsl{in-between},
      and other [temporal] relations.
\index{pconind}{before!temporal}%
\index{pconind}{after!temporal}%
\index{pconind}{in-between!temporal}%
\index{pconind}{temporal!before}%
\index{pconind}{temporal!after}%
\index{pconind}{temporal!in-between}%
\afslut
\pind We have thus deduced that every possible world \nyl
      must ``occur in time'' \nyl
      and that primary entities may exist in\ysfchg{, } before or after states.
\afslut

\mnewfoil

\begynd
\pind From the above we can derive a whole algebra of temporal types
      and operations, for example:
\afslut
\begin{itemize}
\item $\mathbb{TIME}$ and $\mathbb{TIME}$\ $\mathbb{INTERVAL}$ types;\index{pconind}{time!interval}
\item addition of $\mathbb{TIME}$ and $\mathbb{TIME}$\ $\mathbb{INTERVAL}$ to obtain $\mathbb{TIME}$;
\index{pconind}{addition!of time and time intervals}%
\item addition of $\mathbb{TIME}$\ $\mathbb{INTERVAL}$s to obtain $\mathbb{TIME}$ $\mathbb{INTERVAL}$s;
\index{pconind}{addition!of time intervals}%
\item subtraction of two $\mathbb{TIME}$s to become $\mathbb{TIME}$
  $\mathbb{INTERVAL}$s; and
\index{pconind}{subtraction!of time intervals from times}%
\item subtraction of two $\mathbb{TIME}$ $\mathbb{INTERVAL}$s to obtain $\mathbb{TIME}$\ $\mathbb{INTERVAL}$.
\index{pconind}{subtraction!of time intervals}%
\end{itemize}

\nbbbb{The Causality Principle}

\begynd
\pind But what is it that \sfsl{cause}\ysfchg{s } primary entities to undergo
      \sfsl{state changes}\,? \index{pconind}{state!change}%
\begynd
\pind Assertions about how a primary entity is at different times, \nyl
      such assertions must necessarily be logically independent.
\pind That follows from primary entities necessarily must accrue
      incommensurable predicates at different times.
\pind It is therefore logically impossible to conclude from how a
      primary entity is at one time to  how it is at another time.
\pind How, therefore, can assertions about a primary entity at
      different times be about the same entity\,?
\afslut
\afslut

\mnewfoil

\begynd
\pind We can therefore transcendentally deduce that \nyl
      there must be a \sfsl{special implication-%
      relationship} \nyl between assertions about how a primary entity
    \nyl is at different times.
\begynd
\pind Such a \sfsl{special implication-relationship} must depend on
      the \sfsl{empirical circumstances} under which the primary entity
      exists. 
\pind That is, we must deduce the conditions under which it is, at
      all, possible to consistently make statements about primary
      entities going from one state in which it accrues a specific
      predicate to another state in which it accrues a therefrom
      incommensurable predicate.
\pind There must be something in the empirical circumstances which
      implicates the state transition.
\mnewfoil
\pind If the \ysfchg{\dbeat{the} } empirical circumstances are \sfsl{stable} \nyl
      then there is nothing in these circumstances that imply entity changes.
\pind If the primary entity changes, then that assumes that there must
      have been a prior change in the circumstances -- with those
      changes having that consequence.\ \ldots\footnote{We skip some of
      S{\o}rlander's reasoning, \cite[Page\,162, lines 1--12]{kaisorlander2016}}\ 
\pind We name such a change of the circumstances \sfsl{a cause}.\index{pconind}{cause}
\pind And we conclude that every change of a primary entity must have
      a cause.
\pind We also conclude that \sfsl{equivalent causes} imply
      \sfsl{equivalent effects}.
\afslut
\afslut

\mnewfoil

\begynd
\pind This form of implication is called the \sfsl{causality
      principle}. \index{pconind}{causality!principle}\index{pconind}{principle!of
      causality}% 
\begynd 
\pind It assumes logical implication.
\pind But it cannot be reduced to logical implication. 
\pind It is logically necessary that every primary entity -- and
      therefore every possible world -- is subject
      to the \sfsl{causality principle}.
\afslut
\pind In this way Kai S{\o}rlander transcendentally deduce\ysfchg{s } the
      principle of causality.
\pind Every change has a cause.
\pind The same cause under the same circumstances lead\ysfchg{s } to same effects.
\afslut

\nbbbb{Newton's Laws}

\begynd
\pind S{\o}rlander then shows how Newton's laws can be deduced.
\pind These laws, in summary, are:
\begin{itemize}
\item \sort{Newton's First Law:} \index{pconind}{Newton's Law!Number 1}  
    An entity at rest or moving at a constant speed in a straight
    line, will remain at rest or keep moving in a straight line at
    constant speed unless it is acted upon by a force.  
\item \sort{Newton's Second Law:} \index{pconind}{Newton's Law!Number 2}
     When an entity is acted upon by a force, the time rate of change of
     its momentum equals the force.  
\item \sort{Newton's Third Law:} \index{pconind}{Newton's Law!Number 3}
    To every action there is always opposed an equal reaction; or, the
    mutual actions of two bodies upon each other are always equal, and
    directed to contrary entities.
\end{itemize}
\afslut

\nbbb{Kinematics}\label{Kinematics}

\begynd
\pind Above we have deduced that primary entities are in both space
      and time.
\begynd
\pind They have \sfsl{extent}  in both space and time.
\pind That means that they may change with respect to their spatial
      properties: place and form.
\pind The change in place is the fundamental.
\pind A primary entity which changes place is said to be in
      \sfsl{movement}. \index{pconind}{movement}%
\pind A primary entity in {movement} must follow a certain geometric route.
\pind It must move a certain length of route in a certain interval of
      time, i.e., have a \sfsl{velocity:} speed and
      direction. \index{pconind}{velocity}% 
\pind A primary entity which changes velocity has an
      \sfsl{acceleration.} \index{pconind}{acceleration}% 
\pind That is, we have deduced \ysfchg{t}he basics of
      \sfsl{kinematics}. \index{pconind}{kinematics}% 
\afslut 
\afslut

\nbbb{Dynamics}\label{Dynamics}

\begynd
\pind When we, to the above, add that primary entities are in time, \nyl
      then they are subject to causality.
\begynd
\pind That means that we are entering the doctrine of the influence of
      \sfsl{force}s on primary entities.
\pind That is, \sfsl{dynamics}. \index{pconind}{dynamics}%
\pind Kinematics imply that an entity changes if it goes from being at
      rest to mov\ysfchg{ing}, \nyl or if it goes from moving to being at rest.
\pind An entity also changes if it goes from moving at one velocity to
      moving at a different velocity.
\pind We introduce the notion of \sfsl{momentum}. \index{pconind}{momentum}%
\pind An entity has \ysfchg{the } same momentum at two
      times \ysfchg{if } it has the same
      velocity and acceleration.
\afslut 
\afslut

\nbbb{Newton's First Law}

\begynd
\pind When we combine kinematics with causality
\begynd
\pind then we can deduce that if an entity changes momentum \nyl
      then there must be a cause in the circumstances \nyl which
      causally implies the change.
\pind We call that cause a \sfsl{force}. \index{pconind}{force}%
\pind The force must be proportional to the change in momentum.
\pind This implies that an entity which is not subject to an external
      force \nyl remains in the same momentum.
\pind This is \sort{The Law of Inertia}, \sort{Newton\ysfchg{'}s First Law.}
      \index{pconind}{The Law of Inertia}% 
      \index{pconind}{Newton's Law!Number 1}    
\afslut
\afslut

\nbbb{Newton's Second Law}

\begynd
\pind That a certain force is necessary \nyl in order to change an entity's  
      momentum \nyl must imply that such an entity must provide a
      certain \sfsl{resistance} \index{pconind}{resistance} against
      change of momentum.
\begynd
\pind It must have a \sfsl{mass}. \index{pconind}{mass}%
\pind From this it follows that the change of an entity's momentum
\begynd
\pind not only must be proportional to the applied force
\pind but also inversely proportional to that entity's mass.
\afslut
\pind This is \sort{Newton\ysfchg{'}s Second Law}. 
      \index{pconind}{Newton's Law!Number 2}%
\afslut
\afslut

\nbbb{Newton's Third Law}

\begynd
\pind Where do the forces that influence the momentum of entities come from\,?
\pind It must, it can only, be from primary entities.
\begynd
\pind Primary entities must be the source of the forces that influence
      other entities.
\pind Here we shall argue one such reason.
\pind The next section, on universal gravitation, presents a second reason.
\afslut

\mnewfoil

\pind Primary entities may be in one \ysf{an}other's way.
\begynd
\pind Hence they may eventually collide.
\pind If a primary entity has a certain velocity \nyl
      it may collide with another primary entity crossing its way.
\pind In the mutual collision the two entities influence one another
      \nyl such that they change momentum.
\pind They influence each other with forces.
\pind Since \ysfchg{neither } of the two entities ha\ysfchg{s } any special position, i.e.,
      rank, \nyl
\begynd
\pind the forces by means of which they affect one another
\pind must be equal and oppositely directed.
\afslut
\afslut
\pind This is \sort{Newton\ysf{'}s Third Law.} \index{pconind}{Newton's Law!Number 3}%
\afslut
 
\nbbbb{Universal Gravitation}
  \index{pconind}{universal!gravitation}\index{pconind}{gravitation!universal}%

\begynd
\pind But\footnote{This section is from \cite[Pages
     168--173]{kaisorlander2016}}, really, how can primary entities be
     the source of forces  \nyl that affe\ysfchgv{c}t one another\,?
\begynd
\pind We must dig deeper\,!
\pind How can primary entities have mass such that it requires force
      to change their momentum\,?
\pind Our answer is that the reason they have mass \nyl
      must be due to mutual influence between the primary objects themselves.
\pind It must be an influence which is oppositely directed to that \nyl
      which they expose on one another when they collide.
\pind Because this, in principle, applies to all primary entities, \nyl
      these must be characterised by a mutual universal attraction.
\afslut
\mnewfoil
\pind And that is what we call \sfsl{universal gravitation}.
\index{pconind}{universal!gravitation}%
\index{pconind}{gravitation!universal}%
\pind That concept has profound implications.
\afslut

\treprikker

\noindent
\begynd
\pind We shall not go into details here
\begynd
\pind but just, casually, as it w\ysfchg{as}, mention that such concepts as
\pind speed limit, elementary particles and
      Einstein's theories \nyl are ``more-or-less'' transcendentally deduced\,!
\afslut
\afslut

\nbbbb{Purpose, Life and Evolution}\label{Purpose Life and Evolution}

\begynd
\pind We shall briefly summarise S{\o}rlander's analysis and
      deductions with respect to the concepts of
      \sfsl{living species:} \index{pconind}{living species}%
      \sfsl{plants} and \index{pconind}{plant}%
      \sfsl{animals}\index{pconind}{animal}%
      \pos{, the latter including \sfsl{humans}}{}. \index{pconind}{human}%

\begynd
\pind Up till now S{\o}rlander's analyses and deductions have focused
      on the physical world, ``culminating'' in Newton's Laws and
      Einstein's theories.
      
\pind If\footnote{\LLLL We now treat the material of \cite[\sfsl{Chapter 10, Pages
      174--179}]{kaisorlander2016}.} there is to be
      language \index{pconind}{language} and meaning 
      \index{pconind}{meaning} then, as a first condition, there must
      be the possibility that there are primary entities which are not
      locked-in ``only'' in that physical world deduced till now.
\pind This is only possible if such primary entities are additionally
      subject to a \sfsl{purpose-causality},
      \index{pconind}{purpose!causality}\index{pconind}{causality!of
      purpose}%
      one that is so constructed as to \sfsl{strive} to
      \sfsl{maintain} its own \sfsl{existence}.
\afslut
\pind We shall refer to this kind of primary entities as \sfsl{living
     species}. \index{pconind}{living species}%
\afslut

\nbbb{Living Species}\label{filo:Living Species}

\begynd
\pind As living species they must be subject to all the physical
      conditions for existence and mutual influence.
\pind Additionally they must have a form which they are \sfsl{causally
      determined to reach and maintain.}
\pind This development and maintenance must take place in a
      \sfsl{substance exchange} \index{pconind}{substance
      exchange}\index{pconind}{exchange!of substance}%      
      with its surroundings.
\pind Living species need these substances in order to develop and
      maintain their form.

\mnewfoil
\pind It must furthermore be possible to distinguish between two forms
      of living species: 
\begynd
\pind (i) one form which is characterised only by \sfsl{development, form
      and substance exchange}; and
\pind (ii) another form which, additional to (i), is characterised by
      \sfsl{being able to move.}
\afslut
\pind The first form we call \sfsl{plants}. \index{pconind}{plant}%
\pind The second form we call \sfsl{animals}. \index{pconind}{animal}%
\afslut

\nbbb{Animals}\label{filo:Animals}\index{pconind}{animals|(}

\begynd
\pind For animals to move they must 
\begynd
\pind (i) possess \sfsl{sense organs},
      \index{pconind}{sense organs}\index{pconind}{organs!sensory}%
\pind (ii) \sfsl{organs of movement}
      \index{pconind}{organs!of movement}\index{pconind}{movement!organs}%
      and 
\pind (iii) \sfsl{instincts, incentives,} or \sfsl{feelings}.
      \index{pconind}{instinct}\index{pconind}{feeling}%
      \index{pconind}{incentive}%
\afslut
\pind All th\ysfchg{ese } still subject to the physical laws and to satisfy motion.
\afslut

\mnewfoil
\begynd 
\pind This is only possible if animals are \sort{not} built \nyl
      (like the elementary particles of physics) \nyl
      but by special physical units.
\begynd
\pind These cells must satisfy the \sfsl{purpose-causality} of animals.
\pind And we know, now, from the \sfsl{biological sciences}
      \index{pconind}{biological science}%
      \index{pconind}{biology}%
      \index{pconind}{science!of biology}%
      that something like that is indeed the case.
\pind Indeed animals are built from cells all of which possess
      \sfsl{genomes} \index{pconind}{genome}%
      for the whole animal
\pind and, for each such cell, a  proper fraction of its genome \nyl
      controls whether it is part of a sensory organ, or a nerve, or a
      motion organ, or a more specific function. 
\afslut
\pind Thus it has transcendentally been deduced \nyl that such must be the
      case \nyl and biology has confirmed this.
\afslut
\index{pconind}{animals|)}

\nbb{Humans}\label{filo:Humans}

\begynd
\pind We briefly summarise\footnote{\cite[\sfsl{Chapter\,11,
      Pages\,180--183}]{kaisorlander2016}}, in six steps, (i--vi), S{\o}rlander's reasoning
      \nyl that leads from animals, in general, see above, to humans,
      in particular. 
    
\begynd
\pind (i) First the concept of \bbcolor{\sfsl{level of consciousness}}
      \index{pconind}{level!of consciousness}\index{pconind}{consciousness!level}\ is introduced.
\begynd
\pind On the basis of animals being able to \sfsl{learn} from \sfsl{experience}
      \index{pconind}{learn}\index{pconind}{experience}%
\pind the concept of \sfsl{consciousness level} is introduced.
\pind It is argued that \sfsl{neurons} \index{pconind}{neuron} \nyl
      provide part of the basis for \sfsl{learning} and the
      \sfsl{consciousness level}. 
\afslut

\pind (ii) Secondly the concept of \bbcolor{\sfsl{social instincts}}
      \index{pconind}{social instincts}\ is introduced.
\begynd
\pind For animals to interact social instincts are required.
\afslut
\mnewfoil

\pind (iii) Thirdly the concept of \bbcolor{\sfsl{sign language}}
      \index{pconind}{sign language}\ is introduced.
\begynd
\pind In order for animals to interact some such animals, \nyl notably the
      humans, develop a sign language.
\afslut
      
\pind (iv) Fourthly the concept of \bbcolor{\sfsl{language}}
      \index{pconind}{language}\ is introduced.
\begynd
\pind The animals that we call \sfsl{humans} finally develop their
      sign language into a language that can be spoken, heard and understood.
      \pind Such a language, regardless of where it is developed, \nyl
      that is, regardless of which language it is, \nyl
      must assume, i.e., build on the same set of basic concepts \nyl
      as had been uncovered so far in our deductions of what must
      necessarily be in any description of any world.
\afslut
\afslut
\afslut
\mnewfoil

\begynd
\pind We continue \ysfchg{to } summarise\footnote{\cite[\sfsl{Chapter\,12,
      Pages\,184--187}]{kaisorlander2016}} S{\o}rlander's reasoning
      \nyl that leads from generalities about humans to 
      humans with knowledge and responsibility.
      
\begynd
\pind  (v) Fifthly the concept of \bbcolor{\sfsl{knowledge}}
      \index{pconind}{knowledge}\ is introduced.
\begynd
\pind An animal which is \sfsl{conscious} must \sfsl{sense} \nyl
      and must react to what it senses.
\pind To do so it must have \sfsl{incentives} as causal conditions
      for its specific such actions.
\pind If the animal has, possess\ysf{es}, language, then it must be able to
      express that and what it senses and that it acts accordingly,
      and why it does so. 
\pind It must be able to express that it can express this.
\pind That is, that what it expresses, is true.
\pind To express such assertions, with sufficient reasons for why they
      are true, is equivalent to \sfsl{knowing} that they are true.
\afslut
\pind Such animals, as possess\ysfchg{ing } the above ``skills'', become persons, humans.
\afslut
\mnewfoil

\begynd
\pind  (vi) Sixthly the concept of \bbcolor{\sfsl{responsibility}}
      \index{pconind}{responsibility}\ is introduced.
\begynd
\pind Humans conscious of their concrete situation, must also know
      that these situations change.
\pind They are conscious of earlier situations. 
\pind Hence they have \sfsl{memory}.
\pind So that they can formulate \sfsl{experience} with respect to the
      \sfsl{consequences} of their actions.
\pind Thus humans are (also) characterised by being able to
      understand the consequences of future actions.
\pind A person who considers how he ought act, can also be ascribed
      \sfsl{responsibility} --
\pind and can be judged \sfsl{morally.}
\afslut
\afslut

\treprikker

\noindent
\mnewfoil
\begynd
\pind This ends our expos{\'e} of S{\o}rlander's metaphysics \ysf{with
      respect to} living species.
\begynd
\pind That is, we shall cover
\begynd
\pind neither non-human animals, 
\pind nor plants.
\afslut
\afslut
\afslut

\nbbbbb{Philosophy, Science and the Arts}\label{3:Philosophy, Science and the Arts}

\begynd
\pind We quote extensively from \cite[Kai S{\o}rlander,
1997]{kaisorlander1997}\HHHH.
\ysfchgii{%%%%%%%%%%%%%%
\begin{itemize}
\item Page\,178: \textsl{Philosophy, science and the arts are products of the human mind.}
\item Page\,179: \textsl{Philosophy, science and the arts each have their own goals.}
\end{itemize}
}%%%%%%%%%%%%%%%%%%%%%%%
\afslut

\noindent
\ysfchgii{And:} 
\begin{itemize}
\item  {\sort{Philosophers} seek to find the inescapable
  characteristics of any world.}

\item  {\sort{Scientists} seek to determine how the world
  actually \ysfchgv{-}- and our situation in that world -- is.}

\item  {\sort{Artists} seek to create objects for experience.}
\end{itemize}
\noindent
\pos{\psno}{\mnewfoil}
We shall elaborate.
\begynd
\pind \pos{\zzks{180}{Simplifying, but not without an element of truth,
  we can relate the three concepts by the
  \sort{modalities:}}}{Simplifying, but not without an element of
truth, 
  we can relate the three concepts by the \sort{modalities:}}
\begin{itemize}
\item \brcolor{\sfsl{philosophy}} is the \sort{necessary},
\item \brcolor{\sfsl{science}} is the \sort{real}, and
\item \brcolor{\sfsl{art}} is the \sort{possible}.
\end{itemize}
\afslut
\noindent
\begynd
\pind 
\ldots\ Here we have, then, a distinction between philosophy and
science. \ldots\ From \cite{kaisorlander1994} we can conclude the
following about the results of philosophy and science. These results
must be consistent [with one another]. This is a necessary condition
for their being \sfsl{correct}. \ldots\ \ldots\ The \sort{real} must be a
\sfsl{concrete realisation} of the \sort{necessary}.
\afslut

\pos{\pof{}{%%%%%%%%%%%%%%%%%%%%%%%
\nbbbbb{A Bibliographical Note}

\begynd
\pind Of the 30 citations given in Footnote\,\ref{citations-1}, Pages\,\pageref{citations-1}--\pageref{citations-2}
\begynd
\pind \ysfchg{We } have not read 20 of then, 
\pind but have studied some of Kant's, Russell's, Wittgenstein's and
      Popper's writings.
\pind The dictionaries
      \cite{cambridge.dict.phil95,blackwell96,oxford.dict.phil95}, as
      well as \cite{OED},  have
      followed \ysfchg{us } for years.
\afslut
\afslut
}}{}%%%%%%%%%%%%%%%%%%%%%%%%%%%%%%

\nbbbbb{A Word of Caution}

\begynd
\pind The present \pof{chapter}{paper} represents an attempt to
      give an English interpretation of Kai S{\o}rlander's Philosophy.
\pind \ysfchg{We } otherwise refer to \cite{kaisorlander2023}.
\afslut

\label{last-page-filosofi}
\index{pconind}{philosophy!S{\o}rlander's|)}

\label{chap2.tex.Philosophy.n}


 
%%  LocalWords:  endurants philosophia philosophising analyse Kai ing


%%  LocalWords:  rlander Immanuel unstack Formalisation Bool colour
%%  LocalWords:  disjunction mereology Holte Fredsvej perdurant mereo
%%  LocalWords:  progammable designators mereologies perdurants eind
%%  LocalWords:  incommensurable roperty characterisation rlander's
%%  LocalWords:  transcendentality priori ACL Coq HOL STeP PVS prover
%%  LocalWords:  Promela endurant behaviour Lindgreen modelling Tsui
%%  LocalWords:  Bunnin Intransitivity cardinality infinitum etcetera
%%  LocalWords:  endeavour Milet Anaximander apeiron un dif ren www
%%  LocalWords:  Anaximenes Heraklit Efesos Empedokles Parminedes cy
%%  LocalWords:  Demokrit clo pe di phy li gion cepts th physika wi
%%  LocalWords:  Zalta ki conind pconind pdefind html intransitivity
%%  LocalWords:  cmcdefineprimaryobject wrt characterised le les epos
%%  LocalWords:  summarise Sixthly realisation nformal ines possibl
