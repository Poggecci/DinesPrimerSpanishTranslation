

\nbbbbbb{Some Formal Languages}\label{chap2.tex.1}\label{chapter:The Mathematics}
\minitoc

\noindent
\begynd
\pind In this chapter we shall briefly cover an assortment of formal
languages:
\begynd
\pind discrete mathematics,
\pind notations for types and sorts, values in general and function values,
\pind the notion of of recursive function theory: the
      $\lambda$-Calculus \cite{Chu51,Barendregt91} and recursive
      functions \cite{Kleene52,Kleene67,Rogers67}, and 
\pind \texttt{CSP} \citecsp.
\afslut
\afslut

\begin{itemize}
\item[] \centerline{\fbox{Several sections of this chapter are still to be edited}}
\end{itemize}


\nbbbbb{Discrete Mathematics}\label{chap2.tex.Discrete Mathematics}

By discrete mathematics we shall understand such fields as
\begynd
\pind mathematical logic, %\pconindex{mathematical logic}
\pind set theory, %\pconindex{set!theory}\pconindex{theory!sets}
\pind   algebra, %\pconindex{algebra}
\pind   combinatorics, %\pconindex{combinatorics}
\pind   graph theory, %\pconindex{graph!theory}\pconindex{theory!graphs}
\pind   probability theory, %\pconindex{probability!theory}\pconindex{theory!probability}
\pind   number theory, %\pconindex{number!theory}\pconindex{theory!number}
\pind   Euclidean geometry, %\pconindex{Euclidean geometry}
\pind   topology, %\pconindex{topology}
\pind   operations research, and %\pconindex{operations research}
\pind   game theory. %\pconindex{game!theory}\pconindex{theory!game}
\afslut
  
We shall only cover a three of these:  mathematical logic, set theory
and algebra -- and only cursorily in this chapter.

\nbbbb{Logic and Mathematical Logic}\label{Logic and Mathematical Logic}

\bbb{Classical Logic}\label{Classical Logic}

\tbw

\nbbb{Mathematical Logic}\label{Mathematical Logic}

\tbw

\nbbb{Model and Proof Theory}\label{Model and Proof Theory}

\label{truth tables} %% remember to insert those !
\tbw

\nbbb{R{\^{o}}les of Mathematical Logic in Computing}

\tbw


\nbbbb{Set Theory}\label{Set Theory}

\begin{quote}
           \tt
           \dbbook{\hspace*{25mm}}  Band of Musicians 
\hfill  Bevy of Beauties\\
           \dbbook{\hspace*{25mm}}  Bunch of Crooks 
\hfill     Crew of Sailors\\
           \dbbook{\hspace*{25mm}}  Flock of Geese  
\hfill      Fleet of Ships\\
           \dbbook{\hspace*{25mm}}  Gang of Outlaws  
\hfill    Group of People\\
           \dbbook{\hspace*{25mm}}  Herd of Cattle   
\hfill        Mop of Hair\\
           \dbbook{\hspace*{25mm}}  Pack of Dogs  
\hfill   Posse of Vigilantes\\
           \dbbook{\hspace*{25mm}}  Pride of Lions  
\hfill  School of Dolphins\\
           \dbbook{\hspace*{25mm}}  Suite of Bells  
\hfill   Swarm of Flies\\
           \dbbook{\hspace*{25mm}} 
\hfill  Volley of Arrows
\end{quote}

           \vspace*{1mm}
           \sl ~~~ \hfill --- are all examples of Sets!\rm
           \vspace*{1mm}

           \mnewfoil
           
\bookdefn{Set}{
\begynd
\pind By a \textsl{set} we shall, loosely, understand an
unordered collection of distinct elements (i.e., entities),
\pind 
something for which it is meaningful to speak about 
\begynd
\pind (i) an entity \textsl{being a member} of a set (or not) \ISIN, 
\pind (ii) the \textsl{union} (merging) of two or more 
      sets into a set (of all the elements of the argument sets) \UNION, 
\pind (iii) the \textsl{intersection} of two or more sets into a set
      (of those elements which are in all argument sets) \INTER, 
\pind (iv)  the \textsl{complement} of one set with respect to another
      set $\setminus$, 
\pind (v) whether one set is a subset of another set
      \SUBSET\ and \SUBSETEQ\ and  
\pind (vi) the cardinality of a (finite) set (i.e., how many members it
      contains), \kw{card} and a few more% 
\afslut%
\afslut}
\index{sind}{\protect{\ISIN}\ set membership}% 
\index{sind}{\UNION!set union}%
\index{sind}{\INTER\ set intersection}%
\index{sind}{/!set difference}%
\index{sind}{\protect{$\setminus$}!set complement}%
\index{sind}{\SUBSET\ proper subset}%
\index{sind}{\SUBSETEQ\ subset}%
\index{sind}{\EQ\ equality!sets}%
\index{sind}{\NOTEQ\ in-equality!sets}%
\index{sind}{\kw{card}\ set cardinality}%

\mnewfoil

\noindent
\begynd
\pind The concepts of \textsl{sets} and \textsl{set elements} are
  left undefined.
\pind Above we have hinted at some set forming and other operations
over sets and their elements.
\pind What sets ``really are'' is usually defined in mathematics by
establishing what is called an axiom system.
\pind Axiomatically speaking,
  sets and their operations are what a number of axioms of a set theory
  define them to be! 
\newfoil
\pind There are several axiom systems for set theory.
\pind They each define a set theory. 
\pind The different set theories may
      therefore not be exactly the same.
\pind The perhaps best-known axiom system for set theory is that put
    forward by Zermelo/Fraenkel\pos{ (ZF)
    \cite{fraenkel1973,enderton1977}}{}\footnote{See,
    for example:\\  
http://plato.stanford.edu/entries/set-theory/ZF.html\\
http://mathworld.wolfram.com/Zermelo-FraenkelAxioms.html\\
http://planetmath.org/encyclopedia/ZermeloFraenkelAxioms.html\\
http://www.britannica.com/eb/article?tocId=24035, etc.}.  
\afslut

\mnewfoil

\begynd
\pind We refer to Appendix Sects.\,%
  \vref{tseb.rsl.Type Expressions} (set types),
  \vref{mono-rsl.Set Expressions} (set expressions), and
  \vref{mono-rsl.Set Operations} (set operations),
  for an overview of the set
      constructs of \texttt{RSL} \cite{RSL}.
\afslut

\nbbb{R{\^{o}}les of Set Theory in Computing}

\tbw

\nbbbb{Algebra}\label{Algebra}

\bookdefn{Algebra}{By an \textsl{algebra} we, loosely, mean
  a possibly infinite set of entities and a usually finite set of
  operations over these entities}


\noindent
\begynd
\pind Section\,\ref{The Implicit Meaning Theory}
      gave, on Page \,\pageref{algebra.1}, a first example
      of an algebra\,!
\afslut

\begynd
\pind In this \primer\ we shall again-and-again describe domain
      entities in terms of algebras: the types of their entities and
      operations and the definition of their operations, the latter sometimes, as in
      Sect.\,\ref{The Implicit Meaning Theory},  Page
      \,\pageref{algebra.1}, in terms of axioms, sometimes in terms of
      explicit function definitions. 
\afslut

\nbbb{Formal Definition of the Algebra Concept}

\begynd
\pind We shall primarily take an algebraic approach when
      determining, i.e., when deciding upon, the form of, and developing 
      software development descriptions.
\pind An \textsl{algebraic system}
\index{cind}{algebraic!system}\index{cind}{system!algebra}%
      is a set,\footnote{We usually do not say what the elements
      of this set are, it is just a set!\dbeat{ \dbbook{Of course, with our 
      background in programming, we could, for example, 
      state that the set
      elements were values (of some or any kind!).
      Some would call them entities, others might call them
      objects --- and now you see our problem. So, let us
      stick with mathematical practice and just say set.}}}
      $A$ (finite or infinite) of values, and
      a set\footnote{Similarly: Just a set!}, $\Omega$, (usually
      finite), of operations over these values: 
\afslut
\[ (A,\Omega) \] 
\[ A = \{a_1,a_2,...,a_m,...\}, 
   \Omega = \{\omega_1,\omega_2,...,\omega_o\} \] 
\noindent
\begynd
\pind Set $A$ is the \textsl{carrier}
\index{cind}{carrier of algebraic system}%
      of the algebraic system, and   
\pind $\Omega$ is a collection of operations defined on $A$.
\newfoil
\pind Each operation $\omega_i:\Omega$ ($\omega_i$ in $\Omega$, i.e.,
$\omega_i$ of type $\Omega$) is a
      function of some \textsl{arity},
\index{cind}{arity!of function} say $n$, taking operands,
      i.e., argument values
\index{cind}{argument!value}%
\index{cind}{value!argument}%
      in $A$, and yielding a result value in $A$:
\index{cind}{value!result}%
\index{cind}{result!value}%
\index{cind}{arity!of function, meta operator}%

\[ \omega(a_{i_1},a_{i_2},\ldots,a_{i_n}) = a \]


\noindent
\pind That is, $\omega_i$ is of type $A^n\rightarrow
A$.\pt{\footnote{The expression $A^n\rightarrow A$ is not an
    expression of \texttt{RSL.} First, we are explaining basic
    mathematical concepts not in \texttt{RSL} but in an informal
    notation of mathematics already assumed understood. Second, if
    we wish to express in \texttt{RSL} what may seem to be a
    Cartesian of arity $n$, for a known, fixed $n$, then we write it
    out in full:  $A_1\times A_1\times\cdots\times A_n$. If $n$ 
    varies, then it is probably not to be modelled, i.e., thought of, as
    a Cartesian, but rather as a list, $A^{\star}$, where $A$ is then
    the union type of all the $A_i$'s.}}{}
      Different functions (in $\Omega$) may have different
      arities. 
\pind Think of {arity} as a functional, a
      function that applies to functions and yields their
      arity:

\[\mbox{\kw{type } {\sf arity:}}\ \Omega \rightarrow
   \mbox{\kw{Nat}},\ \ \
   \mbox{{\sf arity}}(\omega_i) = n \]
\afslut
\label{chap2.tex.Discrete Mathematics.n}

\nbbb{Concrete Algebras}\label{ch2alg.ca}

\bookdefn{Concrete Algebra}{A \textsl{concrete algebra} 
      has sets of known, specific 
      values as carrier, and a set of specifically given operations}
    
\monoexample{Number Algebras}{Examples of concrete algebras are those
  over number:
\begin{itemize}
\item \textsl{An Integer Algebra:} (\textsf{Integer},$\{+,-,*\}$), an
      infinite carrier algebra whose operations yield all the integers.
\item \textsl{A Natural Numbers Algebra:} (\textsf{NatNumber},$\{${\sf
      gcd,lcm}$\}$) an infinite carrier algebra
      where \textsf{gcd, lcm} are the greatest common divisor,
      respectively the smallest common multiple 
      (viz.: \textsf{gcd(4,6)=2, lcm(4,6)=12}) operations, which
      yield all the natural numbers.
\item \textsl{A Modulo Natural Number Algebra:} ($\Im_m = \{0, 1, 2,
      \ldots,$ $m-1\}, \Omega =  
      \{\oplus, \otimes\}$) is a finite carrier algebra: $\oplus$
      and $\otimes$ are the addition and multiplication
      operations \textsf{modulo $m$.}
    \end{itemize}
  }
  
\nbbb{Abstract Algebras}\label{ch2alg.aa}

\begynd
\pind Whereas concrete algebras are known,
      i.e., effectively constructed, abstract algebras
\index{cind}{abstract!algebra}%
\index{cind}{algebra!abstract}%
      are postulated,  
\index{cind}{postulated!algebra}%
\index{cind}{algebra!postulated}%
      That is, they are what we shall call
      (and define as) `axiomatised' in \pt{Chap.~\ref{1ch8}}{}.  
\afslut

\bookdefn{Abstract Algebra}{An \textsl{abstract algebra} has a sort, i.e., a
  presently further undefined set of entities as carrier, a set of
  operations, and a set of axioms that relate (i.e., constrain)
  properties of carrier elements and operations}
\index{cind}{axiomatised!algebraic system}%

\noindent
\begynd
\pind The algebraic system of an abstract algebra is
      thus defined by a system of postulates, to be known
      henceforth as axioms. 
\index{cind}{axiom}% 
\afslut
 
\nbbb{Heterogeneous Algebras}\label{ch2alg.ha}

\bookdefn{Heterogeneous Algebra}{
\begynd
\pind A \textsl{heterogeneous algebra:}
\index{cind}{heterogeneous algebra}%
\index{cind}{algebra!heterogeneous}%
\[ (\{A_1,A_2,\ldots,A_m\},\Omega\}) \]

\noindent
\begynd
\pind has its carrier set $A$ be expressible as the 
      union of a set of disjoint sub-carriers $A_i$, and
\pind associates with every operation $\omega$ in $\Omega$
      a \textsf{signature}:
\index{cind}{signature!of operator, meta operator}%
\afslut
\afslut

\[ \mbox{\textsf{{signature}}}(\omega) = 
   A_{i_1}\times A_{i_2}\times \cdots\ \times A_{i_n} \rightarrow
   A_{i_{n+1}} \]
}

\noindent
\begynd
\pind has its carrier set $A$ be expressible as the 
      union of a set of disjoint sub-carriers $A_i$, and
\pind associates with every operation $\omega$ in $\Omega$
      a \textsf{signature}:
\index{cind}{signature!of operator, meta operator}%
\afslut
\afslut

\monoexample{Stack Algebra}{%%
Viewing the \textsf{stack} definition of Page\,\pageref{algebra.1} of
Sect.\,\ref{The Implicit Meaning Theory} as a heterogeneous
      algebra, the \textsf{stack} operations
      are (now) of the following signatures:
      $S$ is the stack type, and $E$ is the type of
      stack elements:
\begynd
\pind \textsf{empty:} \kw{Unit} $\rightarrow S$,
\pind \textsf{is\_empty:} $S$\ \RIGHTARROW\ \kw{Bool},
\pind \textsf{push:} $S \times\ E$ \RIGHTARROW\ $S$,
\pind \textsf{top:} $S$ \PARRIGHTARROW\ $E$, and 
\pind \textsf{pop:} $S$ \PARRIGHTARROW\ $S$.
\afslut
}

\nbbb{Universal Algebras}\label{ch2alg.ua}

\bookdefn{Universal Algebra}{
\begynd
\pind A \textsl{universal algebra}
\index{cind}{universal!algebra}%
\index{cind}{algebra!universal}%
      is a carrier and a set of operations
      with no postulates, i.e., the operations are
      not further constrained%
\afslut
}

\bb{The Morphism Concept}\label{ch2.alg.morph}

\begynd
\pind When, in software development we transform abstract
      specifications to more concrete ones, 
      then, usually, an \textsl{algebra morphism}
\index{cind}{morphism}\index{cind}{algebra!morphism}%
      is taking place.

\pind Let there be two  algebras:
\afslut

\[ (A,\Omega),  (A',\Omega') \]

\noindent%
\begynd%
\pind%
A function $\phi: A\rightarrow A'$ is said to be 
      a \textsl{morphism}
\index{cind}{morphism}%
      (also called a \textsl{homomorphism})
\index{cind}{homomorphism}%
      from  $(A,\Omega)$ to $(A',\Omega')$ if for any 
      $\omega\in\Omega$ and for any $a_1$, $a_2$, \ldots,
      $a_n$ in $A$ there is a corresponding $\omega'\in\Omega'$,
      such that:
\afslut
\label{ch2.alg.morph.def}

      \[ M:  \phi(\omega(a_1,a_2,\ldots,a_n))
        =  \omega'(\phi(a_1),\phi(a_2),\ldots,\phi(a_n)) \]

\noindent
\begynd
\pind We say that the homomorphism relation $M$ 
      \textsl{respects}
\index{cind}{homomorphism!respects operations}%
\index{cind}{respect operations}%
      or \textsl{preserves} corresponding operations
      in $\Omega$ and $\Omega'$ (Fig.~\pt{\vref}{\ref}{ch2.morph}). 
\index{cind}{homomorphism!preserves operations}%
\index{cind}{preserve operations}%

\newfoil
    
\DBfigure{morphism}{30mm}
         {Morphism mapping diagram}
         {ch2.morph}
     
\pind $\phi^n$ is the $n$-fold Cartesian power of
      $\phi:A\rightarrow A'$, that is, the map
      $A^n\rightarrow (A')^n$, and is defined by:

\[ \phi^n: (a_1,a_2,\ldots,a_n) \mapsto\
        (\phi(a_1),\phi(a_2),\ldots,\phi(a_n))\]
 
\noindent
\pind If $\phi:A\rightarrow A'$ is a homomorphism of
      $\Omega$-algebras, then, by definition $\phi$
      preserves all the operations of $\Omega$. 
\afslut

\nbbb{R{\^{o}}les of Algebra in Computing}

\tbw

\nbbb{Syntactic Means of Expressing Algebras}

\begynd
\pind To define the various carriers we define their
      \kw{type}s, and 
\pind to define the various operations over these carriers
      we define these as function \kw{value}s.
\newfoil
\pind Schematically:
\afslut

%\RSLatex
%class
%  type
%    A, B, C, D, ...
%  value
%    f: A -> B
%    f(a) is ...
%    g: C -> D
%    g(c) is ...
%    ...
%end
%\endRSLatex
\bp
\kw{class}\\
\>\kw{type}\\
\>\>A, B, C, D, {\DOTDOTDOT}\\
\>\kw{value}\\
\>\>f: A {\RIGHTARROW} B\\
\>\>f(a) {\IS} {\DOTDOTDOT}\\
\>\>g: C {\RIGHTARROW} D\\
\>\>g(c) {\IS} {\DOTDOTDOT}\\
\>\>{\DOTDOTDOT}\\
\kw{end}
\ep

\noindent
\begynd
\pind The above class expression defines carriers \textsf{A, B, C} and
      \textsf{D} (etcetera), and operations  \textsf{f} and  
      \textsf{g} (etcetera).
\afslut

\nbbbbb{Types, Sorts, Values and Functions}\label{chap2.tex.typ-val-fct}

\bbbb{Types and Sorts}

\begynd
\pind \sort{Types}\pdefindex{type} and
      \sort{functions}\pdefindex{function} are
      [usually\footnote{except for `sorts'}] abstract mathematical
      concepts\dbsquare\ 

\pind \sort{Sorts}\pdefindex{sort} are the types of domain
      endurants, and, as such, \sfsl{material
      concepts}\dbsquare\ 
\afslut

\bbbb{Values}

\begynd
\pind \sort{Values}\pdefindex{value} are either physical, i.e.,
      \sfsl{observable}\pconindex{observable} or
      measurable\pconindex{measurable}, hence rationally \sfsl{describable}
      phenomena (entities) of domains or are mathematical, hence
      formally describable abstract, immaterial concepts\dbsquare\
\afslut

\bbbb{Functions}

\begynd
\pind \sort{Function}\pdefindex{function}: A function is a
      mathematical concept. We say that a function is some
      mathematical thing which, when \sfsl{applied} to a
      \sfsl{value} (of its \sfsl{function definition set}) \sfsl{yields} either
      a \sfsl{value} (of its \sfsl{function range}) or
      \sort{chaos}\plitindex{chaos@\sort{chaos}} -- a further undefined
      quantity -- and renders the expression in which this
      \sfsl{function invocation} [\sfsl{function call}] occurs
      meaningless\dbsquare\
     
\pind \sort{Function
      signatures}\pdefindex{function!signature}\pdefindex{signature!function}, 
      \sfsl{syntactically} are pairs of a \sfsl{function
      name}\pconindex{function!name}\pconindex{name!function} and a \sfsl{function 
      type}\pconindex{function!type}.


%\RSLatex
%  type  F >< (Type_Expr >< ( -> | -~-> ) >< Type_Expr)object
%\endRSLatex 
\bp
\>\kw{type}\ \ F {\TIMES} (Type\_Expr {\TIMES} ( {\RIGHTARROW} {\BAR} {\PARRIGHTARROW} ) {\TIMES} Type\_Expr)
\ep

    
      A function
      name\pdefindex{function!name}\pconindex{name!function}, \sfsl{syntactically}, is some
      arbitrarily chosen identifier.
      
      A function type, \sfsl{syntactically}, is a triple of a sort or type expression, a \RIGHTARROW\
      (total function type symbol) or a \PARRIGHTARROW\ (partial
      function type symbol), and a
      sort\pconindex{sort!expression}\pconindex{expression!sort} or
      type expression\pconindex{expression!type}.
      
      Function types, \sfsl{semantically}, designate a type of
      function values\dbsquare\

\noindent
\treprikker
      
      Sort and type expressions will be defined in
      Sect.\,\vref{tseb.rsl.Type Expressions}.
      
      Henceforth we shall use the term `sort' when referring to
      external quality domain endurants and `type' otherwise,
      including, notably, internal qualities.
\afslut
 
\nbbbb{R{\^{o}}les of Types, Sorts, Values and Functions in Computing}

\tbw

\label{chap2.tex.typ-val-fct.n}

\nbbbbb{Recursive Function Theory}\label{chap2.tex.Recursive Function Theory}

\nbbbb{Introduction}\label{chap2.Introduction}

The mathematical concept of recursive function theory stems from work
by \textsf{Kleene, Turing}, and \textsf{Post} starting in the late
1930s and from work by \textsf{Church} on the \texttt{Lambda Calculus}.

\nbbbb{A Lambda Calculus}\label{chap2.A Lambda Calculus}

\begynd
\pind We refer to the appendix on the \texttt{RSL} specification
      language Appendix\,\vref{tseb.rsl.Lambda Calculus + Functions}.
\afslut

\nbbbb{Recursive Functions}\label{chap2.Recursive Funtions}

\begynd
\pind \sfsl{Recursive functions are a class of functions on the natural
      numbers studied in computability
      theory}\footnote{https://plato.stanford.edu/entries/recursive-functions}. 
\pind We refer to 
\begynd
\pind \texttt{plato.stanford.edu/entries/recursive-functions/\#Defi}
      for the definition of \sfsl{primitive recursive functions}, 
\pind \texttt{plato.stanford.edu/entries/recursive-func\-tions/\#De\-fi\_1}
      for the definition of \sfsl{partial recursive functions} and the
      \sfsl{recursive functions}.
\pind Borrowing from \texttt{plato.stanford.edu/entries/recursive-functions/} we sumarise:
\afslut

\nbbb{The Primitive Recursive Functions}

\newcounter{rfi}\newcounter{rfii}\newcounter{rfiii}\newcounter{rfiv}\newcounter{rfv}
\newcommand{\srfc}[1]{\addtocounter{enumi}{1}\setcounter{#1}{\value{enumi}}\arabic{#1}}

\begin{tabular}{lrcrcl}
\srfc{rfi}. \ \ & zero & & & $0$ & \\
\srfc{rfii}. \ \  & successor & & $s(x)$ & = & $x+1$ \\
\srfc{rfiii}. \ \  & $k$ary proj.\ fct.\ on the
  $i$th arg.  & & $\pi^k_i(x_0,...,x_i,...,x_{k-1})$ &= & $x_i$ where $0\leq{i}\leq{k}$ \\ 
\srfc{rfiv}. \ \ &  composition & &
    $h(x_0,...,x_{k-1})$&=&$f(g_0(x_0,...,x_{k-1}),...,g_{j−1}(x_0,...,x_{k-1}))$\\ 
\srfc{rfv}. \ \  & primitive recursion & &$h(x_0,...,x_{k-1},0)$ &=& $f(x_0,...,x_{k-1})$\\
\arabic{rfv}.  \ \ & & & $h(x_0,...,x_{k-1},y+1)$  &= &  $g(x_0,...,x_{k-1},y,h(x_0,...,x_{k-1},y))$\\
\end{tabular}

\begin{itemize}
\item[\arabic{rfi}.]
\item[\arabic{rfii}.]
\item[\arabic{rfiii}.]
\item[\arabic{rfiv}.]
\item[\arabic{rfv}.]
\end{itemize}
 
 


\vspace{1mm}
 
\noindent
\begynd
\pind With the above (=) definitions one can now show
\begynd
\pind the encoding of all the familiar functions over natural numbers.
\afslut
\pind 
\afslut

\nbbb{R{\^{o}}les of Recursive Functions in Computing}

\tbw

\nbbbbb{The \texttt{CSP} Story}\label{The CSP Story}

\begynd
\pind \texttt{CSP} is a wonderful tool, i.e., a language
\begynd
\pind with which to study  and describe \sfsl{communicating sequential
  processes}. 
\afslut
\pind It is the invention of \sfsl{Charles Anthony Richard Hoare}.
\pind Major publications on \texttt{CSP} are
      \cite{Hoa78a,Hoare85+2004,Roscoe97,Schneider99}.
\afslut

\nbbbb{Informal Presentation}
\begynd
\pind \texttt{CSP} processes (models of domain behaviors) $P_i, P_j, ..., P_k$ can proceed in
      parallel:
%\RSLatex
%        P_i || P_j || ... || P_k
%\endRSLatex 
\bp
\>\>\>\>P\_i {\PARL} P\_j {\PARL} {\DOTDOTDOT} {\PARL} P\_k
\ep
\pind Behaviours
\begynd
\pind sometimes synchronise 
\pind and usually communicate.
\afslut
\pos{\psno}{\mnewfoil}
\pind Synchronisation and communication is abstracted as 
\begynd
\pind the sending (\textsf{ch\,!\,m}) and 
\pind receipt (\textsf{ch\,?})
\pind of messages, \textsf{m:M}, 
\pind over channels, \textsf{ch}.
\afslut
%\RSLatex
%    type M
%    channel ch:M
%\endRSLatex 
\bp
\>\>\kw{type} M\\
\>\>\kw{channel} ch:M
\ep
\pos{\psno}{\mnewfoil}
\pind Communication between (unique identifier) indexed behaviours \nyl
      have their channels modelled as similarly indexed channels:
%\RSLatex
%    out:        ch[idx]!m
%    in:          ch[idx]?
%    channel  {ch[ide]:M|ide:IDE}
%\endRSLatex 
\bp
\>\>\kw{out}:\ \ \ \ \ \ \ \ ch{\LBRACKET}idx{\RBRACKET}!m\\
\>\>\kw{in}:\ \ \ \ \ \ \ \ \ \ ch{\LBRACKET}idx{\RBRACKET}?\\
\>\>\kw{channel}\ \ {\LBRACE}ch{\LBRACKET}ide{\RBRACKET}:M{\BAR}ide:IDE{\RBRACE}
\ep
    where \textsf{IDE} typically is some type expression \nyl over
    unique identifier types.
\afslut
\pos{\psno}{\mnewfoil}


\begynd
\pind The expression
%\RSLatex
%   P_i |^| P_j |^| ... |^| P_k
%\endRSLatex 
\bp
\>\ P\_i {\NONDETCHOICE} P\_j {\NONDETCHOICE} {\DOTDOTDOT} {\NONDETCHOICE} P\_k
\ep
\begynd
\pind can be understood as a choice:
\pos{}{\bmcii}
\begynd
\pind either \textsf{P\_i}, 
\pind or \textsf{P\_j},
\pind or ...
\pind or \textsf{P\_k}
\afslut \pos{}{\emcii} as \sfsl{non-deterministically \brcolor{internally}} chosen
\pind with no stipulation as to why\,!
\afslut
\afslut
\pos{\psno}{\mnewfoil}%\pos{\psno}{\mnewfoil}


\begynd
\pind The expression
%\RSLatex
%   P_i |=| P_j |=| ... |=| P_k
%\endRSLatex 
\bp
\>\ P\_i {\DETCHOICE} P\_j {\DETCHOICE} {\DOTDOTDOT} {\DETCHOICE} P\_k
\ep
\begynd
\pind can be understood as a choice:
\pos{}{\bmcii}
\begynd
\pind either \textsf{P\_i}, 
\pind or \textsf{P\_j},
\pind or ...
\pind or \textsf{P\_k}
\afslut \pos{}{\emcii} as \sfsl{deterministically \brcolor{externally}} chosen
\pind on the basis that the one chosen offers to participate
      \pos{}{\\} in either an input, \textsf{ch\,?}, or an output,
      \textsf{ch\,!\,msg}, event.
\pind If more than one   \textsf{P\_i} offers a communication \pos{}{\\} then one
      is arbitrarily chosen.
\pind If no  \textsf{P\_i} offers a communication the behaviour halts \pos{}{\\} 
      till some  \textsf{P\_j} offers a communication.
\afslut
\afslut

\nbbbb{A Syntax for \texttt{CSP}}\label{A Syntax for CSP}


\begynd
\pind We present the syntax for the \texttt{CSP} such as it is used in
this \primer.
\afslut

\begin{center}
  \begin{tabular}{rcll}
     \textsf{DB} & ::= & \textsf{AB  {\PARL} AB {\PARL} ...  {\PARL} AB}& \ \ \ Domain Behaviour \\
     \textsf{AB} & ::= & \textsf{A} ; \textsf{A} ; ... ; \textsf{A}; $\mathcal{B}$ &  \  \ \ Part Behaviour \\
     \textsf{A} & ::= & \kw{stop} & \ \ \ behaviour stops \\
      & $\mid$ & \kw{skip} & \ \ \ no effect \\
      & $\mid$ &  \textsf{A} ; \textsf{A} ; ... ; \textsf{A} &  \ \ \ sequential composition \\
      & $\mid$ & \textsf{A {\PARL} A {\PARL} ...  {\PARL} A}  & \ \ \ parallel composition (interleave) \\
      & $\mid$ & \textsf{{\PARL} $\{$ A(i) {\BAR} i:Index ... } $\}$  & \ \ \ distributed parallel composition (interleave) \\
      & $\mid$ & \textsf{A {\NONDETCHOICE} A {\NONDETCHOICE} ... {\NONDETCHOICE} A} &  \ \ \ internal
      non-deterministic choice \\
      & $\mid$ & \textsf{{\NONDETCHOICE} $\{$ A(i) {\BAR} i:Index
     ... } $\}$  & \ \ \ distributed internal
      non-deterministic choice \\
      & $\mid$ & \textsf{A {\DETCHOICE} A {\DETCHOICE} ...{\DETCHOICE} A}  &  \ \ \ external
      non-deterministic choice \\
      & $\mid$ & \textsf{{\DETCHOICE} $\{$ A(i) {\BAR} i:Index ... } $\}$  & \ \ \ distributed external
      non-deterministic choice \\
      & $\mid$ & \textsf{\kw{if} Bool \kw{then} A \kw{else} A \kw{end}} & \ \ \  Boolean conditional\\
      & $\mid$ & \textsf{\kw{let} v = ch\,? \kw{in} A \kw{end}}  &  \ \ \ input
      value \textsf{v} on \kw{channel} \textsf{ch} \\
      & $\mid$ & \textsf{ch\,!\,e ; A} &  \ \ \ output value of expression \textsf{e} on \kw{channel} \textsf{ch}
  \end{tabular}  
\end{center}

\nbbbb{R{\^{o}}les of \texttt{CSP} in Computing}

\tbw

\nbbbbb{Closing}\label{chap2.tex.Closing}

\label{chap2.tex.n}


%%  LocalWords:  CSP combinatorics cardinality sind Zermelo Fraenkel
%%  LocalWords:  ZF rsl endurants Expr Hoare Behaviours synchronise
%%  LocalWords:  Synchronisation behaviours modelled idx ide msg rcll
%%  LocalWords:  deterministically behaviour Bool
